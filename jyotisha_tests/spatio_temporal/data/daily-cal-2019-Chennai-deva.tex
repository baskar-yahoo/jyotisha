% !Tex program = xelatex
\documentclass[12pt]{article}
\usepackage[dvipsnames]{xcolor} 
\usepackage[paperwidth=135mm,paperheight=180mm,left=4mm,right=4mm,top=8mm,bottom=8mm]{geometry}
\usepackage[xetex]{graphicx}
\usepackage{array}
\usepackage{setspace}
\usepackage{multirow}
\usepackage{pxfonts}
\usepackage{bbding}
\usepackage{wasysym} 
\usepackage{fontspec}
\usepackage{multicol}
\usepackage{supertabular}
\usepackage{fancyhdr}
\pagestyle{fancy}
\fancyhf{}
\rhead{}
\lhead{}
\cfoot{}
\usepackage{../templates/listofitems}
\newcommand{\yearname}{2015}
%%%%%%%%%%%%%%%%%%%%%%%%%%%%%%%%%%%%%%%%%%%%%%%%%%%%%%%%%%%%%%%%%%%%%%%%%%%%%%%
%% MOONPHASE CODE 
%%%%%%%%%%%%%%%%%%%%%%%%%%%%%%%%%%%%%%%%%%%%%%%%%%%%%%%%%%%%%%%%%%%%%%%%%%%%%%%
%Credits: http://tex.stackexchange.com/questions/34785/how-to-typeset-moon-phase-symbols (Jake!)
\usepackage{tikz}
\usetikzlibrary{calendar,fpu}

\tikzset{
    moon colour/.style={
        moon fill/.style={
            fill=#1
        }
    },
    sky colour/.style={
        sky draw/.style={
            draw=#1
        },
        sky fill/.style={
            fill=#1
        }
    },
    southern hemisphere/.style={
        rotate=180
    }
}

\makeatletter
\pgfcalendardatetojulian{2010-01-15}{\c@pgf@counta} % 2010-01-15 07:11 UTC -- http://aa.usno.navy.mil/cgi-bin/aa_moonphases.pl?year=2010&ZZZ=END
\def\synodicmonth{29.530588853}
\newcommand{\moon}[2][]{%
    \edef\checkfordate{\noexpand\in@{-}{#2}}%
    \checkfordate%
    \ifin@%
        \pgfcalendardatetojulian{#2}{\c@pgf@countb}%
        \pgfkeys{/pgf/fpu=true,/pgf/fpu/output format=fixed}%
        \pgfmathsetmacro\dayssincenewmoon{\the\c@pgf@countb-\the\c@pgf@counta-(7/24+11/(24*60))}%
        \pgfmathsetmacro\lunarage{mod(\dayssincenewmoon,\synodicmonth)}
        \pgfkeys{/pgf/fpu=false}%%
    \else%
        \def\lunarage{#2}%
    \fi%
    \pgfmathsetmacro\leftside{ifthenelse(\lunarage<=\synodicmonth/2,cos(360*(\lunarage/\synodicmonth)),1)}%
    \pgfmathsetmacro\rightside{ifthenelse(\lunarage<=\synodicmonth/2,-1,-cos(360*(\lunarage/\synodicmonth))}%
    \tikz [moon colour=white,sky colour=black,#1]{
        \draw [moon fill, sky draw] (0,0) circle [radius=1ex];
        \draw [sky draw, sky fill] (0,1ex)
            arc (90:-90:\rightside ex and 1ex)
            arc (-90:90:\leftside ex and 1ex)
            -- cycle;
    }%
}
%%%%%%%%%%%%%%%%%%%%%%%%%%%%%%%%%%%%%%%%%%%%%%%%%%%%%%%%%%%%%%%%%%%%%%%%%%%%%%%
%% END MOONPHASE CODE
%%%%%%%%%%%%%%%%%%%%%%%%%%%%%%%%%%%%%%%%%%%%%%%%%%%%%%%%%%%%%%%%%%%%%%%%%%%%%%%
% \setlength{\footskip}{2mm}
% PDF SETUP
% ---- FILL IN HERE THE DOC TITLE AND AUTHOR
\defaultfontfeatures{Scale=MatchLowercase,Mapping=tex-text}
\setmainfont{siddhanta.ttf}[Path=../fonts/,Script=Devanagari] 
\setsansfont[Path=../fonts/,Scale=0.95,Numbers=Lining]{AlegreyaSans-Regular.ttf}
% \newfontfamily\noto[Path=../fonts/, Ligatures=TeX]{NotoSansUI-Regular}
%%%%%%% Numbers and counters %%%%%%%
\newcount\num
\newcount\tempone \newcount\temptwo
\newcommand{\devanumber}[1]{%
\num=#1\devanumberrecurse}
\setlength{\textheight}{150mm}
\newcommand{\devanumberrecurse}{%
{\tempone=\num
%  \showthe\tempone\ %
\ifnum\num > 0 
    \divide \num by 10%
    \temptwo=\num \multiply\temptwo by -10%
    \devanumberrecurse%
%   \\stage 2\ %
%   \showthe\temptwo\ %
%   temp1=\number\tempone\ %
%   num=\number\num\ %
    \advance\tempone by \temptwo%
    \devadigit
\fi
}}
\newcommand{\devadigit}{%
\ifcase\tempone०\or१\or२\or३\or४\or५\or६\or७\or८\or९\fi%\number\tempone%
}
\newcommand{\eventsep}{~\raisebox{1pt}{\scriptsize$\Diamondblack$} }
\newcommand{\TO}{\hspace{1pt}\raisebox{1pt}{\footnotesize\RIGHTarrow}\hspace{1pt}}
\newcommand{\To}{\hspace{1pt}\raisebox{1pt}{\footnotesize\RIGHTarrow}\hspace{1pt}}
\newcommand{\Too}{\hspace{1pt}\raisebox{1pt}{\footnotesize\RIGHTarrow\hspace{-5pt}\RIGHTarrow}\hspace{1pt}}
%%%%%%%% Calendar display stuff %%%%%%%%%%%
\newcommand{\samvatsaraName}{}
\newcommand{\solarMonthName}{}
\newcommand{\solarMonthEndTime}{}
\newcommand{\lunarMonthName}{}
\newcommand{\lunarRtu}{}
\newcommand{\solarMonthDate}{}
\newcommand{\vaaraName}{}
\newcommand{\rtuName}{}
\newcommand{\ayanamName}{}

\newcommand{\sunmonth}[9]{%
\renewcommand{\solarMonthName}{#1}
\renewcommand{\solarMonthDate}{#2}
\renewcommand{\solarMonthEndTime}{#3}
\renewcommand{\lunarMonthName}{#4}
\renewcommand{\lunarRtu}{#5}
\renewcommand{\vaaraName}{#6}
\renewcommand{\samvatsaraName}{#7}
\renewcommand{\ayanamName}{#8}
\renewcommand{\rtuName}{#9}
}
\newcommand{\tamil}[1]{%
{\fontspec[Scale=0.8,FakeStretch=0.9,Path=../fonts/]{NotoSansTamil-Regular.ttf} \footnotesize #1}%
}
\newcommand{\kalas}[1]{%
\setsepchar{ }
\readlist\arg{#1}
{\small{\mbox{ब्राह्म\,\textsf{\arg[1]}{\scriptsize\RIGHTarrow}\,सङ्गव\,\textsf{\arg[4]}{\scriptsize\RIGHTarrow}\,मध्याह्न\,\textsf{\arg[7]}{\scriptsize\RIGHTarrow}\,अपराह्ण\,\textsf{\arg[8]}{\scriptsize\RIGHTarrow}\,सायाह्न\,\textsf{\arg[9]}{\scriptsize\RIGHTarrow}}\hfill {दिनान्तः{\scriptsize\RIGHTarrow}\textsf{\arg[14]}}}}\\[-1ex]
{\small{\mbox{प्रातः सन्ध्या \textsf{\arg[2]}{\scriptsize\RIGHTarrow}\textsf{\arg[3]} माध्याह्निक \textsf{\arg[5]}{\scriptsize\RIGHTarrow}\textsf{\arg[6]} 
सायं \textsf{\arg[10]}{\scriptsize\RIGHTarrow}\textsf{\arg[11]}}\hfill शयन \textsf{\arg[12]}{\scriptsize\RIGHTarrow}\textsf{\arg[13]}}\\[-4.5ex]}
}
\newcommand{\sunmoonrsdata}[6]{%
\mbox{\large\sun{\small\UParrow}{#1}~~\sun{#5}~~\sun{\small\DOWNarrow}{#2}\hspace{1ex}}\hfill
\mbox{\large\rightmoon{\small\UParrow}{#3}~\rightmoon{\small\DOWNarrow}{#4}}\\
#6
 }
\newcommand{\sunmoonsrdata}[6]{%
\mbox{\large\sun{\small\UParrow}{#1}~~\sun{#5}~~\sun{\small\DOWNarrow}{#2}\hspace{1ex}}\hfill
\mbox{\large\rightmoon{\small\DOWNarrow}{#4}~\rightmoon{\small\UParrow}{#3}}\\
#6
 }
\newcommand{\ahorAtram}{अहोरात्रम्}
\newcommand{\tithi}[2]{\raisebox{-1pt}{\moon[scale=0.8]{#1}}\hspace{2pt}#2}
\newcommand{\tnykdata}[6]{\large%\fontsize{13pt}{16pt}\selectfont
{#1}\\%Tithi
{नक्षत्रम्–#2 (#3)}\\%Nakshatram and Rashi
{\setstretch{0.55}
\begin{tabular}{@{}r@{}p{108mm}@{}}
योगः–&#4\\[2pt]%Yogam
करणम्–&#5\\%Karanam
\end{tabular}}\mbox{}\\[3pt]
\parbox[c][2ex][c]{0.9\linewidth}{\footnotesize #6}%Lagna, if required
}
\newcommand{\avamA}{
    \raisebox{1.5pt}{\fcolorbox{white}{gray!40}{\scriptsize अवमा}}
}
\newcommand{\tridina}{
    \raisebox{1.5pt}{\fcolorbox{white}{gray!40}{\scriptsize त्रिदिनस्पृक्}}
}
\renewcommand{\time}[2]{#1 (#2)}
\newcommand{\anga}[2]{\mbox{#1\To{}\textsf{#2}}}
\newcommand{\fullanga}[1]{\mbox{#1\To{}\ahorAtram}}
\newcommand{\fulltithi}[1]{\mbox{#1\To{}\ahorAtram\tridina}}
\newcommand{\lagna}[2]{\mbox{#1\RIGHTarrow\textsf{#2}}}
\newcommand{\uanga}[1]{\mbox{#1\Too}}
\newcommand{\rygdata}[3]{%
\begin{minipage}{\linewidth}
\centering
\rule[-1ex]{0.7\textwidth}{.4pt}
\small राहु॰~\textsf{#1}~~यम॰~\textsf{#2}~~गुलिक॰~\textsf{#3}%Rahu Yama Gulika
\end{minipage}
}
\newcommand{\prev}[1]{\textcolor{gray}{#1}}
\newcommand{\caldata}[7]{%
\clearpage
\begin{minipage}{\linewidth}
#3% Calls \sunmonth
\large% Fixes font size
{\centering\begin{tabular}{c|c}
\large \textsf{\yearname} & {\large\samvatsaraName}\\[-1ex]%YYYY
& {\footnotesize \ayanamName \hspace{6pt} \rtuName}\\[0.2ex]
\mbox{\sffamily\fontsize{20}{25}\selectfont {\uppercase{#1}}} & \parbox[c][14pt][c]{3cm}{\centering\LARGE\solarMonthName}\\[-4pt]%mmm
& {\mbox{\small \solarMonthEndTime}}\\[-2pt]
& {\parbox[c][15pt]{52mm}{\centering\lunarMonthName}}\\[-6pt]
& {\parbox[c][10pt]{52mm}{\centering\scriptsize(\lunarRtu)}}\\[-5pt]
\hspace{0.465\linewidth} & \hspace{0.465\linewidth} \\[-6pt]
\mbox{\sffamily\fontsize{96}{115}\selectfont #2} & \mbox{\fontsize{90}{24}\selectfont \devanumber{\solarMonthDate}}\\[1.6ex]%DD
\mbox{\sffamily\fontsize{24}{28}\selectfont\uppercase{#7}} & \parbox[c][24pt][t]{1cm}{\centering\LARGE\vaaraName}\\[1.2ex]%Day of the week
\hline
\end{tabular}
}\mbox{}\\[-4pt]
#4\\[0.5em]%Sun rise, kalas etc
#5\mbox{}\\[1em]%Tithi, Nakshatram, Varam, Yogam
% \vspace{\fill}
{\parbox[b]{0.95\linewidth}{\centering\normalsize\textcolor{RoyalBlue}{#6}}}%Festivals
\end{minipage}
}

\addtolength{\headsep}{-3ex}
\setlength\parindent{0pt}
\pagestyle{empty}
\begin{document}
\mbox{}
\renewcommand{\yearname}{2019}
\begin{center}
{\sffamily \fontsize{80}{80}\selectfont  2019\\[0.5cm]}
\mbox{\fontsize{48}{48}\selectfont विलम्बः–विकारी}\\
\mbox{\fontsize{32}{32}\selectfont कलि } %
{\sffamily \fontsize{43}{43}\selectfont  5119–5120\\[0.5cm]}
\hrule
\vspace{0.2cm}
{\sffamily \fontsize{50}{50}\selectfont  \uppercase{Chennai}\\[0.2cm]}
{\sffamily \fontsize{23}{23}\selectfont  {13.090°N, 80.270°E}\\[0.2cm]}
\hrule
\end{center}
\clearpage\pagestyle{fancy}
\caldata{JANUARY}{1}{\sunmonth{धनुः}{17}{}{मार्गशीर्षः}{हेमन्तऋतुः}{मङ्गलः}{विलम्बः}{दक्षिणायनम्}{हेमन्तऋतुः}}
{\sunmoonsrdata{06:34}{17:49}{03:08*}{14:08}{12:12}
{\kalas{04:52 05:43 09:34 08:49 10:19 16:19 11:04 13:19 15:34 17:04 18:40 21:01 22:36 01:48*}}}
{\tnykdata{\anga{\tithi{26}{कृष्ण-एकादशी}}{\time{47-59}{01:28*}}\hspace{1ex}}%
{\anga{स्वाती}{\time{5-39}{08:42}}\hspace{1ex}}{चन्द्रराशिः—\mbox{तुला\RIGHTarrow{03:20*}}}%
{\anga{धृतिः}{\time{50-40}{02:37*}}\hspace{1ex}\uanga{शूलः}}%
{\anga{बवम्}{\time{17-57}{13:18}}\hspace{1ex}\anga{बालवम्}{\time{47-59}{01:28*}}\hspace{1ex}\uanga{कौलवम्}}{}
}
{सर्व-सफला-एकादशी\eventsep त्रिपुष्कर-योगः~01:28*\RIGHTarrow{}06:35*}
{Tue} 
\cfoot{\rygdata{15:01--16:25}{09:23--10:48}{12:12--13:36}}
\caldata{JANUARY}{2}{\sunmonth{धनुः}{18}{}{मार्गशीर्षः}{हेमन्तऋतुः}{बुधः}{विलम्बः}{दक्षिणायनम्}{हेमन्तऋतुः}}
{\sunmoonsrdata{06:35}{17:50}{04:02*}{14:54}{12:12}
{\kalas{04:53 05:44 09:35 08:50 10:20 16:20 11:05 13:20 15:35 17:05 18:41 21:01 22:37 01:48*}}}
{\tnykdata{\anga{\tithi{27}{कृष्ण-द्वादशी}}{\time{49-37}{02:10*}}\hspace{1ex}}%
{\anga{विशाखा}{\time{8-5}{09:37}}\hspace{1ex}}{चन्द्रराशिः—\mbox{वृश्चिकः}}%
{\anga{शूलः}{\time{49-10}{01:59*}}\hspace{1ex}\uanga{गण्डः}}%
{\anga{कौलवम्}{\time{19-8}{13:46}}\hspace{1ex}\anga{तैतिलम्}{\time{49-37}{02:10*}}\hspace{1ex}\uanga{गरजा}}{}
}
{बुधानुराधा-योगः~09:37\RIGHTarrow{}\eventsep हरिवासरः\RIGHTarrow{}07:36\eventsep काञ्ची ६८ जगद्गुरु श्री-चन्द्रशेखरेन्द्र सरस्वती ७ आराधना~\#{२५}\eventsep पक्षवर्धिनी-महाद्वादशी\eventsep \tamil{உந்து~மதக்களிற்றன்}}
{Wed} 
\cfoot{\rygdata{12:12--13:37}{07:59--09:24}{10:48--12:12}}
\caldata{JANUARY}{3}{\sunmonth{धनुः}{19}{}{मार्गशीर्षः}{हेमन्तऋतुः}{गुरुः}{विलम्बः}{दक्षिणायनम्}{हेमन्तऋतुः}}
{\sunmoonsrdata{06:35}{17:50}{04:55*}{15:43}{12:13}
{\kalas{04:53 05:44 09:35 08:50 10:20 16:20 11:05 13:20 15:35 17:05 18:41 21:02 22:37 01:49*}}}
{\tnykdata{\anga{\tithi{28}{कृष्ण-त्रयोदशी}}{\time{52-21}{03:21*}}\hspace{1ex}}%
{\anga{अनूराधा}{\time{11-47}{11:01}}\hspace{1ex}}{चन्द्रराशिः—\mbox{वृश्चिकः}}%
{\anga{गण्डः}{\time{48-30}{01:43*}}\hspace{1ex}\uanga{वृद्धिः}}%
{\anga{गरजा}{\time{21-37}{14:42}}\hspace{1ex}\anga{वणिजा}{\time{52-21}{03:21*}}\hspace{1ex}\uanga{भद्रा}}{}
}
{प्रदोष-व्रतम्~17:50\RIGHTarrow{}19:26}
{Thu} 
\cfoot{\rygdata{13:37--15:02}{06:35--08:00}{09:24--10:48}}
\caldata{JANUARY}{4}{\sunmonth{धनुः}{20}{}{मार्गशीर्षः}{हेमन्तऋतुः}{शुक्रः}{विलम्बः}{दक्षिणायनम्}{हेमन्तऋतुः}}
{\sunmoonsrdata{06:36}{17:51}{05:46*}{16:33}{12:13}
{\kalas{04:54 05:45 09:36 08:51 10:21 16:21 11:06 13:21 15:36 17:06 18:42 21:02 22:38 01:49*}}}
{\tnykdata{\anga{\tithi{29}{कृष्ण-चतुर्दशी}}{\time{56-8}{04:57*}}\hspace{1ex}}%
{\anga{ज्येष्ठा}{\time{16-39}{12:51}}\hspace{1ex}}{चन्द्रराशिः—\mbox{वृश्चिकः\RIGHTarrow{12:51}}}%
{\anga{वृद्धिः}{\time{48-38}{01:46*}}\hspace{1ex}\uanga{ध्रुवः}}%
{\anga{भद्रा}{\time{25-20}{16:06}}\hspace{1ex}\anga{शकुनिः}{\time{56-8}{04:57*}}\hspace{1ex}\uanga{चतुष्पात्}}{}
}
{अनध्यायः\eventsep मासशिवरात्रिः\eventsep पञ्च-पर्व-पूजा (चतुर्दशी)\eventsep \tamil{தொண்டரடிப்பொடியாழ்வார் திருநக்ஷத்திரம்}}
{Fri} 
\cfoot{\rygdata{10:49--12:13}{15:02--16:26}{08:00--09:24}}
\caldata{JANUARY}{5}{\sunmonth{धनुः}{21}{}{मार्गशीर्षः}{हेमन्तऋतुः}{शनिः}{विलम्बः}{दक्षिणायनम्}{हेमन्तऋतुः}}
{\sunmoonsrdata{06:36}{17:52}{06:36*}{17:24}{12:14}
{\kalas{04:54 05:45 09:36 08:51 10:21 16:21 11:06 13:21 15:36 17:07 18:43 21:03 22:38 01:49*}}}
{\tnykdata{\prev{\anga{\tithi{29}{कृष्ण-चतुर्दशी}}{\time{*56-8}{04:57}}}\hspace{1ex}\fulltithi{\tithi{30}{अमावास्या}}}%
{\anga{मूला}{\time{22-36}{15:05}}\hspace{1ex}}{चन्द्रराशिः—\mbox{धनुः}}%
{\anga{ध्रुवः}{\time{49-27}{02:08*}}\hspace{1ex}\uanga{व्याघातः}}%
{\prev{\anga{शकुनिः}{\time{*56-8}{04:57}}}\hspace{1ex}\anga{चतुष्पात्}{\time{30-7}{17:55}}\hspace{1ex}\uanga{नाग}}{}
}
{अनध्यायः\eventsep काञ्ची १४ जगद्गुरु श्री-विद्याघनेन्द्र सरस्वती आराधना~\#{१७०२}\eventsep काञ्ची ३४ जगद्गुरु श्री-चन्द्रशेखरेन्द्र सरस्वती २ आराधना~\#{१३०९}\eventsep कमला-जयन्ती\eventsep ख्येचिमावस पोष्त/खिचडी-अमावास्या\eventsep पार्वणव्रतम् अमावास्यायाम्\eventsep पञ्च-पर्व-पूजा (अमावास्या)\eventsep सर्व-मार्गशीर्ष-अमावास्या\eventsep श्री-हनूमत्-जयन्ती}
{Sat} 
\cfoot{\rygdata{09:25--10:49}{13:38--15:03}{06:36--08:00}}
\caldata{JANUARY}{6}{\sunmonth{धनुः}{22}{}{मार्गशीर्षः}{हेमन्तऋतुः}{भानुः}{विलम्बः}{दक्षिणायनम्}{हेमन्तऋतुः}}
{\sunmoonsrdata{06:36}{17:52}{---}{18:15}{12:14}
{\kalas{04:54 05:45 09:36 08:51 10:21 16:22 11:07 13:22 15:37 17:07 18:43 21:03 22:39 01:50*}}}
{\tnykdata{\anga{\tithi{30}{अमावास्या}}{\time{0-57}{06:58}}\hspace{1ex}}%
{\anga{पूर्वाषाढा}{\time{29-28}{17:40}}\hspace{1ex}}{चन्द्रराशिः—\mbox{धनुः\RIGHTarrow{00:22*}}}%
{\anga{व्याघातः}{\time{50-53}{02:44*}}\hspace{1ex}\uanga{हर्षणः}}%
{\anga{नाग}{\time{0-57}{06:58}}\hspace{1ex}\anga{किंस्तुघ्नः}{\time{35-14}{20:06}}\hspace{1ex}\uanga{बवम्}}{}
}
{\tamil{சாக்கிய நாயன்மார் (34) குருபூஜை}\eventsep दर्शेष्टिः\eventsep पार्वण-प्रायश्चित्तावकाशः दर्शे\eventsep पिण्ड-पितृ-यज्ञः\eventsep दर्श-स्थालीपाकः}
{Sun} 
\cfoot{\rygdata{16:28--17:52}{12:14--13:39}{15:03--16:28}}
\caldata{JANUARY}{7}{\sunmonth{धनुः}{23}{}{पौषः}{हेमन्तऋतुः}{सोमः}{विलम्बः}{दक्षिणायनम्}{हेमन्तऋतुः}}
{\sunmoonrsdata{06:37}{17:53}{07:22}{19:05}{12:15}
{\kalas{04:55 05:46 09:37 08:52 10:22 16:23 11:07 13:22 15:37 17:08 18:44 21:04 22:39 01:50*}}}
{\tnykdata{\anga{\tithi{1}{शुक्ल-प्रथमा}}{\time{7-10}{09:18}}\hspace{1ex}}%
{\anga{उत्तराषाढा}{\time{36-17}{20:33}}\hspace{1ex}}{चन्द्रराशिः—\mbox{मकरः}}%
{\anga{हर्षणः}{\time{52-48}{03:33*}}\hspace{1ex}\uanga{वज्रम्}}%
{\anga{बवम्}{\time{7-10}{09:18}}\hspace{1ex}\anga{बालवम्}{\time{41-3}{22:35}}\hspace{1ex}\uanga{कौलवम्}}{}
}
{अनध्यायः\eventsep चन्द्र-दर्शनम्~17:53\RIGHTarrow{}19:05}
{Mon} 
\cfoot{\rygdata{08:01--09:26}{10:50--12:15}{13:39--15:04}}
\caldata{JANUARY}{8}{\sunmonth{धनुः}{24}{}{पौषः}{हेमन्तऋतुः}{मङ्गलः}{विलम्बः}{दक्षिणायनम्}{हेमन्तऋतुः}}
{\sunmoonrsdata{06:37}{17:53}{08:05}{19:54}{12:15}
{\kalas{04:55 05:46 09:37 08:52 10:22 16:23 11:07 13:23 15:38 17:08 18:44 21:04 22:40 01:51*}}}
{\tnykdata{\anga{\tithi{2}{शुक्ल-द्वितीया}}{\time{14-3}{11:54}}\hspace{1ex}}%
{\anga{श्रवणः}{\time{43-32}{23:38}}\hspace{1ex}}{चन्द्रराशिः—\mbox{मकरः}}%
{\anga{वज्रम्}{\time{55-2}{04:31*}}\hspace{1ex}\uanga{सिद्धिः}}%
{\anga{कौलवम्}{\time{14-3}{11:54}}\hspace{1ex}\anga{तैतिलम्}{\time{47-21}{01:15*}}\hspace{1ex}\uanga{गरजा}}{}
}
{श्रवण-व्रतम्}
{Tue} 
\cfoot{\rygdata{15:04--16:29}{09:26--10:50}{12:15--13:40}}
\caldata{JANUARY}{9}{\sunmonth{धनुः}{25}{}{पौषः}{हेमन्तऋतुः}{बुधः}{विलम्बः}{दक्षिणायनम्}{हेमन्तऋतुः}}
{\sunmoonrsdata{06:37}{17:54}{08:46}{20:41}{12:15}
{\kalas{04:55 05:46 09:38 08:52 10:23 16:24 11:08 13:23 15:39 17:09 18:45 21:05 22:40 01:51*}}}
{\tnykdata{\anga{\tithi{3}{शुक्ल-तृतीया}}{\time{21-19}{14:38}}\hspace{1ex}}%
{\anga{श्रविष्ठा}{\time{50-57}{02:47*}}\hspace{1ex}}{चन्द्रराशिः—\mbox{मकरः\RIGHTarrow{13:13}}}%
{\prev{\anga{वज्रम्}{\time{*55-2}{04:31}}}\hspace{1ex}\anga{सिद्धिः}{\time{57-21}{05:30*}}\hspace{1ex}\uanga{व्यतीपातः}}%
{\anga{गरजा}{\time{21-19}{14:38}}\hspace{1ex}\anga{वणिजा}{\time{53-50}{04:01*}}\hspace{1ex}\uanga{भद्रा}}{}
}
{प्रवासि-भारतीय-दिवसम्~\#{१६}}
{Wed} 
\cfoot{\rygdata{12:15--13:40}{08:02--09:26}{10:51--12:15}}
\caldata{JANUARY}{10}{\sunmonth{धनुः}{26}{}{पौषः}{हेमन्तऋतुः}{गुरुः}{विलम्बः}{दक्षिणायनम्}{हेमन्तऋतुः}}
{\sunmoonrsdata{06:37}{17:54}{09:24}{21:28}{12:16}
{\kalas{04:56 05:46 09:38 08:53 10:23 16:24 11:08 13:24 15:39 17:09 18:45 21:05 22:41 01:51*}}}
{\tnykdata{\anga{\tithi{4}{शुक्ल-चतुर्थी}}{\time{28-32}{17:22}}\hspace{1ex}}%
{\anga{शतभिषक्}{\time{58-11}{05:52*}}\hspace{1ex}}{चन्द्रराशिः—\mbox{कुम्भः}}%
{\prev{\anga{सिद्धिः}{\time{*57-21}{05:30}}}\hspace{1ex}\anga{व्यतीपातः}{\time{59-31}{06:25*}}\hspace{1ex}\uanga{वरीयान्}}%
{\anga{भद्रा}{\time{28-32}{17:22}}\hspace{1ex}\uanga{बवम्}}{}
}
{महाधनुर्व्यतीपात-स्नानम्\eventsep महाधनुर्व्यतीपात-श्राद्धम्\eventsep शुक्ल-चतुर्थी-व्रतम्}
{Thu} 
\cfoot{\rygdata{13:40--15:05}{06:37--08:02}{09:27--10:51}}
\caldata{JANUARY}{11}{\sunmonth{धनुः}{27}{}{पौषः}{हेमन्तऋतुः}{शुक्रः}{विलम्बः}{दक्षिणायनम्}{हेमन्तऋतुः}}
{\sunmoonrsdata{06:38}{17:55}{10:01}{22:13}{12:16}
{\kalas{04:56 05:47 09:38 08:53 10:23 16:25 11:09 13:24 15:40 17:10 18:46 21:06 22:41 01:52*}}}
{\tnykdata{\anga{\tithi{5}{शुक्ल-पञ्चमी}}{\time{34-41}{19:54}}\hspace{1ex}}%
{\prev{\anga{शतभिषक्}{\time{*58-11}{05:52}}}\hspace{1ex}\fullanga{पूर्वप्रोष्ठपदा}}{चन्द्रराशिः—\mbox{कुम्भः\RIGHTarrow{02:00*}}}%
{\prev{\anga{व्यतीपातः}{\time{*59-31}{06:25}}}\hspace{1ex}\fullanga{वरीयान्}}%
{\anga{बवम्}{\time{0-6}{06:40}}\hspace{1ex}\anga{बालवम्}{\time{34-41}{19:54}}\hspace{1ex}\uanga{कौलवम्}}{}
}
{\tamil{கூடாரவல்லீ}}
{Fri} 
\cfoot{\rygdata{10:52--12:16}{15:06--16:30}{08:02--09:27}}
\caldata{JANUARY}{12}{\sunmonth{धनुः}{28}{}{पौषः}{हेमन्तऋतुः}{शनिः}{विलम्बः}{दक्षिणायनम्}{हेमन्तऋतुः}}
{\sunmoonrsdata{06:38}{17:55}{10:38}{23:00}{12:17}
{\kalas{04:56 05:47 09:39 08:53 10:24 16:25 11:09 13:24 15:40 17:10 18:46 21:06 22:41 01:52*}}}
{\tnykdata{\anga{\tithi{6}{शुक्ल-षष्ठी}}{\time{39-48}{22:05}}\hspace{1ex}}%
{\anga{पूर्वप्रोष्ठपदा}{\time{5-26}{08:41}}\hspace{1ex}}{चन्द्रराशिः—\mbox{मीनः}}%
{\anga{वरीयान्}{\time{1-20}{07:08}}\hspace{1ex}\uanga{परिघः}}%
{\anga{कौलवम्}{\time{6-25}{09:03}}\hspace{1ex}\anga{तैतिलम्}{\time{39-48}{22:05}}\hspace{1ex}\uanga{गरजा}}{}
}
{षष्ठी-व्रतम्\eventsep \tamil{கறவைகள் பின்சென்று}}
{Sat} 
\cfoot{\rygdata{09:27--10:52}{13:41--15:06}{06:38--08:02}}
\caldata{JANUARY}{13}{\sunmonth{धनुः}{29}{}{पौषः}{हेमन्तऋतुः}{भानुः}{विलम्बः}{दक्षिणायनम्}{हेमन्तऋतुः}}
{\sunmoonrsdata{06:38}{17:56}{11:15}{23:47}{12:17}
{\kalas{04:56 05:47 09:39 08:54 10:24 16:26 11:09 13:25 15:40 17:11 18:47 21:07 22:42 01:52*}}}
{\tnykdata{\anga{\tithi{7}{शुक्ल-सप्तमी}}{\time{43-36}{23:42}}\hspace{1ex}}%
{\anga{उत्तरप्रोष्ठपदा}{\time{11-44}{11:03}}\hspace{1ex}}{चन्द्रराशिः—\mbox{मीनः}}%
{\anga{परिघः}{\time{2-20}{07:31}}\hspace{1ex}\uanga{शिवः}}%
{\anga{गरजा}{\time{11-30}{10:58}}\hspace{1ex}\anga{वणिजा}{\time{43-36}{23:42}}\hspace{1ex}\uanga{भद्रा}}{}
}
{विजया-भानुसप्तमी}
{Sun} 
\cfoot{\rygdata{16:31--17:56}{12:17--13:42}{15:07--16:31}}
\caldata{JANUARY}{14}{\sunmonth{धनुः}{30}{\mbox{धनुः{\tiny\RIGHTarrow}{19:28}}}{पौषः}{हेमन्तऋतुः}{सोमः}{विलम्बः}{दक्षिणायनम्}{हेमन्तऋतुः}}
{\sunmoonrsdata{06:38}{17:57}{11:54}{00:36*}{12:17}
{\kalas{04:57 05:47 09:39 08:54 10:24 16:26 11:10 13:25 15:41 17:11 18:47 21:07 22:42 01:53*}}}
{\tnykdata{\anga{\tithi{8}{शुक्ल-अष्टमी}}{\time{45-46}{00:37*}}\hspace{1ex}}%
{\anga{रेवती}{\time{16-26}{12:50}}\hspace{1ex}}{चन्द्रराशिः—\mbox{मीनः\RIGHTarrow{12:50}}}%
{\anga{शिवः}{\time{2-7}{07:26}}\hspace{1ex}\uanga{सिद्धः}}%
{\anga{भद्रा}{\time{14-53}{12:15}}\hspace{1ex}\anga{बवम्}{\time{45-46}{00:37*}}\hspace{1ex}\uanga{बालवम्}}{}
}
{अनध्यायः\eventsep \tamil{போகி}\eventsep धनुर्मास-उषःकाल-पूजा-समापनम्\eventsep \tamil{வாயிலார் நாயன்மார் (51) குருபூஜை}}
{Mon} 
\cfoot{\rygdata{08:03--09:28}{10:53--12:17}{13:42--15:07}}
\caldata{JANUARY}{15}{\sunmonth{मकरः}{1}{}{पौषः}{हेमन्तऋतुः}{मङ्गलः}{विलम्बः}{उत्तरायणम्}{हेमन्तऋतुः}}
{\sunmoonrsdata{06:38}{17:57}{12:35}{01:29*}{12:18}
{\kalas{04:57 05:48 09:39 08:54 10:25 16:27 11:10 13:26 15:41 17:12 18:48 21:08 22:43 01:53*}}}
{\tnykdata{\anga{\tithi{9}{शुक्ल-नवमी}}{\time{46-3}{00:45*}}\hspace{1ex}}%
{\anga{अश्विनी}{\time{19-13}{13:54}}\hspace{1ex}}{चन्द्रराशिः—\mbox{मेषः}}%
{\anga{सिद्धः}{\time{0-25}{06:48}}\hspace{1ex}\anga{साध्यः}{\time{57-25}{05:33*}}\hspace{1ex}\uanga{शुभः}}%
{\anga{बालवम्}{\time{16-17}{12:47}}\hspace{1ex}\anga{कौलवम्}{\time{46-3}{00:45*}}\hspace{1ex}\uanga{तैतिलम्}}{}
}
{भौमाश्विनी-योगः\RIGHTarrow{}13:54\eventsep \tamil{மதுரை மீனாக்ஷீ கோயிலில் கல் யானைக்கு கரும்பு கோடுத்த லீலை}\eventsep मकर-ज्योतिः\eventsep मकर-सङ्क्रमण-पुण्यकालः\eventsep सङ्क्रमण-दिन-पूर्वाह्ण-पुण्यकालः~06:38\RIGHTarrow{}12:18}
{Tue} 
\cfoot{\rygdata{15:07--16:32}{09:28--10:53}{12:18--13:43}}
\caldata{JANUARY}{16}{\sunmonth{मकरः}{2}{}{पौषः}{हेमन्तऋतुः}{बुधः}{विलम्बः}{उत्तरायणम्}{हेमन्तऋतुः}}
{\sunmoonrsdata{06:39}{17:58}{13:21}{02:25*}{12:18}
{\kalas{04:57 05:48 09:40 08:54 10:25 16:27 11:10 13:26 15:42 17:12 18:48 21:08 22:43 01:53*}}}
{\tnykdata{\anga{\tithi{10}{शुक्ल-दशमी}}{\time{44-24}{00:03*}}\hspace{1ex}}%
{\anga{अपभरणी}{\time{19-55}{14:10}}\hspace{1ex}}{चन्द्रराशिः—\mbox{मेषः\RIGHTarrow{20:06}}}%
{\prev{\anga{साध्यः}{\time{*57-25}{05:33}}}\hspace{1ex}\anga{शुभः}{\time{52-57}{03:40*}}\hspace{1ex}\uanga{शुक्लः}}%
{\anga{तैतिलम्}{\time{15-31}{12:30}}\hspace{1ex}\anga{गरजा}{\time{44-24}{00:03*}}\hspace{1ex}\uanga{वणिजा}}{}
}
{इन्द्र-पूजा/गो-पूजा\eventsep कृत्तिका-व्रतम्\eventsep \tamil{கனுப்~பொங்கல்}\eventsep साम्ब-दशमी (सूर्यपूजा)}
{Wed} 
\cfoot{\rygdata{12:18--13:43}{08:03--09:28}{10:53--12:18}}
\caldata{JANUARY}{17}{\sunmonth{मकरः}{3}{}{पौषः}{हेमन्तऋतुः}{गुरुः}{विलम्बः}{उत्तरायणम्}{हेमन्तऋतुः}}
{\sunmoonrsdata{06:39}{17:58}{14:12}{03:25*}{12:19}
{\kalas{04:57 05:48 09:40 08:55 10:25 16:28 11:11 13:26 15:42 17:13 18:49 21:08 22:43 01:54*}}}
{\tnykdata{\anga{\tithi{11}{शुक्ल-एकादशी}}{\time{40-52}{22:34}}\hspace{1ex}}%
{\anga{कृत्तिका}{\time{18-31}{13:38}}\hspace{1ex}}{चन्द्रराशिः—\mbox{वृषभः}}%
{\anga{शुक्लः}{\time{47-0}{01:10*}}\hspace{1ex}\uanga{ब्राह्मः}}%
{\anga{वणिजा}{\time{12-36}{11:24}}\hspace{1ex}\anga{भद्रा}{\time{40-52}{22:34}}\hspace{1ex}\uanga{बवम्}}{}
}
{अनध्यायः\eventsep काञ्ची ५५ जगद्गुरु श्री-चन्द्रचूडेन्द्र सरस्वती ३ आराधना~\#{४९५}\eventsep मन्वादिः-(चाक्षुषः-[६])\eventsep सर्व-पुत्रदा-एकादशी\eventsep मकर-श्रवण-कार्त्तिकोत्सवः}
{Thu} 
\cfoot{\rygdata{13:43--15:08}{06:39--08:04}{09:29--10:54}}
\caldata{JANUARY}{18}{\sunmonth{मकरः}{4}{}{पौषः}{हेमन्तऋतुः}{शुक्रः}{विलम्बः}{उत्तरायणम्}{हेमन्तऋतुः}}
{\sunmoonrsdata{06:39}{17:59}{15:08}{04:27*}{12:19}
{\kalas{04:57 05:48 09:40 08:55 10:25 16:28 11:11 13:27 15:43 17:13 18:49 21:09 22:44 01:54*}}}
{\tnykdata{\anga{\tithi{12}{शुक्ल-द्वादशी}}{\time{35-39}{20:22}}\hspace{1ex}}%
{\anga{रोहिणी}{\time{15-10}{12:23}}\hspace{1ex}}{चन्द्रराशिः—\mbox{वृषभः\RIGHTarrow{23:30}}}%
{\anga{ब्राह्मः}{\time{39-44}{22:06}}\hspace{1ex}\uanga{माहेन्द्रः}}%
{\anga{बवम्}{\time{7-41}{09:33}}\hspace{1ex}\anga{बालवम्}{\time{35-39}{20:22}}\hspace{1ex}\uanga{कौलवम्}}{}
}
{पापनाशिनी-महाद्वादशी\eventsep तैत्तिरीय-उत्सर्गो रोहिण्याम्\eventsep \tamil{தை~வெள்ளிக்கிழமை}}
{Fri} 
\cfoot{\rygdata{10:54--12:19}{15:09--16:34}{08:04--09:29}}
\caldata{JANUARY}{19}{\sunmonth{मकरः}{5}{}{पौषः}{हेमन्तऋतुः}{शनिः}{विलम्बः}{उत्तरायणम्}{हेमन्तऋतुः}}
{\sunmoonrsdata{06:39}{17:59}{16:10}{05:29*}{12:19}
{\kalas{04:58 05:48 09:40 08:55 10:26 16:29 11:11 13:27 15:43 17:14 18:50 21:09 22:44 01:54*}}}
{\tnykdata{\anga{\tithi{13}{शुक्ल-त्रयोदशी}}{\time{28-52}{17:34}}\hspace{1ex}}%
{\anga{मृगशीर्षम्}{\time{10-9}{10:29}}\hspace{1ex}}{चन्द्रराशिः—\mbox{मिथुनम्}}%
{\anga{माहेन्द्रः}{\time{31-20}{18:33}}\hspace{1ex}\uanga{वैधृतिः}}%
{\anga{कौलवम्}{\time{1-1}{07:02}}\hspace{1ex}\anga{तैतिलम्}{\time{28-52}{17:34}}\hspace{1ex}\anga{गरजा}{\time{53-41}{03:59*}}\hspace{1ex}\uanga{वणिजा}}{}
}
{\tamil{அரிவாட்டாய நாயன்மார் (13) குருபூஜை}\eventsep \tamil{கண்ணப்ப நாயன்மார் (10) குருபூஜை}\eventsep \tamil{கபாலீ தெப்போத்ஸவம்}\eventsep शनिवार-शुक्ल-प्रदोष-व्रतम्~17:59\RIGHTarrow{}19:34}
{Sat} 
\cfoot{\rygdata{09:29--10:54}{13:44--15:09}{06:39--08:04}}
\caldata{JANUARY}{20}{\sunmonth{मकरः}{6}{}{पौषः}{हेमन्तऋतुः}{भानुः}{विलम्बः}{उत्तरायणम्}{हेमन्तऋतुः}}
{\sunmoonrsdata{06:39}{18:00}{17:14}{06:29*}{12:19}
{\kalas{04:58 05:48 09:41 08:55 10:26 16:29 11:11 13:27 15:44 17:14 18:50 21:10 22:44 01:54*}}}
{\tnykdata{\anga{\tithi{14}{शुक्ल-चतुर्दशी}}{\time{20-15}{14:19}}\hspace{1ex}}%
{\anga{आर्द्रा}{\time{3-47}{08:05}}\hspace{1ex}\anga{पुनर्वसुः}{\time{56-53}{05:20*}}\hspace{1ex}\avamA{}}{चन्द्रराशिः—\mbox{मिथुनम्\RIGHTarrow{00:03*}}}%
{\anga{वैधृतिः}{\time{21-8}{14:39}}\hspace{1ex}\uanga{विष्कम्भः}}%
{\anga{वणिजा}{\time{20-15}{14:19}}\hspace{1ex}\anga{भद्रा}{\time{45-33}{00:34*}}\hspace{1ex}\uanga{बवम्}}{}
}
{(सायन) विष्णुपदी-पुण्यकालः~08:05\RIGHTarrow{}18:00\eventsep अनध्यायः\eventsep देवी-पर्व-१०\eventsep काञ्ची ८ जगद्गुरु श्री-कैवल्यानन्दयोगेन्द्र सरस्वती आराधना~\#{१९९०}\eventsep \tamil{கபாலீ தெப்போத்ஸவம்}\eventsep पार्वणव्रतम् पूर्णिमायाम्\eventsep पञ्च-पर्व-पूजा (पूर्णिमा)\eventsep सायन-सङ्क्रमण-दिन-अपराह्ण-पुण्यकालः~12:19\RIGHTarrow{}18:00\eventsep मकर-पुष्योत्सवः\eventsep तपो-मासः/शिशिरऋतुः~14:29\RIGHTarrow{}\eventsep वेङ्कटाचले पूर्णिमा-गरुड-सेवा\eventsep वैधृति-श्राद्धम्}
{Sun} 
\cfoot{\rygdata{16:35--18:00}{12:19--13:44}{15:10--16:35}}
\caldata{JANUARY}{21}{\sunmonth{मकरः}{7}{}{पौषः}{हेमन्तऋतुः}{सोमः}{विलम्बः}{उत्तरायणम्}{हेमन्तऋतुः}}
{\sunmoonrsdata{06:39}{18:00}{18:20}{---}{12:20}
{\kalas{04:58 05:48 09:41 08:55 10:26 16:30 11:12 13:28 15:44 17:15 18:51 21:10 22:45 01:55*}}}
{\tnykdata{\anga{\tithi{15}{पौर्णमासी}}{\time{10-51}{10:46}}\hspace{1ex}}%
{\prev{\anga{पुनर्वसुः}{\time{*56-53}{05:20}}}\hspace{1ex}\anga{पुष्यः}{\time{49-57}{02:25*}}\hspace{1ex}}{चन्द्रराशिः—\mbox{कर्कटः}}%
{\anga{विष्कम्भः}{\time{10-9}{10:30}}\hspace{1ex}\anga{प्रीतिः}{\time{59-1}{06:14*}}\hspace{1ex}\uanga{आयुष्मान्}}%
{\anga{बवम्}{\time{10-51}{10:46}}\hspace{1ex}\anga{बालवम्}{\time{36-55}{20:55}}\hspace{1ex}\uanga{कौलवम्}}{}
}
{अनध्यायः\eventsep अनध्यायः\eventsep अनध्यायः\eventsep बदरी ज्योतिर्मठ-प्रतिष्ठापन-जयन्ती~\#{२५०४}\eventsep \tamil{கபாலீ தெப்போத்ஸவம்}\eventsep पार्वण-प्रायश्चित्तावकाशः पौर्णमास्याम्\eventsep पूर्णमासेष्टिः\eventsep पूर्णिमा-व्रतम्\eventsep पूर्ण-स्थालीपाकः\eventsep तैत्तिरीय-उत्सर्गः पौर्णमास्याम्\eventsep तुङ्गभद्रा~शृङ्गगिरि शारदामठ-प्रतिष्ठापन-जयन्ती~\#{२५०२}\eventsep शाकम्भरी-जयन्ती}
{Mon} 
\cfoot{\rygdata{08:04--09:29}{10:55--12:20}{13:45--15:10}}
\caldata{JANUARY}{22}{\sunmonth{मकरः}{8}{}{पौषः}{हेमन्तऋतुः}{मङ्गलः}{विलम्बः}{उत्तरायणम्}{हेमन्तऋतुः}}
{\sunmoonsrdata{06:39}{18:01}{19:24}{07:25}{12:20}
{\kalas{04:58 05:48 09:41 08:55 10:26 16:30 11:12 13:28 15:45 17:15 18:51 21:10 22:45 01:55*}}}
{\tnykdata{\anga{\tithi{16}{कृष्ण-प्रथमा}}{\time{1-7}{07:05}}\hspace{1ex}\anga{\tithi{17}{कृष्ण-द्वितीया}}{\time{52-20}{03:26*}}\hspace{1ex}\avamA{}}%
{\anga{आश्रेषा}{\time{43-0}{23:30}}\hspace{1ex}}{चन्द्रराशिः—\mbox{कर्कटः\RIGHTarrow{23:30}}}%
{\prev{\anga{प्रीतिः}{\time{*59-1}{06:14}}}\hspace{1ex}\anga{आयुष्मान्}{\time{48-58}{02:00*}}\hspace{1ex}\uanga{सौभाग्यः}}%
{\anga{कौलवम्}{\time{1-7}{07:05}}\hspace{1ex}\anga{तैतिलम्}{\time{27-56}{17:14}}\hspace{1ex}\anga{गरजा}{\time{52-20}{03:26*}}\hspace{1ex}\uanga{वणिजा}}{}
}
{अनध्यायः\eventsep दिनक्षयः\eventsep काञ्ची ३७ जगद्गुरु श्री-विद्याघनेन्द्र सरस्वती ३ आराधना~\#{१२३१}\eventsep काञ्ची ६२ जगद्गुरु श्री-चन्द्रशेखरेन्द्र सरस्वती ४ आराधना~\#{२३६}}
{Tue} 
\cfoot{\rygdata{15:10--16:36}{09:29--10:55}{12:20--13:45}}
\caldata{JANUARY}{23}{\sunmonth{मकरः}{9}{}{पौषः}{हेमन्तऋतुः}{बुधः}{विलम्बः}{उत्तरायणम्}{हेमन्तऋतुः}}
{\sunmoonsrdata{06:39}{18:01}{20:25}{08:18}{12:20}
{\kalas{04:58 05:49 09:41 08:55 10:26 16:30 11:12 13:28 15:45 17:16 18:52 21:11 22:45 01:55*}}}
{\tnykdata{\anga{\tithi{18}{कृष्ण-तृतीया}}{\time{44-9}{23:59}}\hspace{1ex}}%
{\anga{मघा}{\time{36-27}{20:45}}\hspace{1ex}}{चन्द्रराशिः—\mbox{सिंहः}}%
{\anga{सौभाग्यः}{\time{39-16}{21:56}}\hspace{1ex}\uanga{शोभनः}}%
{\anga{वणिजा}{\time{18-31}{13:40}}\hspace{1ex}\anga{भद्रा}{\time{44-9}{23:59}}\hspace{1ex}\uanga{बवम्}}{}
}
{अनध्यायः\eventsep \tamil{திருமழிசையாழ்வார் திருநக்ஷத்திரம்}}
{Wed} 
\cfoot{\rygdata{12:20--13:45}{08:04--09:30}{10:55--12:20}}
\caldata{JANUARY}{24}{\sunmonth{मकरः}{10}{}{पौषः}{हेमन्तऋतुः}{गुरुः}{विलम्बः}{उत्तरायणम्}{हेमन्तऋतुः}}
{\sunmoonsrdata{06:39}{18:02}{21:24}{09:06}{12:21}
{\kalas{04:58 05:49 09:41 08:56 10:27 16:31 11:12 13:29 15:45 17:16 18:52 21:11 22:46 01:55*}}}
{\tnykdata{\anga{\tithi{19}{कृष्ण-चतुर्थी}}{\time{36-47}{20:53}}\hspace{1ex}}%
{\anga{पूर्वफल्गुनी}{\time{30-41}{18:19}}\hspace{1ex}}{चन्द्रराशिः—\mbox{सिंहः\RIGHTarrow{23:47}}}%
{\anga{शोभनः}{\time{30-14}{18:08}}\hspace{1ex}\uanga{अतिगण्डः}}%
{\anga{बवम्}{\time{9-50}{10:23}}\hspace{1ex}\anga{बालवम्}{\time{36-47}{20:53}}\hspace{1ex}\uanga{कौलवम्}}{}
}
{लम्बोदर-महागणपति-सङ्कटहर-चतुर्थी-व्रतम्}
{Thu} 
\cfoot{\rygdata{13:46--15:11}{06:39--08:04}{09:30--10:55}}
\caldata{JANUARY}{25}{\sunmonth{मकरः}{11}{}{पौषः}{हेमन्तऋतुः}{शुक्रः}{विलम्बः}{उत्तरायणम्}{हेमन्तऋतुः}}
{\sunmoonsrdata{06:39}{18:02}{22:21}{09:52}{12:21}
{\kalas{04:58 05:49 09:41 08:56 10:27 16:31 11:12 13:29 15:46 17:17 18:53 21:12 22:46 01:55*}}}
{\tnykdata{\anga{\tithi{20}{कृष्ण-पञ्चमी}}{\time{30-36}{18:18}}\hspace{1ex}}%
{\anga{उत्तरफल्गुनी}{\time{25-37}{16:23}}\hspace{1ex}}{चन्द्रराशिः—\mbox{कन्या}}%
{\anga{अतिगण्डः}{\time{21-15}{14:43}}\hspace{1ex}\uanga{सुकर्म}}%
{\anga{कौलवम्}{\time{2-18}{07:32}}\hspace{1ex}\anga{तैतिलम्}{\time{30-36}{18:18}}\hspace{1ex}\anga{गरजा}{\time{56-36}{05:13*}}\hspace{1ex}\uanga{वणिजा}}{}
}
{\tamil{சண்டேஶ்வர நாயன்மார் (20) குருபூஜை}\eventsep \tamil{தை~வெள்ளிக்கிழமை}\eventsep त्यागराज-आराधना/बहुल-पञ्चमी~\#{१७२}}
{Fri} 
\cfoot{\rygdata{10:55--12:21}{15:12--16:37}{08:04--09:30}}
\caldata{JANUARY}{26}{\sunmonth{मकरः}{12}{}{पौषः}{हेमन्तऋतुः}{शनिः}{विलम्बः}{उत्तरायणम्}{हेमन्तऋतुः}}
{\sunmoonsrdata{06:39}{18:03}{23:16}{10:37}{12:21}
{\kalas{04:58 05:49 09:41 08:56 10:27 16:32 11:13 13:29 15:46 17:17 18:53 21:12 22:46 01:55*}}}
{\tnykdata{\anga{\tithi{21}{कृष्ण-षष्ठी}}{\time{25-26}{16:19}}\hspace{1ex}}%
{\anga{हस्तः}{\time{22-4}{15:02}}\hspace{1ex}}{चन्द्रराशिः—\mbox{कन्या\RIGHTarrow{02:37*}}}%
{\anga{सुकर्म}{\time{13-32}{11:48}}\hspace{1ex}\uanga{धृतिः}}%
{\prev{\anga{गरजा}{\time{*56-36}{05:13}}}\hspace{1ex}\anga{वणिजा}{\time{25-26}{16:19}}\hspace{1ex}\anga{भद्रा}{\time{52-41}{03:35*}}\hspace{1ex}\uanga{बवम्}}{}
}
{द्विपुष्कर-योगः~16:19\RIGHTarrow{}06:39*}
{Sat} 
\cfoot{\rygdata{09:30--10:55}{13:46--15:12}{06:39--08:04}}
\caldata{JANUARY}{27}{\sunmonth{मकरः}{13}{}{पौषः}{हेमन्तऋतुः}{भानुः}{विलम्बः}{उत्तरायणम्}{हेमन्तऋतुः}}
{\sunmoonsrdata{06:39}{18:03}{00:10*}{11:21}{12:21}
{\kalas{04:58 05:49 09:42 08:56 10:27 16:32 11:13 13:30 15:46 17:18 18:54 21:12 22:47 01:56*}}}
{\tnykdata{\anga{\tithi{22}{कृष्ण-सप्तमी}}{\time{22-2}{15:02}}\hspace{1ex}}%
{\anga{चित्रा}{\time{20-18}{14:22}}\hspace{1ex}}{चन्द्रराशिः—\mbox{तुला}}%
{\anga{धृतिः}{\time{7-16}{09:25}}\hspace{1ex}\uanga{शूलः}}%
{\anga{बवम्}{\time{22-2}{15:02}}\hspace{1ex}\anga{बालवम्}{\time{50-30}{02:40*}}\hspace{1ex}\uanga{कौलवम्}}{}
}
{अनध्यायः\eventsep भानुसप्तमी\eventsep द्विपुष्कर-योगः~06:39\RIGHTarrow{}14:22\eventsep पञ्च-पर्व-पूजा (अष्टमी)\eventsep पौष-अष्टका-पूर्वेद्युः\eventsep विवेकानन्द-जन्मदिनम्~\#{१५७}}
{Sun} 
\cfoot{\rygdata{16:38--18:03}{12:21--13:47}{15:12--16:38}}
\caldata{JANUARY}{28}{\sunmonth{मकरः}{14}{}{पौषः}{हेमन्तऋतुः}{सोमः}{विलम्बः}{उत्तरायणम्}{हेमन्तऋतुः}}
{\sunmoonsrdata{06:39}{18:04}{01:04*}{12:06}{12:21}
{\kalas{04:58 05:49 09:42 08:56 10:27 16:33 11:13 13:30 15:47 17:18 18:54 21:13 22:47 01:56*}}}
{\tnykdata{\anga{\tithi{23}{कृष्ण-अष्टमी}}{\time{20-35}{14:29}}\hspace{1ex}}%
{\anga{स्वाती}{\time{20-26}{14:26}}\hspace{1ex}}{चन्द्रराशिः—\mbox{तुला}}%
{\anga{शूलः}{\time{2-31}{07:37}}\hspace{1ex}\anga{गण्डः}{\time{59-22}{06:23*}}\hspace{1ex}\uanga{वृद्धिः}}%
{\anga{कौलवम्}{\time{20-35}{14:29}}\hspace{1ex}\anga{तैतिलम्}{\time{50-5}{02:29*}}\hspace{1ex}\uanga{गरजा}}{}
}
{अनध्यायः\eventsep अनध्यायः\eventsep पौष-अष्टका-श्राद्धम्}
{Mon} 
\cfoot{\rygdata{08:04--09:30}{10:56--12:21}{13:47--15:13}}
\caldata{JANUARY}{29}{\sunmonth{मकरः}{15}{}{पौषः}{हेमन्तऋतुः}{मङ्गलः}{विलम्बः}{उत्तरायणम्}{हेमन्तऋतुः}}
{\sunmoonsrdata{06:39}{18:04}{01:58*}{12:52}{12:22}
{\kalas{04:58 05:49 09:42 08:56 10:27 16:33 11:13 13:30 15:47 17:19 18:55 21:13 22:47 01:56*}}}
{\tnykdata{\anga{\tithi{24}{कृष्ण-नवमी}}{\time{21-4}{14:40}}\hspace{1ex}}%
{\anga{विशाखा}{\time{22-27}{15:12}}\hspace{1ex}}{चन्द्रराशिः—\mbox{तुला\RIGHTarrow{08:57}}}%
{\prev{\anga{गण्डः}{\time{*59-22}{06:23}}}\hspace{1ex}\anga{वृद्धिः}{\time{57-45}{05:42*}}\hspace{1ex}\uanga{ध्रुवः}}%
{\anga{गरजा}{\time{21-4}{14:40}}\hspace{1ex}\anga{वणिजा}{\time{51-22}{03:02*}}\hspace{1ex}\uanga{भद्रा}}{}
}
{अनध्यायः\eventsep भीष्म-जयन्ती\eventsep पौष-अन्वष्टका-श्राद्धम्\eventsep \tamil{திருநீலகண்ட நாயன்மார் (2) குருபூஜை}}
{Tue} 
\cfoot{\rygdata{15:13--16:39}{09:30--10:56}{12:22--13:47}}
\caldata{JANUARY}{30}{\sunmonth{मकरः}{16}{}{पौषः}{हेमन्तऋतुः}{बुधः}{विलम्बः}{उत्तरायणम्}{हेमन्तऋतुः}}
{\sunmoonsrdata{06:39}{18:05}{02:51*}{13:40}{12:22}
{\kalas{04:58 05:48 09:42 08:56 10:27 16:33 11:13 13:30 15:48 17:19 18:55 21:13 22:48 01:56*}}}
{\tnykdata{\anga{\tithi{25}{कृष्ण-दशमी}}{\time{23-22}{15:33}}\hspace{1ex}}%
{\anga{अनूराधा}{\time{26-12}{16:38}}\hspace{1ex}}{चन्द्रराशिः—\mbox{वृश्चिकः}}%
{\prev{\anga{वृद्धिः}{\time{*57-45}{05:42}}}\hspace{1ex}\anga{ध्रुवः}{\time{57-17}{05:30*}}\hspace{1ex}\uanga{व्याघातः}}%
{\anga{भद्रा}{\time{23-22}{15:33}}\hspace{1ex}\anga{बवम्}{\time{54-12}{04:13*}}\hspace{1ex}\uanga{बालवम्}}{}
}
{बुधानुराधा-योगः\RIGHTarrow{}16:38\eventsep त्रैलोक्य-गौरी-व्रतम्}
{Wed} 
\cfoot{\rygdata{12:22--13:48}{08:04--09:30}{10:56--12:22}}
\caldata{JANUARY}{31}{\sunmonth{मकरः}{17}{}{पौषः}{हेमन्तऋतुः}{गुरुः}{विलम्बः}{उत्तरायणम्}{हेमन्तऋतुः}}
{\sunmoonsrdata{06:39}{18:05}{03:43*}{14:30}{12:22}
{\kalas{04:58 05:48 09:42 08:56 10:27 16:34 11:13 13:31 15:48 17:19 18:55 21:13 22:48 01:56*}}}
{\tnykdata{\anga{\tithi{26}{कृष्ण-एकादशी}}{\time{27-13}{17:02}}\hspace{1ex}}%
{\anga{ज्येष्ठा}{\time{31-18}{18:38}}\hspace{1ex}}{चन्द्रराशिः—\mbox{वृश्चिकः\RIGHTarrow{18:38}}}%
{\prev{\anga{ध्रुवः}{\time{*57-17}{05:30}}}\hspace{1ex}\anga{व्याघातः}{\time{57-47}{05:43*}}\hspace{1ex}\uanga{हर्षणः}}%
{\prev{\anga{बवम्}{\time{*54-12}{04:13}}}\hspace{1ex}\anga{बालवम्}{\time{27-13}{17:02}}\hspace{1ex}\anga{कौलवम्}{\time{58-21}{05:57*}}\hspace{1ex}\uanga{तैतिलम्}}{}
}
{सर्व-षट्तिला-एकादशी}
{Thu} 
\cfoot{\rygdata{13:48--15:13}{06:39--08:04}{09:30--10:56}}
\caldata{FEBRUARY}{1}{\sunmonth{मकरः}{18}{}{पौषः}{हेमन्तऋतुः}{शुक्रः}{विलम्बः}{उत्तरायणम्}{हेमन्तऋतुः}}
{\sunmoonsrdata{06:39}{18:06}{04:32*}{15:20}{12:22}
{\kalas{04:58 05:48 09:42 08:56 10:27 16:34 11:13 13:31 15:48 17:20 18:56 21:14 22:48 01:56*}}}
{\tnykdata{\anga{\tithi{27}{कृष्ण-द्वादशी}}{\time{32-8}{18:59}}\hspace{1ex}}%
{\anga{मूला}{\time{37-9}{21:05}}\hspace{1ex}}{चन्द्रराशिः—\mbox{धनुः}}%
{\prev{\anga{व्याघातः}{\time{*57-47}{05:43}}}\hspace{1ex}\anga{हर्षणः}{\time{59-3}{06:15*}}\hspace{1ex}\uanga{वज्रम्}}%
{\prev{\anga{कौलवम्}{\time{*58-21}{05:57}}}\hspace{1ex}\anga{तैतिलम्}{\time{32-8}{18:59}}\hspace{1ex}\uanga{गरजा}}{}
}
{सेङ्गालिपुरम्-मुत्तण्णावाळ्-आराधना~\#{१२६}\eventsep \tamil{தை~வெள்ளிக்கிழமை}}
{Fri} 
\cfoot{\rygdata{10:56--12:22}{15:14--16:40}{08:04--09:30}}
\caldata{FEBRUARY}{2}{\sunmonth{मकरः}{19}{}{पौषः}{हेमन्तऋतुः}{शनिः}{विलम्बः}{उत्तरायणम्}{हेमन्तऋतुः}}
{\sunmoonsrdata{06:38}{18:06}{05:19*}{16:11}{12:22}
{\kalas{04:58 05:48 09:42 08:56 10:28 16:34 11:13 13:31 15:48 17:20 18:56 21:14 22:48 01:56*}}}
{\tnykdata{\anga{\tithi{28}{कृष्ण-त्रयोदशी}}{\time{37-41}{21:19}}\hspace{1ex}}%
{\anga{पूर्वाषाढा}{\time{43-49}{23:52}}\hspace{1ex}}{चन्द्रराशिः—\mbox{धनुः\RIGHTarrow{06:37*}}}%
{\prev{\anga{हर्षणः}{\time{*59-3}{06:15}}}\hspace{1ex}\fullanga{वज्रम्}}%
{\anga{गरजा}{\time{3-51}{08:07}}\hspace{1ex}\anga{वणिजा}{\time{37-41}{21:19}}\hspace{1ex}\uanga{भद्रा}}{}
}
{मासशिवरात्रिः\eventsep शनि-प्रदोष-व्रतम्~18:06\RIGHTarrow{}19:40}
{Sat} 
\cfoot{\rygdata{09:30--10:56}{13:48--15:14}{06:38--08:04}}
\caldata{FEBRUARY}{3}{\sunmonth{मकरः}{20}{}{पौषः}{हेमन्तऋतुः}{भानुः}{विलम्बः}{उत्तरायणम्}{हेमन्तऋतुः}}
{\sunmoonsrdata{06:38}{18:06}{06:03*}{17:01}{12:22}
{\kalas{04:58 05:48 09:42 08:56 10:28 16:35 11:13 13:31 15:49 17:21 18:57 21:14 22:48 01:56*}}}
{\tnykdata{\anga{\tithi{29}{कृष्ण-चतुर्दशी}}{\time{43-48}{23:52}}\hspace{1ex}}%
{\anga{उत्तराषाढा}{\time{51-0}{02:53*}}\hspace{1ex}}{चन्द्रराशिः—\mbox{मकरः}}%
{\anga{वज्रम्}{\time{0-57}{07:00}}\hspace{1ex}\uanga{सिद्धिः}}%
{\anga{भद्रा}{\time{10-17}{10:34}}\hspace{1ex}\anga{शकुनिः}{\time{43-48}{23:52}}\hspace{1ex}\uanga{चतुष्पात्}}{}
}
{अनध्यायः\eventsep पञ्च-पर्व-पूजा (चतुर्दशी)\eventsep पौष-यम-तर्पणम्}
{Sun} 
\cfoot{\rygdata{16:40--18:06}{12:22--13:48}{15:14--16:40}}
\caldata{FEBRUARY}{4}{\sunmonth{मकरः}{21}{}{पौषः}{हेमन्तऋतुः}{सोमः}{विलम्बः}{उत्तरायणम्}{हेमन्तऋतुः}}
{\sunmoonsrdata{06:38}{18:07}{---}{17:50}{12:22}
{\kalas{04:58 05:48 09:42 08:56 10:28 16:35 11:14 13:31 15:49 17:21 18:57 21:15 22:48 01:56*}}}
{\tnykdata{\anga{\tithi{30}{अमावास्या}}{\time{50-13}{02:33*}}\hspace{1ex}}%
{\anga{श्रवणः}{\time{58-26}{05:59*}}\hspace{1ex}}{चन्द्रराशिः—\mbox{मकरः}}%
{\anga{सिद्धिः}{\time{3-19}{07:54}}\hspace{1ex}\uanga{व्यतीपातः}}%
{\anga{चतुष्पात्}{\time{17-9}{13:12}}\hspace{1ex}\anga{नाग}{\time{50-13}{02:33*}}\hspace{1ex}\uanga{किंस्तुघ्नः}}{}
}
{अनध्यायः\eventsep महोदय-पुण्यकालः\eventsep पार्वणव्रतम् अमावास्यायाम्\eventsep पञ्च-पर्व-पूजा (अमावास्या)\eventsep पिण्ड-पितृ-यज्ञः\eventsep सोमवती अमावास्या\eventsep सोमश्रावणी-योगः\eventsep सर्व-मौनि (पौष/मकर) अमावास्या (अलभ्यम्–पुष्कला)\eventsep \tamil{திருநெல்வேலி நெல்லையப்பர் பத்ர தீப திருவிழா}\eventsep व्यतीपात-श्राद्धम्\eventsep श्रवण-व्रतम्}
{Mon} 
\cfoot{\rygdata{08:04--09:30}{10:56--12:22}{13:49--15:15}}
\caldata{FEBRUARY}{5}{\sunmonth{मकरः}{22}{}{माघः}{शिशिरऋतुः}{मङ्गलः}{विलम्बः}{उत्तरायणम्}{हेमन्तऋतुः}}
{\sunmoonrsdata{06:38}{18:07}{06:45}{18:38}{12:23}
{\kalas{04:58 05:48 09:42 08:56 10:28 16:35 11:14 13:31 15:49 17:21 18:57 21:15 22:49 01:56*}}}
{\tnykdata{\anga{\tithi{1}{शुक्ल-प्रथमा}}{\time{56-42}{05:15*}}\hspace{1ex}}%
{\prev{\anga{श्रवणः}{\time{*58-26}{05:59}}}\hspace{1ex}\fullanga{श्रविष्ठा}}{चन्द्रराशिः—\mbox{मकरः\RIGHTarrow{19:33}}}%
{\anga{व्यतीपातः}{\time{5-50}{08:52}}\hspace{1ex}\uanga{वरीयान्}}%
{\anga{किंस्तुघ्नः}{\time{24-13}{15:54}}\hspace{1ex}\anga{बवम्}{\time{56-42}{05:15*}}\hspace{1ex}\uanga{बालवम्}}{}
}
{अनध्यायः\eventsep दर्शेष्टिः\eventsep द्विपुष्कर-योगः~05:15*\RIGHTarrow{}06:37*\eventsep पार्वण-प्रायश्चित्तावकाशः दर्शे\eventsep दर्श-स्थालीपाकः\eventsep श्यामळानवरात्र-आरम्भः}
{Tue} 
\cfoot{\rygdata{15:15--16:41}{09:30--10:56}{12:23--13:49}}
\caldata{FEBRUARY}{6}{\sunmonth{मकरः}{23}{}{माघः}{शिशिरऋतुः}{बुधः}{विलम्बः}{उत्तरायणम्}{हेमन्तऋतुः}}
{\sunmoonrsdata{06:38}{18:08}{07:24}{19:25}{12:23}
{\kalas{04:57 05:48 09:42 08:56 10:28 16:36 11:14 13:32 15:50 17:22 18:58 21:15 22:49 01:56*}}}
{\tnykdata{\prev{\anga{\tithi{1}{शुक्ल-प्रथमा}}{\time{*56-42}{05:15}}}\hspace{1ex}\fulltithi{\tithi{2}{शुक्ल-द्वितीया}}}%
{\anga{श्रविष्ठा}{\time{6-26}{09:06}}\hspace{1ex}}{चन्द्रराशिः—\mbox{कुम्भः}}%
{\anga{वरीयान्}{\time{8-22}{09:50}}\hspace{1ex}\uanga{परिघः}}%
{\prev{\anga{बवम्}{\time{*56-42}{05:15}}}\hspace{1ex}\anga{बालवम्}{\time{31-5}{18:35}}\hspace{1ex}\uanga{कौलवम्}}{}
}
{चन्द्र-दर्शनम्~18:08\RIGHTarrow{}19:25}
{Wed} 
\cfoot{\rygdata{12:23--13:49}{08:04--09:30}{10:56--12:23}}
\caldata{FEBRUARY}{7}{\sunmonth{मकरः}{24}{}{माघः}{शिशिरऋतुः}{गुरुः}{विलम्बः}{उत्तरायणम्}{हेमन्तऋतुः}}
{\sunmoonrsdata{06:37}{18:08}{08:01}{20:11}{12:23}
{\kalas{04:57 05:47 09:41 08:55 10:27 16:36 11:14 13:32 15:50 17:22 18:58 21:15 22:49 01:56*}}}
{\tnykdata{\anga{\tithi{2}{शुक्ल-द्वितीया}}{\time{3-15}{07:52}}\hspace{1ex}}%
{\anga{शतभिषक्}{\time{14-18}{12:07}}\hspace{1ex}}{चन्द्रराशिः—\mbox{कुम्भः}}%
{\anga{परिघः}{\time{10-41}{10:43}}\hspace{1ex}\uanga{शिवः}}%
{\anga{कौलवम्}{\time{3-15}{07:52}}\hspace{1ex}\anga{तैतिलम्}{\time{37-9}{21:07}}\hspace{1ex}\uanga{गरजा}}{}
}
{\tamil{அப்பூதியடிகள் நாயன்மார் (25) குருபூஜை}}
{Thu} 
\cfoot{\rygdata{13:49--15:15}{06:37--08:04}{09:30--10:56}}
\caldata{FEBRUARY}{8}{\sunmonth{मकरः}{25}{}{माघः}{शिशिरऋतुः}{शुक्रः}{विलम्बः}{उत्तरायणम्}{हेमन्तऋतुः}}
{\sunmoonrsdata{06:37}{18:08}{08:38}{20:56}{12:23}
{\kalas{04:57 05:47 09:41 08:55 10:27 16:36 11:14 13:32 15:50 17:22 18:58 21:15 22:49 01:56*}}}
{\tnykdata{\anga{\tithi{3}{शुक्ल-तृतीया}}{\time{9-34}{10:18}}\hspace{1ex}}%
{\anga{पूर्वप्रोष्ठपदा}{\time{21-40}{14:56}}\hspace{1ex}}{चन्द्रराशिः—\mbox{कुम्भः\RIGHTarrow{08:15}}}%
{\anga{शिवः}{\time{12-38}{11:28}}\hspace{1ex}\uanga{सिद्धः}}%
{\anga{गरजा}{\time{9-34}{10:18}}\hspace{1ex}\anga{वणिजा}{\time{42-40}{23:24}}\hspace{1ex}\uanga{भद्रा}}{}
}
{\tamil{தை~வெள்ளிக்கிழமை}\eventsep \tamil{திருச்செந்தூர் முருகன் மாசித் திருவிழா தொடக்கம்}\eventsep शान्ता~वरकुन्द-चतुर्थी\eventsep शुक्ल-चतुर्थी-व्रतम्}
{Fri} 
\cfoot{\rygdata{10:56--12:23}{15:15--16:42}{08:03--09:30}}
\caldata{FEBRUARY}{9}{\sunmonth{मकरः}{26}{}{माघः}{शिशिरऋतुः}{शनिः}{विलम्बः}{उत्तरायणम्}{हेमन्तऋतुः}}
{\sunmoonrsdata{06:37}{18:09}{09:14}{21:43}{12:23}
{\kalas{04:57 05:47 09:41 08:55 10:27 16:36 11:14 13:32 15:50 17:23 18:59 21:16 22:49 01:56*}}}
{\tnykdata{\anga{\tithi{4}{शुक्ल-चतुर्थी}}{\time{15-7}{12:26}}\hspace{1ex}}%
{\anga{उत्तरप्रोष्ठपदा}{\time{28-14}{17:28}}\hspace{1ex}}{चन्द्रराशिः—\mbox{मीनः}}%
{\anga{सिद्धः}{\time{13-59}{12:00}}\hspace{1ex}\uanga{साध्यः}}%
{\anga{भद्रा}{\time{15-7}{12:26}}\hspace{1ex}\anga{बवम्}{\time{47-19}{01:21*}}\hspace{1ex}\uanga{बालवम्}}{}
}
{मार्कण्डेय-जयन्ती\eventsep प्रोक्लस्-जन्म~\#{१६०७}\eventsep \tamil{திருச்செந்தூர் முருகன் மாசித் திருவிழா 2ம் நாள்}}
{Sat} 
\cfoot{\rygdata{09:30--10:56}{13:49--15:16}{06:37--08:03}}
\caldata{FEBRUARY}{10}{\sunmonth{मकरः}{27}{}{माघः}{शिशिरऋतुः}{भानुः}{विलम्बः}{उत्तरायणम्}{हेमन्तऋतुः}}
{\sunmoonrsdata{06:37}{18:09}{09:52}{22:31}{12:23}
{\kalas{04:57 05:47 09:41 08:55 10:27 16:37 11:14 13:32 15:50 17:23 18:59 21:16 22:49 01:56*}}}
{\tnykdata{\anga{\tithi{5}{शुक्ल-पञ्चमी}}{\time{19-35}{14:09}}\hspace{1ex}}%
{\anga{रेवती}{\time{33-27}{19:35}}\hspace{1ex}}{चन्द्रराशिः—\mbox{मीनः\RIGHTarrow{19:35}}}%
{\anga{साध्यः}{\time{14-33}{12:13}}\hspace{1ex}\uanga{शुभः}}%
{\anga{बालवम्}{\time{19-35}{14:09}}\hspace{1ex}\anga{कौलवम्}{\time{50-52}{02:49*}}\hspace{1ex}\uanga{तैतिलम्}}{}
}
{\tamil{கலிக்கம்ப நாயன்மார் (43) குருபூஜை}\eventsep माघी-सरस्वती-पूजा\eventsep सर्प-पूजा\eventsep \tamil{திருச்செந்தூர் முருகன் மாசித் திருவிழா 3ம் நாள்—முருகன் பவனி}\eventsep वसन्त-श्री-पञ्चमी\eventsep श्रीराम-वनवास-गमनम्}
{Sun} 
\cfoot{\rygdata{16:42--18:09}{12:23--13:49}{15:16--16:42}}
\caldata{FEBRUARY}{11}{\sunmonth{मकरः}{28}{}{माघः}{शिशिरऋतुः}{सोमः}{विलम्बः}{उत्तरायणम्}{हेमन्तऋतुः}}
{\sunmoonrsdata{06:36}{18:09}{10:32}{23:21}{12:23}
{\kalas{04:57 05:46 09:41 08:55 10:27 16:37 11:13 13:32 15:51 17:23 18:59 21:16 22:49 01:56*}}}
{\tnykdata{\anga{\tithi{6}{शुक्ल-षष्ठी}}{\time{22-40}{15:20}}\hspace{1ex}}%
{\anga{अश्विनी}{\time{37-16}{21:10}}\hspace{1ex}}{चन्द्रराशिः—\mbox{मेषः}}%
{\anga{शुभः}{\time{14-7}{12:03}}\hspace{1ex}\uanga{शुक्लः}}%
{\anga{तैतिलम्}{\time{22-40}{15:20}}\hspace{1ex}\anga{गरजा}{\time{53-1}{03:42*}}\hspace{1ex}\uanga{वणिजा}}{}
}
{षष्ठी-व्रतम्\eventsep \tamil{திருச்செந்தூர் முருகன் மாசித் திருவிழா 4ம் நாள்}\eventsep \tamil{திருநெல்வேலி நெல்லையப்பர் நெல்லுக்கு வேலி கட்டிய லீலை}}
{Mon} 
\cfoot{\rygdata{08:03--09:29}{10:56--12:23}{13:49--15:16}}
\caldata{FEBRUARY}{12}{\sunmonth{मकरः}{29}{}{माघः}{शिशिरऋतुः}{मङ्गलः}{विलम्बः}{उत्तरायणम्}{हेमन्तऋतुः}}
{\sunmoonrsdata{06:36}{18:10}{11:15}{00:14*}{12:23}
{\kalas{04:56 05:46 09:41 08:55 10:27 16:37 11:13 13:32 15:51 17:24 18:59 21:16 22:49 01:56*}}}
{\tnykdata{\anga{\tithi{7}{शुक्ल-सप्तमी}}{\time{24-8}{15:54}}\hspace{1ex}}%
{\anga{अपभरणी}{\time{39-36}{22:09}}\hspace{1ex}}{चन्द्रराशिः—\mbox{मेषः\RIGHTarrow{04:17*}}}%
{\anga{शुक्लः}{\time{12-28}{11:24}}\hspace{1ex}\uanga{ब्राह्मः}}%
{\anga{वणिजा}{\time{24-8}{15:54}}\hspace{1ex}\anga{भद्रा}{\time{53-34}{03:56*}}\hspace{1ex}\uanga{बवम्}}{}
}
{अचला-सप्तमी-व्रतम्\eventsep अनध्यायः\eventsep द्वारका-मठ-प्रतिष्ठापन-जयन्ती~\#{२५०९}\eventsep मन्वादिः-(वैवस्वतः-[७])\eventsep नर्मदा-जयन्ती\eventsep रथ-सप्तमी\eventsep \tamil{திருச்செந்தூர் முருகன் மாசித் திருவிழா 5ம் நாள்}}
{Tue} 
\cfoot{\rygdata{15:16--16:43}{09:29--10:56}{12:23--13:50}}
\caldata{FEBRUARY}{13}{\sunmonth{कुम्भः}{1}{\mbox{मकरः{\tiny\RIGHTarrow}{08:30}}}{माघः}{शिशिरऋतुः}{बुधः}{विलम्बः}{उत्तरायणम्}{शिशिरऋतुः}}
{\sunmoonrsdata{06:36}{18:10}{12:02}{01:10*}{12:23}
{\kalas{04:56 05:46 09:41 08:54 10:27 16:37 11:13 13:32 15:51 17:24 19:00 21:16 22:49 01:56*}}}
{\tnykdata{\anga{\tithi{8}{शुक्ल-अष्टमी}}{\time{23-47}{15:46}}\hspace{1ex}}%
{\anga{कृत्तिका}{\time{40-17}{22:26}}\hspace{1ex}}{चन्द्रराशिः—\mbox{वृषभः}}%
{\anga{ब्राह्मः}{\time{9-27}{10:15}}\hspace{1ex}\uanga{माहेन्द्रः}}%
{\anga{बवम्}{\time{23-47}{15:46}}\hspace{1ex}\anga{बालवम्}{\time{52-22}{03:26*}}\hspace{1ex}\uanga{कौलवम्}}{}
}
{अनध्यायः\eventsep भीष्माष्टमी\eventsep बुधाष्टमी\eventsep कृत्तिका-व्रतम्\eventsep खोडियार-माता-जयन्ती\eventsep कुम्भ-रवि-सङ्क्रमण-विष्णुपदी-पुण्यकालः~06:35\RIGHTarrow{}14:54\eventsep सङ्क्रमण-दिन-पूर्वाह्ण-पुण्यकालः~06:35\RIGHTarrow{}12:23\eventsep \tamil{திருச்செந்தூர் முருகன் மாசித் திருவிழா 6ம் நாள்—வெள்ளித் தேர் பவனி}}
{Wed} 
\cfoot{\rygdata{12:23--13:50}{08:02--09:29}{10:56--12:23}}
\caldata{FEBRUARY}{14}{\sunmonth{कुम्भः}{2}{}{माघः}{शिशिरऋतुः}{गुरुः}{विलम्बः}{उत्तरायणम्}{शिशिरऋतुः}}
{\sunmoonrsdata{06:35}{18:10}{12:54}{02:09*}{12:23}
{\kalas{04:56 05:46 09:41 08:54 10:27 16:38 11:13 13:32 15:51 17:24 19:00 21:16 22:50 01:56*}}}
{\tnykdata{\anga{\tithi{9}{शुक्ल-नवमी}}{\time{21-32}{14:54}}\hspace{1ex}}%
{\anga{रोहिणी}{\time{39-13}{21:59}}\hspace{1ex}}{चन्द्रराशिः—\mbox{वृषभः}}%
{\anga{माहेन्द्रः}{\time{4-59}{08:31}}\hspace{1ex}\anga{वैधृतिः}{\time{59-4}{06:12*}}\hspace{1ex}\uanga{विष्कम्भः}}%
{\anga{कौलवम्}{\time{21-32}{14:54}}\hspace{1ex}\anga{तैतिलम्}{\time{49-23}{02:12*}}\hspace{1ex}\uanga{गरजा}}{}
}
{काञ्ची ११ जगद्गुरु श्री-शिवानन्द चिद्घनेन्द्र सरस्वती आराधना~\#{१८४७}\eventsep मध्व-नवमी\eventsep \tamil{திருச்செந்தூர் முருகன் மாசித் திருவிழா 7ம் நாள்—உருகு சத்தச் சேவை/சிகப்பு சாத்தி அலங்காரம்}\eventsep वैधृति-श्राद्धम्\eventsep श्यामळानवरात्र-समापनम्}
{Thu} 
\cfoot{\rygdata{13:50--15:16}{06:35--08:02}{09:29--10:56}}
\caldata{FEBRUARY}{15}{\sunmonth{कुम्भः}{3}{}{माघः}{शिशिरऋतुः}{शुक्रः}{विलम्बः}{उत्तरायणम्}{शिशिरऋतुः}}
{\sunmoonrsdata{06:35}{18:11}{13:51}{03:09*}{12:23}
{\kalas{04:56 05:45 09:40 08:54 10:27 16:38 11:13 13:32 15:51 17:24 19:00 21:16 22:50 01:55*}}}
{\tnykdata{\anga{\tithi{10}{शुक्ल-दशमी}}{\time{17-23}{13:18}}\hspace{1ex}}%
{\anga{मृगशीर्षम्}{\time{36-27}{20:51}}\hspace{1ex}}{चन्द्रराशिः—\mbox{वृषभः\RIGHTarrow{09:30}}}%
{\prev{\anga{वैधृतिः}{\time{*59-4}{06:12}}}\hspace{1ex}\anga{विष्कम्भः}{\time{52-8}{03:20*}}\hspace{1ex}\uanga{प्रीतिः}}%
{\anga{गरजा}{\time{17-23}{13:18}}\hspace{1ex}\anga{वणिजा}{\time{44-41}{00:15*}}\hspace{1ex}\uanga{भद्रा}}{}
}
{\tamil{திருச்செந்தூர் முருகன் மாசித் திருவிழா 8ம் நாள்—பச்சை சாத்தி அலங்காரம்}}
{Fri} 
\cfoot{\rygdata{10:56--12:23}{15:17--16:44}{08:02--09:29}}
\caldata{FEBRUARY}{16}{\sunmonth{कुम्भः}{4}{}{माघः}{शिशिरऋतुः}{शनिः}{विलम्बः}{उत्तरायणम्}{शिशिरऋतुः}}
{\sunmoonrsdata{06:34}{18:11}{14:52}{04:09*}{12:23}
{\kalas{04:55 05:45 09:40 08:54 10:27 16:38 11:13 13:32 15:52 17:25 19:00 21:17 22:50 01:55*}}}
{\tnykdata{\anga{\tithi{11}{शुक्ल-एकादशी}}{\time{11-31}{11:02}}\hspace{1ex}}%
{\anga{आर्द्रा}{\time{32-7}{19:04}}\hspace{1ex}}{चन्द्रराशिः—\mbox{मिथुनम्}}%
{\anga{प्रीतिः}{\time{43-59}{23:57}}\hspace{1ex}\uanga{आयुष्मान्}}%
{\anga{भद्रा}{\time{11-31}{11:02}}\hspace{1ex}\anga{बवम्}{\time{38-26}{21:40}}\hspace{1ex}\uanga{बालवम्}}{}
}
{सर्व-जया/भैमी-एकादशी\eventsep \tamil{திருச்செந்தூர் முருகன் மாசித் திருவிழா 9ம் நாள்—தங்க கைலாச வாஹனம்}\eventsep त्रिपुष्कर-योगः~19:04\RIGHTarrow{}06:34*}
{Sat} 
\cfoot{\rygdata{09:28--10:56}{13:50--15:17}{06:34--08:01}}
\caldata{FEBRUARY}{17}{\sunmonth{कुम्भः}{5}{}{माघः}{शिशिरऋतुः}{भानुः}{विलम्बः}{उत्तरायणम्}{शिशिरऋतुः}}
{\sunmoonrsdata{06:34}{18:11}{15:56}{05:06*}{12:23}
{\kalas{04:55 05:44 09:40 08:53 10:26 16:38 11:13 13:32 15:52 17:25 19:01 21:17 22:50 01:55*}}}
{\tnykdata{\anga{\tithi{12}{शुक्ल-द्वादशी}}{\time{4-7}{08:10}}\hspace{1ex}\anga{\tithi{13}{शुक्ल-त्रयोदशी}}{\time{55-48}{04:50*}}\hspace{1ex}\avamA{}}%
{\anga{पुनर्वसुः}{\time{26-15}{16:44}}\hspace{1ex}}{चन्द्रराशिः—\mbox{मिथुनम्\RIGHTarrow{11:21}}}%
{\anga{आयुष्मान्}{\time{34-47}{20:10}}\hspace{1ex}\uanga{सौभाग्यः}}%
{\anga{बालवम्}{\time{4-7}{08:10}}\hspace{1ex}\anga{कौलवम्}{\time{30-52}{18:33}}\hspace{1ex}\anga{तैतिलम्}{\time{55-48}{04:50*}}\hspace{1ex}\uanga{गरजा}}{}
}
{भीष्म-द्वादशी\eventsep दिनक्षयः\eventsep जयन्ती-महाद्वादशी\eventsep \tamil{குலஶேகர ஆழ்வார் திருநக்ஷத்திரம்}\eventsep माघ-मास-अन्तिमत्रयतिथि-व्रत-आरम्भः\eventsep रविपुष्य-योगः~16:44\RIGHTarrow{}\eventsep रविवार-शुक्ल-प्रदोष-व्रतम्~18:11\RIGHTarrow{}19:44\eventsep तिलपद्म-द्वादशी/तिलोत्पत्ति\eventsep \tamil{திருச்செந்தூர் முருகன் மாசித் திருவிழா 10ம் நாள்—தேர்}\eventsep त्रिपुष्कर-योगः~06:34\RIGHTarrow{}08:10\eventsep वराह-द्वादशी\eventsep वराह-कल्पादिः}
{Sun} 
\cfoot{\rygdata{16:44--18:11}{12:23--13:50}{15:17--16:44}}
\caldata{FEBRUARY}{18}{\sunmonth{कुम्भः}{6}{}{माघः}{शिशिरऋतुः}{सोमः}{विलम्बः}{उत्तरायणम्}{शिशिरऋतुः}}
{\sunmoonrsdata{06:34}{18:11}{17:00}{06:01*}{12:23}
{\kalas{04:55 05:44 09:40 08:53 10:26 16:38 11:13 13:32 15:52 17:25 19:01 21:17 22:50 01:55*}}}
{\tnykdata{\prev{\anga{\tithi{13}{शुक्ल-त्रयोदशी}}{\time{*55-48}{04:50}}}\hspace{1ex}\anga{\tithi{14}{शुक्ल-चतुर्दशी}}{\time{46-58}{01:11*}}\hspace{1ex}}%
{\anga{पुष्यः}{\time{19-11}{14:00}}\hspace{1ex}}{चन्द्रराशिः—\mbox{कर्कटः}}%
{\anga{सौभाग्यः}{\time{24-29}{16:03}}\hspace{1ex}\uanga{शोभनः}}%
{\prev{\anga{तैतिलम्}{\time{*55-48}{04:50}}}\hspace{1ex}\anga{गरजा}{\time{21-52}{15:02}}\hspace{1ex}\anga{वणिजा}{\time{46-58}{01:11*}}\hspace{1ex}\uanga{भद्रा}}{}
}
{(सायन) षडशीति-पुण्यकालः\eventsep अनध्यायः\eventsep देवी-पर्व-११\eventsep \tamil{நடராஜர் மஹாபிஷேகம்}\eventsep पञ्च-पर्व-पूजा (पूर्णिमा)\eventsep सायन-सङ्क्रमण-दिन-अपराह्ण-पुण्यकालः~12:22\RIGHTarrow{}18:11\eventsep तपस्य-मासः~04:33*\RIGHTarrow{}\eventsep \tamil{திருச்செந்தூர் முருகன் தெப்பம்}}
{Mon} 
\cfoot{\rygdata{08:01--09:28}{10:55--12:23}{13:50--15:17}}
\caldata{FEBRUARY}{19}{\sunmonth{कुम्भः}{7}{}{माघः}{शिशिरऋतुः}{मङ्गलः}{विलम्बः}{उत्तरायणम्}{शिशिरऋतुः}}
{\sunmoonrsdata{06:33}{18:12}{18:03}{---}{12:22}
{\kalas{04:54 05:44 09:39 08:53 10:26 16:39 11:13 13:32 15:52 17:25 19:01 21:17 22:50 01:55*}}}
{\tnykdata{\anga{\tithi{15}{पौर्णमासी}}{\time{37-44}{21:23}}\hspace{1ex}}%
{\anga{आश्रेषा}{\time{11-30}{11:01}}\hspace{1ex}}{चन्द्रराशिः—\mbox{कर्कटः\RIGHTarrow{11:01}}}%
{\anga{शोभनः}{\time{13-25}{11:46}}\hspace{1ex}\uanga{अतिगण्डः}}%
{\anga{भद्रा}{\time{12-12}{11:18}}\hspace{1ex}\anga{बवम्}{\time{37-44}{21:23}}\hspace{1ex}\uanga{बालवम्}}{}
}
{अनध्यायः\eventsep ललिता-जयन्ती\eventsep \tamil{மாசி~செவ்வாய்}\eventsep माघ-मास-अन्तिमत्रयतिथि-व्रत-समापनम्\eventsep माघ-पूर्णिमा\eventsep माघ-पूर्णिमा-स्नानम्\eventsep पार्वणव्रतम् पूर्णिमायाम्\eventsep पूर्णिमा-व्रतम्\eventsep \tamil{திருச்செந்தூர் மாசித் திருவிழா நிறைவு}\eventsep वेङ्कटाचले पूर्णिमा-गरुड-सेवा}
{Tue} 
\cfoot{\rygdata{15:17--16:44}{09:28--10:55}{12:22--13:50}}
\caldata{FEBRUARY}{20}{\sunmonth{कुम्भः}{8}{}{माघः}{शिशिरऋतुः}{बुधः}{विलम्बः}{उत्तरायणम्}{शिशिरऋतुः}}
{\sunmoonsrdata{06:33}{18:12}{19:05}{06:52}{12:22}
{\kalas{04:54 05:43 09:39 08:53 10:26 16:39 11:12 13:32 15:52 17:25 19:01 21:17 22:50 01:55*}}}
{\tnykdata{\anga{\tithi{16}{कृष्ण-प्रथमा}}{\time{28-28}{17:36}}\hspace{1ex}}%
{\anga{मघा}{\time{3-39}{07:58}}\hspace{1ex}\anga{पूर्वफल्गुनी}{\time{56-21}{05:02*}}\hspace{1ex}\avamA{}}{चन्द्रराशिः—\mbox{सिंहः}}%
{\anga{अतिगण्डः}{\time{2-13}{07:25}}\hspace{1ex}\anga{सुकर्म}{\time{51-47}{03:09*}}\hspace{1ex}\uanga{धृतिः}}%
{\anga{बालवम्}{\time{2-24}{07:29}}\hspace{1ex}\anga{कौलवम्}{\time{28-28}{17:36}}\hspace{1ex}\anga{तैतिलम्}{\time{53-17}{03:47*}}\hspace{1ex}\uanga{गरजा}}{}
}
{अनध्यायः\eventsep काञ्ची ५१ जगद्गुरु श्री-विद्यातीर्थेन्द्र सरस्वती आराधना~\#{६३४}\eventsep कुम्भमाघोत्सवः\eventsep पार्वण-प्रायश्चित्तावकाशः पौर्णमास्याम्\eventsep पूर्णमासेष्टिः\eventsep पूर्ण-स्थालीपाकः}
{Wed} 
\cfoot{\rygdata{12:22--13:50}{08:00--09:28}{10:55--12:22}}
\caldata{FEBRUARY}{21}{\sunmonth{कुम्भः}{9}{}{माघः}{शिशिरऋतुः}{गुरुः}{विलम्बः}{उत्तरायणम्}{शिशिरऋतुः}}
{\sunmoonsrdata{06:32}{18:12}{20:04}{07:41}{12:22}
{\kalas{04:54 05:43 09:39 08:52 10:26 16:39 11:12 13:32 15:52 17:26 19:01 21:17 22:50 01:54*}}}
{\tnykdata{\anga{\tithi{17}{कृष्ण-द्वितीया}}{\time{19-15}{14:02}}\hspace{1ex}}%
{\prev{\anga{पूर्वफल्गुनी}{\time{*56-21}{05:02}}}\hspace{1ex}\anga{उत्तरफल्गुनी}{\time{49-57}{02:24*}}\hspace{1ex}}{चन्द्रराशिः—\mbox{सिंहः\RIGHTarrow{10:21}}}%
{\anga{धृतिः}{\time{42-0}{23:08}}\hspace{1ex}\uanga{शूलः}}%
{\anga{गरजा}{\time{19-15}{14:02}}\hspace{1ex}\anga{वणिजा}{\time{45-0}{00:22*}}\hspace{1ex}\uanga{भद्रा}}{}
}
{}
{Thu} 
\cfoot{\rygdata{13:50--15:17}{06:32--08:00}{09:27--10:55}}
\caldata{FEBRUARY}{22}{\sunmonth{कुम्भः}{10}{}{माघः}{शिशिरऋतुः}{शुक्रः}{विलम्बः}{उत्तरायणम्}{शिशिरऋतुः}}
{\sunmoonsrdata{06:32}{18:12}{21:02}{08:27}{12:22}
{\kalas{04:53 05:43 09:39 08:52 10:25 16:39 11:12 13:32 15:52 17:26 19:02 21:17 22:50 01:54*}}}
{\tnykdata{\anga{\tithi{18}{कृष्ण-तृतीया}}{\time{11-2}{10:50}}\hspace{1ex}}%
{\anga{हस्तः}{\time{44-44}{00:16*}}\hspace{1ex}}{चन्द्रराशिः—\mbox{कन्या}}%
{\anga{शूलः}{\time{33-8}{19:30}}\hspace{1ex}\uanga{गण्डः}}%
{\anga{भद्रा}{\time{11-2}{10:50}}\hspace{1ex}\anga{बवम्}{\time{37-50}{21:25}}\hspace{1ex}\uanga{बालवम्}}{}
}
{द्विजप्रिय-महागणपति-सङ्कटहर-चतुर्थी-व्रतम्\eventsep \tamil{எறிபத்த நாயன்மார் (8) குருபூஜை}}
{Fri} 
\cfoot{\rygdata{10:55--12:22}{15:17--16:45}{07:59--09:27}}
\caldata{FEBRUARY}{23}{\sunmonth{कुम्भः}{11}{}{माघः}{शिशिरऋतुः}{शनिः}{विलम्बः}{उत्तरायणम्}{शिशिरऋतुः}}
{\sunmoonsrdata{06:31}{18:13}{21:59}{09:13}{12:22}
{\kalas{04:53 05:42 09:38 08:52 10:25 16:39 11:12 13:32 15:52 17:26 19:02 21:17 22:49 01:54*}}}
{\tnykdata{\anga{\tithi{19}{कृष्ण-चतुर्थी}}{\time{4-15}{08:11}}\hspace{1ex}\anga{\tithi{20}{कृष्ण-पञ्चमी}}{\time{59-17}{06:13*}}\hspace{1ex}\avamA{}}%
{\anga{चित्रा}{\time{41-4}{22:45}}\hspace{1ex}}{चन्द्रराशिः—\mbox{कन्या\RIGHTarrow{11:25}}}%
{\anga{गण्डः}{\time{25-14}{16:21}}\hspace{1ex}\uanga{वृद्धिः}}%
{\anga{बालवम्}{\time{4-15}{08:11}}\hspace{1ex}\anga{कौलवम्}{\time{32-10}{19:06}}\hspace{1ex}\anga{तैतिलम्}{\time{59-17}{06:13*}}\hspace{1ex}\uanga{गरजा}}{}
}
{दिनक्षयः}
{Sat} 
\cfoot{\rygdata{09:27--10:54}{13:50--15:17}{06:31--07:59}}
\caldata{FEBRUARY}{24}{\sunmonth{कुम्भः}{12}{}{माघः}{शिशिरऋतुः}{भानुः}{विलम्बः}{उत्तरायणम्}{शिशिरऋतुः}}
{\sunmoonsrdata{06:31}{18:13}{22:55}{10:00}{12:22}
{\kalas{04:52 05:42 09:38 08:51 10:25 16:39 11:12 13:32 15:52 17:26 19:02 21:17 22:49 01:54*}}}
{\tnykdata{\prev{\anga{\tithi{20}{कृष्ण-पञ्चमी}}{\time{*59-17}{06:13}}}\hspace{1ex}\anga{\tithi{21}{कृष्ण-षष्ठी}}{\time{56-29}{05:04*}}\hspace{1ex}}%
{\anga{स्वाती}{\time{39-15}{22:01}}\hspace{1ex}}{चन्द्रराशिः—\mbox{तुला}}%
{\anga{वृद्धिः}{\time{18-42}{13:49}}\hspace{1ex}\uanga{ध्रुवः}}%
{\prev{\anga{तैतिलम्}{\time{*59-17}{06:13}}}\hspace{1ex}\anga{गरजा}{\time{28-15}{17:32}}\hspace{1ex}\anga{वणिजा}{\time{56-29}{05:04*}}\hspace{1ex}\uanga{भद्रा}}{}
}
{त्रिपुष्कर-योगः~05:04*\RIGHTarrow{}06:30*\eventsep यशोदा-जयन्ती}
{Sun} 
\cfoot{\rygdata{16:45--18:13}{12:22--13:50}{15:17--16:45}}
\caldata{FEBRUARY}{25}{\sunmonth{कुम्भः}{13}{}{माघः}{शिशिरऋतुः}{सोमः}{विलम्बः}{उत्तरायणम्}{शिशिरऋतुः}}
{\sunmoonsrdata{06:30}{18:13}{23:51}{10:47}{12:22}
{\kalas{04:52 05:41 09:38 08:51 10:25 16:39 11:11 13:32 15:52 17:26 19:02 21:17 22:49 01:54*}}}
{\tnykdata{\prev{\anga{\tithi{21}{कृष्ण-षष्ठी}}{\time{*56-29}{05:04}}}\hspace{1ex}\anga{\tithi{22}{कृष्ण-सप्तमी}}{\time{55-48}{04:47*}}\hspace{1ex}}%
{\anga{विशाखा}{\time{39-28}{22:06}}\hspace{1ex}}{चन्द्रराशिः—\mbox{तुला\RIGHTarrow{16:00}}}%
{\anga{ध्रुवः}{\time{13-52}{11:55}}\hspace{1ex}\uanga{व्याघातः}}%
{\prev{\anga{वणिजा}{\time{*56-29}{05:04}}}\hspace{1ex}\anga{भद्रा}{\time{26-24}{16:49}}\hspace{1ex}\anga{बवम्}{\time{55-48}{04:47*}}\hspace{1ex}\uanga{बालवम्}}{}
}
{अनध्यायः\eventsep माघ-अष्टका-पूर्वेद्युः\eventsep निक्षुभार्क-सप्तमी}
{Mon} 
\cfoot{\rygdata{07:58--09:26}{10:54--12:22}{13:50--15:17}}
\caldata{FEBRUARY}{26}{\sunmonth{कुम्भः}{14}{}{माघः}{शिशिरऋतुः}{मङ्गलः}{विलम्बः}{उत्तरायणम्}{शिशिरऋतुः}}
{\sunmoonsrdata{06:30}{18:13}{00:46*}{11:36}{12:21}
{\kalas{04:52 05:41 09:37 08:50 10:24 16:39 11:11 13:32 15:52 17:26 19:02 21:17 22:49 01:53*}}}
{\tnykdata{\prev{\anga{\tithi{22}{कृष्ण-सप्तमी}}{\time{*55-48}{04:47}}}\hspace{1ex}\anga{\tithi{23}{कृष्ण-अष्टमी}}{\time{57-11}{05:20*}}\hspace{1ex}}%
{\anga{अनूराधा}{\time{41-44}{23:01}}\hspace{1ex}}{चन्द्रराशिः—\mbox{वृश्चिकः}}%
{\anga{व्याघातः}{\time{10-47}{10:43}}\hspace{1ex}\uanga{हर्षणः}}%
{\prev{\anga{बवम्}{\time{*55-48}{04:47}}}\hspace{1ex}\anga{बालवम्}{\time{26-45}{16:57}}\hspace{1ex}\anga{कौलवम्}{\time{57-11}{05:20*}}\hspace{1ex}\uanga{तैतिलम्}}{}
}
{अनध्यायः\eventsep अनध्यायः\eventsep काञ्ची ६६ जगद्गुरु श्री-चन्द्रशेखरेन्द्र सरस्वती ६ आराधना~\#{११२}\eventsep \tamil{மாசி~செவ்வாய்}\eventsep माघ-अष्टका-श्राद्धम्\eventsep पञ्च-पर्व-पूजा (अष्टमी)}
{Tue} 
\cfoot{\rygdata{15:17--16:45}{09:26--10:54}{12:22--13:49}}
\caldata{FEBRUARY}{27}{\sunmonth{कुम्भः}{15}{}{माघः}{शिशिरऋतुः}{बुधः}{विलम्बः}{उत्तरायणम्}{शिशिरऋतुः}}
{\sunmoonsrdata{06:29}{18:13}{01:38*}{12:26}{12:21}
{\kalas{04:51 05:40 09:37 08:50 10:24 16:40 11:11 13:32 15:53 17:27 19:02 21:17 22:49 01:53*}}}
{\tnykdata{\prev{\anga{\tithi{23}{कृष्ण-अष्टमी}}{\time{*57-11}{05:20}}}\hspace{1ex}\fulltithi{\tithi{24}{कृष्ण-नवमी}}}%
{\anga{ज्येष्ठा}{\time{45-54}{00:43*}}\hspace{1ex}}{चन्द्रराशिः—\mbox{वृश्चिकः\RIGHTarrow{00:43*}}}%
{\anga{हर्षणः}{\time{9-22}{10:09}}\hspace{1ex}\uanga{वज्रम्}}%
{\prev{\anga{कौलवम्}{\time{*57-11}{05:20}}}\hspace{1ex}\anga{तैतिलम्}{\time{29-13}{17:55}}\hspace{1ex}\uanga{गरजा}}{}
}
{अनध्यायः\eventsep काशि-विश्वनाथ-मन्दिर-स्थापनम्~\#{४३४}\eventsep माघ-अन्वष्टका-श्राद्धम्}
{Wed} 
\cfoot{\rygdata{12:21--13:49}{07:57--09:25}{10:53--12:21}}
\caldata{FEBRUARY}{28}{\sunmonth{कुम्भः}{16}{}{माघः}{शिशिरऋतुः}{गुरुः}{विलम्बः}{उत्तरायणम्}{शिशिरऋतुः}}
{\sunmoonsrdata{06:29}{18:14}{02:29*}{13:16}{12:21}
{\kalas{04:51 05:40 09:37 08:50 10:24 16:40 11:11 13:32 15:53 17:27 19:03 21:17 22:49 01:53*}}}
{\tnykdata{\anga{\tithi{24}{कृष्ण-नवमी}}{\time{0-30}{06:41}}\hspace{1ex}}%
{\anga{मूला}{\time{51-39}{03:04*}}\hspace{1ex}}{चन्द्रराशिः—\mbox{धनुः}}%
{\anga{वज्रम्}{\time{9-25}{10:10}}\hspace{1ex}\uanga{सिद्धिः}}%
{\anga{गरजा}{\time{0-30}{06:41}}\hspace{1ex}\anga{वणिजा}{\time{33-21}{19:36}}\hspace{1ex}\uanga{भद्रा}}{}
}
{}
{Thu} 
\cfoot{\rygdata{13:49--15:17}{06:29--07:57}{09:25--10:53}}
\caldata{MARCH}{1}{\sunmonth{कुम्भः}{17}{}{माघः}{शिशिरऋतुः}{शुक्रः}{विलम्बः}{उत्तरायणम्}{शिशिरऋतुः}}
{\sunmoonsrdata{06:28}{18:14}{03:17*}{14:07}{12:21}
{\kalas{04:50 05:39 09:36 08:49 10:23 16:40 11:10 13:32 15:53 17:27 19:03 21:17 22:49 01:52*}}}
{\tnykdata{\anga{\tithi{25}{कृष्ण-दशमी}}{\time{5-33}{08:39}}\hspace{1ex}}%
{\anga{पूर्वाषाढा}{\time{58-33}{05:53*}}\hspace{1ex}}{चन्द्रराशिः—\mbox{धनुः}}%
{\anga{सिद्धिः}{\time{10-39}{10:39}}\hspace{1ex}\uanga{व्यतीपातः}}%
{\anga{भद्रा}{\time{5-33}{08:39}}\hspace{1ex}\anga{बवम्}{\time{38-47}{21:49}}\hspace{1ex}\uanga{बालवम्}}{}
}
{\tamil{காரி நாயன்மார் (48) குருபூஜை}\eventsep व्यतीपात-श्राद्धम्}
{Fri} 
\cfoot{\rygdata{10:53--12:21}{15:17--16:46}{07:56--09:25}}
\caldata{MARCH}{2}{\sunmonth{कुम्भः}{18}{}{माघः}{शिशिरऋतुः}{शनिः}{विलम्बः}{उत्तरायणम्}{शिशिरऋतुः}}
{\sunmoonsrdata{06:28}{18:14}{04:02*}{14:57}{12:21}
{\kalas{04:50 05:39 09:36 08:49 10:23 16:40 11:10 13:31 15:53 17:27 19:03 21:17 22:49 01:52*}}}
{\tnykdata{\anga{\tithi{26}{कृष्ण-एकादशी}}{\time{11-44}{11:04}}\hspace{1ex}}%
{\prev{\anga{पूर्वाषाढा}{\time{*58-33}{05:53}}}\hspace{1ex}\fullanga{उत्तराषाढा}}{चन्द्रराशिः—\mbox{धनुः\RIGHTarrow{12:38}}}%
{\anga{व्यतीपातः}{\time{12-43}{11:27}}\hspace{1ex}\uanga{वरीयान्}}%
{\anga{बालवम्}{\time{11-44}{11:04}}\hspace{1ex}\anga{कौलवम्}{\time{45-6}{00:23*}}\hspace{1ex}\uanga{तैतिलम्}}{}
}
{सर्व-विजया-एकादशी\eventsep त्रिपुष्कर-योगः~11:04\RIGHTarrow{}06:27*}
{Sat} 
\cfoot{\rygdata{09:24--10:53}{13:49--15:17}{06:28--07:56}}
\caldata{MARCH}{3}{\sunmonth{कुम्भः}{19}{}{माघः}{शिशिरऋतुः}{भानुः}{विलम्बः}{उत्तरायणम्}{शिशिरऋतुः}}
{\sunmoonsrdata{06:27}{18:14}{04:44*}{15:47}{12:21}
{\kalas{04:49 05:38 09:36 08:49 10:23 16:40 11:10 13:31 15:53 17:27 19:03 21:17 22:49 01:52*}}}
{\tnykdata{\anga{\tithi{27}{कृष्ण-द्वादशी}}{\time{18-33}{13:44}}\hspace{1ex}}%
{\anga{उत्तराषाढा}{\time{6-23}{08:58}}\hspace{1ex}}{चन्द्रराशिः—\mbox{मकरः}}%
{\anga{वरीयान्}{\time{15-14}{12:26}}\hspace{1ex}\uanga{परिघः}}%
{\anga{तैतिलम्}{\time{18-33}{13:44}}\hspace{1ex}\anga{गरजा}{\time{51-48}{03:06*}}\hspace{1ex}\uanga{वणिजा}}{}
}
{काञ्ची ७० जगद्गुरु श्री-शङ्कर विजयेन्द्र सरस्वती जयन्ती~\#{५१}\eventsep प्रदोष-व्रतम्~18:14\RIGHTarrow{}19:46\eventsep त्रिपुष्कर-योगः~06:27\RIGHTarrow{}08:58\eventsep विजया/श्रवण-महाद्वादशी\eventsep श्रवण-व्रतम्}
{Sun} 
\cfoot{\rygdata{16:46--18:14}{12:21--13:49}{15:17--16:46}}
\caldata{MARCH}{4}{\sunmonth{कुम्भः}{20}{}{माघः}{शिशिरऋतुः}{सोमः}{विलम्बः}{उत्तरायणम्}{शिशिरऋतुः}}
{\sunmoonsrdata{06:27}{18:14}{05:24*}{16:35}{12:20}
{\kalas{04:49 05:38 09:35 08:48 10:22 16:40 11:10 13:31 15:53 17:27 19:03 21:17 22:49 01:52*}}}
{\tnykdata{\anga{\tithi{28}{कृष्ण-त्रयोदशी}}{\time{25-30}{16:28}}\hspace{1ex}}%
{\anga{श्रवणः}{\time{14-28}{12:08}}\hspace{1ex}}{चन्द्रराशिः—\mbox{मकरः\RIGHTarrow{01:42*}}}%
{\anga{परिघः}{\time{17-54}{13:29}}\hspace{1ex}\uanga{शिवः}}%
{\anga{वणिजा}{\time{25-30}{16:28}}\hspace{1ex}\anga{भद्रा}{\time{58-28}{05:49*}}\hspace{1ex}\uanga{शकुनिः}}{}
}
{मासशिवरात्रिः\eventsep महाशिवरात्रिः\eventsep पञ्च-पर्व-पूजा (चतुर्दशी)\eventsep सोमश्रावणी-योगः\RIGHTarrow{}12:08}
{Mon} 
\cfoot{\rygdata{07:55--09:23}{10:52--12:20}{13:49--15:17}}
\caldata{MARCH}{5}{\sunmonth{कुम्भः}{21}{}{माघः}{शिशिरऋतुः}{मङ्गलः}{विलम्बः}{उत्तरायणम्}{शिशिरऋतुः}}
{\sunmoonsrdata{06:26}{18:14}{06:02*}{17:22}{12:20}
{\kalas{04:48 05:37 09:35 08:48 10:22 16:40 11:09 13:31 15:53 17:27 19:03 21:17 22:49 01:51*}}}
{\tnykdata{\anga{\tithi{29}{कृष्ण-चतुर्दशी}}{\time{32-9}{19:07}}\hspace{1ex}}%
{\anga{श्रविष्ठा}{\time{22-23}{15:15}}\hspace{1ex}}{चन्द्रराशिः—\mbox{कुम्भः}}%
{\anga{शिवः}{\time{20-25}{14:28}}\hspace{1ex}\uanga{सिद्धः}}%
{\prev{\anga{भद्रा}{\time{*58-28}{05:49}}}\hspace{1ex}\anga{शकुनिः}{\time{32-9}{19:07}}\hspace{1ex}\uanga{चतुष्पात्}}{}
}
{अनध्यायः\eventsep कृष्णाङ्गारक-चतुर्दशी-पुण्यकालः/यम-तर्पणम्\eventsep \tamil{மாசி~செவ்வாய்}}
{Tue} 
\cfoot{\rygdata{15:17--16:46}{09:23--10:52}{12:20--13:49}}
\caldata{MARCH}{6}{\sunmonth{कुम्भः}{22}{}{माघः}{शिशिरऋतुः}{बुधः}{विलम्बः}{उत्तरायणम्}{शिशिरऋतुः}}
{\sunmoonsrdata{06:25}{18:14}{---}{18:08}{12:20}
{\kalas{04:48 05:37 09:34 08:47 10:22 16:40 11:09 13:31 15:53 17:27 19:03 21:17 22:48 01:51*}}}
{\tnykdata{\anga{\tithi{30}{अमावास्या}}{\time{38-10}{21:33}}\hspace{1ex}}%
{\anga{शतभिषक्}{\time{29-50}{18:11}}\hspace{1ex}}{चन्द्रराशिः—\mbox{कुम्भः}}%
{\anga{सिद्धः}{\time{22-35}{15:20}}\hspace{1ex}\uanga{साध्यः}}%
{\anga{चतुष्पात्}{\time{4-56}{08:22}}\hspace{1ex}\anga{नाग}{\time{38-10}{21:33}}\hspace{1ex}\uanga{किंस्तुघ्नः}}{}
}
{अनध्यायः\eventsep अनध्यायः\eventsep दर्भ-सङ्ग्रह आर्तव-चान्द्रः ११\eventsep कलियुगादिः\eventsep \tamil{கொச்செங்கட் சோழ நாயன்மார் (60) குருபூஜை}\eventsep माघ-स्नानपूर्तिः\eventsep पार्वणव्रतम् अमावास्यायाम्\eventsep पञ्च-पर्व-पूजा (अमावास्या)\eventsep पिण्ड-पितृ-यज्ञः\eventsep पुरन्दरदास-आराधना~\#{४५५}\eventsep सर्व-माघ-अमावास्या (अलभ्यम्–माघ-शतभिषक्)}
{Wed} 
\cfoot{\rygdata{12:20--13:49}{07:54--09:23}{10:51--12:20}}
\caldata{MARCH}{7}{\sunmonth{कुम्भः}{23}{}{फाल्गुनः}{शिशिरऋतुः}{गुरुः}{विलम्बः}{उत्तरायणम्}{शिशिरऋतुः}}
{\sunmoonrsdata{06:25}{18:15}{06:39}{18:54}{12:20}
{\kalas{04:47 05:36 09:34 08:47 10:21 16:40 11:09 13:31 15:53 17:27 19:03 21:17 22:48 01:51*}}}
{\tnykdata{\anga{\tithi{1}{शुक्ल-प्रथमा}}{\time{43-31}{23:44}}\hspace{1ex}}%
{\anga{पूर्वप्रोष्ठपदा}{\time{36-28}{20:52}}\hspace{1ex}}{चन्द्रराशिः—\mbox{कुम्भः\RIGHTarrow{14:13}}}%
{\anga{साध्यः}{\time{24-16}{15:59}}\hspace{1ex}\uanga{शुभः}}%
{\anga{किंस्तुघ्नः}{\time{10-49}{10:41}}\hspace{1ex}\anga{बवम्}{\time{43-31}{23:44}}\hspace{1ex}\uanga{बालवम्}}{}
}
{पयोव्रत-आरम्भः\eventsep अनध्यायः\eventsep दर्शेष्टिः\eventsep काञ्ची ६७ जगद्गुरु श्री-महादेवेन्द्र सरस्वती ५ आराधना~\#{११२}\eventsep पार्वण-प्रायश्चित्तावकाशः दर्शे\eventsep दर्श-स्थालीपाकः}
{Thu} 
\cfoot{\rygdata{13:48--15:17}{06:25--07:54}{09:22--10:51}}
\caldata{MARCH}{8}{\sunmonth{कुम्भः}{24}{}{फाल्गुनः}{शिशिरऋतुः}{शुक्रः}{विलम्बः}{उत्तरायणम्}{शिशिरऋतुः}}
{\sunmoonrsdata{06:24}{18:15}{07:15}{19:41}{12:19}
{\kalas{04:47 05:36 09:34 08:46 10:21 16:40 11:08 13:31 15:53 17:27 19:03 21:17 22:48 01:50*}}}
{\tnykdata{\anga{\tithi{2}{शुक्ल-द्वितीया}}{\time{48-4}{01:34*}}\hspace{1ex}}%
{\anga{उत्तरप्रोष्ठपदा}{\time{42-20}{23:15}}\hspace{1ex}}{चन्द्रराशिः—\mbox{मीनः}}%
{\anga{शुभः}{\time{25-21}{16:25}}\hspace{1ex}\uanga{शुक्लः}}%
{\anga{बालवम्}{\time{15-55}{12:41}}\hspace{1ex}\anga{कौलवम्}{\time{48-4}{01:34*}}\hspace{1ex}\uanga{तैतिलम्}}{}
}
{चन्द्र-दर्शनम्~18:15\RIGHTarrow{}19:41\eventsep काञ्ची ६८ जगद्गुरु श्री-चन्द्रशेखरेन्द्र सरस्वती ७ आश्रम-स्वीकार-दिनम्~\#{१११}\eventsep फूलेरा-दूज्\eventsep रामकृष्ण-परमहंस-जयन्ती~\#{१८४}}
{Fri} 
\cfoot{\rygdata{10:51--12:20}{15:17--16:46}{07:53--09:22}}
\caldata{MARCH}{9}{\sunmonth{कुम्भः}{25}{}{फाल्गुनः}{शिशिरऋतुः}{शनिः}{विलम्बः}{उत्तरायणम्}{शिशिरऋतुः}}
{\sunmoonrsdata{06:24}{18:15}{07:53}{20:28}{12:19}
{\kalas{04:46 05:35 09:33 08:46 10:21 16:40 11:08 13:30 15:53 17:28 19:03 21:17 22:48 01:50*}}}
{\tnykdata{\anga{\tithi{3}{शुक्ल-तृतीया}}{\time{51-44}{03:02*}}\hspace{1ex}}%
{\anga{रेवती}{\time{47-22}{01:17*}}\hspace{1ex}}{चन्द्रराशिः—\mbox{मीनः\RIGHTarrow{01:17*}}}%
{\anga{शुक्लः}{\time{25-46}{16:35}}\hspace{1ex}\uanga{ब्राह्मः}}%
{\anga{तैतिलम्}{\time{20-8}{14:21}}\hspace{1ex}\anga{गरजा}{\time{51-44}{03:02*}}\hspace{1ex}\uanga{वणिजा}}{}
}
{}
{Sat} 
\cfoot{\rygdata{09:21--10:50}{13:48--15:17}{06:24--07:53}}
\caldata{MARCH}{10}{\sunmonth{कुम्भः}{26}{}{फाल्गुनः}{शिशिरऋतुः}{भानुः}{विलम्बः}{उत्तरायणम्}{शिशिरऋतुः}}
{\sunmoonrsdata{06:23}{18:15}{08:32}{21:18}{12:19}
{\kalas{04:46 05:34 09:33 08:45 10:20 16:40 11:08 13:30 15:53 17:28 19:03 21:17 22:48 01:50*}}}
{\tnykdata{\anga{\tithi{4}{शुक्ल-चतुर्थी}}{\time{54-23}{04:06*}}\hspace{1ex}}%
{\anga{अश्विनी}{\time{51-27}{02:55*}}\hspace{1ex}}{चन्द्रराशिः—\mbox{मेषः}}%
{\anga{ब्राह्मः}{\time{25-26}{16:27}}\hspace{1ex}\uanga{माहेन्द्रः}}%
{\anga{वणिजा}{\time{23-21}{15:38}}\hspace{1ex}\anga{भद्रा}{\time{54-23}{04:06*}}\hspace{1ex}\uanga{बवम्}}{}
}
{शुक्ल-चतुर्थी-व्रतम्}
{Sun} 
\cfoot{\rygdata{16:46--18:15}{12:19--13:48}{15:17--16:46}}
\caldata{MARCH}{11}{\sunmonth{कुम्भः}{27}{}{फाल्गुनः}{शिशिरऋतुः}{सोमः}{विलम्बः}{उत्तरायणम्}{शिशिरऋतुः}}
{\sunmoonrsdata{06:22}{18:15}{09:13}{22:09}{12:19}
{\kalas{04:45 05:34 09:32 08:45 10:20 16:40 11:07 13:30 15:52 17:28 19:04 21:17 22:48 01:49*}}}
{\tnykdata{\prev{\anga{\tithi{4}{शुक्ल-चतुर्थी}}{\time{*54-23}{04:06}}}\hspace{1ex}\anga{\tithi{5}{शुक्ल-पञ्चमी}}{\time{55-55}{04:43*}}\hspace{1ex}}%
{\anga{अपभरणी}{\time{54-28}{04:08*}}\hspace{1ex}}{चन्द्रराशिः—\mbox{मेषः}}%
{\anga{माहेन्द्रः}{\time{24-18}{16:00}}\hspace{1ex}\uanga{वैधृतिः}}%
{\prev{\anga{भद्रा}{\time{*54-23}{04:06}}}\hspace{1ex}\anga{बवम्}{\time{25-30}{16:28}}\hspace{1ex}\anga{बालवम्}{\time{55-55}{04:43*}}\hspace{1ex}\uanga{कौलवम्}}{}
}
{कपालीश्वर-ध्वजारोहणम्}
{Mon} 
\cfoot{\rygdata{07:51--09:21}{10:50--12:19}{13:48--15:17}}
\caldata{MARCH}{12}{\sunmonth{कुम्भः}{28}{}{फाल्गुनः}{शिशिरऋतुः}{मङ्गलः}{विलम्बः}{उत्तरायणम्}{शिशिरऋतुः}}
{\sunmoonrsdata{06:22}{18:15}{09:58}{23:03}{12:18}
{\kalas{04:45 05:33 09:32 08:44 10:20 16:40 11:07 13:30 15:52 17:28 19:04 21:17 22:47 01:49*}}}
{\tnykdata{\prev{\anga{\tithi{5}{शुक्ल-पञ्चमी}}{\time{*55-55}{04:43}}}\hspace{1ex}\anga{\tithi{6}{शुक्ल-षष्ठी}}{\time{56-12}{04:50*}}\hspace{1ex}}%
{\prev{\anga{अपभरणी}{\time{*54-28}{04:08}}}\hspace{1ex}\anga{कृत्तिका}{\time{56-17}{04:52*}}\hspace{1ex}}{चन्द्रराशिः—\mbox{मेषः\RIGHTarrow{10:22}}}%
{\anga{वैधृतिः}{\time{22-13}{15:10}}\hspace{1ex}\uanga{विष्कम्भः}}%
{\prev{\anga{बालवम्}{\time{*55-55}{04:43}}}\hspace{1ex}\anga{कौलवम्}{\time{26-25}{16:50}}\hspace{1ex}\anga{तैतिलम्}{\time{56-12}{04:50*}}\hspace{1ex}\uanga{गरजा}}{}
}
{षष्ठी-व्रतम्\eventsep कृत्तिका-व्रतम्\eventsep \tamil{கபாலீ ஸூர்ய~சந்த்ர~வட்டம்}\eventsep \tamil{மாசி~செவ்வாய்}\eventsep त्रिपुष्कर-योगः~04:49*\RIGHTarrow{}04:51*\eventsep वैधृति-श्राद्धम्}
{Tue} 
\cfoot{\rygdata{15:17--16:46}{09:20--10:49}{12:19--13:48}}
\caldata{MARCH}{13}{\sunmonth{कुम्भः}{29}{}{फाल्गुनः}{शिशिरऋतुः}{बुधः}{विलम्बः}{उत्तरायणम्}{शिशिरऋतुः}}
{\sunmoonrsdata{06:21}{18:15}{10:47}{00:00*}{12:18}
{\kalas{04:44 05:33 09:31 08:44 10:19 16:40 11:07 13:30 15:52 17:28 19:04 21:17 22:47 01:49*}}}
{\tnykdata{\prev{\anga{\tithi{6}{शुक्ल-षष्ठी}}{\time{*56-12}{04:50}}}\hspace{1ex}\anga{\tithi{7}{शुक्ल-सप्तमी}}{\time{55-8}{04:23*}}\hspace{1ex}}%
{\prev{\anga{कृत्तिका}{\time{*56-17}{04:52}}}\hspace{1ex}\anga{रोहिणी}{\time{56-48}{05:03*}}\hspace{1ex}}{चन्द्रराशिः—\mbox{वृषभः}}%
{\anga{विष्कम्भः}{\time{19-8}{13:57}}\hspace{1ex}\uanga{प्रीतिः}}%
{\prev{\anga{तैतिलम्}{\time{*56-12}{04:50}}}\hspace{1ex}\anga{गरजा}{\time{26-0}{16:40}}\hspace{1ex}\anga{वणिजा}{\time{55-8}{04:23*}}\hspace{1ex}\uanga{भद्रा}}{}
}
{कपाल्यधिकार-नन्दी\eventsep \tamil{கபாலீ பூதண் பூதகீ}\eventsep नन्दा-सप्तमी\eventsep श्री-राघवेन्द्र-स्वामि-जयन्ती~\#{४२५}}
{Wed} 
\cfoot{\rygdata{12:18--13:48}{07:50--09:20}{10:49--12:18}}
\caldata{MARCH}{14}{\sunmonth{कुम्भः}{30}{\mbox{कुम्भः{\tiny\RIGHTarrow}{05:24*}}}{फाल्गुनः}{शिशिरऋतुः}{गुरुः}{विलम्बः}{उत्तरायणम्}{शिशिरऋतुः}}
{\sunmoonrsdata{06:20}{18:15}{11:41}{00:57*}{12:18}
{\kalas{04:44 05:32 09:31 08:43 10:19 16:40 11:06 13:29 15:52 17:28 19:04 21:16 22:47 01:48*}}}
{\tnykdata{\prev{\anga{\tithi{7}{शुक्ल-सप्तमी}}{\time{*55-8}{04:23}}}\hspace{1ex}\anga{\tithi{8}{शुक्ल-अष्टमी}}{\time{52-36}{03:21*}}\hspace{1ex}}%
{\prev{\anga{रोहिणी}{\time{*56-48}{05:03}}}\hspace{1ex}\anga{मृगशीर्षम्}{\time{55-52}{04:40*}}\hspace{1ex}}{चन्द्रराशिः—\mbox{वृषभः\RIGHTarrow{16:56}}}%
{\anga{प्रीतिः}{\time{14-56}{12:17}}\hspace{1ex}\uanga{आयुष्मान्}}%
{\prev{\anga{वणिजा}{\time{*55-8}{04:23}}}\hspace{1ex}\anga{भद्रा}{\time{24-10}{15:57}}\hspace{1ex}\anga{बवम्}{\time{52-36}{03:21*}}\hspace{1ex}\uanga{बालवम्}}{}
}
{अनध्यायः}
{Thu} 
\cfoot{\rygdata{13:47--15:17}{06:20--07:50}{09:19--10:49}}
\caldata{MARCH}{15}{\sunmonth{मीनः}{1}{}{फाल्गुनः}{शिशिरऋतुः}{शुक्रः}{विलम्बः}{उत्तरायणम्}{शिशिरऋतुः}}
{\sunmoonrsdata{06:20}{18:15}{12:38}{01:55*}{12:18}
{\kalas{04:43 05:31 09:31 08:43 10:18 16:40 11:06 13:29 15:52 17:28 19:04 21:16 22:47 01:48*}}}
{\tnykdata{\anga{\tithi{9}{शुक्ल-नवमी}}{\time{48-35}{01:44*}}\hspace{1ex}}%
{\prev{\anga{मृगशीर्षम्}{\time{*55-52}{04:40}}}\hspace{1ex}\anga{आर्द्रा}{\time{53-30}{03:43*}}\hspace{1ex}}{चन्द्रराशिः—\mbox{मिथुनम्}}%
{\anga{आयुष्मान्}{\time{9-35}{10:08}}\hspace{1ex}\uanga{सौभाग्यः}}%
{\anga{बालवम्}{\time{20-50}{14:37}}\hspace{1ex}\anga{कौलवम्}{\time{48-35}{01:44*}}\hspace{1ex}\uanga{तैतिलम्}}{}
}
{सावित्री-व्रतम्\eventsep \tamil{கபாலீ சவுடல் விமானம்}\eventsep कपालि-वृषभ-वाहनम्\eventsep मीन-रवि-सङ्क्रमण-षडशीति-पुण्यकालः~06:20\RIGHTarrow{}18:15\eventsep रवि-सङ्क्रमण-पुण्यकालः~06:20\RIGHTarrow{}11:47\eventsep सङ्क्रमण-दिन-पूर्वाह्ण-पुण्यकालः~06:20\RIGHTarrow{}12:18}
{Fri} 
\cfoot{\rygdata{10:48--12:18}{15:16--16:46}{07:49--09:19}}
\caldata{MARCH}{16}{\sunmonth{मीनः}{2}{}{फाल्गुनः}{शिशिरऋतुः}{शनिः}{विलम्बः}{उत्तरायणम्}{शिशिरऋतुः}}
{\sunmoonrsdata{06:19}{18:16}{13:39}{02:52*}{12:17}
{\kalas{04:43 05:31 09:30 08:42 10:18 16:40 11:06 13:29 15:52 17:28 19:04 21:16 22:47 01:47*}}}
{\tnykdata{\anga{\tithi{10}{शुक्ल-दशमी}}{\time{43-9}{23:33}}\hspace{1ex}}%
{\anga{पुनर्वसुः}{\time{49-43}{02:11*}}\hspace{1ex}}{चन्द्रराशिः—\mbox{मिथुनम्\RIGHTarrow{20:37}}}%
{\anga{सौभाग्यः}{\time{3-3}{07:32}}\hspace{1ex}\anga{शोभनः}{\time{55-27}{04:29*}}\hspace{1ex}\uanga{अतिगण्डः}}%
{\anga{तैतिलम्}{\time{16-3}{12:43}}\hspace{1ex}\anga{गरजा}{\time{43-9}{23:33}}\hspace{1ex}\uanga{वणिजा}}{}
}
{\tamil{கபாலீ பல்லக்கு விழா}\eventsep वेङ्कटाचले प्लवोत्सव-प्रारम्भः}
{Sat} 
\cfoot{\rygdata{09:18--10:48}{13:47--15:16}{06:19--07:49}}
\caldata{MARCH}{17}{\sunmonth{मीनः}{3}{}{फाल्गुनः}{शिशिरऋतुः}{भानुः}{विलम्बः}{उत्तरायणम्}{शिशिरऋतुः}}
{\sunmoonrsdata{06:19}{18:16}{14:41}{03:46*}{12:17}
{\kalas{04:42 05:30 09:30 08:42 10:18 16:40 11:05 13:29 15:52 17:28 19:04 21:16 22:46 01:47*}}}
{\tnykdata{\anga{\tithi{11}{शुक्ल-एकादशी}}{\time{36-25}{20:50}}\hspace{1ex}}%
{\anga{पुष्यः}{\time{44-42}{00:10*}}\hspace{1ex}}{चन्द्रराशिः—\mbox{कर्कटः}}%
{\prev{\anga{शोभनः}{\time{*55-27}{04:29}}}\hspace{1ex}\anga{अतिगण्डः}{\time{46-51}{01:01*}}\hspace{1ex}\uanga{सुकर्म}}%
{\anga{वणिजा}{\time{9-53}{10:15}}\hspace{1ex}\anga{भद्रा}{\time{36-25}{20:50}}\hspace{1ex}\uanga{बवम्}}{}
}
{कपालीश्वरयात्रा\eventsep \tamil{முனையடுவார் நாயன்மார் (52) குருபூஜை}\eventsep रंगभरी एकादशी\eventsep रविपुष्य-योगः\eventsep सर्व-आमलकी-एकादशी\eventsep वेङ्कटाचले प्लवोत्सवः}
{Sun} 
\cfoot{\rygdata{16:46--18:16}{12:17--13:47}{15:16--16:46}}
\caldata{MARCH}{18}{\sunmonth{मीनः}{4}{}{फाल्गुनः}{शिशिरऋतुः}{सोमः}{विलम्बः}{उत्तरायणम्}{शिशिरऋतुः}}
{\sunmoonrsdata{06:18}{18:16}{15:43}{04:37*}{12:17}
{\kalas{04:42 05:30 09:29 08:41 10:17 16:40 11:05 13:29 15:52 17:28 19:04 21:16 22:46 01:47*}}}
{\tnykdata{\anga{\tithi{12}{शुक्ल-द्वादशी}}{\time{28-38}{17:43}}\hspace{1ex}}%
{\anga{आश्रेषा}{\time{38-40}{21:45}}\hspace{1ex}}{चन्द्रराशिः—\mbox{कर्कटः\RIGHTarrow{21:45}}}%
{\anga{सुकर्म}{\time{37-26}{21:15}}\hspace{1ex}\uanga{धृतिः}}%
{\anga{बवम्}{\time{2-34}{07:20}}\hspace{1ex}\anga{बालवम्}{\time{28-38}{17:43}}\hspace{1ex}\anga{कौलवम्}{\time{54-23}{04:02*}}\hspace{1ex}\uanga{तैतिलम्}}{}
}
{पयोव्रत-समापनम्\eventsep \tamil{கபாலீ அறுபத்து மூவர்}\eventsep नरसिंह-द्वादशी\eventsep सोम-प्रदोष-व्रतम्~18:16\RIGHTarrow{}19:46\eventsep वेङ्कटाचले प्लवोत्सवः}
{Mon} 
\cfoot{\rygdata{07:48--09:17}{10:47--12:17}{13:46--15:16}}
\caldata{MARCH}{19}{\sunmonth{मीनः}{5}{}{फाल्गुनः}{शिशिरऋतुः}{मङ्गलः}{विलम्बः}{उत्तरायणम्}{शिशिरऋतुः}}
{\sunmoonrsdata{06:17}{18:16}{16:44}{05:26*}{12:16}
{\kalas{04:41 05:29 09:29 08:41 10:17 16:40 11:05 13:28 15:52 17:28 19:04 21:16 22:46 01:46*}}}
{\tnykdata{\anga{\tithi{13}{शुक्ल-त्रयोदशी}}{\time{20-4}{14:18}}\hspace{1ex}}%
{\anga{मघा}{\time{31-58}{19:03}}\hspace{1ex}}{चन्द्रराशिः—\mbox{सिंहः}}%
{\anga{धृतिः}{\time{27-26}{17:14}}\hspace{1ex}\uanga{शूलः}}%
{\prev{\anga{कौलवम्}{\time{*54-23}{04:02}}}\hspace{1ex}\anga{तैतिलम्}{\time{20-4}{14:18}}\hspace{1ex}\anga{गरजा}{\time{45-39}{00:32*}}\hspace{1ex}\uanga{वणिजा}}{}
}
{काञ्ची ६९ जगद्गुरु श्री-जयेन्द्र सरस्वती आराधना~\#{१}\eventsep काम-दहनम्\eventsep वेङ्कटाचले प्लवोत्सवः}
{Tue} 
\cfoot{\rygdata{15:16--16:46}{09:17--10:47}{12:16--13:46}}
\caldata{MARCH}{20}{\sunmonth{मीनः}{6}{}{फाल्गुनः}{शिशिरऋतुः}{बुधः}{विलम्बः}{उत्तरायणम्}{शिशिरऋतुः}}
{\sunmoonrsdata{06:16}{18:16}{17:44}{06:14*}{12:16}
{\kalas{04:40 05:28 09:28 08:40 10:16 16:40 11:04 13:28 15:52 17:28 19:04 21:16 22:46 01:46*}}}
{\tnykdata{\anga{\tithi{14}{शुक्ल-चतुर्दशी}}{\time{11-10}{10:45}}\hspace{1ex}}%
{\anga{पूर्वफल्गुनी}{\time{24-58}{16:16}}\hspace{1ex}}{चन्द्रराशिः—\mbox{सिंहः\RIGHTarrow{21:34}}}%
{\anga{शूलः}{\time{17-10}{13:08}}\hspace{1ex}\uanga{गण्डः}}%
{\anga{वणिजा}{\time{11-10}{10:45}}\hspace{1ex}\anga{भद्रा}{\time{36-44}{20:58}}\hspace{1ex}\uanga{बवम्}}{}
}
{अनध्यायः\eventsep अनध्यायः\eventsep अनध्यायः\eventsep अनध्यायः\eventsep होलिका-पूर्णिमा\eventsep \tamil{கற்பகாம்பாள்–கபாலீஶ்வரர் திருக்கல்யாணம்}\eventsep मधु-मासः/वसन्तऋतुः~03:28*\RIGHTarrow{}\eventsep मन्वादिः-(रुद्र-सावर्णिः-[१२])\eventsep पार्वणव्रतम् पूर्णिमायाम्\eventsep पञ्च-पर्व-पूजा (पूर्णिमा)\eventsep सायन-सङ्क्रमण-दिन-अपराह्ण-पुण्यकालः~12:16\RIGHTarrow{}18:16\eventsep वेङ्कटाचले पूर्णिमा-गरुड-सेवा\eventsep वेङ्कटाचले प्लवोत्सव-समापनम्\eventsep विषु-पुण्यकालः}
{Wed} 
\cfoot{\rygdata{12:16--13:46}{07:46--09:16}{10:46--12:16}}
\caldata{MARCH}{21}{\sunmonth{मीनः}{7}{}{फाल्गुनः}{शिशिरऋतुः}{गुरुः}{विलम्बः}{उत्तरायणम्}{शिशिरऋतुः}}
{\sunmoonrsdata{06:16}{18:16}{18:43}{---}{12:16}
{\kalas{04:40 05:28 09:28 08:40 10:16 16:40 11:04 13:28 15:52 17:28 19:04 21:16 22:46 01:45*}}}
{\tnykdata{\anga{\tithi{15}{पौर्णमासी}}{\time{2-21}{07:12}}\hspace{1ex}\anga{\tithi{16}{कृष्ण-प्रथमा}}{\time{54-2}{03:52*}}\hspace{1ex}\avamA{}}%
{\anga{उत्तरफल्गुनी}{\time{18-10}{13:32}}\hspace{1ex}}{चन्द्रराशिः—\mbox{कन्या}}%
{\anga{गण्डः}{\time{7-1}{09:04}}\hspace{1ex}\anga{वृद्धिः}{\time{57-20}{05:11*}}\hspace{1ex}\uanga{ध्रुवः}}%
{\anga{बवम्}{\time{2-21}{07:12}}\hspace{1ex}\anga{बालवम्}{\time{28-5}{17:30}}\hspace{1ex}\anga{कौलवम्}{\time{54-2}{03:52*}}\hspace{1ex}\uanga{तैतिलम्}}{}
}
{आम्र-कुसुम-प्राशनम्\eventsep अनध्यायः\eventsep अनध्यायः\eventsep चैतन्य-महाप्रभु-जयन्ती~\#{५३४}\eventsep दिनक्षयः\eventsep होलि\eventsep उमा-कपालीश्वर-दर्शनम्\eventsep \tamil{கபாலீ விடையாற்றி தொடக்கம்}\eventsep पार्वण-प्रायश्चित्तावकाशः पौर्णमास्याम्\eventsep पूर्णमासेष्टिः\eventsep पूर्णिमा-व्रतम्\eventsep मीनोत्तरफाल्गुनोत्सवः\eventsep पूर्ण-स्थालीपाकः\eventsep विषुवदिनम्}
{Thu} 
\cfoot{\rygdata{13:46--15:16}{06:16--07:46}{09:16--10:46}}
\caldata{MARCH}{22}{\sunmonth{मीनः}{8}{}{फाल्गुनः}{शिशिरऋतुः}{शुक्रः}{विलम्बः}{उत्तरायणम्}{शिशिरऋतुः}}
{\sunmoonsrdata{06:15}{18:16}{19:42}{07:01}{12:16}
{\kalas{04:39 05:27 09:27 08:39 10:15 16:40 11:03 13:28 15:52 17:28 19:04 21:16 22:45 01:45*}}}
{\tnykdata{\prev{\anga{\tithi{16}{कृष्ण-प्रथमा}}{\time{*54-2}{03:52}}}\hspace{1ex}\anga{\tithi{17}{कृष्ण-द्वितीया}}{\time{46-40}{00:55*}}\hspace{1ex}}%
{\anga{हस्तः}{\time{12-2}{11:04}}\hspace{1ex}}{चन्द्रराशिः—\mbox{कन्या\RIGHTarrow{22:00}}}%
{\prev{\anga{वृद्धिः}{\time{*57-20}{05:11}}}\hspace{1ex}\anga{ध्रुवः}{\time{48-27}{01:38*}}\hspace{1ex}\uanga{व्याघातः}}%
{\prev{\anga{कौलवम्}{\time{*54-2}{03:52}}}\hspace{1ex}\anga{तैतिलम्}{\time{20-11}{14:20}}\hspace{1ex}\anga{गरजा}{\time{46-40}{00:55*}}\hspace{1ex}\uanga{वणिजा}}{}
}
{अनध्यायः\eventsep चातुर्मास्य-द्वितीया}
{Fri} 
\cfoot{\rygdata{10:45--12:16}{15:16--16:46}{07:45--09:15}}
\caldata{MARCH}{23}{\sunmonth{मीनः}{9}{}{फाल्गुनः}{शिशिरऋतुः}{शनिः}{विलम्बः}{उत्तरायणम्}{शिशिरऋतुः}}
{\sunmoonsrdata{06:14}{18:16}{20:40}{07:48}{12:15}
{\kalas{04:39 05:27 09:27 08:39 10:15 16:40 11:03 13:27 15:52 17:28 19:04 21:15 22:45 01:45*}}}
{\tnykdata{\anga{\tithi{18}{कृष्ण-तृतीया}}{\time{40-41}{22:32}}\hspace{1ex}}%
{\anga{चित्रा}{\time{7-1}{09:04}}\hspace{1ex}}{चन्द्रराशिः—\mbox{तुला}}%
{\anga{व्याघातः}{\time{40-43}{22:33}}\hspace{1ex}\uanga{हर्षणः}}%
{\anga{वणिजा}{\time{13-28}{11:39}}\hspace{1ex}\anga{भद्रा}{\time{40-41}{22:32}}\hspace{1ex}\uanga{बवम्}}{}
}
{अनध्यायः\eventsep ब्रह्म-कल्पादिः\eventsep छत्रपति-शिवाजी-जयन्ती~\#{३९०}\eventsep काञ्ची ६९ जगद्गुरु श्री-जयेन्द्र सरस्वती आश्रम-स्वीकार-दिनम्~\#{६६}\eventsep \tamil{காரைக்கால் அம்மையார் (24) குருபூஜை}}
{Sat} 
\cfoot{\rygdata{09:15--10:45}{13:45--15:16}{06:14--07:45}}
\caldata{MARCH}{24}{\sunmonth{मीनः}{10}{}{फाल्गुनः}{शिशिरऋतुः}{भानुः}{विलम्बः}{उत्तरायणम्}{शिशिरऋतुः}}
{\sunmoonsrdata{06:14}{18:16}{21:38}{08:36}{12:15}
{\kalas{04:38 05:26 09:26 08:38 10:15 16:40 11:03 13:27 15:52 17:28 19:04 21:15 22:45 01:44*}}}
{\tnykdata{\anga{\tithi{19}{कृष्ण-चतुर्थी}}{\time{36-28}{20:51}}\hspace{1ex}}%
{\anga{स्वाती}{\time{3-33}{07:40}}\hspace{1ex}}{चन्द्रराशिः—\mbox{तुला\RIGHTarrow{01:06*}}}%
{\anga{हर्षणः}{\time{34-24}{20:02}}\hspace{1ex}\uanga{वज्रम्}}%
{\anga{बवम्}{\time{8-22}{09:36}}\hspace{1ex}\anga{बालवम्}{\time{36-28}{20:51}}\hspace{1ex}\uanga{कौलवम्}}{}
}
{रविवार-भालचन्द्र-महागणपति सङ्कटहर-चतुर्थी-व्रतम्}
{Sun} 
\cfoot{\rygdata{16:46--18:16}{12:15--13:45}{15:16--16:46}}
\caldata{MARCH}{25}{\sunmonth{मीनः}{11}{}{फाल्गुनः}{शिशिरऋतुः}{सोमः}{विलम्बः}{उत्तरायणम्}{शिशिरऋतुः}}
{\sunmoonsrdata{06:13}{18:16}{22:35}{09:26}{12:15}
{\kalas{04:38 05:25 09:26 08:38 10:14 16:40 11:02 13:27 15:52 17:28 19:04 21:15 22:45 01:44*}}}
{\tnykdata{\anga{\tithi{20}{कृष्ण-पञ्चमी}}{\time{34-20}{20:00}}\hspace{1ex}}%
{\anga{विशाखा}{\time{1-59}{07:01}}\hspace{1ex}}{चन्द्रराशिः—\mbox{वृश्चिकः}}%
{\anga{वज्रम्}{\time{29-44}{18:10}}\hspace{1ex}\uanga{सिद्धिः}}%
{\anga{कौलवम्}{\time{5-12}{08:19}}\hspace{1ex}\anga{तैतिलम्}{\time{34-20}{20:00}}\hspace{1ex}\uanga{गरजा}}{}
}
{रङ्ग-पञ्चमी}
{Mon} 
\cfoot{\rygdata{07:44--09:14}{10:44--12:15}{13:45--15:15}}
\caldata{MARCH}{26}{\sunmonth{मीनः}{12}{}{फाल्गुनः}{शिशिरऋतुः}{मङ्गलः}{विलम्बः}{उत्तरायणम्}{शिशिरऋतुः}}
{\sunmoonsrdata{06:12}{18:16}{23:30}{10:17}{12:14}
{\kalas{04:37 05:25 09:26 08:37 10:14 16:40 11:02 13:27 15:51 17:28 19:04 21:15 22:45 01:43*}}}
{\tnykdata{\anga{\tithi{21}{कृष्ण-षष्ठी}}{\time{34-24}{20:01}}\hspace{1ex}}%
{\anga{अनूराधा}{\time{2-31}{07:13}}\hspace{1ex}}{चन्द्रराशिः—\mbox{वृश्चिकः}}%
{\anga{सिद्धिः}{\time{26-51}{17:00}}\hspace{1ex}\uanga{व्यतीपातः}}%
{\anga{गरजा}{\time{4-12}{07:54}}\hspace{1ex}\anga{वणिजा}{\time{34-24}{20:01}}\hspace{1ex}\uanga{भद्रा}}{}
}
{}
{Tue} 
\cfoot{\rygdata{15:15--16:46}{09:13--10:44}{12:14--13:45}}
\caldata{MARCH}{27}{\sunmonth{मीनः}{13}{}{फाल्गुनः}{शिशिरऋतुः}{बुधः}{विलम्बः}{उत्तरायणम्}{शिशिरऋतुः}}
{\sunmoonsrdata{06:12}{18:16}{00:23*}{11:09}{12:14}
{\kalas{04:36 05:24 09:25 08:37 10:13 16:40 11:02 13:26 15:51 17:28 19:04 21:15 22:44 01:43*}}}
{\tnykdata{\anga{\tithi{22}{कृष्ण-सप्तमी}}{\time{36-38}{20:55}}\hspace{1ex}}%
{\anga{ज्येष्ठा}{\time{5-11}{08:17}}\hspace{1ex}}{चन्द्रराशिः—\mbox{वृश्चिकः\RIGHTarrow{08:17}}}%
{\anga{व्यतीपातः}{\time{25-39}{16:31}}\hspace{1ex}\uanga{वरीयान्}}%
{\anga{भद्रा}{\time{5-23}{08:22}}\hspace{1ex}\anga{बवम्}{\time{36-38}{20:55}}\hspace{1ex}\uanga{बालवम्}}{}
}
{फाल्गुन-अष्टका-पूर्वेद्युः\eventsep व्यतीपात-श्राद्धम्}
{Wed} 
\cfoot{\rygdata{12:14--13:45}{07:42--09:13}{10:43--12:14}}
\caldata{MARCH}{28}{\sunmonth{मीनः}{14}{}{फाल्गुनः}{शिशिरऋतुः}{गुरुः}{विलम्बः}{उत्तरायणम्}{शिशिरऋतुः}}
{\sunmoonsrdata{06:11}{18:16}{01:12*}{12:01}{12:14}
{\kalas{04:36 05:24 09:25 08:36 10:13 16:40 11:01 13:26 15:51 17:28 19:04 21:15 22:44 01:43*}}}
{\tnykdata{\anga{\tithi{23}{कृष्ण-अष्टमी}}{\time{40-49}{22:34}}\hspace{1ex}}%
{\anga{मूला}{\time{9-49}{10:09}}\hspace{1ex}}{चन्द्रराशिः—\mbox{धनुः}}%
{\anga{वरीयान्}{\time{25-56}{16:38}}\hspace{1ex}\uanga{परिघः}}%
{\anga{बालवम्}{\time{8-36}{09:39}}\hspace{1ex}\anga{कौलवम्}{\time{40-49}{22:34}}\hspace{1ex}\uanga{तैतिलम्}}{}
}
{अनध्यायः\eventsep पञ्च-पर्व-पूजा (अष्टमी)\eventsep फाल्गुन-अष्टका-श्राद्धम्}
{Thu} 
\cfoot{\rygdata{13:44--15:15}{06:11--07:42}{09:12--10:43}}
\caldata{MARCH}{29}{\sunmonth{मीनः}{15}{}{फाल्गुनः}{शिशिरऋतुः}{शुक्रः}{विलम्बः}{उत्तरायणम्}{शिशिरऋतुः}}
{\sunmoonsrdata{06:10}{18:16}{01:59*}{12:52}{12:13}
{\kalas{04:35 05:23 09:24 08:36 10:13 16:40 11:01 13:26 15:51 17:28 19:04 21:15 22:44 01:42*}}}
{\tnykdata{\anga{\tithi{24}{कृष्ण-नवमी}}{\time{46-27}{00:48*}}\hspace{1ex}}%
{\anga{पूर्वाषाढा}{\time{16-3}{12:39}}\hspace{1ex}}{चन्द्रराशिः—\mbox{धनुः\RIGHTarrow{19:21}}}%
{\anga{परिघः}{\time{27-25}{17:14}}\hspace{1ex}\uanga{शिवः}}%
{\anga{तैतिलम्}{\time{13-30}{11:37}}\hspace{1ex}\anga{गरजा}{\time{46-27}{00:48*}}\hspace{1ex}\uanga{वणिजा}}{}
}
{गुरु-सङ्क्रान्तिः~(वृश्चिकः\To{}धनुः)~13:31\RIGHTarrow{}\eventsep फाल्गुन-अन्वष्टका-श्राद्धम्}
{Fri} 
\cfoot{\rygdata{10:43--12:13}{15:15--16:46}{07:41--09:12}}
\caldata{MARCH}{30}{\sunmonth{मीनः}{16}{}{फाल्गुनः}{शिशिरऋतुः}{शनिः}{विलम्बः}{उत्तरायणम्}{शिशिरऋतुः}}
{\sunmoonsrdata{06:10}{18:16}{02:42*}{13:42}{12:13}
{\kalas{04:35 05:22 09:24 08:35 10:12 16:40 11:00 13:26 15:51 17:28 19:04 21:15 22:44 01:42*}}}
{\tnykdata{\anga{\tithi{25}{कृष्ण-दशमी}}{\time{52-59}{03:23*}}\hspace{1ex}}%
{\anga{उत्तराषाढा}{\time{23-21}{15:36}}\hspace{1ex}}{चन्द्रराशिः—\mbox{मकरः}}%
{\anga{शिवः}{\time{29-41}{18:09}}\hspace{1ex}\uanga{सिद्धः}}%
{\anga{वणिजा}{\time{19-33}{14:04}}\hspace{1ex}\anga{भद्रा}{\time{52-59}{03:23*}}\hspace{1ex}\uanga{बवम्}}{}
}
{}
{Sat} 
\cfoot{\rygdata{09:11--10:42}{13:44--15:15}{06:10--07:41}}
\caldata{MARCH}{31}{\sunmonth{मीनः}{17}{}{फाल्गुनः}{शिशिरऋतुः}{भानुः}{विलम्बः}{उत्तरायणम्}{शिशिरऋतुः}}
{\sunmoonsrdata{06:09}{18:17}{03:22*}{14:30}{12:13}
{\kalas{04:34 05:22 09:23 08:35 10:12 16:40 11:00 13:26 15:51 17:28 19:04 21:15 22:44 01:42*}}}
{\tnykdata{\anga{\tithi{26}{कृष्ण-एकादशी}}{\time{59-48}{06:04*}}\hspace{1ex}}%
{\anga{श्रवणः}{\time{31-10}{18:45}}\hspace{1ex}}{चन्द्रराशिः—\mbox{मकरः}}%
{\anga{सिद्धः}{\time{32-20}{19:12}}\hspace{1ex}\uanga{साध्यः}}%
{\anga{बवम्}{\time{26-9}{16:43}}\hspace{1ex}\anga{बालवम्}{\time{59-48}{06:04*}}\hspace{1ex}\uanga{कौलवम्}}{}
}
{देवी-पर्व-१२\eventsep द्विपुष्कर-योगः~06:04*\RIGHTarrow{}06:09*\eventsep \tamil{கபாலீ விடையாற்றி நிறைவு}\eventsep स्मार्त-पापमोचनी-एकादशी\eventsep श्रवण-व्रतम्}
{Sun} 
\cfoot{\rygdata{16:46--18:17}{12:13--13:44}{15:15--16:46}}
\caldata{APRIL}{1}{\sunmonth{मीनः}{18}{}{फाल्गुनः}{शिशिरऋतुः}{सोमः}{विलम्बः}{उत्तरायणम्}{शिशिरऋतुः}}
{\sunmoonsrdata{06:09}{18:17}{04:01*}{15:18}{12:13}
{\kalas{04:34 05:21 09:23 08:34 10:11 16:40 11:00 13:25 15:51 17:28 19:04 21:14 22:43 01:41*}}}
{\tnykdata{\prev{\anga{\tithi{26}{कृष्ण-एकादशी}}{\time{*59-48}{06:04}}}\hspace{1ex}\fulltithi{\tithi{27}{कृष्ण-द्वादशी}}}%
{\anga{श्रविष्ठा}{\time{39-5}{21:52}}\hspace{1ex}}{चन्द्रराशिः—\mbox{मकरः\RIGHTarrow{08:19}}}%
{\anga{साध्यः}{\time{34-57}{20:14}}\hspace{1ex}\uanga{शुभः}}%
{\prev{\anga{बालवम्}{\time{*59-48}{06:04}}}\hspace{1ex}\anga{कौलवम्}{\time{32-47}{19:23}}\hspace{1ex}\uanga{तैतिलम्}}{}
}
{हरिवासरः\RIGHTarrow{}12:44\eventsep वैष्णव-पापमोचनी-एकादशी\eventsep व्यञ्जुली-महाद्वादशी}
{Mon} 
\cfoot{\rygdata{07:40--09:11}{10:42--12:13}{13:44--15:15}}
\caldata{APRIL}{2}{\sunmonth{मीनः}{19}{}{फाल्गुनः}{शिशिरऋतुः}{मङ्गलः}{विलम्बः}{उत्तरायणम्}{शिशिरऋतुः}}
{\sunmoonsrdata{06:08}{18:17}{04:38*}{16:04}{12:12}
{\kalas{04:33 05:20 09:22 08:34 10:11 16:40 10:59 13:25 15:51 17:28 19:04 21:14 22:43 01:41*}}}
{\tnykdata{\anga{\tithi{27}{कृष्ण-द्वादशी}}{\time{6-11}{08:38}}\hspace{1ex}}%
{\anga{शतभिषक्}{\time{46-29}{00:47*}}\hspace{1ex}}{चन्द्रराशिः—\mbox{कुम्भः}}%
{\anga{शुभः}{\time{37-10}{21:07}}\hspace{1ex}\uanga{शुक्लः}}%
{\anga{तैतिलम्}{\time{6-11}{08:38}}\hspace{1ex}\anga{गरजा}{\time{38-59}{21:50}}\hspace{1ex}\uanga{वणिजा}}{}
}
{\tamil{தண்டியடிகள் நாயன்மார் (31) குருபூஜை}\eventsep प्रदोष-व्रतम्~18:17\RIGHTarrow{}19:45}
{Tue} 
\cfoot{\rygdata{15:14--16:46}{09:10--10:41}{12:12--13:43}}
\caldata{APRIL}{3}{\sunmonth{मीनः}{20}{}{फाल्गुनः}{शिशिरऋतुः}{बुधः}{विलम्बः}{उत्तरायणम्}{शिशिरऋतुः}}
{\sunmoonsrdata{06:07}{18:17}{05:15*}{16:51}{12:12}
{\kalas{04:33 05:20 09:22 08:33 10:10 16:40 10:59 13:25 15:51 17:28 19:04 21:14 22:43 01:40*}}}
{\tnykdata{\anga{\tithi{28}{कृष्ण-त्रयोदशी}}{\time{11-52}{10:56}}\hspace{1ex}}%
{\anga{पूर्वप्रोष्ठपदा}{\time{53-4}{03:23*}}\hspace{1ex}}{चन्द्रराशिः—\mbox{कुम्भः\RIGHTarrow{20:46}}}%
{\anga{शुक्लः}{\time{38-47}{21:45}}\hspace{1ex}\uanga{ब्राह्मः}}%
{\anga{वणिजा}{\time{11-52}{10:56}}\hspace{1ex}\anga{भद्रा}{\time{44-21}{23:57}}\hspace{1ex}\uanga{शकुनिः}}{}
}
{मासशिवरात्रिः\eventsep पञ्च-पर्व-पूजा (चतुर्दशी)}
{Wed} 
\cfoot{\rygdata{12:12--13:43}{07:38--09:10}{10:41--12:12}}
\caldata{APRIL}{4}{\sunmonth{मीनः}{21}{}{फाल्गुनः}{शिशिरऋतुः}{गुरुः}{विलम्बः}{उत्तरायणम्}{शिशिरऋतुः}}
{\sunmoonsrdata{06:07}{18:17}{05:53*}{17:37}{12:12}
{\kalas{04:32 05:19 09:21 08:33 10:10 16:39 10:59 13:25 15:51 17:28 19:04 21:14 22:43 01:40*}}}
{\tnykdata{\anga{\tithi{29}{कृष्ण-चतुर्दशी}}{\time{16-36}{12:51}}\hspace{1ex}}%
{\anga{उत्तरप्रोष्ठपदा}{\time{58-39}{05:34*}}\hspace{1ex}}{चन्द्रराशिः—\mbox{मीनः}}%
{\anga{ब्राह्मः}{\time{39-36}{22:04}}\hspace{1ex}\uanga{माहेन्द्रः}}%
{\anga{शकुनिः}{\time{16-36}{12:51}}\hspace{1ex}\anga{चतुष्पात्}{\time{48-41}{01:39*}}\hspace{1ex}\uanga{नाग}}{}
}
{अनध्यायः\eventsep अनध्यायः\eventsep काञ्ची ६५ जगद्गुरु श्री-सुदर्शन महादेवेन्द्र सरस्वती आराधना~\#{१२९}\eventsep मन्वादिः-(रैवतः-[५])\eventsep पञ्च-पर्व-पूजा (अमावास्या)\eventsep सर्व-फाल्गुन-अमावास्या (अलभ्यम्–पुष्कला)}
{Thu} 
\cfoot{\rygdata{13:43--15:14}{06:07--07:38}{09:09--10:40}}
\caldata{APRIL}{5}{\sunmonth{मीनः}{22}{}{फाल्गुनः}{शिशिरऋतुः}{शुक्रः}{विलम्बः}{उत्तरायणम्}{शिशिरऋतुः}}
{\sunmoonsrdata{06:06}{18:17}{---}{18:25}{12:11}
{\kalas{04:31 05:19 09:21 08:32 10:10 16:39 10:58 13:24 15:51 17:28 19:04 21:14 22:43 01:40*}}}
{\tnykdata{\anga{\tithi{30}{अमावास्या}}{\time{20-16}{14:20}}\hspace{1ex}}%
{\prev{\anga{उत्तरप्रोष्ठपदा}{\time{*58-39}{05:34}}}\hspace{1ex}\fullanga{रेवती}}{चन्द्रराशिः—\mbox{मीनः}}%
{\anga{माहेन्द्रः}{\time{39-36}{22:04}}\hspace{1ex}\uanga{वैधृतिः}}%
{\anga{नाग}{\time{20-16}{14:20}}\hspace{1ex}\anga{किंस्तुघ्नः}{\time{51-55}{02:55*}}\hspace{1ex}\uanga{बवम्}}{}
}
{अनध्यायः\eventsep बोधायन-कात्यायन-इष्टिः\eventsep भृगुरेवती-योगः\eventsep पार्वणव्रतम् अमावास्यायाम्\eventsep पिण्ड-पितृ-यज्ञः}
{Fri} 
\cfoot{\rygdata{10:40--12:11}{15:14--16:45}{07:37--09:09}}
\caldata{APRIL}{6}{\sunmonth{मीनः}{23}{}{चैत्रः}{वसन्तऋतुः}{शनिः}{विलम्बः}{उत्तरायणम्}{शिशिरऋतुः}}
{\sunmoonrsdata{06:05}{18:17}{06:31}{19:14}{12:11}
{\kalas{04:31 05:18 09:20 08:32 10:09 16:39 10:58 13:24 15:51 17:28 19:04 21:14 22:42 01:39*}}}
{\tnykdata{\anga{\tithi{1}{शुक्ल-प्रथमा}}{\time{22-52}{15:23}}\hspace{1ex}}%
{\anga{रेवती}{\time{3-5}{07:21}}\hspace{1ex}}{चन्द्रराशिः—\mbox{मीनः\RIGHTarrow{07:21}}}%
{\anga{वैधृतिः}{\time{38-46}{21:44}}\hspace{1ex}\uanga{विष्कम्भः}}%
{\anga{बवम्}{\time{22-52}{15:23}}\hspace{1ex}\anga{बालवम्}{\time{54-5}{03:45*}}\hspace{1ex}\uanga{कौलवम्}}{}
}
{अनध्यायः\eventsep चन्द्र-दर्शनम्~18:17\RIGHTarrow{}19:14\eventsep दर्शेष्टिः\eventsep काञ्ची १५ जगद्गुरु श्री-गङ्गाधरेन्द्र सरस्वती आराधना~\#{१६९१}\eventsep काञ्ची २७ जगद्गुरु श्री-चिद्विलासेन्द्र सरस्वती आराधना~\#{१४४३}\eventsep काञ्ची ५२ जगद्गुरु श्री-शङ्करानन्देन्द्र सरस्वती आराधना~\#{६०३}\eventsep पार्वण-प्रायश्चित्तावकाशः दर्शे\eventsep दर्श-स्थालीपाकः\eventsep वैधृति-श्राद्धम्\eventsep वसन्तनवरात्र-आरम्भः\eventsep युगादिः/चान्द्रमान-संवत्सरारम्भः\eventsep श्वेतवराह-कल्पादिः}
{Sat} 
\cfoot{\rygdata{09:08--10:40}{13:43--15:14}{06:05--07:37}}
\caldata{APRIL}{7}{\sunmonth{मीनः}{24}{}{चैत्रः}{वसन्तऋतुः}{भानुः}{विलम्बः}{उत्तरायणम्}{शिशिरऋतुः}}
{\sunmoonrsdata{06:05}{18:17}{07:13}{20:06}{12:11}
{\kalas{04:30 05:17 09:20 08:31 10:09 16:39 10:58 13:24 15:51 17:28 19:04 21:14 22:42 01:39*}}}
{\tnykdata{\anga{\tithi{2}{शुक्ल-द्वितीया}}{\time{24-25}{16:01}}\hspace{1ex}}%
{\anga{अश्विनी}{\time{6-28}{08:43}}\hspace{1ex}}{चन्द्रराशिः—\mbox{मेषः}}%
{\anga{विष्कम्भः}{\time{37-7}{21:05}}\hspace{1ex}\uanga{प्रीतिः}}%
{\prev{\anga{बालवम्}{\time{*54-5}{03:45}}}\hspace{1ex}\anga{कौलवम्}{\time{24-25}{16:01}}\hspace{1ex}\anga{तैतिलम्}{\time{55-12}{04:11*}}\hspace{1ex}\uanga{गरजा}}{}
}
{आन्दोलन-तृतीया\eventsep बालेन्दुव्रतम्\eventsep काञ्ची ६० जगद्गुरु श्री-अद्वैतात्मप्रकाशेन्द्र सरस्वती आराधना~\#{३१६}}
{Sun} 
\cfoot{\rygdata{16:45--18:17}{12:11--13:42}{15:14--16:45}}
\caldata{APRIL}{8}{\sunmonth{मीनः}{25}{}{चैत्रः}{वसन्तऋतुः}{सोमः}{विलम्बः}{उत्तरायणम्}{शिशिरऋतुः}}
{\sunmoonrsdata{06:04}{18:17}{07:57}{20:59}{12:11}
{\kalas{04:30 05:17 09:20 08:31 10:08 16:39 10:57 13:24 15:50 17:28 19:04 21:14 22:42 01:39*}}}
{\tnykdata{\anga{\tithi{3}{शुक्ल-तृतीया}}{\time{25-0}{16:15}}\hspace{1ex}}%
{\anga{अपभरणी}{\time{8-53}{09:41}}\hspace{1ex}}{चन्द्रराशिः—\mbox{मेषः\RIGHTarrow{15:52}}}%
{\anga{प्रीतिः}{\time{34-41}{20:08}}\hspace{1ex}\uanga{आयुष्मान्}}%
{\prev{\anga{तैतिलम्}{\time{*55-12}{04:11}}}\hspace{1ex}\anga{गरजा}{\time{25-0}{16:15}}\hspace{1ex}\anga{वणिजा}{\time{55-21}{04:14*}}\hspace{1ex}\uanga{भद्रा}}{}
}
{अनध्यायः\eventsep अरुन्धती-व्रतम्\eventsep गौरी-तृतीया/सौभाग्य-गौरी-व्रतम्\eventsep कृत्तिका-व्रतम्\eventsep महातारा-जयन्ती\eventsep मन्वादिः-(उत्तमः-[३])\eventsep पार्वतीश्वरयोरान्दोलनव्रतम्}
{Mon} 
\cfoot{\rygdata{07:36--09:07}{10:39--12:11}{13:42--15:14}}
\caldata{APRIL}{9}{\sunmonth{मीनः}{26}{}{चैत्रः}{वसन्तऋतुः}{मङ्गलः}{विलम्बः}{उत्तरायणम्}{शिशिरऋतुः}}
{\sunmoonrsdata{06:03}{18:17}{08:45}{21:55}{12:10}
{\kalas{04:29 05:16 09:19 08:30 10:08 16:39 10:57 13:24 15:50 17:28 19:04 21:14 22:42 01:38*}}}
{\tnykdata{\anga{\tithi{4}{शुक्ल-चतुर्थी}}{\time{24-40}{16:07}}\hspace{1ex}}%
{\anga{कृत्तिका}{\time{10-23}{10:18}}\hspace{1ex}}{चन्द्रराशिः—\mbox{वृषभः}}%
{\anga{आयुष्मान्}{\time{31-31}{18:53}}\hspace{1ex}\uanga{सौभाग्यः}}%
{\prev{\anga{वणिजा}{\time{*55-21}{04:14}}}\hspace{1ex}\anga{भद्रा}{\time{24-40}{16:07}}\hspace{1ex}\anga{बवम्}{\time{54-32}{03:54*}}\hspace{1ex}\uanga{बालवम्}}{}
}
{मुत्तुस्वामि-दीक्षित-जयन्ती~\#{२४५}\eventsep सुखा~अङ्गारकी-चतुर्थी\eventsep शुक्ल-चतुर्थी-व्रतम्}
{Tue} 
\cfoot{\rygdata{15:14--16:45}{09:07--10:39}{12:10--13:42}}
\caldata{APRIL}{10}{\sunmonth{मीनः}{27}{}{चैत्रः}{वसन्तऋतुः}{बुधः}{विलम्बः}{उत्तरायणम्}{शिशिरऋतुः}}
{\sunmoonrsdata{06:03}{18:17}{09:37}{22:52}{12:10}
{\kalas{04:29 05:16 09:19 08:30 10:08 16:39 10:57 13:23 15:50 17:28 19:04 21:13 22:42 01:38*}}}
{\tnykdata{\anga{\tithi{5}{शुक्ल-पञ्चमी}}{\time{23-24}{15:36}}\hspace{1ex}}%
{\anga{रोहिणी}{\time{10-59}{10:32}}\hspace{1ex}}{चन्द्रराशिः—\mbox{वृषभः\RIGHTarrow{22:31}}}%
{\anga{सौभाग्यः}{\time{27-40}{17:20}}\hspace{1ex}\uanga{शोभनः}}%
{\prev{\anga{बवम्}{\time{*54-32}{03:54}}}\hspace{1ex}\anga{बालवम्}{\time{23-24}{15:36}}\hspace{1ex}\anga{कौलवम्}{\time{52-44}{03:12*}}\hspace{1ex}\uanga{तैतिलम्}}{}
}
{हय-पूजा\eventsep कूर्म-कल्पादिः\eventsep लक्ष्मी-पञ्चमी\eventsep \tamil{நேச நாயன்மார் (59) குருபூஜை}\eventsep शालिहोत्र-व्रत-आरम्भः}
{Wed} 
\cfoot{\rygdata{12:10--13:42}{07:35--09:06}{10:38--12:10}}
\caldata{APRIL}{11}{\sunmonth{मीनः}{28}{}{चैत्रः}{वसन्तऋतुः}{गुरुः}{विलम्बः}{उत्तरायणम्}{शिशिरऋतुः}}
{\sunmoonrsdata{06:02}{18:17}{10:33}{23:49}{12:10}
{\kalas{04:28 05:15 09:18 08:29 10:07 16:39 10:56 13:23 15:50 17:28 19:04 21:13 22:41 01:37*}}}
{\tnykdata{\anga{\tithi{6}{शुक्ल-षष्ठी}}{\time{21-11}{14:42}}\hspace{1ex}}%
{\anga{मृगशीर्षम्}{\time{10-40}{10:24}}\hspace{1ex}}{चन्द्रराशिः—\mbox{मिथुनम्}}%
{\anga{शोभनः}{\time{23-9}{15:30}}\hspace{1ex}\uanga{अतिगण्डः}}%
{\anga{तैतिलम्}{\time{21-11}{14:42}}\hspace{1ex}\anga{गरजा}{\time{49-56}{02:06*}}\hspace{1ex}\uanga{वणिजा}}{}
}
{षष्ठी-व्रतम्\eventsep यमुना-जयन्ती}
{Thu} 
\cfoot{\rygdata{13:42--15:14}{06:02--07:34}{09:06--10:38}}
\caldata{APRIL}{12}{\sunmonth{मीनः}{29}{}{चैत्रः}{वसन्तऋतुः}{शुक्रः}{विलम्बः}{उत्तरायणम्}{शिशिरऋतुः}}
{\sunmoonrsdata{06:02}{18:18}{11:31}{00:45*}{12:10}
{\kalas{04:28 05:15 09:18 08:29 10:07 16:39 10:56 13:23 15:50 17:28 19:04 21:13 22:41 01:37*}}}
{\tnykdata{\anga{\tithi{7}{शुक्ल-सप्तमी}}{\time{18-0}{13:23}}\hspace{1ex}}%
{\anga{आर्द्रा}{\time{9-24}{09:52}}\hspace{1ex}}{चन्द्रराशिः—\mbox{मिथुनम्\RIGHTarrow{03:13*}}}%
{\anga{अतिगण्डः}{\time{17-52}{13:20}}\hspace{1ex}\uanga{सुकर्म}}%
{\anga{वणिजा}{\time{18-0}{13:23}}\hspace{1ex}\anga{भद्रा}{\time{46-6}{00:35*}}\hspace{1ex}\uanga{बवम्}}{}
}
{\tamil{கணநாத நாயன்மார் (38) குருபூஜை}\eventsep सूर्यस्य दमनकपूजा}
{Fri} 
\cfoot{\rygdata{10:38--12:09}{15:13--16:45}{07:34--09:06}}
\caldata{APRIL}{13}{\sunmonth{मीनः}{30}{}{चैत्रः}{वसन्तऋतुः}{शनिः}{विलम्बः}{उत्तरायणम्}{शिशिरऋतुः}}
{\sunmoonrsdata{06:01}{18:18}{12:31}{01:38*}{12:09}
{\kalas{04:27 05:14 09:17 08:28 10:06 16:39 10:56 13:23 15:50 17:28 19:04 21:13 22:41 01:37*}}}
{\tnykdata{\anga{\tithi{8}{शुक्ल-अष्टमी}}{\time{13-51}{11:41}}\hspace{1ex}}%
{\anga{पुनर्वसुः}{\time{7-10}{08:57}}\hspace{1ex}}{चन्द्रराशिः—\mbox{कर्कटः}}%
{\anga{सुकर्म}{\time{11-47}{10:51}}\hspace{1ex}\uanga{धृतिः}}%
{\anga{बवम्}{\time{13-51}{11:41}}\hspace{1ex}\anga{बालवम्}{\time{41-14}{22:41}}\hspace{1ex}\uanga{कौलवम्}}{}
}
{अनध्यायः\eventsep अशोकाष्टमी\eventsep भवान्युत्पत्तिः\eventsep ब्रह्मपुत्रस्नानम्\eventsep काञ्ची ४३ जगद्गुरु श्री-आनन्दघनेन्द्र सरस्वती आराधना~\#{१००६}\eventsep श्रीरामनवमी}
{Sat} 
\cfoot{\rygdata{09:05--10:37}{13:41--15:13}{06:01--07:33}}
\caldata{APRIL}{14}{\sunmonth{मेषः}{1}{\mbox{मीनः{\tiny\RIGHTarrow}{13:54}}}{चैत्रः}{वसन्तऋतुः}{भानुः}{विकारी}{उत्तरायणम्}{वसन्तऋतुः}}
{\sunmoonrsdata{06:00}{18:18}{13:31}{02:29*}{12:09}
{\kalas{04:27 05:13 09:17 08:28 10:06 16:39 10:55 13:23 15:50 17:29 19:04 21:13 22:41 01:36*}}}
{\tnykdata{\anga{\tithi{9}{शुक्ल-नवमी}}{\time{8-44}{09:35}}\hspace{1ex}}%
{\anga{पुष्यः}{\time{3-59}{07:38}}\hspace{1ex}\anga{आश्रेषा}{\time{59-55}{05:58*}}\hspace{1ex}\avamA{}}{चन्द्रराशिः—\mbox{कर्कटः\RIGHTarrow{05:58*}}}%
{\anga{धृतिः}{\time{4-58}{08:02}}\hspace{1ex}\anga{शूलः}{\time{57-17}{04:56*}}\hspace{1ex}\uanga{गण्डः}}%
{\anga{कौलवम्}{\time{8-44}{09:35}}\hspace{1ex}\anga{तैतिलम्}{\time{35-24}{20:24}}\hspace{1ex}\uanga{गरजा}}{}
}
{खाल्सारम्भः~\#{३२१}\eventsep मेष-सङ्क्रान्तिः (विकारि-संवत्सरः)\eventsep मेष-सङ्क्रमण-पुण्यकालः~09:54\RIGHTarrow{}17:54\eventsep निम्ब-कुसुम-भक्षणम्\eventsep पञ्चाङ्ग-पठनम्\eventsep रवि-सङ्क्रमण-पुण्यकालः~07:30\RIGHTarrow{}18:18\eventsep रविपुष्य-योगः\RIGHTarrow{}07:38\eventsep सङ्क्रमण-दिन-अपराह्ण-पुण्यकालः~12:09\RIGHTarrow{}18:18\eventsep वसन्तनवरात्र-समापनम्\eventsep \tamil{விஷுக்கனி}}
{Sun} 
\cfoot{\rygdata{16:45--18:18}{12:09--13:41}{15:13--16:45}}
\caldata{APRIL}{15}{\sunmonth{मेषः}{2}{}{चैत्रः}{वसन्तऋतुः}{सोमः}{विकारी}{उत्तरायणम्}{वसन्तऋतुः}}
{\sunmoonrsdata{06:00}{18:18}{14:30}{03:17*}{12:09}
{\kalas{04:26 05:13 09:16 08:27 10:06 16:39 10:55 13:23 15:50 17:29 19:05 21:13 22:41 01:36*}}}
{\tnykdata{\anga{\tithi{10}{शुक्ल-दशमी}}{\time{2-46}{07:08}}\hspace{1ex}\anga{\tithi{11}{शुक्ल-एकादशी}}{\time{55-52}{04:23*}}\hspace{1ex}\avamA{}}%
{\prev{\anga{आश्रेषा}{\time{*59-55}{05:58}}}\hspace{1ex}\anga{मघा}{\time{54-52}{03:59*}}\hspace{1ex}}{चन्द्रराशिः—\mbox{सिंहः}}%
{\prev{\anga{शूलः}{\time{*57-17}{04:56}}}\hspace{1ex}\anga{गण्डः}{\time{48-43}{01:36*}}\hspace{1ex}\uanga{वृद्धिः}}%
{\anga{गरजा}{\time{2-46}{07:08}}\hspace{1ex}\anga{वणिजा}{\time{28-45}{17:47}}\hspace{1ex}\anga{भद्रा}{\time{55-52}{04:23*}}\hspace{1ex}\uanga{बवम्}}{}
}
{ऋषीणां दमनकपूजा\eventsep धर्मराज-दशमी\eventsep दिनक्षयः\eventsep समुद्र-मन्थनम्\eventsep स्मार्त-कामदा-एकादशी\eventsep श्रीकृष्णदोलोत्सवः}
{Mon} 
\cfoot{\rygdata{07:32--09:04}{10:36--12:09}{13:41--15:13}}
\caldata{APRIL}{16}{\sunmonth{मेषः}{3}{}{चैत्रः}{वसन्तऋतुः}{मङ्गलः}{विकारी}{उत्तरायणम्}{वसन्तऋतुः}}
{\sunmoonrsdata{05:59}{18:18}{15:28}{04:03*}{12:09}
{\kalas{04:26 05:12 09:16 08:27 10:05 16:39 10:55 13:22 15:50 17:29 19:05 21:13 22:41 01:36*}}}
{\tnykdata{\prev{\anga{\tithi{11}{शुक्ल-एकादशी}}{\time{*55-52}{04:23}}}\hspace{1ex}\anga{\tithi{12}{शुक्ल-द्वादशी}}{\time{48-18}{01:26*}}\hspace{1ex}}%
{\prev{\anga{मघा}{\time{*54-52}{03:59}}}\hspace{1ex}\anga{पूर्वफल्गुनी}{\time{49-18}{01:49*}}\hspace{1ex}}{चन्द्रराशिः—\mbox{सिंहः}}%
{\anga{वृद्धिः}{\time{39-41}{22:04}}\hspace{1ex}\uanga{ध्रुवः}}%
{\prev{\anga{भद्रा}{\time{*55-52}{04:23}}}\hspace{1ex}\anga{बवम्}{\time{21-46}{14:55}}\hspace{1ex}\anga{बालवम्}{\time{48-18}{01:26*}}\hspace{1ex}\uanga{कौलवम्}}{}
}
{भ्रातृप्राप्ति-व्रत-आरम्भः\eventsep दमनकारोपण-द्वादशी\eventsep हरिवासरः\RIGHTarrow{}09:39\eventsep तुलसी-जननं-क्षीरसागरतः\eventsep वेङ्कटाचले वसन्तोत्सव-प्रारम्भः\eventsep वैष्णव-कामदा-एकादशी\eventsep विष्णु-दमनकोत्सवः}
{Tue} 
\cfoot{\rygdata{15:13--16:45}{09:04--10:36}{12:08--13:41}}
\caldata{APRIL}{17}{\sunmonth{मेषः}{4}{}{चैत्रः}{वसन्तऋतुः}{बुधः}{विकारी}{उत्तरायणम्}{वसन्तऋतुः}}
{\sunmoonrsdata{05:58}{18:18}{16:26}{04:50*}{12:08}
{\kalas{04:25 05:12 09:16 08:26 10:05 16:39 10:54 13:22 15:50 17:29 19:05 21:13 22:40 01:35*}}}
{\tnykdata{\anga{\tithi{13}{शुक्ल-त्रयोदशी}}{\time{40-32}{22:24}}\hspace{1ex}}%
{\anga{उत्तरफल्गुनी}{\time{43-33}{23:34}}\hspace{1ex}}{चन्द्रराशिः—\mbox{सिंहः\RIGHTarrow{07:15}}}%
{\anga{ध्रुवः}{\time{30-26}{18:28}}\hspace{1ex}\uanga{व्याघातः}}%
{\anga{कौलवम्}{\time{14-27}{11:55}}\hspace{1ex}\anga{तैतिलम्}{\time{40-32}{22:24}}\hspace{1ex}\uanga{गरजा}}{}
}
{दमनक-चोरी-उत्सवः\eventsep मदन-त्रयोदशी\eventsep प्रदोष-व्रतम्~18:18\RIGHTarrow{}19:45\eventsep वेङ्कटाचले वसन्तोत्सवः}
{Wed} 
\cfoot{\rygdata{12:08--13:41}{07:31--09:03}{10:36--12:08}}
\caldata{APRIL}{18}{\sunmonth{मेषः}{5}{}{चैत्रः}{वसन्तऋतुः}{गुरुः}{विकारी}{उत्तरायणम्}{वसन्तऋतुः}}
{\sunmoonrsdata{05:58}{18:18}{17:24}{05:36*}{12:08}
{\kalas{04:25 05:11 09:15 08:26 10:05 16:39 10:54 13:22 15:50 17:29 19:05 21:13 22:40 01:35*}}}
{\tnykdata{\anga{\tithi{14}{शुक्ल-चतुर्दशी}}{\time{32-54}{19:26}}\hspace{1ex}}%
{\anga{हस्तः}{\time{37-58}{21:24}}\hspace{1ex}}{चन्द्रराशिः—\mbox{कन्या}}%
{\anga{व्याघातः}{\time{21-44}{14:54}}\hspace{1ex}\uanga{हर्षणः}}%
{\anga{गरजा}{\time{7-7}{08:54}}\hspace{1ex}\anga{वणिजा}{\time{32-54}{19:26}}\hspace{1ex}\uanga{भद्रा}}{}
}
{अनध्यायः\eventsep चित्रा-पूर्णिमा\eventsep दमनक-चतुर्दशी\eventsep मदन-चतुर्दशी\eventsep नृसिंह-दोलोत्सवः\eventsep पञ्च-पर्व-पूजा (पूर्णिमा)\eventsep प्रोक्लस्-मृत्युः~\#{१५३४}\eventsep वेङ्कटाचले पूर्णिमा-गरुड-सेवा\eventsep वेङ्कटाचले वसन्तोत्सव-समापनम्}
{Thu} 
\cfoot{\rygdata{13:41--15:13}{05:58--07:31}{09:03--10:36}}
\caldata{APRIL}{19}{\sunmonth{मेषः}{6}{}{चैत्रः}{वसन्तऋतुः}{शुक्रः}{विकारी}{उत्तरायणम्}{वसन्तऋतुः}}
{\sunmoonrsdata{05:57}{18:18}{18:22}{---}{12:08}
{\kalas{04:24 05:11 09:15 08:26 10:04 16:39 10:54 13:22 15:50 17:29 19:05 21:13 22:40 01:35*}}}
{\tnykdata{\anga{\tithi{15}{पौर्णमासी}}{\time{26-5}{16:42}}\hspace{1ex}}%
{\anga{चित्रा}{\time{32-59}{19:28}}\hspace{1ex}}{चन्द्रराशिः—\mbox{कन्या\RIGHTarrow{08:23}}}%
{\anga{हर्षणः}{\time{13-27}{11:30}}\hspace{1ex}\uanga{वज्रम्}}%
{\anga{भद्रा}{\time{0-10}{06:02}}\hspace{1ex}\anga{बवम्}{\time{26-5}{16:42}}\hspace{1ex}\anga{बालवम्}{\time{53-35}{03:28*}}\hspace{1ex}\uanga{कौलवम्}}{}
}
{अनध्यायः\eventsep अनध्यायः\eventsep चैत्र-पूर्णिमा\eventsep चित्रगुप्त-व्रतम्\eventsep गजेन्द्र-मोक्षः\eventsep \tamil{இசைஞானியார் நாயன்மார் (63) குருபூஜை}\eventsep \tamil{மதுரகவி ஆழ்வார் திருநக்ஷத்திரம்}\eventsep मन्वादिः-(रौच्यः-[१३])\eventsep पार्वणव्रतम् पूर्णिमायाम्\eventsep पूर्णिमा-व्रतम्\eventsep श्री-हनूमत्-जयन्ती}
{Fri} 
\cfoot{\rygdata{10:35--12:08}{15:13--16:46}{07:30--09:03}}
\caldata{APRIL}{20}{\sunmonth{मेषः}{7}{}{चैत्रः}{वसन्तऋतुः}{शनिः}{विकारी}{उत्तरायणम्}{वसन्तऋतुः}}
{\sunmoonsrdata{05:57}{18:18}{19:21}{06:24}{12:08}
{\kalas{04:24 05:10 09:15 08:25 10:04 16:39 10:53 13:22 15:50 17:29 19:05 21:13 22:40 01:35*}}}
{\tnykdata{\anga{\tithi{16}{कृष्ण-प्रथमा}}{\time{20-22}{14:21}}\hspace{1ex}}%
{\anga{स्वाती}{\time{29-7}{17:57}}\hspace{1ex}}{चन्द्रराशिः—\mbox{तुला}}%
{\anga{वज्रम्}{\time{5-52}{08:22}}\hspace{1ex}\anga{सिद्धिः}{\time{59-16}{05:39*}}\hspace{1ex}\uanga{व्यतीपातः}}%
{\anga{कौलवम्}{\time{20-22}{14:21}}\hspace{1ex}\anga{तैतिलम्}{\time{48-12}{01:22*}}\hspace{1ex}\uanga{गरजा}}{}
}
{(सायन) विष्णुपदी-पुण्यकालः~08:01\RIGHTarrow{}18:18\eventsep अनध्यायः\eventsep माधव-मासः~14:25\RIGHTarrow{}\eventsep पार्वण-प्रायश्चित्तावकाशः पौर्णमास्याम्\eventsep पूर्णमासेष्टिः\eventsep सायन-सङ्क्रमण-दिन-अपराह्ण-पुण्यकालः~12:08\RIGHTarrow{}18:18\eventsep पूर्ण-स्थालीपाकः\eventsep \tamil{திருக்குறிப்புத் தொண்ட நாயன்மார் (19) குருபூஜை}\eventsep त्रिपुष्कर-योगः~17:57\RIGHTarrow{}05:56*}
{Sat} 
\cfoot{\rygdata{09:02--10:35}{13:40--15:13}{05:57--07:30}}
\caldata{APRIL}{21}{\sunmonth{मेषः}{8}{}{चैत्रः}{वसन्तऋतुः}{भानुः}{विकारी}{उत्तरायणम्}{वसन्तऋतुः}}
{\sunmoonsrdata{05:56}{18:19}{20:19}{07:13}{12:07}
{\kalas{04:23 05:10 09:14 08:25 10:04 16:39 10:53 13:22 15:50 17:29 19:05 21:13 22:40 01:34*}}}
{\tnykdata{\anga{\tithi{17}{कृष्ण-द्वितीया}}{\time{16-0}{12:32}}\hspace{1ex}}%
{\anga{विशाखा}{\time{26-47}{16:59}}\hspace{1ex}}{चन्द्रराशिः—\mbox{तुला\RIGHTarrow{11:10}}}%
{\prev{\anga{सिद्धिः}{\time{*59-16}{05:39}}}\hspace{1ex}\anga{व्यतीपातः}{\time{53-38}{03:28*}}\hspace{1ex}\uanga{वरीयान्}}%
{\anga{गरजा}{\time{16-0}{12:32}}\hspace{1ex}\anga{वणिजा}{\time{44-23}{23:53}}\hspace{1ex}\uanga{भद्रा}}{}
}
{पातार्क-योगः\eventsep त्रिपुष्कर-योगः~05:56\RIGHTarrow{}12:32\eventsep व्यतीपात-श्राद्धम्}
{Sun} 
\cfoot{\rygdata{16:46--18:19}{12:07--13:40}{15:13--16:46}}
\caldata{APRIL}{22}{\sunmonth{मेषः}{9}{}{चैत्रः}{वसन्तऋतुः}{सोमः}{विकारी}{उत्तरायणम्}{वसन्तऋतुः}}
{\sunmoonsrdata{05:56}{18:19}{21:16}{08:04}{12:07}
{\kalas{04:23 05:09 09:14 08:24 10:03 16:40 10:53 13:21 15:50 17:29 19:05 21:13 22:40 01:34*}}}
{\tnykdata{\anga{\tithi{18}{कृष्ण-तृतीया}}{\time{13-17}{11:25}}\hspace{1ex}}%
{\anga{अनूराधा}{\time{26-10}{16:44}}\hspace{1ex}}{चन्द्रराशिः—\mbox{वृश्चिकः}}%
{\anga{वरीयान्}{\time{49-33}{01:53*}}\hspace{1ex}\uanga{परिघः}}%
{\anga{भद्रा}{\time{13-17}{11:25}}\hspace{1ex}\anga{बवम्}{\time{42-28}{23:08}}\hspace{1ex}\uanga{बालवम्}}{}
}
{विकट-महागणपति-सङ्कटहर-चतुर्थी-व्रतम्}
{Mon} 
\cfoot{\rygdata{07:29--09:02}{10:34--12:07}{13:40--15:13}}
\caldata{APRIL}{23}{\sunmonth{मेषः}{10}{}{चैत्रः}{वसन्तऋतुः}{मङ्गलः}{विकारी}{उत्तरायणम्}{वसन्तऋतुः}}
{\sunmoonsrdata{05:55}{18:19}{22:12}{08:57}{12:07}
{\kalas{04:22 05:09 09:13 08:24 10:03 16:40 10:53 13:21 15:50 17:29 19:05 21:13 22:40 01:34*}}}
{\tnykdata{\anga{\tithi{19}{कृष्ण-चतुर्थी}}{\time{12-26}{11:04}}\hspace{1ex}}%
{\anga{ज्येष्ठा}{\time{27-25}{17:15}}\hspace{1ex}}{चन्द्रराशिः—\mbox{वृश्चिकः\RIGHTarrow{17:15}}}%
{\anga{परिघः}{\time{47-6}{00:56*}}\hspace{1ex}\uanga{शिवः}}%
{\anga{बालवम्}{\time{12-26}{11:04}}\hspace{1ex}\anga{कौलवम्}{\time{42-37}{23:12}}\hspace{1ex}\uanga{तैतिलम्}}{}
}
{अङ्गारकी-चतुर्थी\eventsep गुरु-सङ्क्रान्तिः~(धनुः\To{}वृश्चिकः)~07:42\RIGHTarrow{}\eventsep वराह-जयन्ती}
{Tue} 
\cfoot{\rygdata{15:13--16:46}{09:01--10:34}{12:07--13:40}}
\caldata{APRIL}{24}{\sunmonth{मेषः}{11}{}{चैत्रः}{वसन्तऋतुः}{बुधः}{विकारी}{उत्तरायणम्}{वसन्तऋतुः}}
{\sunmoonsrdata{05:55}{18:19}{23:04}{09:50}{12:07}
{\kalas{04:22 05:08 09:13 08:24 10:03 16:40 10:52 13:21 15:50 17:29 19:05 21:13 22:40 01:34*}}}
{\tnykdata{\anga{\tithi{20}{कृष्ण-पञ्चमी}}{\time{13-35}{11:32}}\hspace{1ex}}%
{\anga{मूला}{\time{30-37}{18:33}}\hspace{1ex}}{चन्द्रराशिः—\mbox{धनुः}}%
{\anga{शिवः}{\time{46-17}{00:36*}}\hspace{1ex}\uanga{सिद्धः}}%
{\anga{तैतिलम्}{\time{13-35}{11:32}}\hspace{1ex}\anga{गरजा}{\time{44-52}{00:04*}}\hspace{1ex}\uanga{वणिजा}}{}
}
{}
{Wed} 
\cfoot{\rygdata{12:07--13:40}{07:28--09:01}{10:34--12:07}}
\caldata{APRIL}{25}{\sunmonth{मेषः}{12}{}{चैत्रः}{वसन्तऋतुः}{गुरुः}{विकारी}{उत्तरायणम्}{वसन्तऋतुः}}
{\sunmoonsrdata{05:54}{18:19}{23:53}{10:43}{12:07}
{\kalas{04:22 05:08 09:13 08:23 10:02 16:40 10:52 13:21 15:50 17:29 19:05 21:13 22:40 01:33*}}}
{\tnykdata{\anga{\tithi{21}{कृष्ण-षष्ठी}}{\time{16-35}{12:46}}\hspace{1ex}}%
{\anga{पूर्वाषाढा}{\time{35-53}{20:35}}\hspace{1ex}}{चन्द्रराशिः—\mbox{धनुः\RIGHTarrow{03:12*}}}%
{\anga{सिद्धः}{\time{46-54}{00:51*}}\hspace{1ex}\uanga{साध्यः}}%
{\anga{वणिजा}{\time{16-35}{12:46}}\hspace{1ex}\anga{भद्रा}{\time{48-59}{01:39*}}\hspace{1ex}\uanga{बवम्}}{}
}
{}
{Thu} 
\cfoot{\rygdata{13:40--15:13}{05:54--07:27}{09:00--10:34}}
\caldata{APRIL}{26}{\sunmonth{मेषः}{13}{}{चैत्रः}{वसन्तऋतुः}{शुक्रः}{विकारी}{उत्तरायणम्}{वसन्तऋतुः}}
{\sunmoonsrdata{05:54}{18:19}{00:38*}{11:34}{12:06}
{\kalas{04:21 05:07 09:12 08:23 10:02 16:40 10:52 13:21 15:50 17:29 19:05 21:13 22:39 01:33*}}}
{\tnykdata{\anga{\tithi{22}{कृष्ण-सप्तमी}}{\time{21-10}{14:40}}\hspace{1ex}}%
{\anga{उत्तराषाढा}{\time{42-40}{23:12}}\hspace{1ex}}{चन्द्रराशिः—\mbox{मकरः}}%
{\anga{साध्यः}{\time{48-39}{01:31*}}\hspace{1ex}\uanga{शुभः}}%
{\anga{बवम्}{\time{21-10}{14:40}}\hspace{1ex}\anga{बालवम्}{\time{54-34}{03:48*}}\hspace{1ex}\uanga{कौलवम्}}{}
}
{पञ्च-पर्व-पूजा (अष्टमी)}
{Fri} 
\cfoot{\rygdata{10:33--12:06}{15:13--16:46}{07:27--09:00}}
\caldata{APRIL}{27}{\sunmonth{मेषः}{14}{}{चैत्रः}{वसन्तऋतुः}{शनिः}{विकारी}{उत्तरायणम्}{वसन्तऋतुः}}
{\sunmoonsrdata{05:53}{18:19}{01:19*}{12:24}{12:06}
{\kalas{04:21 05:07 09:12 08:22 10:02 16:40 10:52 13:21 15:50 17:30 19:06 21:13 22:39 01:33*}}}
{\tnykdata{\anga{\tithi{23}{कृष्ण-अष्टमी}}{\time{26-50}{17:01}}\hspace{1ex}}%
{\anga{श्रवणः}{\time{50-23}{02:11*}}\hspace{1ex}}{चन्द्रराशिः—\mbox{मकरः}}%
{\anga{शुभः}{\time{51-6}{02:27*}}\hspace{1ex}\uanga{शुक्लः}}%
{\prev{\anga{बालवम्}{\time{*54-34}{03:48}}}\hspace{1ex}\anga{कौलवम्}{\time{26-50}{17:01}}\hspace{1ex}\uanga{तैतिलम्}}{}
}
{अनध्यायः\eventsep काञ्ची ५६ जगद्गुरु श्री-सर्वज्ञ सदाशिव बोधेन्द्र सरस्वती आराधना~\#{४८१}\eventsep \tamil{நடராஜர் சித்திரை ஓணம் மஹாபிஷேகம்}\eventsep श्रवण-व्रतम्}
{Sat} 
\cfoot{\rygdata{09:00--10:33}{13:40--15:13}{05:53--07:26}}
\caldata{APRIL}{28}{\sunmonth{मेषः}{15}{}{चैत्रः}{वसन्तऋतुः}{भानुः}{विकारी}{उत्तरायणम्}{वसन्तऋतुः}}
{\sunmoonsrdata{05:53}{18:20}{01:59*}{13:12}{12:06}
{\kalas{04:20 05:07 09:12 08:22 10:02 16:40 10:51 13:21 15:50 17:30 19:06 21:13 22:39 01:33*}}}
{\tnykdata{\anga{\tithi{24}{कृष्ण-नवमी}}{\time{33-13}{19:34}}\hspace{1ex}}%
{\anga{श्रविष्ठा}{\time{58-24}{05:16*}}\hspace{1ex}}{चन्द्रराशिः—\mbox{मकरः\RIGHTarrow{15:43}}}%
{\anga{शुक्लः}{\time{53-46}{03:29*}}\hspace{1ex}\uanga{ब्राह्मः}}%
{\anga{तैतिलम्}{\time{0-57}{06:17}}\hspace{1ex}\anga{गरजा}{\time{33-13}{19:34}}\hspace{1ex}\uanga{वणिजा}}{}
}
{}
{Sun} 
\cfoot{\rygdata{16:46--18:20}{12:06--13:39}{15:13--16:46}}
\caldata{APRIL}{29}{\sunmonth{मेषः}{16}{}{चैत्रः}{वसन्तऋतुः}{सोमः}{विकारी}{उत्तरायणम्}{वसन्तऋतुः}}
{\sunmoonsrdata{05:52}{18:20}{02:36*}{13:59}{12:06}
{\kalas{04:20 05:06 09:12 08:22 10:01 16:40 10:51 13:21 15:50 17:30 19:06 21:13 22:39 01:32*}}}
{\tnykdata{\anga{\tithi{25}{कृष्ण-दशमी}}{\time{39-43}{22:04}}\hspace{1ex}}%
{\prev{\anga{श्रविष्ठा}{\time{*58-24}{05:16}}}\hspace{1ex}\fullanga{शतभिषक्}}{चन्द्रराशिः—\mbox{कुम्भः}}%
{\anga{ब्राह्मः}{\time{56-13}{04:25*}}\hspace{1ex}\uanga{माहेन्द्रः}}%
{\anga{वणिजा}{\time{7-8}{08:50}}\hspace{1ex}\anga{भद्रा}{\time{39-43}{22:04}}\hspace{1ex}\uanga{बवम्}}{}
}
{\tamil{திருநாவுக்கரச நாயன்மார் (21) குருபூஜை}}
{Mon} 
\cfoot{\rygdata{07:26--08:59}{10:33--12:06}{13:39--15:13}}
\caldata{APRIL}{30}{\sunmonth{मेषः}{17}{}{चैत्रः}{वसन्तऋतुः}{मङ्गलः}{विकारी}{उत्तरायणम्}{वसन्तऋतुः}}
{\sunmoonsrdata{05:52}{18:20}{03:13*}{14:45}{12:06}
{\kalas{04:20 05:06 09:11 08:21 10:01 16:40 10:51 13:21 15:50 17:30 19:06 21:13 22:39 01:32*}}}
{\tnykdata{\anga{\tithi{26}{कृष्ण-एकादशी}}{\time{45-30}{00:18*}}\hspace{1ex}}%
{\anga{शतभिषक्}{\time{5-39}{08:13}}\hspace{1ex}}{चन्द्रराशिः—\mbox{कुम्भः\RIGHTarrow{04:13*}}}%
{\prev{\anga{ब्राह्मः}{\time{*56-13}{04:25}}}\hspace{1ex}\anga{माहेन्द्रः}{\time{58-2}{05:06*}}\hspace{1ex}\uanga{वैधृतिः}}%
{\anga{बवम्}{\time{12-54}{11:14}}\hspace{1ex}\anga{बालवम्}{\time{45-30}{00:18*}}\hspace{1ex}\uanga{कौलवम्}}{}
}
{सर्व-वरूथिनी-एकादशी\eventsep त्रिपुष्कर-योगः~00:17*\RIGHTarrow{}05:51*\eventsep वल्लभाचार्य-जयन्ती~\#{५४१}}
{Tue} 
\cfoot{\rygdata{15:13--16:46}{08:59--10:32}{12:06--13:39}}
\caldata{MAY}{1}{\sunmonth{मेषः}{18}{}{चैत्रः}{वसन्तऋतुः}{बुधः}{विकारी}{उत्तरायणम्}{वसन्तऋतुः}}
{\sunmoonsrdata{05:51}{18:20}{03:50*}{15:31}{12:06}
{\kalas{04:19 05:05 09:11 08:21 10:01 16:40 10:51 13:21 15:50 17:30 19:06 21:13 22:39 01:32*}}}
{\tnykdata{\anga{\tithi{27}{कृष्ण-द्वादशी}}{\time{50-11}{02:05*}}\hspace{1ex}}%
{\anga{पूर्वप्रोष्ठपदा}{\time{11-58}{10:50}}\hspace{1ex}}{चन्द्रराशिः—\mbox{मीनः}}%
{\prev{\anga{माहेन्द्रः}{\time{*58-2}{05:06}}}\hspace{1ex}\anga{वैधृतिः}{\time{58-57}{05:27*}}\hspace{1ex}\uanga{विष्कम्भः}}%
{\anga{कौलवम्}{\time{17-46}{13:15}}\hspace{1ex}\anga{तैतिलम्}{\time{50-11}{02:05*}}\hspace{1ex}\uanga{गरजा}}{}
}
{हरिवासरः\RIGHTarrow{}06:47\eventsep वैधृति-श्राद्धम्}
{Wed} 
\cfoot{\rygdata{12:06--13:39}{07:25--08:59}{10:32--12:06}}
\caldata{MAY}{2}{\sunmonth{मेषः}{19}{}{चैत्रः}{वसन्तऋतुः}{गुरुः}{विकारी}{उत्तरायणम्}{वसन्तऋतुः}}
{\sunmoonsrdata{05:51}{18:20}{04:29*}{16:19}{12:06}
{\kalas{04:19 05:05 09:11 08:21 10:01 16:40 10:51 13:21 15:50 17:30 19:06 21:13 22:39 01:32*}}}
{\tnykdata{\anga{\tithi{28}{कृष्ण-त्रयोदशी}}{\time{53-29}{03:21*}}\hspace{1ex}}%
{\anga{उत्तरप्रोष्ठपदा}{\time{17-10}{13:00}}\hspace{1ex}}{चन्द्रराशिः—\mbox{मीनः}}%
{\prev{\anga{वैधृतिः}{\time{*58-57}{05:27}}}\hspace{1ex}\anga{विष्कम्भः}{\time{58-49}{05:24*}}\hspace{1ex}\uanga{प्रीतिः}}%
{\anga{गरजा}{\time{21-27}{14:47}}\hspace{1ex}\anga{वणिजा}{\time{53-29}{03:21*}}\hspace{1ex}\uanga{भद्रा}}{}
}
{देवी-पर्व-१\eventsep मत्स्य-जयन्ती\eventsep प्रदोष-व्रतम्~18:20\RIGHTarrow{}19:46\eventsep रमण-महर्षि-आराधना~\#{६९}}
{Thu} 
\cfoot{\rygdata{13:39--15:13}{05:51--07:25}{08:58--10:32}}
\caldata{MAY}{3}{\sunmonth{मेषः}{20}{}{चैत्रः}{वसन्तऋतुः}{शुक्रः}{विकारी}{उत्तरायणम्}{वसन्तऋतुः}}
{\sunmoonsrdata{05:51}{18:20}{05:09*}{17:08}{12:05}
{\kalas{04:19 05:05 09:11 08:21 10:01 16:40 10:51 13:20 15:50 17:30 19:06 21:13 22:39 01:32*}}}
{\tnykdata{\anga{\tithi{29}{कृष्ण-चतुर्दशी}}{\time{55-21}{04:04*}}\hspace{1ex}}%
{\anga{रेवती}{\time{21-6}{14:38}}\hspace{1ex}}{चन्द्रराशिः—\mbox{मीनः\RIGHTarrow{14:38}}}%
{\prev{\anga{विष्कम्भः}{\time{*58-49}{05:24}}}\hspace{1ex}\anga{प्रीतिः}{\time{57-33}{04:54*}}\hspace{1ex}\uanga{आयुष्मान्}}%
{\anga{भद्रा}{\time{23-49}{15:46}}\hspace{1ex}\anga{शकुनिः}{\time{55-21}{04:04*}}\hspace{1ex}\uanga{चतुष्पात्}}{}
}
{अनध्यायः\eventsep भृगुरेवती-योगः\RIGHTarrow{}14:38\eventsep गङ्गा-स्नानम्\eventsep मासशिवरात्रिः\eventsep पञ्च-पर्व-पूजा (चतुर्दशी)}
{Fri} 
\cfoot{\rygdata{10:32--12:06}{15:13--16:47}{07:24--08:58}}
\caldata{MAY}{4}{\sunmonth{मेषः}{21}{}{चैत्रः}{वसन्तऋतुः}{शनिः}{विकारी}{उत्तरायणम्}{वसन्तऋतुः}}
{\sunmoonsrdata{05:50}{18:21}{---}{17:59}{12:05}
{\kalas{04:18 05:04 09:10 08:20 10:00 16:40 10:50 13:20 15:50 17:31 19:06 21:13 22:39 01:31*}}}
{\tnykdata{\prev{\anga{\tithi{29}{कृष्ण-चतुर्दशी}}{\time{*55-21}{04:04}}}\hspace{1ex}\anga{\tithi{30}{अमावास्या}}{\time{55-52}{04:15*}}\hspace{1ex}}%
{\anga{अश्विनी}{\time{23-47}{15:45}}\hspace{1ex}}{चन्द्रराशिः—\mbox{मेषः}}%
{\prev{\anga{प्रीतिः}{\time{*57-33}{04:54}}}\hspace{1ex}\anga{आयुष्मान्}{\time{55-15}{04:01*}}\hspace{1ex}\uanga{सौभाग्यः}}%
{\prev{\anga{शकुनिः}{\time{*55-21}{04:04}}}\hspace{1ex}\anga{चतुष्पात्}{\time{24-54}{16:13}}\hspace{1ex}\anga{नाग}{\time{55-52}{04:15*}}\hspace{1ex}\uanga{किंस्तुघ्नः}}{}
}
{अग्निनक्षत्र-आरम्भः~02:33*\RIGHTarrow{}\eventsep अनध्यायः\eventsep भार्गव-राम-पूजा\eventsep काञ्ची ४७ जगद्गुरु श्री-चन्द्रशेखरेन्द्र सरस्वती ३ आराधना~\#{८५४}\eventsep पार्वणव्रतम् अमावास्यायाम्\eventsep पञ्च-पर्व-पूजा (अमावास्या)\eventsep पिण्ड-पितृ-यज्ञः\eventsep सर्व-चैत्र-अमावास्या\eventsep वह्नि-व्रतम्}
{Sat} 
\cfoot{\rygdata{08:58--10:32}{13:39--15:13}{05:50--07:24}}
\caldata{MAY}{5}{\sunmonth{मेषः}{22}{}{वैशाखः}{वसन्तऋतुः}{भानुः}{विकारी}{उत्तरायणम्}{वसन्तऋतुः}}
{\sunmoonrsdata{05:50}{18:21}{05:53}{18:52}{12:05}
{\kalas{04:18 05:04 09:10 08:20 10:00 16:41 10:50 13:20 15:51 17:31 19:07 21:13 22:39 01:31*}}}
{\tnykdata{\prev{\anga{\tithi{30}{अमावास्या}}{\time{*55-52}{04:15}}}\hspace{1ex}\anga{\tithi{1}{शुक्ल-प्रथमा}}{\time{55-9}{03:58*}}\hspace{1ex}}%
{\anga{अपभरणी}{\time{25-16}{16:23}}\hspace{1ex}}{चन्द्रराशिः—\mbox{मेषः\RIGHTarrow{22:28}}}%
{\prev{\anga{आयुष्मान्}{\time{*55-15}{04:01}}}\hspace{1ex}\anga{सौभाग्यः}{\time{51-58}{02:45*}}\hspace{1ex}\uanga{शोभनः}}%
{\prev{\anga{नाग}{\time{*55-52}{04:15}}}\hspace{1ex}\anga{किंस्तुघ्नः}{\time{24-46}{16:10}}\hspace{1ex}\anga{बवम्}{\time{55-9}{03:58*}}\hspace{1ex}\uanga{बालवम्}}{}
}
{अनध्यायः\eventsep \tamil{சிறுத்தொண்ட நாயன்மார் (36) குருபூஜை}\eventsep दर्शेष्टिः\eventsep कृत्तिका-व्रतम्\eventsep पार्वण-प्रायश्चित्तावकाशः दर्शे\eventsep पराशर-महर्षि-जयन्ती\eventsep दर्श-स्थालीपाकः\eventsep त्रिपुष्कर-योगः~03:58*\RIGHTarrow{}05:49*\eventsep वैशाख-मास-आरम्भः}
{Sun} 
\cfoot{\rygdata{16:47--18:21}{12:05--13:39}{15:13--16:47}}
\caldata{MAY}{6}{\sunmonth{मेषः}{23}{}{वैशाखः}{वसन्तऋतुः}{सोमः}{विकारी}{उत्तरायणम्}{वसन्तऋतुः}}
{\sunmoonrsdata{05:49}{18:21}{06:40}{19:49}{12:05}
{\kalas{04:18 05:03 09:10 08:20 10:00 16:41 10:50 13:20 15:51 17:31 19:07 21:13 22:39 01:31*}}}
{\tnykdata{\prev{\anga{\tithi{1}{शुक्ल-प्रथमा}}{\time{*55-9}{03:58}}}\hspace{1ex}\anga{\tithi{2}{शुक्ल-द्वितीया}}{\time{53-23}{03:18*}}\hspace{1ex}}%
{\anga{कृत्तिका}{\time{25-45}{16:35}}\hspace{1ex}}{चन्द्रराशिः—\mbox{वृषभः}}%
{\anga{शोभनः}{\time{47-50}{01:10*}}\hspace{1ex}\uanga{अतिगण्डः}}%
{\prev{\anga{बवम्}{\time{*55-9}{03:58}}}\hspace{1ex}\anga{बालवम्}{\time{23-36}{15:41}}\hspace{1ex}\anga{कौलवम्}{\time{53-23}{03:18*}}\hspace{1ex}\uanga{तैतिलम्}}{}
}
{चन्द्र-दर्शनम्~18:21\RIGHTarrow{}19:49\eventsep श्यामा-शास्त्री-जयन्ती~\#{२५८}}
{Mon} 
\cfoot{\rygdata{07:23--08:57}{10:31--12:05}{13:39--15:13}}
\caldata{MAY}{7}{\sunmonth{मेषः}{24}{}{वैशाखः}{वसन्तऋतुः}{मङ्गलः}{विकारी}{उत्तरायणम्}{वसन्तऋतुः}}
{\sunmoonrsdata{05:49}{18:21}{07:32}{20:47}{12:05}
{\kalas{04:17 05:03 09:10 08:19 10:00 16:41 10:50 13:20 15:51 17:31 19:07 21:13 22:39 01:31*}}}
{\tnykdata{\anga{\tithi{3}{शुक्ल-तृतीया}}{\time{50-44}{02:17*}}\hspace{1ex}}%
{\anga{रोहिणी}{\time{25-22}{16:25}}\hspace{1ex}}{चन्द्रराशिः—\mbox{वृषभः\RIGHTarrow{04:14*}}}%
{\anga{अतिगण्डः}{\time{43-0}{23:19}}\hspace{1ex}\uanga{सुकर्म}}%
{\anga{तैतिलम्}{\time{21-33}{14:49}}\hspace{1ex}\anga{गरजा}{\time{50-44}{02:17*}}\hspace{1ex}\uanga{वणिजा}}{}
}
{अक्षय-तृतीया\eventsep अनध्यायः\eventsep बलराम-जयन्ती\eventsep चन्दन-पूजा\eventsep देवी-पर्व-२\eventsep कृतयुगादिः\eventsep \tamil{மங்கையர்க்கரசியார் நாயன்மார் (50) குருபூஜை}\eventsep राज-मातङ्गी-जयन्ती}
{Tue} 
\cfoot{\rygdata{15:13--16:47}{08:57--10:31}{12:05--13:39}}
\caldata{MAY}{8}{\sunmonth{मेषः}{25}{}{वैशाखः}{वसन्तऋतुः}{बुधः}{विकारी}{उत्तरायणम्}{वसन्तऋतुः}}
{\sunmoonrsdata{05:49}{18:21}{08:28}{21:44}{12:05}
{\kalas{04:17 05:03 09:09 08:19 10:00 16:41 10:50 13:20 15:51 17:31 19:07 21:13 22:39 01:31*}}}
{\tnykdata{\anga{\tithi{4}{शुक्ल-चतुर्थी}}{\time{47-21}{00:59*}}\hspace{1ex}}%
{\anga{मृगशीर्षम्}{\time{24-17}{15:58}}\hspace{1ex}}{चन्द्रराशिः—\mbox{मिथुनम्}}%
{\anga{सुकर्म}{\time{37-33}{21:14}}\hspace{1ex}\uanga{धृतिः}}%
{\anga{वणिजा}{\time{18-46}{13:40}}\hspace{1ex}\anga{भद्रा}{\time{47-21}{00:59*}}\hspace{1ex}\uanga{बवम्}}{}
}
{बगलामुखी-जयन्ती\eventsep वार्ता-गौरी-व्रतम्\eventsep शुक्ल-चतुर्थी-व्रतम्}
{Wed} 
\cfoot{\rygdata{12:05--13:39}{07:23--08:57}{10:31--12:05}}
\caldata{MAY}{9}{\sunmonth{मेषः}{26}{}{वैशाखः}{वसन्तऋतुः}{गुरुः}{विकारी}{उत्तरायणम्}{वसन्तऋतुः}}
{\sunmoonrsdata{05:48}{18:22}{09:26}{22:41}{12:05}
{\kalas{04:17 05:03 09:09 08:19 10:00 16:41 10:50 13:20 15:51 17:31 19:07 21:13 22:39 01:31*}}}
{\tnykdata{\anga{\tithi{5}{शुक्ल-पञ्चमी}}{\time{43-19}{23:27}}\hspace{1ex}}%
{\anga{आर्द्रा}{\time{22-34}{15:15}}\hspace{1ex}}{चन्द्रराशिः—\mbox{मिथुनम्}}%
{\anga{धृतिः}{\time{31-33}{18:57}}\hspace{1ex}\uanga{शूलः}}%
{\anga{बवम्}{\time{15-22}{12:14}}\hspace{1ex}\anga{बालवम्}{\time{43-19}{23:27}}\hspace{1ex}\uanga{कौलवम्}}{}
}
{लावण्य-गौरी-व्रतम्\eventsep रामानुज-जन्म-नक्षत्रम्~\#{१००३}\eventsep सूरदास-जयन्ती~\#{५४२}\eventsep सर्प-पूजा\eventsep \tamil{விறன்மிண்ட நாயன்மார் (6) குருபூஜை}\eventsep शङ्कर-जयन्ती~\#{२५२८}}
{Thu} 
\cfoot{\rygdata{13:39--15:13}{05:48--07:23}{08:57--10:31}}
\caldata{MAY}{10}{\sunmonth{मेषः}{27}{}{वैशाखः}{वसन्तऋतुः}{शुक्रः}{विकारी}{उत्तरायणम्}{वसन्तऋतुः}}
{\sunmoonrsdata{05:48}{18:22}{10:26}{23:35}{12:05}
{\kalas{04:17 05:02 09:09 08:19 09:59 16:41 10:50 13:20 15:51 17:32 19:08 21:13 22:39 01:31*}}}
{\tnykdata{\anga{\tithi{6}{शुक्ल-षष्ठी}}{\time{38-43}{21:41}}\hspace{1ex}}%
{\anga{पुनर्वसुः}{\time{20-21}{14:19}}\hspace{1ex}}{चन्द्रराशिः—\mbox{मिथुनम्\RIGHTarrow{08:35}}}%
{\anga{शूलः}{\time{25-32}{16:30}}\hspace{1ex}\uanga{गण्डः}}%
{\anga{कौलवम्}{\time{11-26}{10:35}}\hspace{1ex}\anga{तैतिलम्}{\time{38-43}{21:41}}\hspace{1ex}\uanga{गरजा}}{}
}
{षष्ठी-व्रतम्\eventsep काञ्ची ४० जगद्गुरु श्री-महादेवेन्द्र सरस्वती २ आराधना~\#{११०५}\eventsep रामानुज-जयन्ती~\#{१००३}}
{Fri} 
\cfoot{\rygdata{10:31--12:05}{15:13--16:48}{07:22--08:57}}
\caldata{MAY}{11}{\sunmonth{मेषः}{28}{}{वैशाखः}{वसन्तऋतुः}{शनिः}{विकारी}{उत्तरायणम्}{वसन्तऋतुः}}
{\sunmoonrsdata{05:48}{18:22}{11:25}{00:26*}{12:05}
{\kalas{04:16 05:02 09:09 08:19 09:59 16:41 10:50 13:20 15:51 17:32 19:08 21:13 22:39 01:31*}}}
{\tnykdata{\anga{\tithi{7}{शुक्ल-सप्तमी}}{\time{33-35}{19:44}}\hspace{1ex}}%
{\anga{पुष्यः}{\time{17-39}{13:12}}\hspace{1ex}}{चन्द्रराशिः—\mbox{कर्कटः}}%
{\anga{गण्डः}{\time{19-17}{13:53}}\hspace{1ex}\uanga{वृद्धिः}}%
{\anga{गरजा}{\time{7-0}{08:44}}\hspace{1ex}\anga{वणिजा}{\time{33-35}{19:44}}\hspace{1ex}\uanga{भद्रा}}{}
}
{गङ्गा-सप्तमी\eventsep त्यागराज-जयन्ती~\#{२५३}\eventsep शर्करा-सप्तमी}
{Sat} 
\cfoot{\rygdata{08:56--10:31}{13:39--15:13}{05:48--07:22}}
\caldata{MAY}{12}{\sunmonth{मेषः}{29}{}{वैशाखः}{वसन्तऋतुः}{भानुः}{विकारी}{उत्तरायणम्}{वसन्तऋतुः}}
{\sunmoonrsdata{05:48}{18:22}{12:24}{01:13*}{12:05}
{\kalas{04:16 05:02 09:09 08:18 09:59 16:42 10:49 13:20 15:51 17:32 19:08 21:14 22:39 01:30*}}}
{\tnykdata{\anga{\tithi{8}{शुक्ल-अष्टमी}}{\time{28-11}{17:37}}\hspace{1ex}}%
{\anga{आश्रेषा}{\time{14-31}{11:53}}\hspace{1ex}}{चन्द्रराशिः—\mbox{कर्कटः\RIGHTarrow{11:53}}}%
{\anga{वृद्धिः}{\time{12-41}{11:07}}\hspace{1ex}\uanga{ध्रुवः}}%
{\anga{भद्रा}{\time{2-9}{06:42}}\hspace{1ex}\anga{बवम्}{\time{28-11}{17:37}}\hspace{1ex}\anga{बालवम्}{\time{56-36}{04:30*}}\hspace{1ex}\uanga{कौलवम्}}{}
}
{अनध्यायः\eventsep काञ्ची २६ जगद्गुरु श्री-प्रज्ञाघनेन्द्र सरस्वती आराधना~\#{१४५६}}
{Sun} 
\cfoot{\rygdata{16:48--18:22}{12:05--13:39}{15:14--16:48}}
\caldata{MAY}{13}{\sunmonth{मेषः}{30}{}{वैशाखः}{वसन्तऋतुः}{सोमः}{विकारी}{उत्तरायणम्}{वसन्तऋतुः}}
{\sunmoonrsdata{05:47}{18:23}{13:21}{01:59*}{12:05}
{\kalas{04:16 05:02 09:09 08:18 09:59 16:42 10:49 13:20 15:51 17:32 19:08 21:14 22:39 01:30*}}}
{\tnykdata{\anga{\tithi{9}{शुक्ल-नवमी}}{\time{22-47}{15:21}}\hspace{1ex}}%
{\anga{मघा}{\time{11-3}{10:25}}\hspace{1ex}}{चन्द्रराशिः—\mbox{सिंहः}}%
{\anga{ध्रुवः}{\time{5-49}{08:14}}\hspace{1ex}\anga{व्याघातः}{\time{58-35}{05:15*}}\hspace{1ex}\uanga{हर्षणः}}%
{\prev{\anga{बालवम्}{\time{*56-36}{04:30}}}\hspace{1ex}\anga{कौलवम्}{\time{22-47}{15:21}}\hspace{1ex}\anga{तैतिलम्}{\time{50-30}{02:11*}}\hspace{1ex}\uanga{गरजा}}{}
}
{नॆरूर्-श्री-सदाशिव-ब्रह्मेन्द्र-आराधना~\#{१०५}\eventsep पुरी गोवर्धन-मठ-प्रतिष्ठापन-जयन्ती~\#{२५०४}\eventsep सीतानवमी\eventsep सिंहाचलं-चन्दन-महोत्सवः\eventsep वेङ्कटाचले पद्मावती-परिणयोत्सव-प्रारम्भः (गज-वाहनम्)\eventsep वसिष्ठ-महर्षि-जयन्ती}
{Mon} 
\cfoot{\rygdata{07:22--08:56}{10:31--12:05}{13:39--15:14}}
\caldata{MAY}{14}{\sunmonth{मेषः}{31}{}{वैशाखः}{वसन्तऋतुः}{मङ्गलः}{विकारी}{उत्तरायणम्}{वसन्तऋतुः}}
{\sunmoonrsdata{05:47}{18:23}{14:17}{02:44*}{12:05}
{\kalas{04:16 05:01 09:09 08:18 09:59 16:42 10:49 13:20 15:52 17:32 19:08 21:14 22:39 01:30*}}}
{\tnykdata{\anga{\tithi{10}{शुक्ल-दशमी}}{\time{17-9}{12:59}}\hspace{1ex}}%
{\anga{पूर्वफल्गुनी}{\time{7-18}{08:51}}\hspace{1ex}}{चन्द्रराशिः—\mbox{सिंहः\RIGHTarrow{14:27}}}%
{\prev{\anga{व्याघातः}{\time{*58-35}{05:15}}}\hspace{1ex}\anga{हर्षणः}{\time{50-37}{02:13*}}\hspace{1ex}\uanga{वज्रम्}}%
{\anga{गरजा}{\time{17-9}{12:59}}\hspace{1ex}\anga{वणिजा}{\time{44-13}{23:47}}\hspace{1ex}\uanga{भद्रा}}{}
}
{काञ्ची १ जगद्गुरु श्री-आदि-शङ्कर भगवत्पाद आराधना~\#{२४९५}\eventsep \tamil{மீனாக்ஷீ திருக்கல்யாணம்}\eventsep निमिषाम्बा-जयन्ती\eventsep वेङ्कटाचले पद्मावती-परिणयम् (अश्व-वाहनम्)\eventsep श्री-वासवी-जयन्ती}
{Tue} 
\cfoot{\rygdata{15:14--16:48}{08:56--10:30}{12:05--13:39}}
\caldata{MAY}{15}{\sunmonth{वृषभः}{1}{\mbox{मेषः{\tiny\RIGHTarrow}{10:44}}}{वैशाखः}{वसन्तऋतुः}{बुधः}{विकारी}{उत्तरायणम्}{वसन्तऋतुः}}
{\sunmoonrsdata{05:47}{18:23}{15:13}{03:29*}{12:05}
{\kalas{04:16 05:01 09:08 08:18 09:59 16:42 10:49 13:21 15:52 17:33 19:09 21:14 22:39 01:30*}}}
{\tnykdata{\anga{\tithi{11}{शुक्ल-एकादशी}}{\time{11-27}{10:35}}\hspace{1ex}}%
{\anga{उत्तरफल्गुनी}{\time{3-28}{07:14}}\hspace{1ex}\anga{हस्तः}{\time{59-42}{05:40*}}\hspace{1ex}\avamA{}}{चन्द्रराशिः—\mbox{कन्या}}%
{\anga{वज्रम्}{\time{42-41}{23:12}}\hspace{1ex}\uanga{सिद्धिः}}%
{\anga{भद्रा}{\time{11-27}{10:35}}\hspace{1ex}\anga{बवम्}{\time{37-57}{21:24}}\hspace{1ex}\uanga{बालवम्}}{}
}
{बुध-जयन्ती\eventsep सङ्क्रमण-दिन-पूर्वाह्ण-पुण्यकालः~05:47\RIGHTarrow{}12:05\eventsep सर्व-मोहिनी-एकादशी\eventsep वेङ्कटाचले पद्मावती-परिणयोत्सव-समापनम् (गरुड-वाहनम्)\eventsep वृषभ-रवि-सङ्क्रमण-विष्णुपदी-पुण्यकालः~05:47\RIGHTarrow{}17:08}
{Wed} 
\cfoot{\rygdata{12:05--13:39}{07:21--08:56}{10:30--12:05}}
\caldata{MAY}{16}{\sunmonth{वृषभः}{2}{}{वैशाखः}{वसन्तऋतुः}{गुरुः}{विकारी}{उत्तरायणम्}{वसन्तऋतुः}}
{\sunmoonrsdata{05:47}{18:23}{16:09}{04:15*}{12:05}
{\kalas{04:15 05:01 09:08 08:18 09:59 16:42 10:49 13:21 15:52 17:33 19:09 21:14 22:39 01:30*}}}
{\tnykdata{\anga{\tithi{12}{शुक्ल-द्वादशी}}{\time{5-53}{08:15}}\hspace{1ex}}%
{\prev{\anga{हस्तः}{\time{*59-42}{05:40}}}\hspace{1ex}\anga{चित्रा}{\time{55-58}{04:15*}}\hspace{1ex}}{चन्द्रराशिः—\mbox{कन्या\RIGHTarrow{16:56}}}%
{\anga{सिद्धिः}{\time{35-1}{20:18}}\hspace{1ex}\uanga{व्यतीपातः}}%
{\anga{बालवम्}{\time{5-53}{08:15}}\hspace{1ex}\anga{कौलवम्}{\time{31-58}{19:08}}\hspace{1ex}\uanga{तैतिलम्}}{}
}
{गिरिजा-विवाहः\eventsep मधुसूदन-पूजा\eventsep परशुराम-द्वादशी\eventsep प्रदोष-व्रतम्~18:23\RIGHTarrow{}19:49\eventsep रुक्मिणी-द्वादशी}
{Thu} 
\cfoot{\rygdata{13:40--15:14}{05:47--07:21}{08:56--10:30}}
\caldata{MAY}{17}{\sunmonth{वृषभः}{3}{}{वैशाखः}{वसन्तऋतुः}{शुक्रः}{विकारी}{उत्तरायणम्}{वसन्तऋतुः}}
{\sunmoonrsdata{05:46}{18:24}{17:06}{05:02*}{12:05}
{\kalas{04:15 05:01 09:08 08:18 09:59 16:43 10:49 13:21 15:52 17:33 19:09 21:14 22:40 01:30*}}}
{\tnykdata{\anga{\tithi{13}{शुक्ल-त्रयोदशी}}{\time{0-43}{06:04}}\hspace{1ex}\anga{\tithi{14}{शुक्ल-चतुर्दशी}}{\time{55-47}{04:10*}}\hspace{1ex}\avamA{}}%
{\prev{\anga{चित्रा}{\time{*55-58}{04:15}}}\hspace{1ex}\anga{स्वाती}{\time{52-56}{03:05*}}\hspace{1ex}}{चन्द्रराशिः—\mbox{तुला}}%
{\anga{व्यतीपातः}{\time{28-1}{17:34}}\hspace{1ex}\uanga{वरीयान्}}%
{\anga{तैतिलम्}{\time{0-43}{06:04}}\hspace{1ex}\anga{गरजा}{\time{26-53}{17:05}}\hspace{1ex}\anga{वणिजा}{\time{55-47}{04:10*}}\hspace{1ex}\uanga{भद्रा}}{}
}
{अनध्यायः\eventsep छिन्नमस्ता-जयन्ती\eventsep दिनक्षयः\eventsep काञ्ची ३९ जगद्गुरु श्री-सच्चिद्विलासेन्द्र सरस्वती आराधना~\#{११४७}\eventsep नृसिंह-जयन्ती\eventsep वैशाख-मास-अन्तिमत्रयतिथि-व्रत-आरम्भः\eventsep व्यतीपात-श्राद्धम्}
{Fri} 
\cfoot{\rygdata{10:30--12:05}{15:14--16:49}{07:21--08:56}}
\caldata{MAY}{18}{\sunmonth{वृषभः}{4}{}{वैशाखः}{वसन्तऋतुः}{शनिः}{विकारी}{उत्तरायणम्}{वसन्तऋतुः}}
{\sunmoonrsdata{05:46}{18:24}{18:04}{---}{12:05}
{\kalas{04:15 05:01 09:08 08:18 09:59 16:43 10:49 13:21 15:52 17:33 19:09 21:14 22:40 01:30*}}}
{\tnykdata{\prev{\anga{\tithi{14}{शुक्ल-चतुर्दशी}}{\time{*55-47}{04:10}}}\hspace{1ex}\anga{\tithi{15}{पौर्णमासी}}{\time{51-51}{02:41*}}\hspace{1ex}}%
{\anga{विशाखा}{\time{50-56}{02:20*}}\hspace{1ex}}{चन्द्रराशिः—\mbox{तुला\RIGHTarrow{20:29}}}%
{\anga{वरीयान्}{\time{22-12}{15:07}}\hspace{1ex}\uanga{परिघः}}%
{\prev{\anga{वणिजा}{\time{*55-47}{04:10}}}\hspace{1ex}\anga{भद्रा}{\time{22-48}{15:22}}\hspace{1ex}\anga{बवम्}{\time{51-51}{02:41*}}\hspace{1ex}\uanga{बालवम्}}{}
}
{अनध्यायः\eventsep अन्नमाचार्य-जयन्ती\eventsep अर्धनारीश्वर-व्रतम्\eventsep काञ्ची कामकोटि-मठ-प्रतिष्ठापन-जयन्ती~\#{२५०१}\eventsep कृत्तिकावैषाखोत्सवः\eventsep \tamil{நம்மாழ்வார் திருநக்ஷத்திரம்}\eventsep पार्वणव्रतम् पूर्णिमायाम्\eventsep पूर्णिमा-व्रतम्\eventsep पञ्च-पर्व-पूजा (पूर्णिमा)\eventsep सम्पत्-गौरी-व्रतम्\eventsep वेङ्कटाचले पूर्णिमा-गरुड-सेवा\eventsep वैशाख-मास-अन्तिमत्रयतिथि-व्रत-समापनम्\eventsep वैशाख-पूर्णिमा-स्नानम्\eventsep शरभ-जयन्ती}
{Sat} 
\cfoot{\rygdata{08:55--10:30}{13:40--15:14}{05:46--07:21}}
\caldata{MAY}{19}{\sunmonth{वृषभः}{5}{}{वैशाखः}{वसन्तऋतुः}{भानुः}{विकारी}{उत्तरायणम्}{वसन्तऋतुः}}
{\sunmoonsrdata{05:46}{18:24}{19:02}{05:52}{12:05}
{\kalas{04:15 05:00 09:08 08:17 09:59 16:43 10:49 13:21 15:52 17:34 19:10 21:14 22:40 01:30*}}}
{\tnykdata{\anga{\tithi{16}{कृष्ण-प्रथमा}}{\time{49-17}{01:42*}}\hspace{1ex}}%
{\anga{अनूराधा}{\time{50-18}{02:06*}}\hspace{1ex}}{चन्द्रराशिः—\mbox{वृश्चिकः}}%
{\anga{परिघः}{\time{17-17}{13:03}}\hspace{1ex}\uanga{शिवः}}%
{\anga{बालवम्}{\time{19-50}{14:07}}\hspace{1ex}\anga{कौलवम्}{\time{49-17}{01:42*}}\hspace{1ex}\uanga{तैतिलम्}}{}
}
{अनध्यायः\eventsep काञ्ची ६८ जगद्गुरु श्री-चन्द्रशेखरेन्द्र सरस्वती ७ जयन्ती~\#{१२६}\eventsep पार्वण-प्रायश्चित्तावकाशः पौर्णमास्याम्\eventsep पूर्णमासेष्टिः\eventsep पूर्ण-स्थालीपाकः}
{Sun} 
\cfoot{\rygdata{16:49--18:24}{12:05--13:40}{15:15--16:49}}
\caldata{MAY}{20}{\sunmonth{वृषभः}{6}{}{वैशाखः}{वसन्तऋतुः}{सोमः}{विकारी}{उत्तरायणम्}{वसन्तऋतुः}}
{\sunmoonsrdata{05:46}{18:24}{19:58}{06:45}{12:05}
{\kalas{04:15 05:00 09:08 08:17 09:59 16:43 10:49 13:21 15:53 17:34 19:10 21:15 22:40 01:30*}}}
{\tnykdata{\anga{\tithi{17}{कृष्ण-द्वितीया}}{\time{48-20}{01:21*}}\hspace{1ex}}%
{\anga{ज्येष्ठा}{\time{51-16}{02:27*}}\hspace{1ex}}{चन्द्रराशिः—\mbox{वृश्चिकः\RIGHTarrow{02:27*}}}%
{\anga{शिवः}{\time{13-27}{11:26}}\hspace{1ex}\uanga{सिद्धः}}%
{\anga{तैतिलम्}{\time{18-14}{13:27}}\hspace{1ex}\anga{गरजा}{\time{48-20}{01:21*}}\hspace{1ex}\uanga{वणिजा}}{}
}
{नारद-जयन्ती}
{Mon} 
\cfoot{\rygdata{07:21--08:55}{10:30--12:05}{13:40--15:15}}
\caldata{MAY}{21}{\sunmonth{वृषभः}{7}{}{वैशाखः}{वसन्तऋतुः}{मङ्गलः}{विकारी}{उत्तरायणम्}{वसन्तऋतुः}}
{\sunmoonsrdata{05:46}{18:25}{20:53}{07:38}{12:05}
{\kalas{04:15 05:00 09:08 08:17 09:59 16:43 10:49 13:21 15:53 17:34 19:10 21:15 22:40 01:30*}}}
{\tnykdata{\anga{\tithi{18}{कृष्ण-तृतीया}}{\time{49-11}{01:40*}}\hspace{1ex}}%
{\anga{मूला}{\time{54-0}{03:29*}}\hspace{1ex}}{चन्द्रराशिः—\mbox{धनुः}}%
{\anga{सिद्धः}{\time{10-51}{10:20}}\hspace{1ex}\uanga{साध्यः}}%
{\anga{वणिजा}{\time{18-10}{13:25}}\hspace{1ex}\anga{भद्रा}{\time{49-11}{01:40*}}\hspace{1ex}\uanga{बवम्}}{}
}
{(सायन) षडशीति-पुण्यकालः~13:29\RIGHTarrow{}18:25\eventsep काञ्ची जगद्गुरु श्री-शङ्कर विजयेन्द्र सरस्वती आश्रम-स्वीकार-दिनम्~\#{३७}\eventsep \tamil{முருக நாயன்மார் (16) குருபூஜை}\eventsep पार्थिव-कल्पादिः\eventsep सायन-रवि-सङ्क्रमण-पुण्यकालः~07:05\RIGHTarrow{}18:25\eventsep सायन-सङ्क्रमण-दिन-अपराह्ण-पुण्यकालः~12:05\RIGHTarrow{}18:25\eventsep \tamil{திருஞானஸம்பந்தமூர்த்தி நாயன்மார் (28) குருபூஜை}\eventsep \tamil{திருநீலகண்ட யாழ்ப்பாண நாயன்மார் (61) குருபூஜை}\eventsep \tamil{திருநீலநக்க நாயன்மார் (26) குருபூஜை}\eventsep शुक्र-मासः/ग्रीष्मऋतुः~13:29\RIGHTarrow{}}
{Tue} 
\cfoot{\rygdata{15:15--16:50}{08:55--10:30}{12:05--13:40}}
\caldata{MAY}{22}{\sunmonth{वृषभः}{8}{}{वैशाखः}{वसन्तऋतुः}{बुधः}{विकारी}{उत्तरायणम्}{वसन्तऋतुः}}
{\sunmoonsrdata{05:45}{18:25}{21:44}{08:32}{12:05}
{\kalas{04:15 05:00 09:08 08:17 09:59 16:44 10:49 13:21 15:53 17:34 19:10 21:15 22:40 01:30*}}}
{\tnykdata{\anga{\tithi{19}{कृष्ण-चतुर्थी}}{\time{51-51}{02:41*}}\hspace{1ex}}%
{\prev{\anga{मूला}{\time{*54-0}{03:29}}}\hspace{1ex}\anga{पूर्वाषाढा}{\time{58-29}{05:11*}}\hspace{1ex}}{चन्द्रराशिः—\mbox{धनुः}}%
{\anga{साध्यः}{\time{9-32}{09:47}}\hspace{1ex}\uanga{शुभः}}%
{\anga{बवम्}{\time{19-44}{14:05}}\hspace{1ex}\anga{बालवम्}{\time{51-51}{02:41*}}\hspace{1ex}\uanga{कौलवम्}}{}
}
{एकदन्त-महागणपति-सङ्कटहर-चतुर्थी-व्रतम्\eventsep काञ्ची ३० जगद्गुरु श्री-बोधेन्द्र सरस्वती २ आराधना~\#{१३६५}\eventsep \tamil{மயிலை~வெள்ளீஶ்வரர்~ப்ரஹ்மோத்ஸவம்}\eventsep सावित्री-व्रतम्}
{Wed} 
\cfoot{\rygdata{12:05--13:40}{07:20--08:55}{10:30--12:05}}
\caldata{MAY}{23}{\sunmonth{वृषभः}{9}{}{वैशाखः}{वसन्तऋतुः}{गुरुः}{विकारी}{उत्तरायणम्}{वसन्तऋतुः}}
{\sunmoonsrdata{05:45}{18:25}{22:31}{09:25}{12:05}
{\kalas{04:15 05:00 09:08 08:17 09:59 16:44 10:49 13:21 15:53 17:35 19:11 21:15 22:40 01:30*}}}
{\tnykdata{\anga{\tithi{20}{कृष्ण-पञ्चमी}}{\time{56-10}{04:18*}}\hspace{1ex}}%
{\prev{\anga{पूर्वाषाढा}{\time{*58-29}{05:11}}}\hspace{1ex}\fullanga{उत्तराषाढा}}{चन्द्रराशिः—\mbox{धनुः\RIGHTarrow{11:42}}}%
{\anga{शुभः}{\time{9-28}{09:45}}\hspace{1ex}\uanga{शुक्लः}}%
{\anga{कौलवम्}{\time{22-53}{15:25}}\hspace{1ex}\anga{तैतिलम्}{\time{56-10}{04:18*}}\hspace{1ex}\uanga{गरजा}}{}
}
{कश्यप-महर्षि-जयन्ती}
{Thu} 
\cfoot{\rygdata{13:40--15:15}{05:45--07:20}{08:55--10:30}}
\caldata{MAY}{24}{\sunmonth{वृषभः}{10}{}{वैशाखः}{वसन्तऋतुः}{शुक्रः}{विकारी}{उत्तरायणम्}{वसन्तऋतुः}}
{\sunmoonsrdata{05:45}{18:26}{23:15}{10:16}{12:05}
{\kalas{04:15 05:00 09:08 08:17 09:59 16:44 10:49 13:21 15:54 17:35 19:11 21:15 22:40 01:30*}}}
{\tnykdata{\prev{\anga{\tithi{20}{कृष्ण-पञ्चमी}}{\time{*56-10}{04:18}}}\hspace{1ex}\fulltithi{\tithi{21}{कृष्ण-षष्ठी}}}%
{\anga{उत्तराषाढा}{\time{4-4}{07:28}}\hspace{1ex}}{चन्द्रराशिः—\mbox{मकरः}}%
{\anga{शुक्लः}{\time{10-29}{10:11}}\hspace{1ex}\uanga{ब्राह्मः}}%
{\prev{\anga{तैतिलम्}{\time{*56-10}{04:18}}}\hspace{1ex}\anga{गरजा}{\time{27-21}{17:19}}\hspace{1ex}\uanga{वणिजा}}{}
}
{श्रवण-व्रतम्}
{Fri} 
\cfoot{\rygdata{10:30--12:05}{15:15--16:51}{07:20--08:55}}
\caldata{MAY}{25}{\sunmonth{वृषभः}{11}{}{वैशाखः}{वसन्तऋतुः}{शनिः}{विकारी}{उत्तरायणम्}{वसन्तऋतुः}}
{\sunmoonsrdata{05:45}{18:26}{23:55}{11:05}{12:05}
{\kalas{04:14 05:00 09:08 08:17 09:59 16:44 10:49 13:21 15:54 17:35 19:11 21:16 22:40 01:30*}}}
{\tnykdata{\anga{\tithi{21}{कृष्ण-षष्ठी}}{\time{1-35}{06:25}}\hspace{1ex}}%
{\anga{श्रवणः}{\time{10-33}{10:13}}\hspace{1ex}}{चन्द्रराशिः—\mbox{मकरः\RIGHTarrow{23:41}}}%
{\anga{ब्राह्मः}{\time{12-18}{10:57}}\hspace{1ex}\uanga{माहेन्द्रः}}%
{\anga{वणिजा}{\time{1-35}{06:25}}\hspace{1ex}\anga{भद्रा}{\time{33-5}{19:36}}\hspace{1ex}\uanga{बवम्}}{}
}
{द्विपुष्कर-योगः~10:13\RIGHTarrow{}05:45*}
{Sat} 
\cfoot{\rygdata{08:55--10:30}{13:41--15:16}{05:45--07:20}}
\caldata{MAY}{26}{\sunmonth{वृषभः}{12}{}{वैशाखः}{वसन्तऋतुः}{भानुः}{विकारी}{उत्तरायणम्}{वसन्तऋतुः}}
{\sunmoonsrdata{05:45}{18:26}{00:33*}{11:52}{12:06}
{\kalas{04:14 05:00 09:08 08:17 09:59 16:45 10:49 13:22 15:54 17:35 19:11 21:16 22:41 01:30*}}}
{\tnykdata{\anga{\tithi{22}{कृष्ण-सप्तमी}}{\time{7-15}{08:49}}\hspace{1ex}}%
{\anga{श्रविष्ठा}{\time{17-36}{13:12}}\hspace{1ex}}{चन्द्रराशिः—\mbox{कुम्भः}}%
{\anga{माहेन्द्रः}{\time{14-34}{11:55}}\hspace{1ex}\uanga{वैधृतिः}}%
{\anga{बवम्}{\time{7-15}{08:49}}\hspace{1ex}\anga{बालवम्}{\time{39-35}{22:03}}\hspace{1ex}\uanga{कौलवम्}}{}
}
{भानुसप्तमी\eventsep द्विपुष्कर-योगः~05:45\RIGHTarrow{}08:49\eventsep पञ्च-पर्व-पूजा (अष्टमी)\eventsep वैधृति-श्राद्धम्}
{Sun} 
\cfoot{\rygdata{16:51--18:26}{12:06--13:41}{15:16--16:51}}
\caldata{MAY}{27}{\sunmonth{वृषभः}{13}{}{वैशाखः}{वसन्तऋतुः}{सोमः}{विकारी}{उत्तरायणम्}{वसन्तऋतुः}}
{\sunmoonsrdata{05:45}{18:26}{01:10*}{12:39}{12:06}
{\kalas{04:14 05:00 09:08 08:17 09:59 16:45 10:50 13:22 15:54 17:36 19:12 21:16 22:41 01:30*}}}
{\tnykdata{\anga{\tithi{23}{कृष्ण-अष्टमी}}{\time{13-2}{11:16}}\hspace{1ex}}%
{\anga{शतभिषक्}{\time{24-39}{16:11}}\hspace{1ex}}{चन्द्रराशिः—\mbox{कुम्भः}}%
{\anga{वैधृतिः}{\time{16-52}{12:53}}\hspace{1ex}\uanga{विष्कम्भः}}%
{\anga{कौलवम्}{\time{13-2}{11:16}}\hspace{1ex}\anga{तैतिलम्}{\time{45-53}{00:26*}}\hspace{1ex}\uanga{गरजा}}{}
}
{अनध्यायः\eventsep काञ्ची ७ जगद्गुरु श्री-आनन्दज्ञानेन्द्र सरस्वती आराधना~\#{२०७४}}
{Mon} 
\cfoot{\rygdata{07:20--08:55}{10:30--12:06}{13:41--15:16}}
\caldata{MAY}{28}{\sunmonth{वृषभः}{14}{}{वैशाखः}{वसन्तऋतुः}{मङ्गलः}{विकारी}{उत्तरायणम्}{वसन्तऋतुः}}
{\sunmoonsrdata{05:45}{18:27}{01:47*}{13:25}{12:06}
{\kalas{04:14 05:00 09:08 08:17 09:59 16:45 10:50 13:22 15:54 17:36 19:12 21:16 22:41 01:31*}}}
{\tnykdata{\anga{\tithi{24}{कृष्ण-नवमी}}{\time{18-20}{13:31}}\hspace{1ex}}%
{\anga{पूर्वप्रोष्ठपदा}{\time{31-18}{18:56}}\hspace{1ex}}{चन्द्रराशिः—\mbox{कुम्भः\RIGHTarrow{12:17}}}%
{\anga{विष्कम्भः}{\time{18-47}{13:42}}\hspace{1ex}\uanga{प्रीतिः}}%
{\anga{गरजा}{\time{18-20}{13:31}}\hspace{1ex}\anga{वणिजा}{\time{51-22}{02:30*}}\hspace{1ex}\uanga{भद्रा}}{}
}
{}
{Tue} 
\cfoot{\rygdata{15:16--16:52}{08:55--10:31}{12:06--13:41}}
\caldata{MAY}{29}{\sunmonth{वृषभः}{15}{}{वैशाखः}{वसन्तऋतुः}{बुधः}{विकारी}{उत्तरायणम्}{वसन्तऋतुः}}
{\sunmoonsrdata{05:45}{18:27}{02:25*}{14:11}{12:06}
{\kalas{04:14 04:59 09:08 08:17 09:59 16:45 10:50 13:22 15:55 17:36 19:12 21:16 22:41 01:31*}}}
{\tnykdata{\anga{\tithi{25}{कृष्ण-दशमी}}{\time{22-41}{15:21}}\hspace{1ex}}%
{\anga{उत्तरप्रोष्ठपदा}{\time{37-28}{21:16}}\hspace{1ex}}{चन्द्रराशिः—\mbox{मीनः}}%
{\anga{प्रीतिः}{\time{19-58}{14:12}}\hspace{1ex}\uanga{आयुष्मान्}}%
{\anga{भद्रा}{\time{22-41}{15:21}}\hspace{1ex}\anga{बवम्}{\time{55-32}{04:04*}}\hspace{1ex}\uanga{बालवम्}}{}
}
{आयुष्मद्-बव-सौम्य-संयोगः~15:21\RIGHTarrow{}18:27\eventsep अग्निनक्षत्र-समापनम्\RIGHTarrow{}07:28\eventsep श्री-हनूमत्-जयन्ती (आन्ध्र-सम्प्रदायः)}
{Wed} 
\cfoot{\rygdata{12:06--13:41}{07:20--08:55}{10:31--12:06}}
\caldata{MAY}{30}{\sunmonth{वृषभः}{16}{}{वैशाखः}{वसन्तऋतुः}{गुरुः}{विकारी}{उत्तरायणम्}{वसन्तऋतुः}}
{\sunmoonsrdata{05:45}{18:27}{03:04*}{14:59}{12:06}
{\kalas{04:14 04:59 09:08 08:17 09:59 16:46 10:50 13:22 15:55 17:36 19:13 21:17 22:41 01:31*}}}
{\tnykdata{\anga{\tithi{26}{कृष्ण-एकादशी}}{\time{25-41}{16:38}}\hspace{1ex}}%
{\anga{रेवती}{\time{42-9}{23:02}}\hspace{1ex}}{चन्द्रराशिः—\mbox{मीनः\RIGHTarrow{23:02}}}%
{\anga{आयुष्मान्}{\time{20-7}{14:16}}\hspace{1ex}\uanga{सौभाग्यः}}%
{\prev{\anga{बवम्}{\time{*55-32}{04:04}}}\hspace{1ex}\anga{बालवम्}{\time{25-41}{16:38}}\hspace{1ex}\anga{कौलवम्}{\time{58-6}{05:02*}}\hspace{1ex}\uanga{तैतिलम्}}{}
}
{भद्रकाळी-जयन्ती\eventsep सर्व-अपरा-एकादशी}
{Thu} 
\cfoot{\rygdata{13:41--15:17}{05:45--07:20}{08:55--10:31}}
\caldata{MAY}{31}{\sunmonth{वृषभः}{17}{}{वैशाखः}{वसन्तऋतुः}{शुक्रः}{विकारी}{उत्तरायणम्}{वसन्तऋतुः}}
{\sunmoonsrdata{05:45}{18:28}{03:46*}{15:49}{12:06}
{\kalas{04:14 04:59 09:08 08:17 09:59 16:46 10:50 13:22 15:55 17:37 19:13 21:17 22:41 01:31*}}}
{\tnykdata{\anga{\tithi{27}{कृष्ण-द्वादशी}}{\time{27-12}{17:16}}\hspace{1ex}}%
{\anga{अश्विनी}{\time{45-9}{00:10*}}\hspace{1ex}}{चन्द्रराशिः—\mbox{मेषः}}%
{\anga{सौभाग्यः}{\time{19-8}{13:52}}\hspace{1ex}\uanga{शोभनः}}%
{\prev{\anga{कौलवम्}{\time{*58-6}{05:02}}}\hspace{1ex}\anga{तैतिलम्}{\time{27-12}{17:16}}\hspace{1ex}\anga{गरजा}{\time{58-57}{05:21*}}\hspace{1ex}\uanga{वणिजा}}{}
}
{प्रदोष-व्रतम्~18:28\RIGHTarrow{}19:52}
{Fri} 
\cfoot{\rygdata{10:31--12:06}{15:17--16:52}{07:20--08:55}}
\caldata{JUNE}{1}{\sunmonth{वृषभः}{18}{}{वैशाखः}{वसन्तऋतुः}{शनिः}{विकारी}{उत्तरायणम्}{वसन्तऋतुः}}
{\sunmoonsrdata{05:45}{18:28}{04:32*}{16:42}{12:06}
{\kalas{04:14 04:59 09:08 08:17 09:59 16:46 10:50 13:23 15:55 17:37 19:13 21:17 22:42 01:31*}}}
{\tnykdata{\anga{\tithi{28}{कृष्ण-त्रयोदशी}}{\time{27-11}{17:16}}\hspace{1ex}}%
{\anga{अपभरणी}{\time{46-31}{00:41*}}\hspace{1ex}}{चन्द्रराशिः—\mbox{मेषः}}%
{\anga{शोभनः}{\time{16-57}{12:56}}\hspace{1ex}\uanga{अतिगण्डः}}%
{\prev{\anga{गरजा}{\time{*58-57}{05:21}}}\hspace{1ex}\anga{वणिजा}{\time{27-11}{17:16}}\hspace{1ex}\anga{भद्रा}{\time{58-7}{05:02*}}\hspace{1ex}\uanga{शकुनिः}}{}
}
{\tamil{கழற்சிங்க நாயன்மார் (53) குருபூஜை}\eventsep मासशिवरात्रिः\eventsep पञ्च-पर्व-पूजा (चतुर्दशी)}
{Sat} 
\cfoot{\rygdata{08:55--10:31}{13:42--15:17}{05:45--07:20}}
\caldata{JUNE}{2}{\sunmonth{वृषभः}{19}{}{वैशाखः}{वसन्तऋतुः}{भानुः}{विकारी}{उत्तरायणम्}{वसन्तऋतुः}}
{\sunmoonsrdata{05:45}{18:28}{05:23*}{17:37}{12:06}
{\kalas{04:14 05:00 09:08 08:17 09:59 16:46 10:50 13:23 15:56 17:37 19:13 21:17 22:42 01:31*}}}
{\tnykdata{\anga{\tithi{29}{कृष्ण-चतुर्दशी}}{\time{25-44}{16:40}}\hspace{1ex}}%
{\anga{कृत्तिका}{\time{46-20}{00:37*}}\hspace{1ex}}{चन्द्रराशिः—\mbox{मेषः\RIGHTarrow{06:43}}}%
{\anga{अतिगण्डः}{\time{13-36}{11:31}}\hspace{1ex}\uanga{सुकर्म}}%
{\prev{\anga{भद्रा}{\time{*58-7}{05:02}}}\hspace{1ex}\anga{शकुनिः}{\time{25-44}{16:40}}\hspace{1ex}\anga{चतुष्पात्}{\time{55-45}{04:09*}}\hspace{1ex}\uanga{नाग}}{}
}
{अनध्यायः\eventsep बोधायन-कात्यायन-वैशाख-अमावास्या\eventsep काञ्ची ३ जगद्गुरु श्री-सर्वज्ञात्मेन्द्र सरस्वती आराधना~\#{२३८३}\eventsep कृत्तिका-व्रतम्\eventsep पञ्च-पर्व-पूजा (अमावास्या)}
{Sun} 
\cfoot{\rygdata{16:53--18:28}{12:06--13:42}{15:17--16:53}}
\caldata{JUNE}{3}{\sunmonth{वृषभः}{20}{}{वैशाखः}{वसन्तऋतुः}{सोमः}{विकारी}{उत्तरायणम्}{वसन्तऋतुः}}
{\sunmoonsrdata{05:45}{18:28}{---}{18:36}{12:07}
{\kalas{04:15 05:00 09:08 08:17 09:59 16:47 10:50 13:23 15:56 17:38 19:14 21:18 22:42 01:31*}}}
{\tnykdata{\anga{\tithi{30}{अमावास्या}}{\time{23-2}{15:31}}\hspace{1ex}}%
{\anga{रोहिणी}{\time{44-51}{00:04*}}\hspace{1ex}}{चन्द्रराशिः—\mbox{वृषभः}}%
{\anga{सुकर्म}{\time{9-14}{09:40}}\hspace{1ex}\uanga{धृतिः}}%
{\prev{\anga{चतुष्पात्}{\time{*55-45}{04:09}}}\hspace{1ex}\anga{नाग}{\time{23-2}{15:31}}\hspace{1ex}\anga{किंस्तुघ्नः}{\time{52-7}{02:47*}}\hspace{1ex}\uanga{बवम्}}{}
}
{अनध्यायः\eventsep बोधायन-कात्यायन-इष्टिः\eventsep पार्वणव्रतम् अमावास्यायाम्\eventsep पिण्ड-पितृ-यज्ञः\eventsep सोमवती अमावास्या\eventsep वैशाख-अमावास्या (अलभ्यम्–पुष्कला)\eventsep वैशाख-मास-समापनम्\eventsep वैशाख-स्नानपूर्तिः\eventsep शनि-जयन्ती\eventsep शुक-महर्षि-जयन्ती}
{Mon} 
\cfoot{\rygdata{07:20--08:56}{10:31--12:07}{13:42--15:18}}
\caldata{JUNE}{4}{\sunmonth{वृषभः}{21}{}{ज्यैष्ठः}{ग्रीष्मऋतुः}{मङ्गलः}{विकारी}{उत्तरायणम्}{वसन्तऋतुः}}
{\sunmoonrsdata{05:45}{18:29}{06:18}{19:35}{12:07}
{\kalas{04:15 05:00 09:08 08:18 09:59 16:47 10:50 13:23 15:56 17:38 19:14 21:18 22:42 01:31*}}}
{\tnykdata{\anga{\tithi{1}{शुक्ल-प्रथमा}}{\time{19-19}{13:57}}\hspace{1ex}}%
{\anga{मृगशीर्षम्}{\time{42-19}{23:07}}\hspace{1ex}}{चन्द्रराशिः—\mbox{वृषभः\RIGHTarrow{11:38}}}%
{\anga{धृतिः}{\time{3-59}{07:26}}\hspace{1ex}\anga{शूलः}{\time{57-48}{04:55*}}\hspace{1ex}\uanga{गण्डः}}%
{\anga{बवम्}{\time{19-19}{13:57}}\hspace{1ex}\anga{बालवम्}{\time{47-27}{01:02*}}\hspace{1ex}\uanga{कौलवम्}}{}
}
{अनध्यायः\eventsep भद्र-चतुष्टय-व्रतम्\eventsep चन्द्र-दर्शनम्~18:29\RIGHTarrow{}19:35\eventsep दर्शेष्टिः\eventsep द्विपुष्कर-योगः~13:57\RIGHTarrow{}23:07\eventsep करवीर-व्रतम्\eventsep पार्वण-प्रायश्चित्तावकाशः दर्शे\eventsep पुन्नाग-गौरी-व्रतम्\eventsep दर्श-स्थालीपाकः\eventsep शृङ्गेरी ३२ जगद्गुरु श्री-नृसिंह भारती आराधना}
{Tue} 
\cfoot{\rygdata{15:18--16:53}{08:56--10:31}{12:07--13:42}}
\caldata{JUNE}{5}{\sunmonth{वृषभः}{22}{}{ज्यैष्ठः}{ग्रीष्मऋतुः}{बुधः}{विकारी}{उत्तरायणम्}{वसन्तऋतुः}}
{\sunmoonrsdata{05:45}{18:29}{07:17}{20:34}{12:07}
{\kalas{04:15 05:00 09:09 08:18 10:00 16:47 10:51 13:23 15:56 17:38 19:14 21:18 22:42 01:31*}}}
{\tnykdata{\anga{\tithi{2}{शुक्ल-द्वितीया}}{\time{14-50}{12:03}}\hspace{1ex}}%
{\anga{आर्द्रा}{\time{39-1}{21:52}}\hspace{1ex}}{चन्द्रराशिः—\mbox{मिथुनम्}}%
{\prev{\anga{शूलः}{\time{*57-48}{04:55}}}\hspace{1ex}\anga{गण्डः}{\time{50-31}{02:11*}}\hspace{1ex}\uanga{वृद्धिः}}%
{\anga{कौलवम्}{\time{14-50}{12:03}}\hspace{1ex}\anga{तैतिलम्}{\time{42-2}{23:00}}\hspace{1ex}\uanga{गरजा}}{}
}
{}
{Wed} 
\cfoot{\rygdata{12:07--13:42}{07:20--08:56}{10:31--12:07}}
\caldata{JUNE}{6}{\sunmonth{वृषभः}{23}{}{ज्यैष्ठः}{ग्रीष्मऋतुः}{गुरुः}{विकारी}{उत्तरायणम्}{वसन्तऋतुः}}
{\sunmoonrsdata{05:45}{18:29}{08:18}{21:30}{12:07}
{\kalas{04:15 05:00 09:09 08:18 10:00 16:48 10:51 13:24 15:57 17:38 19:14 21:18 22:43 01:32*}}}
{\tnykdata{\anga{\tithi{3}{शुक्ल-तृतीया}}{\time{9-48}{09:55}}\hspace{1ex}}%
{\anga{पुनर्वसुः}{\time{35-12}{20:27}}\hspace{1ex}}{चन्द्रराशिः—\mbox{मिथुनम्\RIGHTarrow{14:49}}}%
{\anga{वृद्धिः}{\time{42-50}{23:18}}\hspace{1ex}\uanga{ध्रुवः}}%
{\anga{गरजा}{\time{9-48}{09:55}}\hspace{1ex}\anga{वणिजा}{\time{36-6}{20:47}}\hspace{1ex}\uanga{भद्रा}}{}
}
{रम्भा-तृतीया\eventsep शुक्ल-चतुर्थी-व्रतम्}
{Thu} 
\cfoot{\rygdata{13:43--15:18}{05:45--07:20}{08:56--10:32}}
\caldata{JUNE}{7}{\sunmonth{वृषभः}{24}{}{ज्यैष्ठः}{ग्रीष्मऋतुः}{शुक्रः}{विकारी}{उत्तरायणम्}{वसन्तऋतुः}}
{\sunmoonrsdata{05:45}{18:30}{09:19}{22:23}{12:07}
{\kalas{04:15 05:00 09:09 08:18 10:00 16:48 10:51 13:24 15:57 17:39 19:15 21:19 22:43 01:32*}}}
{\tnykdata{\anga{\tithi{4}{शुक्ल-चतुर्थी}}{\time{4-25}{07:38}}\hspace{1ex}\anga{\tithi{5}{शुक्ल-पञ्चमी}}{\time{58-43}{05:16*}}\hspace{1ex}\avamA{}}%
{\anga{पुष्यः}{\time{31-5}{18:54}}\hspace{1ex}}{चन्द्रराशिः—\mbox{कर्कटः}}%
{\anga{ध्रुवः}{\time{34-56}{20:21}}\hspace{1ex}\uanga{व्याघातः}}%
{\anga{भद्रा}{\time{4-25}{07:38}}\hspace{1ex}\anga{बवम्}{\time{29-54}{18:27}}\hspace{1ex}\anga{बालवम्}{\time{58-43}{05:16*}}\hspace{1ex}\uanga{कौलवम्}}{}
}
{दिनक्षयः\eventsep कदली-गौरी-व्रतम्/पूजा\eventsep \tamil{நமிநந்தியடிகள் நாயன்மார் (27) குருபூஜை}\eventsep उमा-अवतारः\eventsep श्रीनिवासमङ्गापुरे साक्षात्कार-वैभवोत्सव-आरम्भः}
{Fri} 
\cfoot{\rygdata{10:32--12:07}{15:19--16:54}{07:20--08:56}}
\caldata{JUNE}{8}{\sunmonth{वृषभः}{25}{}{ज्यैष्ठः}{ग्रीष्मऋतुः}{शनिः}{विकारी}{उत्तरायणम्}{वसन्तऋतुः}}
{\sunmoonrsdata{05:45}{18:30}{10:19}{23:12}{12:07}
{\kalas{04:15 05:00 09:09 08:18 10:00 16:48 10:51 13:24 15:57 17:39 19:15 21:19 22:43 01:32*}}}
{\tnykdata{\prev{\anga{\tithi{5}{शुक्ल-पञ्चमी}}{\time{*58-43}{05:16}}}\hspace{1ex}\anga{\tithi{6}{शुक्ल-षष्ठी}}{\time{52-25}{02:55*}}\hspace{1ex}}%
{\anga{आश्रेषा}{\time{27-15}{17:20}}\hspace{1ex}}{चन्द्रराशिः—\mbox{कर्कटः\RIGHTarrow{17:20}}}%
{\anga{व्याघातः}{\time{27-20}{17:22}}\hspace{1ex}\uanga{हर्षणः}}%
{\prev{\anga{बालवम्}{\time{*58-43}{05:16}}}\hspace{1ex}\anga{कौलवम्}{\time{24-19}{16:05}}\hspace{1ex}\anga{तैतिलम्}{\time{52-25}{02:55*}}\hspace{1ex}\uanga{गरजा}}{}
}
{आरण्य-गौरी-व्रतम्\eventsep षष्ठी-व्रतम्\eventsep काञ्ची ५० जगद्गुरु श्री-चन्द्रचूडेन्द्र सरस्वती २ आराधना~\#{७२३}\eventsep काञ्ची ६ जगद्गुरु श्री-शुद्धानन्देन्द्र सरस्वती आराधना~\#{२१४३}\eventsep \tamil{ஸோமாஸிமார நாயன்மார் (33) குருபூஜை}\eventsep विन्ध्यावासिनी-देवी-पूजा\eventsep श्रीनिवासमङ्गापुरे साक्षात्कार-वैभवोत्सवः}
{Sat} 
\cfoot{\rygdata{08:56--10:32}{13:43--15:19}{05:45--07:21}}
\caldata{JUNE}{9}{\sunmonth{वृषभः}{26}{}{ज्यैष्ठः}{ग्रीष्मऋतुः}{भानुः}{विकारी}{उत्तरायणम्}{वसन्तऋतुः}}
{\sunmoonrsdata{05:45}{18:30}{11:17}{23:58}{12:08}
{\kalas{04:15 05:00 09:09 08:18 10:00 16:48 10:51 13:24 15:57 17:39 19:15 21:19 22:43 01:32*}}}
{\tnykdata{\anga{\tithi{7}{शुक्ल-सप्तमी}}{\time{46-15}{00:36*}}\hspace{1ex}}%
{\anga{मघा}{\time{23-36}{15:47}}\hspace{1ex}}{चन्द्रराशिः—\mbox{सिंहः}}%
{\anga{हर्षणः}{\time{20-21}{14:24}}\hspace{1ex}\uanga{वज्रम्}}%
{\anga{गरजा}{\time{18-48}{13:45}}\hspace{1ex}\anga{वणिजा}{\time{46-15}{00:36*}}\hspace{1ex}\uanga{भद्रा}}{}
}
{धूमावती-जयन्ती\eventsep काञ्ची २३ जगद्गुरु श्री-सच्चित्सुखेन्द्र सरस्वती आराधना~\#{१५०८}\eventsep वरुण-पूजा\eventsep विजया-भानुसप्तमी\eventsep श्रीनिवासमङ्गापुरे साक्षात्कार-वैभवोत्सव-समापनम्}
{Sun} 
\cfoot{\rygdata{16:55--18:30}{12:08--13:43}{15:19--16:55}}
\caldata{JUNE}{10}{\sunmonth{वृषभः}{27}{}{ज्यैष्ठः}{ग्रीष्मऋतुः}{सोमः}{विकारी}{उत्तरायणम्}{वसन्तऋतुः}}
{\sunmoonrsdata{05:45}{18:31}{12:13}{00:43*}{12:08}
{\kalas{04:15 05:00 09:09 08:18 10:00 16:49 10:51 13:24 15:58 17:39 19:16 21:19 22:44 01:32*}}}
{\tnykdata{\anga{\tithi{8}{शुक्ल-अष्टमी}}{\time{40-20}{22:23}}\hspace{1ex}}%
{\anga{पूर्वफल्गुनी}{\time{20-8}{14:19}}\hspace{1ex}}{चन्द्रराशिः—\mbox{सिंहः\RIGHTarrow{19:58}}}%
{\anga{वज्रम्}{\time{13-32}{11:31}}\hspace{1ex}\uanga{सिद्धिः}}%
{\anga{भद्रा}{\time{13-28}{11:29}}\hspace{1ex}\anga{बवम्}{\time{40-20}{22:23}}\hspace{1ex}\uanga{बालवम्}}{}
}
{अनध्यायः\eventsep ज्येष्ठाष्टमी\eventsep काञ्ची १६ जगद्गुरु श्री-उज्ज्वल शङ्करेन्द्र सरस्वती आराधना~\#{१६५३}}
{Mon} 
\cfoot{\rygdata{07:21--08:56}{10:32--12:08}{13:44--15:19}}
\caldata{JUNE}{11}{\sunmonth{वृषभः}{28}{}{ज्यैष्ठः}{ग्रीष्मऋतुः}{मङ्गलः}{विकारी}{उत्तरायणम्}{वसन्तऋतुः}}
{\sunmoonrsdata{05:45}{18:31}{13:08}{01:27*}{12:08}
{\kalas{04:15 05:00 09:09 08:18 10:00 16:49 10:52 13:25 15:58 17:40 19:16 21:20 22:44 01:32*}}}
{\tnykdata{\anga{\tithi{9}{शुक्ल-नवमी}}{\time{34-49}{20:19}}\hspace{1ex}}%
{\anga{उत्तरफल्गुनी}{\time{16-59}{12:59}}\hspace{1ex}}{चन्द्रराशिः—\mbox{कन्या}}%
{\anga{सिद्धिः}{\time{6-57}{08:43}}\hspace{1ex}\uanga{व्यतीपातः}}%
{\anga{बालवम्}{\time{8-25}{09:20}}\hspace{1ex}\anga{कौलवम्}{\time{34-49}{20:19}}\hspace{1ex}\uanga{तैतिलम्}}{}
}
{ब्रह्माणी-पूजा\eventsep देवी-पर्व-३\eventsep काञ्ची ६१ जगद्गुरु श्री-महादेवेन्द्र सरस्वती ४ आराधना~\#{२७४}\eventsep महेश-नवमी\eventsep व्यतीपात-श्राद्धम्\eventsep शुक्ल-देवी-पूजा}
{Tue} 
\cfoot{\rygdata{15:20--16:55}{08:57--10:32}{12:08--13:44}}
\caldata{JUNE}{12}{\sunmonth{वृषभः}{29}{}{ज्यैष्ठः}{ग्रीष्मऋतुः}{बुधः}{विकारी}{उत्तरायणम्}{वसन्तऋतुः}}
{\sunmoonrsdata{05:45}{18:31}{14:03}{02:11*}{12:08}
{\kalas{04:16 05:00 09:10 08:19 10:01 16:49 10:52 13:25 15:58 17:40 19:16 21:20 22:44 01:33*}}}
{\tnykdata{\anga{\tithi{10}{शुक्ल-दशमी}}{\time{29-49}{18:27}}\hspace{1ex}}%
{\anga{हस्तः}{\time{14-14}{11:49}}\hspace{1ex}}{चन्द्रराशिः—\mbox{कन्या\RIGHTarrow{23:19}}}%
{\anga{व्यतीपातः}{\time{0-41}{06:03}}\hspace{1ex}\anga{वरीयान्}{\time{54-10}{03:34*}}\hspace{1ex}\uanga{परिघः}}%
{\anga{तैतिलम्}{\time{3-45}{07:21}}\hspace{1ex}\anga{गरजा}{\time{29-49}{18:27}}\hspace{1ex}\anga{वणिजा}{\time{59-34}{05:36*}}\hspace{1ex}\uanga{भद्रा}}{}
}
{दशहरा/गङ्गावतरणम्/दशपापहरा-दशमी\eventsep काञ्ची १७ जगद्गुरु श्री-सदाशिवेन्द्र सरस्वती आराधना~\#{१६४५}\eventsep काञ्ची ५३ जगद्गुरु श्री-पूर्णानन्द सदाशिवेन्द्र सरस्वती आराधना~\#{५२२}\eventsep रामेश्वर-दर्शनम्}
{Wed} 
\cfoot{\rygdata{12:08--13:44}{07:21--08:57}{10:33--12:08}}
\caldata{JUNE}{13}{\sunmonth{वृषभः}{30}{}{ज्यैष्ठः}{ग्रीष्मऋतुः}{गुरुः}{विकारी}{उत्तरायणम्}{वसन्तऋतुः}}
{\sunmoonrsdata{05:46}{18:31}{14:58}{02:57*}{12:09}
{\kalas{04:16 05:01 09:10 08:19 10:01 16:49 10:52 13:25 15:58 17:40 19:16 21:20 22:44 01:33*}}}
{\tnykdata{\anga{\tithi{11}{शुक्ल-एकादशी}}{\time{25-59}{16:49}}\hspace{1ex}}%
{\anga{चित्रा}{\time{12-2}{10:53}}\hspace{1ex}}{चन्द्रराशिः—\mbox{तुला}}%
{\prev{\anga{वरीयान्}{\time{*54-10}{03:34}}}\hspace{1ex}\anga{परिघः}{\time{48-7}{01:19*}}\hspace{1ex}\uanga{शिवः}}%
{\prev{\anga{वणिजा}{\time{*59-34}{05:36}}}\hspace{1ex}\anga{भद्रा}{\time{25-59}{16:49}}\hspace{1ex}\anga{बवम्}{\time{55-36}{04:07*}}\hspace{1ex}\uanga{बालवम्}}{}
}
{अलर्मेल्मङ्गापुरे प्लवोत्सव-प्रारम्भः\eventsep सर्व-पाण्डव-निर्जला-एकादशी}
{Thu} 
\cfoot{\rygdata{13:44--15:20}{05:46--07:21}{08:57--10:33}}
\caldata{JUNE}{14}{\sunmonth{वृषभः}{31}{}{ज्यैष्ठः}{ग्रीष्मऋतुः}{शुक्रः}{विकारी}{उत्तरायणम्}{वसन्तऋतुः}}
{\sunmoonrsdata{05:46}{18:32}{15:55}{03:45*}{12:09}
{\kalas{04:16 05:01 09:10 08:19 10:01 16:50 10:52 13:25 15:59 17:41 19:17 21:20 22:44 01:33*}}}
{\tnykdata{\anga{\tithi{12}{शुक्ल-द्वादशी}}{\time{22-52}{15:30}}\hspace{1ex}}%
{\anga{स्वाती}{\time{10-31}{10:15}}\hspace{1ex}}{चन्द्रराशिः—\mbox{तुला\RIGHTarrow{03:59*}}}%
{\anga{शिवः}{\time{42-48}{23:19}}\hspace{1ex}\uanga{सिद्धः}}%
{\prev{\anga{बवम्}{\time{*55-36}{04:07}}}\hspace{1ex}\anga{बालवम्}{\time{22-52}{15:30}}\hspace{1ex}\anga{कौलवम्}{\time{52-32}{02:58*}}\hspace{1ex}\uanga{तैतिलम्}}{}
}
{अलर्मेल्मङ्गापुरे प्लवोत्सवः\eventsep चम्पक-द्वादशी\eventsep गवामयन-द्वादशी\eventsep काञ्ची २ जगद्गुरु श्री-सुरेश्वराचार्य आराधना~\#{२४२५}\eventsep रामलक्ष्मण-द्वादशी\eventsep \tamil{வைகாசி~விஶாகம்}\eventsep शुक्रवार-शुक्ल-प्रदोष-व्रतम्~18:32\RIGHTarrow{}19:56}
{Fri} 
\cfoot{\rygdata{10:33--12:09}{15:20--16:56}{07:21--08:57}}
\caldata{JUNE}{15}{\sunmonth{मिथुनम्}{1}{\mbox{वृषभः{\tiny\RIGHTarrow}{17:18}}}{ज्यैष्ठः}{ग्रीष्मऋतुः}{शनिः}{विकारी}{उत्तरायणम्}{ग्रीष्मऋतुः}}
{\sunmoonrsdata{05:46}{18:32}{16:51}{04:36*}{12:09}
{\kalas{04:16 05:01 09:10 08:19 10:01 16:50 10:52 13:25 15:59 17:41 19:17 21:20 22:45 01:33*}}}
{\tnykdata{\anga{\tithi{13}{शुक्ल-त्रयोदशी}}{\time{20-38}{14:33}}\hspace{1ex}}%
{\anga{विशाखा}{\time{9-50}{09:57}}\hspace{1ex}}{चन्द्रराशिः—\mbox{वृश्चिकः}}%
{\anga{सिद्धः}{\time{38-19}{21:39}}\hspace{1ex}\uanga{साध्यः}}%
{\anga{तैतिलम्}{\time{20-38}{14:33}}\hspace{1ex}\anga{गरजा}{\time{50-33}{02:14*}}\hspace{1ex}\uanga{वणिजा}}{}
}
{अलर्मेल्मङ्गापुरे प्लवोत्सवः\eventsep छत्रपति-शिवाजी-राज्याभिषेकः~\#{३४६}\eventsep दुर्गन्ध-दौर्भाग्य-नाशक-त्रयोदशी\eventsep मिथुन-रवि-सङ्क्रमण-षडशीति-पुण्यकालः~17:18\RIGHTarrow{}18:32\eventsep रवि-सङ्क्रमण-पुण्यकालः~10:54\RIGHTarrow{}18:32\eventsep सङ्क्रमण-दिन-अपराह्ण-पुण्यकालः~12:09\RIGHTarrow{}18:32\eventsep वेङ्कटाचले ज्येष्ठ-अभिद्येयकाभिषेकः (वज्र-कवचम्)\eventsep विद्यारण्य-स्वामि-आराधना~\#{६२८}}
{Sat} 
\cfoot{\rygdata{08:57--10:33}{13:45--15:20}{05:46--07:22}}
\caldata{JUNE}{16}{\sunmonth{मिथुनम्}{2}{}{ज्यैष्ठः}{ग्रीष्मऋतुः}{भानुः}{विकारी}{उत्तरायणम्}{ग्रीष्मऋतुः}}
{\sunmoonrsdata{05:46}{18:32}{17:48}{05:28*}{12:09}
{\kalas{04:16 05:01 09:10 08:19 10:01 16:50 10:53 13:26 15:59 17:41 19:17 21:21 22:45 01:33*}}}
{\tnykdata{\anga{\tithi{14}{शुक्ल-चतुर्दशी}}{\time{19-24}{14:02}}\hspace{1ex}}%
{\anga{अनूराधा}{\time{10-7}{10:05}}\hspace{1ex}}{चन्द्रराशिः—\mbox{वृश्चिकः}}%
{\anga{साध्यः}{\time{34-48}{20:20}}\hspace{1ex}\uanga{शुभः}}%
{\anga{वणिजा}{\time{19-24}{14:02}}\hspace{1ex}\anga{भद्रा}{\time{49-47}{01:57*}}\hspace{1ex}\uanga{बवम्}}{}
}
{अलर्मेल्मङ्गापुरे प्लवोत्सवः\eventsep अनध्यायः\eventsep अनध्यायः\eventsep मन्वादिः-(भौत्यः-[१४])\eventsep पञ्च-पर्व-पूजा (पूर्णिमा)\eventsep वेङ्कटाचले ज्येष्ठ-अभिद्येयकाभिषेकः (मुत्यल-कवचम्)\eventsep वेङ्कटाचले पूर्णिमा-गरुड-सेवा}
{Sun} 
\cfoot{\rygdata{16:56--18:32}{12:09--13:45}{15:21--16:56}}
\caldata{JUNE}{17}{\sunmonth{मिथुनम्}{3}{}{ज्यैष्ठः}{ग्रीष्मऋतुः}{सोमः}{विकारी}{उत्तरायणम्}{ग्रीष्मऋतुः}}
{\sunmoonrsdata{05:46}{18:32}{18:43}{---}{12:09}
{\kalas{04:16 05:01 09:11 08:19 10:02 16:50 10:53 13:26 15:59 17:41 19:17 21:21 22:45 01:34*}}}
{\tnykdata{\anga{\tithi{15}{पौर्णमासी}}{\time{19-20}{14:00}}\hspace{1ex}}%
{\anga{ज्येष्ठा}{\time{11-32}{10:41}}\hspace{1ex}}{चन्द्रराशिः—\mbox{वृश्चिकः\RIGHTarrow{10:41}}}%
{\anga{शुभः}{\time{32-21}{19:26}}\hspace{1ex}\uanga{शुक्लः}}%
{\anga{बवम्}{\time{19-20}{14:00}}\hspace{1ex}\anga{बालवम्}{\time{50-25}{02:11*}}\hspace{1ex}\uanga{कौलवम्}}{}
}
{अलर्मेल्मङ्गापुरे प्लवोत्सव-समापनम्\eventsep अनध्यायः\eventsep ऎरुवक-पूर्णिमा\eventsep कबीरदास-जयन्ती\eventsep पार्वणव्रतम् पूर्णिमायाम्\eventsep पूर्णिमा-व्रतम्\eventsep वेङ्कटाचले ज्येष्ठ-अभिद्येयकाभिषेकः (स्वर्ण-कवचम्)\eventsep वट-पूर्णिमा/वट-सावित्री-व्रतम्}
{Mon} 
\cfoot{\rygdata{07:22--08:58}{10:34--12:09}{13:45--15:21}}
\caldata{JUNE}{18}{\sunmonth{मिथुनम्}{4}{}{ज्यैष्ठः}{ग्रीष्मऋतुः}{मङ्गलः}{विकारी}{उत्तरायणम्}{ग्रीष्मऋतुः}}
{\sunmoonsrdata{05:46}{18:33}{19:35}{06:21}{12:10}
{\kalas{04:17 05:01 09:11 08:20 10:02 16:51 10:53 13:26 15:59 17:42 19:18 21:21 22:45 01:34*}}}
{\tnykdata{\anga{\tithi{16}{कृष्ण-प्रथमा}}{\time{20-31}{14:31}}\hspace{1ex}}%
{\anga{मूला}{\time{14-10}{11:48}}\hspace{1ex}}{चन्द्रराशिः—\mbox{धनुः}}%
{\anga{शुक्लः}{\time{31-3}{18:57}}\hspace{1ex}\uanga{ब्राह्मः}}%
{\anga{कौलवम्}{\time{20-31}{14:31}}\hspace{1ex}\anga{तैतिलम्}{\time{52-30}{02:58*}}\hspace{1ex}\uanga{गरजा}}{}
}
{अनध्यायः\eventsep पार्वण-प्रायश्चित्तावकाशः पौर्णमास्याम्\eventsep पूर्णमासेष्टिः\eventsep पूर्ण-स्थालीपाकः}
{Tue} 
\cfoot{\rygdata{15:21--16:57}{08:58--10:34}{12:10--13:45}}
\caldata{JUNE}{19}{\sunmonth{मिथुनम्}{5}{}{ज्यैष्ठः}{ग्रीष्मऋतुः}{बुधः}{विकारी}{उत्तरायणम्}{ग्रीष्मऋतुः}}
{\sunmoonsrdata{05:47}{18:33}{20:24}{07:15}{12:10}
{\kalas{04:17 05:02 09:11 08:20 10:02 16:51 10:53 13:26 16:00 17:42 19:18 21:21 22:46 01:34*}}}
{\tnykdata{\anga{\tithi{17}{कृष्ण-द्वितीया}}{\time{22-59}{15:34}}\hspace{1ex}}%
{\anga{पूर्वाषाढा}{\time{18-2}{13:27}}\hspace{1ex}}{चन्द्रराशिः—\mbox{धनुः\RIGHTarrow{19:57}}}%
{\anga{ब्राह्मः}{\time{30-53}{18:53}}\hspace{1ex}\uanga{माहेन्द्रः}}%
{\anga{गरजा}{\time{22-59}{15:34}}\hspace{1ex}\anga{वणिजा}{\time{56-0}{04:17*}}\hspace{1ex}\uanga{भद्रा}}{}
}
{}
{Wed} 
\cfoot{\rygdata{12:10--13:45}{07:22--08:58}{10:34--12:10}}
\caldata{JUNE}{20}{\sunmonth{मिथुनम्}{6}{}{ज्यैष्ठः}{ग्रीष्मऋतुः}{गुरुः}{विकारी}{उत्तरायणम्}{ग्रीष्मऋतुः}}
{\sunmoonsrdata{05:47}{18:33}{21:09}{08:07}{12:10}
{\kalas{04:17 05:02 09:11 08:20 10:02 16:51 10:53 13:27 16:00 17:42 19:18 21:22 22:46 01:34*}}}
{\tnykdata{\anga{\tithi{18}{कृष्ण-तृतीया}}{\time{26-39}{17:08}}\hspace{1ex}}%
{\anga{उत्तराषाढा}{\time{23-6}{15:37}}\hspace{1ex}}{चन्द्रराशिः—\mbox{मकरः}}%
{\anga{माहेन्द्रः}{\time{31-46}{19:13}}\hspace{1ex}\uanga{वैधृतिः}}%
{\prev{\anga{वणिजा}{\time{*56-0}{04:17}}}\hspace{1ex}\anga{भद्रा}{\time{26-39}{17:08}}\hspace{1ex}\uanga{बवम्}}{}
}
{कृष्णपिङ्गल-महागणपति-सङ्कटहर-चतुर्थी-व्रतम्}
{Thu} 
\cfoot{\rygdata{13:46--15:22}{05:47--07:23}{08:58--10:34}}
\caldata{JUNE}{21}{\sunmonth{मिथुनम्}{7}{}{ज्यैष्ठः}{ग्रीष्मऋतुः}{शुक्रः}{विकारी}{उत्तरायणम्}{ग्रीष्मऋतुः}}
{\sunmoonsrdata{05:47}{18:33}{21:51}{08:57}{12:10}
{\kalas{04:17 05:02 09:11 08:20 10:02 16:51 10:54 13:27 16:00 17:42 19:18 21:22 22:46 01:35*}}}
{\tnykdata{\anga{\tithi{19}{कृष्ण-चतुर्थी}}{\time{31-33}{19:08}}\hspace{1ex}}%
{\anga{श्रवणः}{\time{29-10}{18:12}}\hspace{1ex}}{चन्द्रराशिः—\mbox{मकरः}}%
{\anga{वैधृतिः}{\time{33-33}{19:53}}\hspace{1ex}\uanga{विष्कम्भः}}%
{\anga{बवम्}{\time{0-43}{06:05}}\hspace{1ex}\anga{बालवम्}{\time{31-33}{19:08}}\hspace{1ex}\uanga{कौलवम्}}{}
}
{अनध्यायः\eventsep दक्षिणायन-पुण्यकालः~09:24\RIGHTarrow{}18:33\eventsep सायन-रवि-सङ्क्रमण-पुण्यकालः~15:00\RIGHTarrow{}18:33\eventsep सायन-सङ्क्रमण-दिन-अपराह्ण-पुण्यकालः~12:10\RIGHTarrow{}18:33\eventsep वैधृति-श्राद्धम्\eventsep श्रवण-व्रतम्\eventsep शुचि-मासः/दक्षिणायनम्~21:24\RIGHTarrow{}}
{Fri} 
\cfoot{\rygdata{10:34--12:10}{15:22--16:58}{07:23--08:59}}
\caldata{JUNE}{22}{\sunmonth{मिथुनम्}{8}{}{ज्यैष्ठः}{ग्रीष्मऋतुः}{शनिः}{विकारी}{उत्तरायणम्}{ग्रीष्मऋतुः}}
{\sunmoonsrdata{05:47}{18:34}{22:30}{09:46}{12:10}
{\kalas{04:17 05:02 09:12 08:20 10:03 16:51 10:54 13:27 16:00 17:43 19:19 21:22 22:46 01:35*}}}
{\tnykdata{\anga{\tithi{20}{कृष्ण-पञ्चमी}}{\time{37-43}{21:27}}\hspace{1ex}}%
{\anga{श्रविष्ठा}{\time{36-45}{21:05}}\hspace{1ex}}{चन्द्रराशिः—\mbox{मकरः\RIGHTarrow{07:37}}}%
{\anga{विष्कम्भः}{\time{35-55}{20:47}}\hspace{1ex}\uanga{प्रीतिः}}%
{\anga{कौलवम्}{\time{5-49}{08:16}}\hspace{1ex}\anga{तैतिलम्}{\time{37-43}{21:27}}\hspace{1ex}\uanga{गरजा}}{}
}
{दक्षिणायनारम्भः}
{Sat} 
\cfoot{\rygdata{08:59--10:35}{13:46--15:22}{05:47--07:23}}
\caldata{JUNE}{23}{\sunmonth{मिथुनम्}{9}{}{ज्यैष्ठः}{ग्रीष्मऋतुः}{भानुः}{विकारी}{उत्तरायणम्}{ग्रीष्मऋतुः}}
{\sunmoonsrdata{05:47}{18:34}{23:08}{10:32}{12:11}
{\kalas{04:18 05:02 09:12 08:21 10:03 16:52 10:54 13:27 16:01 17:43 19:19 21:22 22:46 01:35*}}}
{\tnykdata{\anga{\tithi{21}{कृष्ण-षष्ठी}}{\time{44-11}{23:53}}\hspace{1ex}}%
{\anga{शतभिषक्}{\time{44-45}{00:05*}}\hspace{1ex}}{चन्द्रराशिः—\mbox{कुम्भः}}%
{\anga{प्रीतिः}{\time{38-33}{21:46}}\hspace{1ex}\uanga{आयुष्मान्}}%
{\anga{गरजा}{\time{11-26}{10:40}}\hspace{1ex}\anga{वणिजा}{\time{44-11}{23:53}}\hspace{1ex}\uanga{भद्रा}}{}
}
{त्रिपुष्कर-योगः~00:05*\RIGHTarrow{}05:48*}
{Sun} 
\cfoot{\rygdata{16:58--18:34}{12:11--13:46}{15:22--16:58}}
\caldata{JUNE}{24}{\sunmonth{मिथुनम्}{10}{}{ज्यैष्ठः}{ग्रीष्मऋतुः}{सोमः}{विकारी}{उत्तरायणम्}{ग्रीष्मऋतुः}}
{\sunmoonsrdata{05:48}{18:34}{23:44}{11:18}{12:11}
{\kalas{04:18 05:03 09:12 08:21 10:03 16:52 10:54 13:27 16:01 17:43 19:19 21:22 22:47 01:35*}}}
{\tnykdata{\anga{\tithi{22}{कृष्ण-सप्तमी}}{\time{50-24}{02:12*}}\hspace{1ex}}%
{\anga{पूर्वप्रोष्ठपदा}{\time{52-30}{03:00*}}\hspace{1ex}}{चन्द्रराशिः—\mbox{कुम्भः\RIGHTarrow{20:17}}}%
{\anga{आयुष्मान्}{\time{41-2}{22:42}}\hspace{1ex}\uanga{सौभाग्यः}}%
{\anga{भद्रा}{\time{17-4}{13:04}}\hspace{1ex}\anga{बवम्}{\time{50-24}{02:12*}}\hspace{1ex}\uanga{बालवम्}}{}
}
{}
{Mon} 
\cfoot{\rygdata{07:23--08:59}{10:35--12:11}{13:47--15:22}}
\caldata{JUNE}{25}{\sunmonth{मिथुनम्}{11}{}{ज्यैष्ठः}{ग्रीष्मऋतुः}{मङ्गलः}{विकारी}{उत्तरायणम्}{ग्रीष्मऋतुः}}
{\sunmoonsrdata{05:48}{18:34}{00:21*}{12:04}{12:11}
{\kalas{04:18 05:03 09:12 08:21 10:03 16:52 10:54 13:28 16:01 17:43 19:19 21:23 22:47 01:35*}}}
{\tnykdata{\anga{\tithi{23}{कृष्ण-अष्टमी}}{\time{55-46}{04:13*}}\hspace{1ex}}%
{\anga{उत्तरप्रोष्ठपदा}{\time{59-25}{05:35*}}\hspace{1ex}}{चन्द्रराशिः—\mbox{मीनः}}%
{\anga{सौभाग्यः}{\time{42-55}{23:24}}\hspace{1ex}\uanga{शोभनः}}%
{\anga{बालवम्}{\time{22-14}{15:16}}\hspace{1ex}\anga{कौलवम्}{\time{55-46}{04:13*}}\hspace{1ex}\uanga{तैतिलम्}}{}
}
{अनध्यायः\eventsep पञ्च-पर्व-पूजा (अष्टमी)\eventsep तिन्दुकाष्टमी\eventsep त्रिलोचनाष्टमी\eventsep विनायकाष्टमी\eventsep शीतलाष्टमी}
{Tue} 
\cfoot{\rygdata{15:23--16:58}{09:00--10:35}{12:11--13:47}}
\caldata{JUNE}{26}{\sunmonth{मिथुनम्}{12}{}{ज्यैष्ठः}{ग्रीष्मऋतुः}{बुधः}{विकारी}{उत्तरायणम्}{ग्रीष्मऋतुः}}
{\sunmoonsrdata{05:48}{18:34}{00:59*}{12:50}{12:11}
{\kalas{04:18 05:03 09:12 08:21 10:04 16:52 10:55 13:28 16:01 17:43 19:19 21:23 22:47 01:36*}}}
{\tnykdata{\prev{\anga{\tithi{23}{कृष्ण-अष्टमी}}{\time{*55-46}{04:13}}}\hspace{1ex}\anga{\tithi{24}{कृष्ण-नवमी}}{\time{59-48}{05:44*}}\hspace{1ex}}%
{\prev{\anga{उत्तरप्रोष्ठपदा}{\time{*59-25}{05:35}}}\hspace{1ex}\fullanga{रेवती}}{चन्द्रराशिः—\mbox{मीनः}}%
{\anga{शोभनः}{\time{43-49}{23:45}}\hspace{1ex}\uanga{अतिगण्डः}}%
{\prev{\anga{कौलवम्}{\time{*55-46}{04:13}}}\hspace{1ex}\anga{तैतिलम्}{\time{26-25}{17:03}}\hspace{1ex}\anga{गरजा}{\time{59-48}{05:44*}}\hspace{1ex}\uanga{वणिजा}}{}
}
{\tamil{ஏயர்கோன் கலிக்காம நாயன்மார் (29) குருபூஜை}\eventsep दुर्गा-स्वापनम्}
{Wed} 
\cfoot{\rygdata{12:11--13:47}{07:24--09:00}{10:35--12:11}}
\caldata{JUNE}{27}{\sunmonth{मिथुनम्}{13}{}{ज्यैष्ठः}{ग्रीष्मऋतुः}{गुरुः}{विकारी}{उत्तरायणम्}{ग्रीष्मऋतुः}}
{\sunmoonsrdata{05:48}{18:34}{01:39*}{13:39}{12:11}
{\kalas{04:19 05:03 09:13 08:22 10:04 16:52 10:55 13:28 16:01 17:43 19:19 21:23 22:47 01:36*}}}
{\tnykdata{\prev{\anga{\tithi{24}{कृष्ण-नवमी}}{\time{*59-48}{05:44}}}\hspace{1ex}\fulltithi{\tithi{25}{कृष्ण-दशमी}}}%
{\anga{रेवती}{\time{4-25}{07:41}}\hspace{1ex}}{चन्द्रराशिः—\mbox{मीनः\RIGHTarrow{07:41}}}%
{\anga{अतिगण्डः}{\time{43-28}{23:37}}\hspace{1ex}\uanga{सुकर्म}}%
{\prev{\anga{गरजा}{\time{*59-48}{05:44}}}\hspace{1ex}\anga{वणिजा}{\time{29-14}{18:15}}\hspace{1ex}\uanga{भद्रा}}{}
}
{}
{Thu} 
\cfoot{\rygdata{13:47--15:23}{05:48--07:24}{09:00--10:36}}
\caldata{JUNE}{28}{\sunmonth{मिथुनम्}{14}{}{ज्यैष्ठः}{ग्रीष्मऋतुः}{शुक्रः}{विकारी}{उत्तरायणम्}{ग्रीष्मऋतुः}}
{\sunmoonsrdata{05:49}{18:35}{02:22*}{14:29}{12:12}
{\kalas{04:19 05:04 09:13 08:22 10:04 16:53 10:55 13:28 16:01 17:44 19:20 21:23 22:48 01:36*}}}
{\tnykdata{\anga{\tithi{25}{कृष्ण-दशमी}}{\time{1-51}{06:36}}\hspace{1ex}}%
{\anga{अश्विनी}{\time{7-51}{09:09}}\hspace{1ex}}{चन्द्रराशिः—\mbox{मेषः}}%
{\anga{सुकर्म}{\time{41-38}{22:56}}\hspace{1ex}\uanga{धृतिः}}%
{\anga{भद्रा}{\time{1-51}{06:36}}\hspace{1ex}\anga{बवम्}{\time{30-30}{18:46}}\hspace{1ex}\uanga{बालवम्}}{}
}
{चिदम्बरे ध्वजारोहणम्/पञ्चमूर्ति-रथोत्सवः}
{Fri} 
\cfoot{\rygdata{10:36--12:12}{15:23--16:59}{07:24--09:00}}
\caldata{JUNE}{29}{\sunmonth{मिथुनम्}{15}{}{ज्यैष्ठः}{ग्रीष्मऋतुः}{शनिः}{विकारी}{उत्तरायणम्}{ग्रीष्मऋतुः}}
{\sunmoonsrdata{05:49}{18:35}{03:11*}{15:23}{12:12}
{\kalas{04:19 05:04 09:13 08:22 10:04 16:53 10:55 13:28 16:02 17:44 19:20 21:23 22:48 01:36*}}}
{\tnykdata{\anga{\tithi{26}{कृष्ण-एकादशी}}{\time{2-12}{06:45}}\hspace{1ex}}%
{\anga{अपभरणी}{\time{9-39}{09:56}}\hspace{1ex}}{चन्द्रराशिः—\mbox{मेषः\RIGHTarrow{16:00}}}%
{\anga{धृतिः}{\time{38-16}{21:41}}\hspace{1ex}\uanga{शूलः}}%
{\anga{बालवम्}{\time{2-12}{06:45}}\hspace{1ex}\anga{कौलवम्}{\time{29-56}{18:33}}\hspace{1ex}\uanga{तैतिलम्}}{}
}
{चिदम्बरे रजत-चन्द्रप्रभ-वाहनम्\eventsep कृत्तिका-व्रतम्\eventsep कूर्म-जयन्ती\eventsep सर्व-योगिनी-एकादशी\eventsep त्रिपुष्कर-योगः~09:55\RIGHTarrow{}05:49*}
{Sat} 
\cfoot{\rygdata{09:00--10:36}{13:48--15:23}{05:49--07:25}}
\caldata{JUNE}{30}{\sunmonth{मिथुनम्}{16}{}{ज्यैष्ठः}{ग्रीष्मऋतुः}{भानुः}{विकारी}{उत्तरायणम्}{ग्रीष्मऋतुः}}
{\sunmoonsrdata{05:49}{18:35}{04:04*}{16:20}{12:12}
{\kalas{04:19 05:04 09:13 08:22 10:04 16:53 10:55 13:29 16:02 17:44 19:20 21:24 22:48 01:36*}}}
{\tnykdata{\anga{\tithi{27}{कृष्ण-द्वादशी}}{\time{0-51}{06:11}}\hspace{1ex}\anga{\tithi{28}{कृष्ण-त्रयोदशी}}{\time{57-37}{04:56*}}\hspace{1ex}\avamA{}}%
{\anga{कृत्तिका}{\time{9-47}{09:59}}\hspace{1ex}}{चन्द्रराशिः—\mbox{वृषभः}}%
{\anga{शूलः}{\time{33-25}{19:52}}\hspace{1ex}\uanga{गण्डः}}%
{\anga{तैतिलम्}{\time{0-51}{06:11}}\hspace{1ex}\anga{गरजा}{\time{27-47}{17:38}}\hspace{1ex}\anga{वणिजा}{\time{57-37}{04:56*}}\hspace{1ex}\uanga{भद्रा}}{}
}
{चिदम्बरे स्वर्ण-सूर्यप्रभ-वाहनम्\eventsep दिनक्षयः\eventsep प्रदोष-व्रतम्~18:35\RIGHTarrow{}19:59\eventsep त्रिपुष्कर-योगः~05:49\RIGHTarrow{}06:11}
{Sun} 
\cfoot{\rygdata{16:59--18:35}{12:12--13:48}{15:24--16:59}}
\caldata{JULY}{1}{\sunmonth{मिथुनम्}{17}{}{ज्यैष्ठः}{ग्रीष्मऋतुः}{सोमः}{विकारी}{उत्तरायणम्}{ग्रीष्मऋतुः}}
{\sunmoonsrdata{05:49}{18:35}{05:02*}{17:20}{12:12}
{\kalas{04:19 05:04 09:14 08:22 10:05 16:53 10:56 13:29 16:02 17:44 19:20 21:24 22:48 01:37*}}}
{\tnykdata{\prev{\anga{\tithi{28}{कृष्ण-त्रयोदशी}}{\time{*57-37}{04:56}}}\hspace{1ex}\anga{\tithi{29}{कृष्ण-चतुर्दशी}}{\time{52-42}{03:06*}}\hspace{1ex}}%
{\anga{रोहिणी}{\time{8-21}{09:23}}\hspace{1ex}}{चन्द्रराशिः—\mbox{वृषभः\RIGHTarrow{20:51}}}%
{\anga{गण्डः}{\time{27-31}{17:32}}\hspace{1ex}\uanga{वृद्धिः}}%
{\prev{\anga{वणिजा}{\time{*57-37}{04:56}}}\hspace{1ex}\anga{भद्रा}{\time{24-6}{16:05}}\hspace{1ex}\anga{शकुनिः}{\time{52-42}{03:06*}}\hspace{1ex}\uanga{चतुष्पात्}}{}
}
{अनध्यायः\eventsep चिदम्बरे रजत-भूत-वाहनम्\eventsep मासशिवरात्रिः\eventsep पञ्च-पर्व-पूजा (चतुर्दशी)\eventsep सोममृगशीर्ष-योगः~09:23\RIGHTarrow{}}
{Mon} 
\cfoot{\rygdata{07:25--09:01}{10:37--12:12}{13:48--15:24}}
\caldata{JULY}{2}{\sunmonth{मिथुनम्}{18}{}{ज्यैष्ठः}{ग्रीष्मऋतुः}{मङ्गलः}{विकारी}{उत्तरायणम्}{ग्रीष्मऋतुः}}
{\sunmoonsrdata{05:50}{18:35}{---}{18:20}{12:12}
{\kalas{04:20 05:05 09:14 08:23 10:05 16:53 10:56 13:29 16:02 17:44 19:20 21:24 22:48 01:37*}}}
{\tnykdata{\anga{\tithi{30}{अमावास्या}}{\time{46-28}{00:46*}}\hspace{1ex}}%
{\anga{मृगशीर्षम्}{\time{5-35}{08:12}}\hspace{1ex}}{चन्द्रराशिः—\mbox{मिथुनम्}}%
{\anga{वृद्धिः}{\time{20-59}{14:45}}\hspace{1ex}\uanga{ध्रुवः}}%
{\anga{चतुष्पात्}{\time{19-10}{13:59}}\hspace{1ex}\anga{नाग}{\time{46-28}{00:46*}}\hspace{1ex}\uanga{किंस्तुघ्नः}}{}
}
{अनध्यायः\eventsep भोगशायि-पूजा\eventsep चिदम्बरे रजत-ऋषभ-वाहनम्\eventsep पार्वणव्रतम् अमावास्यायाम्\eventsep पञ्च-पर्व-पूजा (अमावास्या)\eventsep पिण्ड-पितृ-यज्ञः\eventsep सर्व-ज्यैष्ठ-अमावास्या (अलभ्यम्–आर्द्रा, पुष्कला)}
{Tue} 
\cfoot{\rygdata{15:24--16:59}{09:01--10:37}{12:12--13:48}}
\caldata{JULY}{3}{\sunmonth{मिथुनम्}{19}{}{आषाढः}{ग्रीष्मऋतुः}{बुधः}{विकारी}{उत्तरायणम्}{ग्रीष्मऋतुः}}
{\sunmoonrsdata{05:50}{18:35}{06:03}{19:19}{12:13}
{\kalas{04:20 05:05 09:14 08:23 10:05 16:53 10:56 13:29 16:02 17:44 19:20 21:24 22:48 01:37*}}}
{\tnykdata{\anga{\tithi{1}{शुक्ल-प्रथमा}}{\time{39-18}{22:05}}\hspace{1ex}}%
{\anga{आर्द्रा}{\time{1-43}{06:34}}\hspace{1ex}\anga{पुनर्वसुः}{\time{56-44}{04:37*}}\hspace{1ex}\avamA{}}{चन्द्रराशिः—\mbox{मिथुनम्\RIGHTarrow{23:08}}}%
{\anga{ध्रुवः}{\time{13-39}{11:38}}\hspace{1ex}\uanga{व्याघातः}}%
{\anga{किंस्तुघ्नः}{\time{13-13}{11:27}}\hspace{1ex}\anga{बवम्}{\time{39-18}{22:05}}\hspace{1ex}\uanga{बालवम्}}{}
}
{अनध्यायः\eventsep चिदम्बरे रजत-गजवाहनम्\eventsep दर्शेष्टिः\eventsep काञ्ची २५ जगद्गुरु श्री-सच्चिदानन्दघनेन्द्र सरस्वती आराधना~\#{१४७२}\eventsep पार्वण-प्रायश्चित्तावकाशः दर्शे\eventsep दर्श-स्थालीपाकः\eventsep वाराही-नवरात्र-आरम्भः}
{Wed} 
\cfoot{\rygdata{12:13--13:48}{07:26--09:01}{10:37--12:13}}
\caldata{JULY}{4}{\sunmonth{मिथुनम्}{20}{}{आषाढः}{ग्रीष्मऋतुः}{गुरुः}{विकारी}{उत्तरायणम्}{ग्रीष्मऋतुः}}
{\sunmoonrsdata{05:50}{18:35}{07:06}{20:15}{12:13}
{\kalas{04:20 05:05 09:14 08:23 10:05 16:53 10:56 13:29 16:02 17:44 19:20 21:24 22:49 01:37*}}}
{\tnykdata{\anga{\tithi{2}{शुक्ल-द्वितीया}}{\time{31-31}{19:10}}\hspace{1ex}}%
{\prev{\anga{पुनर्वसुः}{\time{*56-44}{04:37}}}\hspace{1ex}\anga{पुष्यः}{\time{51-0}{02:28*}}\hspace{1ex}}{चन्द्रराशिः—\mbox{कर्कटः}}%
{\anga{व्याघातः}{\time{5-44}{08:17}}\hspace{1ex}\anga{हर्षणः}{\time{57-11}{04:47*}}\hspace{1ex}\uanga{वज्रम्}}%
{\anga{बालवम्}{\time{6-35}{08:38}}\hspace{1ex}\anga{कौलवम्}{\time{31-31}{19:10}}\hspace{1ex}\anga{तैतिलम्}{\time{59-30}{05:39*}}\hspace{1ex}\uanga{गरजा}}{}
}
{अमृतलक्ष्मी-व्रतम्\eventsep चन्द्र-दर्शनम्~18:35\RIGHTarrow{}20:15\eventsep चिदम्बरे कैलास-वाहनम्\eventsep गुरुपुष्य-योगः\eventsep जगन्नाथ-रथ-यात्रा}
{Thu} 
\cfoot{\rygdata{13:49--15:24}{05:50--07:26}{09:02--10:37}}
\caldata{JULY}{5}{\sunmonth{मिथुनम्}{21}{}{आषाढः}{ग्रीष्मऋतुः}{शुक्रः}{विकारी}{उत्तरायणम्}{ग्रीष्मऋतुः}}
{\sunmoonrsdata{05:50}{18:36}{08:08}{21:07}{12:13}
{\kalas{04:20 05:06 09:14 08:24 10:05 16:53 10:56 13:30 16:02 17:45 19:20 21:24 22:49 01:38*}}}
{\tnykdata{\anga{\tithi{3}{शुक्ल-तृतीया}}{\time{24-14}{16:09}}\hspace{1ex}}%
{\anga{आश्रेषा}{\time{45-7}{00:16*}}\hspace{1ex}}{चन्द्रराशिः—\mbox{कर्कटः\RIGHTarrow{00:16*}}}%
{\prev{\anga{हर्षणः}{\time{*57-11}{04:47}}}\hspace{1ex}\anga{वज्रम्}{\time{47-45}{01:15*}}\hspace{1ex}\uanga{सिद्धिः}}%
{\prev{\anga{तैतिलम्}{\time{*59-30}{05:39}}}\hspace{1ex}\anga{गरजा}{\time{24-14}{16:09}}\hspace{1ex}\anga{वणिजा}{\time{51-27}{02:39*}}\hspace{1ex}\uanga{भद्रा}}{}
}
{चिदम्बरे भिक्षाटन-स्वर्णरथः\eventsep \tamil{புகழ்த்துணை நாயன்மார் (56) குருபூஜை}}
{Fri} 
\cfoot{\rygdata{10:37--12:13}{15:24--17:00}{07:26--09:02}}
\caldata{JULY}{6}{\sunmonth{मिथुनम्}{22}{}{आषाढः}{ग्रीष्मऋतुः}{शनिः}{विकारी}{उत्तरायणम्}{ग्रीष्मऋतुः}}
{\sunmoonrsdata{05:51}{18:36}{09:09}{21:55}{12:13}
{\kalas{04:21 05:06 09:15 08:24 10:06 16:54 10:57 13:30 16:03 17:45 19:21 21:24 22:49 01:38*}}}
{\tnykdata{\anga{\tithi{4}{शुक्ल-चतुर्थी}}{\time{17-12}{13:09}}\hspace{1ex}}%
{\anga{मघा}{\time{39-26}{22:08}}\hspace{1ex}}{चन्द्रराशिः—\mbox{सिंहः}}%
{\anga{सिद्धिः}{\time{38-31}{21:47}}\hspace{1ex}\uanga{व्यतीपातः}}%
{\anga{भद्रा}{\time{17-12}{13:09}}\hspace{1ex}\anga{बवम्}{\time{43-38}{23:43}}\hspace{1ex}\uanga{बालवम्}}{}
}
{चिदम्बरे रथोत्सवः\eventsep \tamil{மாணிக்கவாசகர் குருபூஜை}\eventsep शुक्ल-चतुर्थी-व्रतम्}
{Sat} 
\cfoot{\rygdata{09:02--10:38}{13:49--15:24}{05:51--07:26}}
\caldata{JULY}{7}{\sunmonth{मिथुनम्}{23}{}{आषाढः}{ग्रीष्मऋतुः}{भानुः}{विकारी}{उत्तरायणम्}{ग्रीष्मऋतुः}}
{\sunmoonrsdata{05:51}{18:36}{10:07}{22:41}{12:13}
{\kalas{04:21 05:06 09:15 08:24 10:06 16:54 10:57 13:30 16:03 17:45 19:21 21:25 22:49 01:38*}}}
{\tnykdata{\anga{\tithi{5}{शुक्ल-पञ्चमी}}{\time{10-29}{10:19}}\hspace{1ex}}%
{\anga{पूर्वफल्गुनी}{\time{34-15}{20:12}}\hspace{1ex}}{चन्द्रराशिः—\mbox{सिंहः\RIGHTarrow{01:45*}}}%
{\anga{व्यतीपातः}{\time{29-42}{18:28}}\hspace{1ex}\uanga{वरीयान्}}%
{\anga{बालवम्}{\time{10-29}{10:19}}\hspace{1ex}\anga{कौलवम्}{\time{36-19}{20:58}}\hspace{1ex}\uanga{तैतिलम्}}{}
}
{\tamil{அமரநீதி நாயன்மார் (7) குருபூஜை}\eventsep चिदम्बरे नटराजस्य राजसभायां महाभिषेकः\eventsep काञ्ची ३५ जगद्गुरु श्री-चित्सुखेन्द्र सरस्वती आराधना~\#{१२८३}\eventsep कुमार-षष्ठी-व्रतम्\eventsep \tamil{நடராஜர் ஆனி திருமஞ்சனம்}\eventsep पातार्क-योगः\eventsep स्कन्द-पञ्चमी\eventsep व्यतीपात-श्राद्धम्\eventsep शमी-गौरी-व्रतम्}
{Sun} 
\cfoot{\rygdata{17:00--18:36}{12:13--13:49}{15:24--17:00}}
\caldata{JULY}{8}{\sunmonth{मिथुनम्}{24}{}{आषाढः}{ग्रीष्मऋतुः}{सोमः}{विकारी}{उत्तरायणम्}{ग्रीष्मऋतुः}}
{\sunmoonrsdata{05:51}{18:36}{11:04}{23:26}{12:13}
{\kalas{04:21 05:06 09:15 08:24 10:06 16:54 10:57 13:30 16:03 17:45 19:21 21:25 22:49 01:38*}}}
{\tnykdata{\anga{\tithi{6}{शुक्ल-षष्ठी}}{\time{4-20}{07:42}}\hspace{1ex}\anga{\tithi{7}{शुक्ल-सप्तमी}}{\time{58-47}{05:25*}}\hspace{1ex}\avamA{}}%
{\anga{उत्तरफल्गुनी}{\time{29-50}{18:32}}\hspace{1ex}}{चन्द्रराशिः—\mbox{कन्या}}%
{\anga{वरीयान्}{\time{22-26}{15:23}}\hspace{1ex}\uanga{परिघः}}%
{\anga{तैतिलम्}{\time{4-20}{07:42}}\hspace{1ex}\anga{गरजा}{\time{29-48}{18:31}}\hspace{1ex}\anga{वणिजा}{\time{58-47}{05:25*}}\hspace{1ex}\uanga{भद्रा}}{}
}
{चिदम्बरे मुत्तुप्पल्लक्कु\eventsep दिनक्षयः\eventsep वैवस्वत-सप्तमी}
{Mon} 
\cfoot{\rygdata{07:27--09:02}{10:38--12:13}{13:49--15:25}}
\caldata{JULY}{9}{\sunmonth{मिथुनम्}{25}{}{आषाढः}{ग्रीष्मऋतुः}{मङ्गलः}{विकारी}{उत्तरायणम्}{ग्रीष्मऋतुः}}
{\sunmoonrsdata{05:52}{18:36}{11:59}{00:10*}{12:14}
{\kalas{04:21 05:07 09:15 08:24 10:06 16:54 10:57 13:30 16:03 17:45 19:21 21:25 22:49 01:38*}}}
{\tnykdata{\prev{\anga{\tithi{7}{शुक्ल-सप्तमी}}{\time{*58-47}{05:25}}}\hspace{1ex}\anga{\tithi{8}{शुक्ल-अष्टमी}}{\time{53-43}{03:30*}}\hspace{1ex}}%
{\anga{हस्तः}{\time{26-45}{17:13}}\hspace{1ex}}{चन्द्रराशिः—\mbox{कन्या\RIGHTarrow{04:43*}}}%
{\anga{परिघः}{\time{15-49}{12:35}}\hspace{1ex}\uanga{शिवः}}%
{\prev{\anga{वणिजा}{\time{*58-47}{05:25}}}\hspace{1ex}\anga{भद्रा}{\time{24-50}{16:24}}\hspace{1ex}\anga{बवम्}{\time{53-43}{03:30*}}\hspace{1ex}\uanga{बालवम्}}{}
}
{अनध्यायः\eventsep महिषघ्नी-पूजा}
{Tue} 
\cfoot{\rygdata{15:25--17:00}{09:03--10:38}{12:14--13:49}}
\caldata{JULY}{10}{\sunmonth{मिथुनम्}{26}{}{आषाढः}{ग्रीष्मऋतुः}{बुधः}{विकारी}{उत्तरायणम्}{ग्रीष्मऋतुः}}
{\sunmoonrsdata{05:52}{18:36}{12:54}{00:56*}{12:14}
{\kalas{04:22 05:07 09:16 08:25 10:06 16:54 10:57 13:30 16:03 17:45 19:21 21:25 22:49 01:38*}}}
{\tnykdata{\anga{\tithi{9}{शुक्ल-नवमी}}{\time{49-48}{02:02*}}\hspace{1ex}}%
{\anga{चित्रा}{\time{24-39}{16:20}}\hspace{1ex}}{चन्द्रराशिः—\mbox{तुला}}%
{\anga{शिवः}{\time{10-0}{10:07}}\hspace{1ex}\uanga{सिद्धः}}%
{\anga{बालवम्}{\time{20-51}{14:43}}\hspace{1ex}\anga{कौलवम्}{\time{49-48}{02:02*}}\hspace{1ex}\uanga{तैतिलम्}}{}
}
{ऐन्द्री-दुर्गा-पूजा\eventsep काञ्ची १२ जगद्गुरु श्री-चन्द्रशेखरेन्द्र सरस्वती आराधना~\#{१७८५}\eventsep सुदर्शन-जयन्ती\eventsep उपेन्द्र-नवमी\eventsep वाराही-नवरात्र-समापनम्}
{Wed} 
\cfoot{\rygdata{12:14--13:49}{07:27--09:03}{10:38--12:14}}
\caldata{JULY}{11}{\sunmonth{मिथुनम्}{27}{}{आषाढः}{ग्रीष्मऋतुः}{गुरुः}{विकारी}{उत्तरायणम्}{ग्रीष्मऋतुः}}
{\sunmoonrsdata{05:52}{18:36}{13:50}{01:42*}{12:14}
{\kalas{04:22 05:07 09:16 08:25 10:07 16:54 10:58 13:30 16:03 17:45 19:21 21:25 22:50 01:39*}}}
{\tnykdata{\anga{\tithi{10}{शुक्ल-दशमी}}{\time{47-7}{01:02*}}\hspace{1ex}}%
{\anga{स्वाती}{\time{23-36}{15:53}}\hspace{1ex}}{चन्द्रराशिः—\mbox{तुला}}%
{\anga{सिद्धः}{\time{5-1}{08:00}}\hspace{1ex}\uanga{साध्यः}}%
{\anga{तैतिलम्}{\time{17-56}{13:29}}\hspace{1ex}\anga{गरजा}{\time{47-7}{01:02*}}\hspace{1ex}\uanga{वणिजा}}{}
}
{आशा-दशमी\eventsep अनध्यायः\eventsep काञ्ची ४८ जगद्गुरु श्री-अद्वैतानन्दबोधेन्द्र सरस्वती आराधना~\#{८२०}\eventsep मन्वादिः-(सूर्य-सावर्णिः-[८])\eventsep \tamil{பெரியாழ்வார் திருநக்ஷத்திரம்}}
{Thu} 
\cfoot{\rygdata{13:49--15:25}{05:52--07:28}{09:03--10:38}}
\caldata{JULY}{12}{\sunmonth{मिथुनम्}{28}{}{आषाढः}{ग्रीष्मऋतुः}{शुक्रः}{विकारी}{उत्तरायणम्}{ग्रीष्मऋतुः}}
{\sunmoonrsdata{05:52}{18:36}{14:45}{02:31*}{12:14}
{\kalas{04:22 05:07 09:16 08:25 10:07 16:54 10:58 13:30 16:03 17:45 19:21 21:25 22:50 01:39*}}}
{\tnykdata{\anga{\tithi{11}{शुक्ल-एकादशी}}{\time{45-44}{00:31*}}\hspace{1ex}}%
{\anga{विशाखा}{\time{23-40}{15:55}}\hspace{1ex}}{चन्द्रराशिः—\mbox{तुला\RIGHTarrow{09:52}}}%
{\anga{साध्यः}{\time{0-56}{06:16}}\hspace{1ex}\anga{शुभः}{\time{57-29}{04:56*}}\hspace{1ex}\uanga{शुक्लः}}%
{\anga{वणिजा}{\time{16-7}{12:43}}\hspace{1ex}\anga{भद्रा}{\time{45-44}{00:31*}}\hspace{1ex}\uanga{बवम्}}{}
}
{अनध्यायः\eventsep गोपद्म-व्रत-आरम्भः\eventsep सर्व-शयन-एकादशी\eventsep विष्णु-शयनोत्सवः}
{Fri} 
\cfoot{\rygdata{10:39--12:14}{15:25--17:00}{07:28--09:03}}
\caldata{JULY}{13}{\sunmonth{मिथुनम्}{29}{}{आषाढः}{ग्रीष्मऋतुः}{शनिः}{विकारी}{उत्तरायणम्}{ग्रीष्मऋतुः}}
{\sunmoonrsdata{05:53}{18:36}{15:41}{03:22*}{12:14}
{\kalas{04:22 05:08 09:16 08:25 10:07 16:54 10:58 13:31 16:03 17:45 19:21 21:25 22:50 01:39*}}}
{\tnykdata{\anga{\tithi{12}{शुक्ल-द्वादशी}}{\time{45-37}{00:28*}}\hspace{1ex}}%
{\anga{अनूराधा}{\time{24-51}{16:25}}\hspace{1ex}}{चन्द्रराशिः—\mbox{वृश्चिकः}}%
{\prev{\anga{शुभः}{\time{*57-29}{04:56}}}\hspace{1ex}\anga{शुक्लः}{\time{54-55}{03:59*}}\hspace{1ex}\uanga{ब्राह्मः}}%
{\anga{बवम्}{\time{15-27}{12:26}}\hspace{1ex}\anga{बालवम्}{\time{45-37}{00:28*}}\hspace{1ex}\uanga{कौलवम्}}{}
}
{चातुर्मास्यव्रत-आरम्भः\eventsep हरिवासरः\RIGHTarrow{}06:27\eventsep काञ्ची ३१ जगद्गुरु श्री-ब्रह्मानन्दघनेन्द्र सरस्वती आराधना~\#{१३५२}\eventsep काञ्ची ६३ जगद्गुरु श्री-महादेवेन्द्र सरस्वती ५ आराधना~\#{२०६}\eventsep वासुदेव-द्वादशी\eventsep शाकव्रत-आरम्भः}
{Sat} 
\cfoot{\rygdata{09:03--10:39}{13:50--15:25}{05:53--07:28}}
\caldata{JULY}{14}{\sunmonth{मिथुनम्}{30}{}{आषाढः}{ग्रीष्मऋतुः}{भानुः}{विकारी}{उत्तरायणम्}{ग्रीष्मऋतुः}}
{\sunmoonrsdata{05:53}{18:36}{16:36}{04:15*}{12:14}
{\kalas{04:23 05:08 09:16 08:26 10:07 16:54 10:58 13:31 16:03 17:45 19:21 21:25 22:50 01:39*}}}
{\tnykdata{\anga{\tithi{13}{शुक्ल-त्रयोदशी}}{\time{46-46}{00:54*}}\hspace{1ex}}%
{\anga{ज्येष्ठा}{\time{27-10}{17:24}}\hspace{1ex}}{चन्द्रराशिः—\mbox{वृश्चिकः\RIGHTarrow{17:24}}}%
{\prev{\anga{शुक्लः}{\time{*54-55}{03:59}}}\hspace{1ex}\anga{ब्राह्मः}{\time{53-21}{03:23*}}\hspace{1ex}\uanga{माहेन्द्रः}}%
{\anga{कौलवम्}{\time{15-55}{12:38}}\hspace{1ex}\anga{तैतिलम्}{\time{46-46}{00:54*}}\hspace{1ex}\uanga{गरजा}}{}
}
{ज्येष्ठाभिषेकः\eventsep रविवार-शुक्ल-प्रदोष-व्रतम्~18:35\RIGHTarrow{}20:00}
{Sun} 
\cfoot{\rygdata{17:00--18:36}{12:14--13:50}{15:25--17:00}}
\caldata{JULY}{15}{\sunmonth{मिथुनम्}{31}{}{आषाढः}{ग्रीष्मऋतुः}{सोमः}{विकारी}{उत्तरायणम्}{ग्रीष्मऋतुः}}
{\sunmoonrsdata{05:53}{18:36}{17:29}{05:08*}{12:14}
{\kalas{04:23 05:08 09:16 08:26 10:07 16:54 10:58 13:31 16:03 17:45 19:21 21:25 22:50 01:39*}}}
{\tnykdata{\anga{\tithi{14}{शुक्ल-चतुर्दशी}}{\time{49-7}{01:48*}}\hspace{1ex}}%
{\anga{मूला}{\time{30-36}{18:49}}\hspace{1ex}}{चन्द्रराशिः—\mbox{धनुः}}%
{\anga{माहेन्द्रः}{\time{52-45}{03:10*}}\hspace{1ex}\uanga{वैधृतिः}}%
{\anga{गरजा}{\time{17-29}{13:18}}\hspace{1ex}\anga{वणिजा}{\time{49-7}{01:48*}}\hspace{1ex}\uanga{भद्रा}}{}
}
{अनध्यायः\eventsep पवित्र-चतुर्दशी}
{Mon} 
\cfoot{\rygdata{07:28--09:04}{10:39--12:14}{13:50--15:25}}
\caldata{JULY}{16}{\sunmonth{मिथुनम्}{32}{\mbox{मिथुनम्{\tiny\RIGHTarrow}{04:09*}}}{आषाढः}{ग्रीष्मऋतुः}{मङ्गलः}{विकारी}{उत्तरायणम्}{ग्रीष्मऋतुः}}
{\sunmoonrsdata{05:54}{18:35}{18:19}{---}{12:14}
{\kalas{04:23 05:08 09:17 08:26 10:07 16:54 10:58 13:31 16:03 17:45 19:21 21:25 22:50 01:39*}}}
{\tnykdata{\anga{\tithi{15}{पौर्णमासी}}{\time{52-39}{03:08*}}\hspace{1ex}}%
{\anga{पूर्वाषाढा}{\time{35-32}{20:41}}\hspace{1ex}}{चन्द्रराशिः—\mbox{धनुः\RIGHTarrow{03:12*}}}%
{\anga{वैधृतिः}{\time{53-3}{03:17*}}\hspace{1ex}\uanga{विष्कम्भः}}%
{\anga{भद्रा}{\time{20-7}{14:25}}\hspace{1ex}\anga{बवम्}{\time{52-39}{03:08*}}\hspace{1ex}\uanga{बालवम्}}{}
}
{आषाढ-पूर्णिमा-स्नानम्\eventsep अनध्यायः\eventsep अनध्यायः\eventsep चन्द्र-ग्रहणम्-(केतुग्रस्त)~01:31*\RIGHTarrow{}04:29*\eventsep गुरु-पूर्णिमा/व्यास-पूजा\eventsep काञ्ची १० जगद्गुरु श्री-सुरेश्वरेन्द्र सरस्वती आराधना~\#{१८९३}\eventsep कोकिल-व्रतम्\eventsep कर्कट-सङ्क्रमण-पुण्यकालः~16:09\RIGHTarrow{}18:35\eventsep मन्वादिः-(ब्रह्म-सावर्णिः-[१०])\eventsep पार्वणव्रतम् पूर्णिमायाम्\eventsep पूर्णिमा-व्रतम्\eventsep पञ्च-पर्व-पूजा (पूर्णिमा)\eventsep सङ्क्रमण-दिन-अपराह्ण-पुण्यकालः~12:14\RIGHTarrow{}18:35\eventsep वेङ्कटाचले पूर्णिमा-गरुड-सेवा\eventsep वैधृति-श्राद्धम्\eventsep यतिचातुर्मास्यव्रत-आरम्भः\eventsep शिव-शयनोत्सवः}
{Tue} 
\cfoot{\rygdata{15:25--17:00}{09:04--10:39}{12:14--13:50}}
\caldata{JULY}{17}{\sunmonth{कर्कटः}{1}{}{आषाढः}{ग्रीष्मऋतुः}{बुधः}{विकारी}{दक्षिणायनम्}{ग्रीष्मऋतुः}}
{\sunmoonsrdata{05:54}{18:35}{19:05}{06:00}{12:15}
{\kalas{04:23 05:09 09:17 08:26 10:08 16:54 10:58 13:31 16:03 17:45 19:21 21:25 22:50 01:40*}}}
{\tnykdata{\anga{\tithi{16}{कृष्ण-प्रथमा}}{\time{57-13}{04:51*}}\hspace{1ex}}%
{\anga{उत्तराषाढा}{\time{41-31}{22:56}}\hspace{1ex}}{चन्द्रराशिः—\mbox{मकरः}}%
{\anga{विष्कम्भः}{\time{54-8}{03:42*}}\hspace{1ex}\uanga{प्रीतिः}}%
{\anga{बालवम्}{\time{23-44}{15:57}}\hspace{1ex}\anga{कौलवम्}{\time{57-13}{04:51*}}\hspace{1ex}\uanga{तैतिलम्}}{}
}
{अनध्यायः\eventsep अनध्यायः\eventsep काञ्ची ५४ जगद्गुरु श्री-व्यासाचल महादेवेन्द्र सरस्वती आराधना~\#{५१३}\eventsep पार्वण-प्रायश्चित्तावकाशः पौर्णमास्याम्\eventsep पूर्णमासेष्टिः\eventsep सर्वनदी-रजस्वला\eventsep पूर्ण-स्थालीपाकः}
{Wed} 
\cfoot{\rygdata{12:15--13:50}{07:29--09:04}{10:39--12:15}}
\caldata{JULY}{18}{\sunmonth{कर्कटः}{2}{}{आषाढः}{ग्रीष्मऋतुः}{गुरुः}{विकारी}{दक्षिणायनम्}{ग्रीष्मऋतुः}}
{\sunmoonsrdata{05:54}{18:35}{19:48}{06:51}{12:15}
{\kalas{04:24 05:09 09:17 08:26 10:08 16:54 10:59 13:31 16:03 17:45 19:20 21:25 22:50 01:40*}}}
{\tnykdata{\prev{\anga{\tithi{16}{कृष्ण-प्रथमा}}{\time{*57-13}{04:51}}}\hspace{1ex}\fulltithi{\tithi{17}{कृष्ण-द्वितीया}}}%
{\anga{श्रवणः}{\time{48-23}{01:32*}}\hspace{1ex}}{चन्द्रराशिः—\mbox{मकरः}}%
{\prev{\anga{विष्कम्भः}{\time{*54-8}{03:42}}}\hspace{1ex}\anga{प्रीतिः}{\time{55-55}{04:22*}}\hspace{1ex}\uanga{आयुष्मान्}}%
{\prev{\anga{कौलवम्}{\time{*57-13}{04:51}}}\hspace{1ex}\anga{तैतिलम्}{\time{28-15}{17:51}}\hspace{1ex}\uanga{गरजा}}{}
}
{अष्टनाग-पूजा\eventsep अनध्यायः\eventsep अशून्यशयन-व्रतम्\eventsep चातुर्मास्य-द्वितीया\eventsep सर्वनदी-रजस्वला\eventsep श्रवण-व्रतम्}
{Thu} 
\cfoot{\rygdata{13:50--15:25}{05:54--07:29}{09:04--10:39}}
\caldata{JULY}{19}{\sunmonth{कर्कटः}{3}{}{आषाढः}{ग्रीष्मऋतुः}{शुक्रः}{विकारी}{दक्षिणायनम्}{ग्रीष्मऋतुः}}
{\sunmoonsrdata{05:54}{18:35}{20:28}{07:40}{12:15}
{\kalas{04:24 05:09 09:17 08:27 10:08 16:54 10:59 13:31 16:03 17:44 19:20 21:25 22:50 01:40*}}}
{\tnykdata{\anga{\tithi{17}{कृष्ण-द्वितीया}}{\time{2-23}{06:55}}\hspace{1ex}}%
{\anga{श्रविष्ठा}{\time{55-56}{04:23*}}\hspace{1ex}}{चन्द्रराशिः—\mbox{मकरः\RIGHTarrow{14:56}}}%
{\prev{\anga{प्रीतिः}{\time{*55-55}{04:22}}}\hspace{1ex}\anga{आयुष्मान्}{\time{58-13}{05:15*}}\hspace{1ex}\uanga{सौभाग्यः}}%
{\anga{गरजा}{\time{2-23}{06:55}}\hspace{1ex}\anga{वणिजा}{\time{33-51}{20:03}}\hspace{1ex}\uanga{भद्रा}}{}
}
{\tamil{ஆடி~வெள்ளிக்கிழமை}\eventsep अनध्यायः\eventsep सर्वनदी-रजस्वला}
{Fri} 
\cfoot{\rygdata{10:40--12:15}{15:25--17:00}{07:30--09:05}}
\caldata{JULY}{20}{\sunmonth{कर्कटः}{4}{}{आषाढः}{ग्रीष्मऋतुः}{शनिः}{विकारी}{दक्षिणायनम्}{ग्रीष्मऋतुः}}
{\sunmoonsrdata{05:55}{18:35}{21:06}{08:27}{12:15}
{\kalas{04:24 05:09 09:17 08:27 10:08 16:54 10:59 13:31 16:03 17:44 19:20 21:25 22:50 01:40*}}}
{\tnykdata{\anga{\tithi{18}{कृष्ण-तृतीया}}{\time{7-50}{09:13}}\hspace{1ex}}%
{\prev{\anga{श्रविष्ठा}{\time{*55-56}{04:23}}}\hspace{1ex}\fullanga{शतभिषक्}}{चन्द्रराशिः—\mbox{कुम्भः}}%
{\prev{\anga{आयुष्मान्}{\time{*58-13}{05:15}}}\hspace{1ex}\fullanga{सौभाग्यः}}%
{\anga{भद्रा}{\time{7-50}{09:13}}\hspace{1ex}\anga{बवम्}{\time{40-11}{22:26}}\hspace{1ex}\uanga{बालवम्}}{}
}
{गजानन-महागणपति-सङ्कटहर-चतुर्थी-व्रतम्}
{Sat} 
\cfoot{\rygdata{09:05--10:40}{13:50--15:25}{05:55--07:30}}
\caldata{JULY}{21}{\sunmonth{कर्कटः}{5}{}{आषाढः}{ग्रीष्मऋतुः}{भानुः}{विकारी}{दक्षिणायनम्}{ग्रीष्मऋतुः}}
{\sunmoonsrdata{05:55}{18:35}{21:43}{09:13}{12:15}
{\kalas{04:24 05:10 09:18 08:27 10:08 16:53 10:59 13:31 16:03 17:44 19:20 21:25 22:50 01:40*}}}
{\tnykdata{\anga{\tithi{19}{कृष्ण-चतुर्थी}}{\time{13-35}{11:39}}\hspace{1ex}}%
{\anga{शतभिषक्}{\time{3-27}{07:22}}\hspace{1ex}}{चन्द्रराशिः—\mbox{कुम्भः\RIGHTarrow{03:38*}}}%
{\anga{सौभाग्यः}{\time{0-44}{06:14}}\hspace{1ex}\uanga{शोभनः}}%
{\anga{बालवम्}{\time{13-35}{11:39}}\hspace{1ex}\anga{कौलवम्}{\time{46-39}{00:52*}}\hspace{1ex}\uanga{तैतिलम्}}{}
}
{}
{Sun} 
\cfoot{\rygdata{17:00--18:35}{12:15--13:50}{15:25--17:00}}
\caldata{JULY}{22}{\sunmonth{कर्कटः}{6}{}{आषाढः}{ग्रीष्मऋतुः}{सोमः}{विकारी}{दक्षिणायनम्}{ग्रीष्मऋतुः}}
{\sunmoonsrdata{05:55}{18:35}{22:19}{09:59}{12:15}
{\kalas{04:24 05:10 09:18 08:27 10:08 16:53 10:59 13:31 16:03 17:44 19:20 21:25 22:50 01:40*}}}
{\tnykdata{\anga{\tithi{20}{कृष्ण-पञ्चमी}}{\time{19-18}{14:04}}\hspace{1ex}}%
{\anga{पूर्वप्रोष्ठपदा}{\time{10-32}{10:22}}\hspace{1ex}}{चन्द्रराशिः—\mbox{मीनः}}%
{\anga{शोभनः}{\time{3-4}{07:13}}\hspace{1ex}\uanga{अतिगण्डः}}%
{\anga{तैतिलम्}{\time{19-18}{14:04}}\hspace{1ex}\anga{गरजा}{\time{52-48}{03:12*}}\hspace{1ex}\uanga{वणिजा}}{}
}
{}
{Mon} 
\cfoot{\rygdata{07:30--09:05}{10:40--12:15}{13:50--15:25}}
\caldata{JULY}{23}{\sunmonth{कर्कटः}{7}{}{आषाढः}{ग्रीष्मऋतुः}{मङ्गलः}{विकारी}{दक्षिणायनम्}{ग्रीष्मऋतुः}}
{\sunmoonsrdata{05:55}{18:34}{22:56}{10:44}{12:15}
{\kalas{04:25 05:10 09:18 08:27 10:08 16:53 10:59 13:31 16:03 17:44 19:20 21:25 22:50 01:40*}}}
{\tnykdata{\anga{\tithi{21}{कृष्ण-षष्ठी}}{\time{24-31}{16:16}}\hspace{1ex}}%
{\anga{उत्तरप्रोष्ठपदा}{\time{17-13}{13:11}}\hspace{1ex}}{चन्द्रराशिः—\mbox{मीनः}}%
{\anga{अतिगण्डः}{\time{5-7}{08:05}}\hspace{1ex}\uanga{सुकर्म}}%
{\anga{वणिजा}{\time{24-31}{16:16}}\hspace{1ex}\anga{भद्रा}{\time{58-10}{05:14*}}\hspace{1ex}\uanga{बवम्}}{}
}
{(सायन) विष्णुपदी-पुण्यकालः~05:55\RIGHTarrow{}14:44\eventsep नभो-मासः/वर्षऋतुः~08:20\RIGHTarrow{}\eventsep सायन-सङ्क्रमण-दिन-पूर्वाह्ण-पुण्यकालः~05:55\RIGHTarrow{}12:15}
{Tue} 
\cfoot{\rygdata{15:25--17:00}{09:05--10:40}{12:15--13:50}}
\caldata{JULY}{24}{\sunmonth{कर्कटः}{8}{}{आषाढः}{ग्रीष्मऋतुः}{बुधः}{विकारी}{दक्षिणायनम्}{ग्रीष्मऋतुः}}
{\sunmoonsrdata{05:56}{18:34}{23:34}{11:31}{12:15}
{\kalas{04:25 05:10 09:18 08:27 10:08 16:53 10:59 13:31 16:03 17:44 19:20 21:25 22:50 01:40*}}}
{\tnykdata{\anga{\tithi{22}{कृष्ण-सप्तमी}}{\time{28-50}{18:05}}\hspace{1ex}}%
{\anga{रेवती}{\time{23-5}{15:40}}\hspace{1ex}}{चन्द्रराशिः—\mbox{मीनः\RIGHTarrow{15:40}}}%
{\anga{सुकर्म}{\time{6-34}{08:42}}\hspace{1ex}\uanga{धृतिः}}%
{\prev{\anga{भद्रा}{\time{*58-10}{05:14}}}\hspace{1ex}\anga{बवम्}{\time{28-50}{18:05}}\hspace{1ex}\uanga{बालवम्}}{}
}
{चामुण्डेश्वरी-जयन्ती\eventsep पञ्च-पर्व-पूजा (अष्टमी)}
{Wed} 
\cfoot{\rygdata{12:15--13:50}{07:30--09:05}{10:40--12:15}}
\caldata{JULY}{25}{\sunmonth{कर्कटः}{9}{}{आषाढः}{ग्रीष्मऋतुः}{गुरुः}{विकारी}{दक्षिणायनम्}{ग्रीष्मऋतुः}}
{\sunmoonsrdata{05:56}{18:34}{00:15*}{12:20}{12:15}
{\kalas{04:25 05:10 09:18 08:28 10:09 16:53 10:59 13:31 16:02 17:44 19:20 21:25 22:50 01:40*}}}
{\tnykdata{\anga{\tithi{23}{कृष्ण-अष्टमी}}{\time{32-3}{19:21}}\hspace{1ex}}%
{\anga{अश्विनी}{\time{27-43}{17:36}}\hspace{1ex}}{चन्द्रराशिः—\mbox{मेषः}}%
{\anga{धृतिः}{\time{7-7}{08:56}}\hspace{1ex}\uanga{शूलः}}%
{\anga{बालवम्}{\time{2-2}{06:48}}\hspace{1ex}\anga{कौलवम्}{\time{32-3}{19:21}}\hspace{1ex}\uanga{तैतिलम्}}{}
}
{अनध्यायः}
{Thu} 
\cfoot{\rygdata{13:50--15:25}{05:56--07:31}{09:05--10:40}}
\caldata{JULY}{26}{\sunmonth{कर्कटः}{10}{}{आषाढः}{ग्रीष्मऋतुः}{शुक्रः}{विकारी}{दक्षिणायनम्}{ग्रीष्मऋतुः}}
{\sunmoonsrdata{05:56}{18:34}{01:00*}{13:11}{12:15}
{\kalas{04:25 05:11 09:18 08:28 10:09 16:53 10:59 13:31 16:02 17:43 19:19 21:24 22:50 01:40*}}}
{\tnykdata{\anga{\tithi{24}{कृष्ण-नवमी}}{\time{33-36}{19:56}}\hspace{1ex}}%
{\anga{अपभरणी}{\time{30-53}{18:54}}\hspace{1ex}}{चन्द्रराशिः—\mbox{मेषः\RIGHTarrow{01:07*}}}%
{\anga{शूलः}{\time{6-30}{08:41}}\hspace{1ex}\uanga{गण्डः}}%
{\anga{तैतिलम्}{\time{4-16}{07:44}}\hspace{1ex}\anga{गरजा}{\time{33-36}{19:56}}\hspace{1ex}\uanga{वणिजा}}{}
}
{\tamil{ஆடி~வெள்ளிக்கிழமை}}
{Fri} 
\cfoot{\rygdata{10:40--12:15}{15:24--16:59}{07:31--09:06}}
\caldata{JULY}{27}{\sunmonth{कर्कटः}{11}{}{आषाढः}{ग्रीष्मऋतुः}{शनिः}{विकारी}{दक्षिणायनम्}{ग्रीष्मऋतुः}}
{\sunmoonsrdata{05:56}{18:34}{01:50*}{14:06}{12:15}
{\kalas{04:25 05:11 09:18 08:28 10:09 16:53 10:59 13:31 16:02 17:43 19:19 21:24 22:50 01:40*}}}
{\tnykdata{\anga{\tithi{25}{कृष्ण-दशमी}}{\time{33-10}{19:46}}\hspace{1ex}}%
{\anga{कृत्तिका}{\time{32-22}{19:28}}\hspace{1ex}}{चन्द्रराशिः—\mbox{वृषभः}}%
{\anga{गण्डः}{\time{4-32}{07:51}}\hspace{1ex}\uanga{वृद्धिः}}%
{\anga{वणिजा}{\time{4-46}{07:57}}\hspace{1ex}\anga{भद्रा}{\time{33-10}{19:46}}\hspace{1ex}\uanga{बवम्}}{}
}
{कर्कट-कार्त्तिक-पूजा\eventsep कृत्तिका-व्रतम्\eventsep \tamil{மூர்த்தி நாயன்மார் (15) குருபூஜை}\eventsep \tamil{புகழ்ச்சோழ நாயன்மார் (40) குருபூஜை}}
{Sat} 
\cfoot{\rygdata{09:06--10:40}{13:50--15:24}{05:56--07:31}}
\caldata{JULY}{28}{\sunmonth{कर्कटः}{12}{}{आषाढः}{ग्रीष्मऋतुः}{भानुः}{विकारी}{दक्षिणायनम्}{ग्रीष्मऋतुः}}
{\sunmoonsrdata{05:56}{18:33}{02:45*}{15:03}{12:15}
{\kalas{04:25 05:11 09:18 08:28 10:09 16:52 10:59 13:31 16:02 17:43 19:19 21:24 22:50 01:40*}}}
{\tnykdata{\anga{\tithi{26}{कृष्ण-एकादशी}}{\time{30-42}{18:49}}\hspace{1ex}}%
{\anga{रोहिणी}{\time{31-51}{19:16}}\hspace{1ex}}{चन्द्रराशिः—\mbox{वृषभः}}%
{\anga{वृद्धिः}{\time{1-5}{06:24}}\hspace{1ex}\anga{ध्रुवः}{\time{55-47}{04:21*}}\hspace{1ex}\uanga{व्याघातः}}%
{\anga{बवम्}{\time{3-26}{07:23}}\hspace{1ex}\anga{बालवम्}{\time{30-42}{18:49}}\hspace{1ex}\uanga{कौलवम्}}{}
}
{देवी-पर्व-४\eventsep द्विपुष्कर-योगः~19:16\RIGHTarrow{}05:57*\eventsep सर्व-कामिका-एकादशी\eventsep \tamil{திருப்பாணாழ்வார் திருநக்ஷத்திரம்}}
{Sun} 
\cfoot{\rygdata{16:59--18:33}{12:15--13:50}{15:24--16:59}}
\caldata{JULY}{29}{\sunmonth{कर्कटः}{13}{}{आषाढः}{ग्रीष्मऋतुः}{सोमः}{विकारी}{दक्षिणायनम्}{ग्रीष्मऋतुः}}
{\sunmoonsrdata{05:57}{18:33}{03:44*}{16:02}{12:15}
{\kalas{04:26 05:11 09:18 08:28 10:09 16:52 10:59 13:31 16:02 17:43 19:19 21:24 22:50 01:40*}}}
{\tnykdata{\anga{\tithi{27}{कृष्ण-द्वादशी}}{\time{26-39}{17:09}}\hspace{1ex}}%
{\anga{मृगशीर्षम्}{\time{29-28}{18:20}}\hspace{1ex}}{चन्द्रराशिः—\mbox{वृषभः\RIGHTarrow{06:53}}}%
{\prev{\anga{ध्रुवः}{\time{*55-47}{04:21}}}\hspace{1ex}\anga{व्याघातः}{\time{48-51}{01:43*}}\hspace{1ex}\uanga{हर्षणः}}%
{\anga{कौलवम्}{\time{0-18}{06:04}}\hspace{1ex}\anga{तैतिलम्}{\time{26-39}{17:09}}\hspace{1ex}\anga{गरजा}{\time{55-1}{04:03*}}\hspace{1ex}\uanga{वणिजा}}{}
}
{सोम-प्रदोष-व्रतम्~18:33\RIGHTarrow{}19:59\eventsep सोममृगशीर्ष-योगः\RIGHTarrow{}18:20}
{Mon} 
\cfoot{\rygdata{07:31--09:06}{10:40--12:15}{13:49--15:24}}
\caldata{JULY}{30}{\sunmonth{कर्कटः}{14}{}{आषाढः}{ग्रीष्मऋतुः}{मङ्गलः}{विकारी}{दक्षिणायनम्}{ग्रीष्मऋतुः}}
{\sunmoonsrdata{05:57}{18:33}{04:46*}{17:01}{12:15}
{\kalas{04:26 05:11 09:19 08:28 10:09 16:52 10:59 13:31 16:02 17:42 19:18 21:24 22:49 01:41*}}}
{\tnykdata{\anga{\tithi{28}{कृष्ण-त्रयोदशी}}{\time{21-7}{14:49}}\hspace{1ex}}%
{\anga{आर्द्रा}{\time{25-43}{16:45}}\hspace{1ex}}{चन्द्रराशिः—\mbox{मिथुनम्}}%
{\anga{हर्षणः}{\time{40-35}{22:35}}\hspace{1ex}\uanga{वज्रम्}}%
{\prev{\anga{गरजा}{\time{*55-1}{04:03}}}\hspace{1ex}\anga{वणिजा}{\time{21-7}{14:49}}\hspace{1ex}\anga{भद्रा}{\time{48-8}{01:27*}}\hspace{1ex}\uanga{शकुनिः}}{}
}
{\tamil{கூற்றுவ நாயன்மார் (39) குருபூஜை}\eventsep मासशिवरात्रिः\eventsep पञ्च-पर्व-पूजा (चतुर्दशी)}
{Tue} 
\cfoot{\rygdata{15:24--16:58}{09:06--10:40}{12:15--13:49}}
\caldata{JULY}{31}{\sunmonth{कर्कटः}{15}{}{आषाढः}{ग्रीष्मऋतुः}{बुधः}{विकारी}{दक्षिणायनम्}{ग्रीष्मऋतुः}}
{\sunmoonsrdata{05:57}{18:32}{05:50*}{18:00}{12:15}
{\kalas{04:26 05:12 09:19 08:28 10:09 16:52 10:59 13:30 16:01 17:42 19:18 21:24 22:49 01:41*}}}
{\tnykdata{\anga{\tithi{29}{कृष्ण-चतुर्दशी}}{\time{14-17}{11:57}}\hspace{1ex}}%
{\anga{पुनर्वसुः}{\time{20-42}{14:39}}\hspace{1ex}}{चन्द्रराशिः—\mbox{मिथुनम्\RIGHTarrow{09:13}}}%
{\anga{वज्रम्}{\time{31-17}{19:02}}\hspace{1ex}\uanga{सिद्धिः}}%
{\anga{शकुनिः}{\time{14-17}{11:57}}\hspace{1ex}\anga{चतुष्पात्}{\time{40-2}{22:22}}\hspace{1ex}\uanga{नाग}}{}
}
{अनध्यायः\eventsep काञ्ची ३८ जगद्गुरु श्री-अभिनवशङ्करेन्द्र सरस्वती आराधना~\#{११८०}\eventsep काञ्ची ४६ जगद्गुरु श्री-सान्द्रानन्दबोधेन्द्र सरस्वती आराधना~\#{९२२}\eventsep पार्वणव्रतम् अमावास्यायाम्\eventsep पञ्च-पर्व-पूजा (अमावास्या)\eventsep सर्व-आषाढ (कर्कट) अमावास्या (अलभ्यम्–पुनर्वसुः)}
{Wed} 
\cfoot{\rygdata{12:15--13:49}{07:32--09:06}{10:40--12:15}}
\caldata{AUGUST}{1}{\sunmonth{कर्कटः}{16}{}{आषाढः}{ग्रीष्मऋतुः}{गुरुः}{विकारी}{दक्षिणायनम्}{ग्रीष्मऋतुः}}
{\sunmoonsrdata{05:57}{18:32}{---}{18:55}{12:15}
{\kalas{04:26 05:12 09:19 08:28 10:09 16:52 10:59 13:30 16:01 17:42 19:18 21:24 22:49 01:41*}}}
{\tnykdata{\anga{\tithi{30}{अमावास्या}}{\time{6-31}{08:42}}\hspace{1ex}\anga{\tithi{1}{शुक्ल-प्रथमा}}{\time{57-57}{05:11*}}\hspace{1ex}\avamA{}}%
{\anga{पुष्यः}{\time{14-47}{12:09}}\hspace{1ex}}{चन्द्रराशिः—\mbox{कर्कटः}}%
{\anga{सिद्धिः}{\time{22-2}{15:12}}\hspace{1ex}\uanga{व्यतीपातः}}%
{\anga{नाग}{\time{6-31}{08:42}}\hspace{1ex}\anga{किंस्तुघ्नः}{\time{31-6}{18:58}}\hspace{1ex}\anga{बवम्}{\time{57-57}{05:11*}}\hspace{1ex}\uanga{बालवम्}}{}
}
{आषाढ-स्नानपूर्तिः\eventsep अनध्यायः\eventsep अनध्यायः\eventsep दीप-पूजा\eventsep दर्शेष्टिः\eventsep दिनक्षयः\eventsep गुरुपुष्य-योगः\RIGHTarrow{}12:09\eventsep काञ्ची ४१ जगद्गुरु श्री-गङ्गाधरेन्द्र सरस्वती २ आराधना~\#{१०७०}\eventsep पार्वण-प्रायश्चित्तावकाशः दर्शे\eventsep पति-सञ्जीवनी-व्रतम्\eventsep पिण्ड-पितृ-यज्ञः\eventsep दर्श-स्थालीपाकः\eventsep व्यतीपात-श्राद्धम्}
{Thu} 
\cfoot{\rygdata{13:49--15:23}{05:57--07:32}{09:06--10:40}}
\caldata{AUGUST}{2}{\sunmonth{कर्कटः}{17}{}{श्रावणः}{वर्षऋतुः}{शुक्रः}{विकारी}{दक्षिणायनम्}{ग्रीष्मऋतुः}}
{\sunmoonrsdata{05:58}{18:32}{06:53}{19:46}{12:15}
{\kalas{04:26 05:12 09:19 08:28 10:09 16:51 10:59 13:30 16:01 17:42 19:18 21:23 22:49 01:41*}}}
{\tnykdata{\prev{\anga{\tithi{1}{शुक्ल-प्रथमा}}{\time{*57-57}{05:11}}}\hspace{1ex}\anga{\tithi{2}{शुक्ल-द्वितीया}}{\time{48-32}{01:36*}}\hspace{1ex}}%
{\anga{आश्रेषा}{\time{8-19}{09:27}}\hspace{1ex}}{चन्द्रराशिः—\mbox{कर्कटः\RIGHTarrow{09:27}}}%
{\anga{व्यतीपातः}{\time{12-31}{11:12}}\hspace{1ex}\uanga{वरीयान्}}%
{\prev{\anga{बवम्}{\time{*57-57}{05:11}}}\hspace{1ex}\anga{बालवम्}{\time{22-30}{15:24}}\hspace{1ex}\anga{कौलवम्}{\time{48-32}{01:36*}}\hspace{1ex}\uanga{तैतिलम्}}{}
}
{\tamil{ஆடி~வெள்ளிக்கிழமை}\eventsep चन्द्र-दर्शनम्~18:32\RIGHTarrow{}19:46\eventsep मनोरथ-द्वितीया\eventsep सत्यनारायण-जयन्ती}
{Fri} 
\cfoot{\rygdata{10:40--12:15}{15:23--16:58}{07:32--09:06}}
\caldata{AUGUST}{3}{\sunmonth{कर्कटः}{18}{}{श्रावणः}{वर्षऋतुः}{शनिः}{विकारी}{दक्षिणायनम्}{ग्रीष्मऋतुः}}
{\sunmoonrsdata{05:58}{18:31}{07:54}{20:35}{12:15}
{\kalas{04:26 05:12 09:19 08:29 10:09 16:51 10:59 13:30 16:01 17:41 19:17 21:23 22:49 01:41*}}}
{\tnykdata{\anga{\tithi{3}{शुक्ल-तृतीया}}{\time{39-20}{22:05}}\hspace{1ex}}%
{\anga{मघा}{\time{1-44}{06:42}}\hspace{1ex}\anga{पूर्वफल्गुनी}{\time{54-59}{04:03*}}\hspace{1ex}\avamA{}}{चन्द्रराशिः—\mbox{सिंहः}}%
{\anga{वरीयान्}{\time{2-54}{07:11}}\hspace{1ex}\anga{परिघः}{\time{52-54}{03:15*}}\hspace{1ex}\uanga{शिवः}}%
{\anga{तैतिलम्}{\time{13-59}{11:49}}\hspace{1ex}\anga{गरजा}{\time{39-20}{22:05}}\hspace{1ex}\uanga{वणिजा}}{}
}
{\tamil{ஆடிப்~பெருக்கு}\eventsep हरियाली-तृतीया\eventsep मधुश्रावणि-व्रतम्\eventsep पार्वती-पवित्रारोपणम्\eventsep स्वर्ण-गौरी-व्रतम्\eventsep \tamil{திருவாடிப்பூரம்}}
{Sat} 
\cfoot{\rygdata{09:06--10:40}{13:49--15:23}{05:58--07:32}}
\caldata{AUGUST}{4}{\sunmonth{कर्कटः}{19}{}{श्रावणः}{वर्षऋतुः}{भानुः}{विकारी}{दक्षिणायनम्}{ग्रीष्मऋतुः}}
{\sunmoonrsdata{05:58}{18:31}{08:54}{21:22}{12:15}
{\kalas{04:26 05:12 09:19 08:29 10:09 16:51 10:59 13:30 16:01 17:41 19:17 21:23 22:49 01:41*}}}
{\tnykdata{\anga{\tithi{4}{शुक्ल-चतुर्थी}}{\time{30-45}{18:49}}\hspace{1ex}}%
{\prev{\anga{पूर्वफल्गुनी}{\time{*54-59}{04:03}}}\hspace{1ex}\anga{उत्तरफल्गुनी}{\time{48-47}{01:42*}}\hspace{1ex}}{चन्द्रराशिः—\mbox{सिंहः\RIGHTarrow{09:26}}}%
{\anga{शिवः}{\time{43-12}{23:34}}\hspace{1ex}\uanga{सिद्धः}}%
{\anga{वणिजा}{\time{5-50}{08:25}}\hspace{1ex}\anga{भद्रा}{\time{30-45}{18:49}}\hspace{1ex}\anga{बवम्}{\time{58-15}{05:18*}}\hspace{1ex}\uanga{बालवम्}}{}
}
{एकविंशति-दिवस-गणपति-व्रत-आरम्भः\eventsep दूर्वा-गणपति-व्रतम्\eventsep शुक्ल-चतुर्थी-व्रतम्}
{Sun} 
\cfoot{\rygdata{16:57--18:31}{12:15--13:49}{15:23--16:57}}
\caldata{AUGUST}{5}{\sunmonth{कर्कटः}{20}{}{श्रावणः}{वर्षऋतुः}{सोमः}{विकारी}{दक्षिणायनम्}{ग्रीष्मऋतुः}}
{\sunmoonrsdata{05:58}{18:31}{09:51}{22:07}{12:14}
{\kalas{04:27 05:12 09:19 08:29 10:09 16:50 10:59 13:30 16:00 17:41 19:17 21:23 22:49 01:40*}}}
{\tnykdata{\anga{\tithi{5}{शुक्ल-पञ्चमी}}{\time{23-45}{15:54}}\hspace{1ex}}%
{\anga{हस्तः}{\time{43-42}{23:45}}\hspace{1ex}}{चन्द्रराशिः—\mbox{कन्या}}%
{\anga{सिद्धः}{\time{34-24}{20:12}}\hspace{1ex}\uanga{साध्यः}}%
{\prev{\anga{बवम्}{\time{*58-15}{05:18}}}\hspace{1ex}\anga{बालवम्}{\time{23-45}{15:54}}\hspace{1ex}\anga{कौलवम्}{\time{51-15}{02:38*}}\hspace{1ex}\uanga{तैतिलम्}}{}
}
{गरुड-पञ्चमी\eventsep नाग-पञ्चमी\eventsep तैत्तिरीय-उपाकर्म हस्ते}
{Mon} 
\cfoot{\rygdata{07:32--09:06}{10:40--12:14}{13:49--15:23}}
\caldata{AUGUST}{6}{\sunmonth{कर्कटः}{21}{}{श्रावणः}{वर्षऋतुः}{मङ्गलः}{विकारी}{दक्षिणायनम्}{ग्रीष्मऋतुः}}
{\sunmoonrsdata{05:58}{18:30}{10:48}{22:53}{12:14}
{\kalas{04:27 05:13 09:19 08:29 10:09 16:50 10:59 13:30 16:00 17:40 19:16 21:22 22:48 01:40*}}}
{\tnykdata{\anga{\tithi{6}{शुक्ल-षष्ठी}}{\time{18-0}{13:30}}\hspace{1ex}}%
{\anga{चित्रा}{\time{40-2}{22:21}}\hspace{1ex}}{चन्द्रराशिः—\mbox{कन्या\RIGHTarrow{10:59}}}%
{\anga{साध्यः}{\time{27-1}{17:16}}\hspace{1ex}\uanga{शुभः}}%
{\anga{तैतिलम्}{\time{18-0}{13:30}}\hspace{1ex}\anga{गरजा}{\time{45-42}{00:31*}}\hspace{1ex}\uanga{वणिजा}}{}
}
{षष्ठी-व्रतम्\eventsep द्विपुष्कर-योगः~13:30\RIGHTarrow{}22:21\eventsep \tamil{பெருமிழலைக் குறும்ப நாயன்மார் (23) குருபூஜை}\eventsep सूपौदन-व्रतम्}
{Tue} 
\cfoot{\rygdata{15:22--16:56}{09:06--10:40}{12:14--13:48}}
\caldata{AUGUST}{7}{\sunmonth{कर्कटः}{22}{}{श्रावणः}{वर्षऋतुः}{बुधः}{विकारी}{दक्षिणायनम्}{ग्रीष्मऋतुः}}
{\sunmoonrsdata{05:58}{18:30}{11:45}{23:40}{12:14}
{\kalas{04:27 05:13 09:19 08:29 10:09 16:50 10:59 13:29 16:00 17:40 19:16 21:22 22:48 01:40*}}}
{\tnykdata{\anga{\tithi{7}{शुक्ल-सप्तमी}}{\time{13-39}{11:41}}\hspace{1ex}}%
{\anga{स्वाती}{\time{37-58}{21:33}}\hspace{1ex}}{चन्द्रराशिः—\mbox{तुला}}%
{\anga{शुभः}{\time{21-11}{14:49}}\hspace{1ex}\uanga{शुक्लः}}%
{\anga{वणिजा}{\time{13-39}{11:41}}\hspace{1ex}\anga{भद्रा}{\time{41-46}{23:00}}\hspace{1ex}\uanga{बवम्}}{}
}
{अव्यङ्ग-सप्तमी\eventsep द्वादश-सप्तमी\eventsep \tamil{கழறிற்றறிவார்/சேரமான் பெருமாள் நாயன்மார் (37) குருபூஜை}\eventsep पापनाशनी-सप्तमी\eventsep \tamil{ஸுந்தரமூர்த்தி நாயன்மார் (1) குருபூஜை/திருவாடி ஸ்வாதி}\eventsep तुलसीदास-जयन्ती\eventsep शीतला-सप्तमी}
{Wed} 
\cfoot{\rygdata{12:14--13:48}{07:32--09:06}{10:40--12:14}}
\caldata{AUGUST}{8}{\sunmonth{कर्कटः}{23}{}{श्रावणः}{वर्षऋतुः}{गुरुः}{विकारी}{दक्षिणायनम्}{ग्रीष्मऋतुः}}
{\sunmoonrsdata{05:59}{18:30}{12:41}{00:29*}{12:14}
{\kalas{04:27 05:13 09:19 08:29 10:09 16:50 10:59 13:29 15:59 17:39 19:16 21:22 22:48 01:40*}}}
{\tnykdata{\anga{\tithi{8}{शुक्ल-अष्टमी}}{\time{10-50}{10:30}}\hspace{1ex}}%
{\anga{विशाखा}{\time{37-38}{21:25}}\hspace{1ex}}{चन्द्रराशिः—\mbox{तुला\RIGHTarrow{15:23}}}%
{\anga{शुक्लः}{\time{16-36}{12:55}}\hspace{1ex}\uanga{ब्राह्मः}}%
{\anga{बवम्}{\time{10-50}{10:30}}\hspace{1ex}\anga{बालवम्}{\time{39-35}{22:10}}\hspace{1ex}\uanga{कौलवम्}}{}
}
{अनध्यायः\eventsep दुर्गा-व्रत-आरम्भः}
{Thu} 
\cfoot{\rygdata{13:48--15:22}{05:59--07:33}{09:06--10:40}}
\caldata{AUGUST}{9}{\sunmonth{कर्कटः}{24}{}{श्रावणः}{वर्षऋतुः}{शुक्रः}{विकारी}{दक्षिणायनम्}{ग्रीष्मऋतुः}}
{\sunmoonrsdata{05:59}{18:29}{13:37}{01:19*}{12:14}
{\kalas{04:27 05:13 09:19 08:29 10:09 16:49 10:59 13:29 15:59 17:39 19:15 21:22 22:48 01:40*}}}
{\tnykdata{\anga{\tithi{9}{शुक्ल-नवमी}}{\time{9-38}{10:00}}\hspace{1ex}}%
{\anga{अनूराधा}{\time{38-58}{21:56}}\hspace{1ex}}{चन्द्रराशिः—\mbox{वृश्चिकः}}%
{\anga{ब्राह्मः}{\time{13-18}{11:32}}\hspace{1ex}\uanga{माहेन्द्रः}}%
{\anga{कौलवम्}{\time{9-38}{10:00}}\hspace{1ex}\anga{तैतिलम्}{\time{39-8}{21:59}}\hspace{1ex}\uanga{गरजा}}{}
}
{\tamil{ஆடி~வெள்ளிக்கிழமை}\eventsep काञ्ची ५७ जगद्गुरु श्री-परमशिवेन्द्र सरस्वती २ आराधना~\#{४३४}\eventsep कौमारी-पूजा\eventsep सेङ्गालिपुरम् अनन्तराम-दीक्षित-जयन्ती~\#{११७}\eventsep वरलक्ष्मी-व्रतम्}
{Fri} 
\cfoot{\rygdata{10:40--12:14}{15:22--16:55}{07:33--09:06}}
\caldata{AUGUST}{10}{\sunmonth{कर्कटः}{25}{}{श्रावणः}{वर्षऋतुः}{शनिः}{विकारी}{दक्षिणायनम्}{ग्रीष्मऋतुः}}
{\sunmoonrsdata{05:59}{18:29}{14:32}{02:11*}{12:14}
{\kalas{04:27 05:13 09:19 08:29 10:09 16:49 10:59 13:29 15:59 17:39 19:15 21:21 22:48 01:40*}}}
{\tnykdata{\anga{\tithi{10}{शुक्ल-दशमी}}{\time{9-58}{10:08}}\hspace{1ex}}%
{\anga{ज्येष्ठा}{\time{41-55}{23:03}}\hspace{1ex}}{चन्द्रराशिः—\mbox{वृश्चिकः\RIGHTarrow{23:03}}}%
{\anga{माहेन्द्रः}{\time{11-13}{10:39}}\hspace{1ex}\uanga{वैधृतिः}}%
{\anga{गरजा}{\time{9-58}{10:08}}\hspace{1ex}\anga{वणिजा}{\time{40-18}{22:26}}\hspace{1ex}\uanga{भद्रा}}{}
}
{काञ्ची २९ जगद्गुरु श्री-पूर्णबोधेन्द्र सरस्वती आराधना~\#{१४०२}\eventsep \tamil{கோட்புலி நாயன்மார் (57) குருபூஜை}\eventsep \tamil{கலிய நாயன்மார் (44) குருபூஜை}\eventsep वैधृति-श्राद्धम्}
{Sat} 
\cfoot{\rygdata{09:06--10:40}{13:48--15:21}{05:59--07:33}}
\caldata{AUGUST}{11}{\sunmonth{कर्कटः}{26}{}{श्रावणः}{वर्षऋतुः}{भानुः}{विकारी}{दक्षिणायनम्}{ग्रीष्मऋतुः}}
{\sunmoonrsdata{05:59}{18:28}{15:25}{03:03*}{12:14}
{\kalas{04:27 05:13 09:19 08:29 10:09 16:48 10:59 13:29 15:58 17:38 19:14 21:21 22:47 01:40*}}}
{\tnykdata{\anga{\tithi{11}{शुक्ल-एकादशी}}{\time{11-43}{10:52}}\hspace{1ex}}%
{\anga{मूला}{\time{46-14}{00:42*}}\hspace{1ex}}{चन्द्रराशिः—\mbox{धनुः}}%
{\anga{वैधृतिः}{\time{10-14}{10:15}}\hspace{1ex}\uanga{विष्कम्भः}}%
{\anga{भद्रा}{\time{11-43}{10:52}}\hspace{1ex}\anga{बवम्}{\time{42-54}{23:26}}\hspace{1ex}\uanga{बालवम्}}{}
}
{सर्व-पवित्रोपान-एकादशी}
{Sun} 
\cfoot{\rygdata{16:55--18:28}{12:14--13:47}{15:21--16:55}}
\caldata{AUGUST}{12}{\sunmonth{कर्कटः}{27}{}{श्रावणः}{वर्षऋतुः}{सोमः}{विकारी}{दक्षिणायनम्}{ग्रीष्मऋतुः}}
{\sunmoonrsdata{05:59}{18:28}{16:15}{03:55*}{12:14}
{\kalas{04:27 05:13 09:19 08:29 10:09 16:48 10:59 13:28 15:58 17:38 19:14 21:21 22:47 01:40*}}}
{\tnykdata{\anga{\tithi{12}{शुक्ल-द्वादशी}}{\time{14-42}{12:06}}\hspace{1ex}}%
{\anga{पूर्वाषाढा}{\time{51-43}{02:49*}}\hspace{1ex}}{चन्द्रराशिः—\mbox{धनुः}}%
{\anga{विष्कम्भः}{\time{10-13}{10:14}}\hspace{1ex}\uanga{प्रीतिः}}%
{\anga{बालवम्}{\time{14-42}{12:06}}\hspace{1ex}\anga{कौलवम्}{\time{46-43}{00:54*}}\hspace{1ex}\uanga{तैतिलम्}}{}
}
{दामोदर-द्वादशी\eventsep दधि-व्रत-आरम्भः\eventsep सोम-प्रदोष-व्रतम्~18:28\RIGHTarrow{}19:54\eventsep शाकव्रत-समापनम्}
{Mon} 
\cfoot{\rygdata{07:33--09:06}{10:40--12:13}{13:47--15:21}}
\caldata{AUGUST}{13}{\sunmonth{कर्कटः}{28}{}{श्रावणः}{वर्षऋतुः}{मङ्गलः}{विकारी}{दक्षिणायनम्}{ग्रीष्मऋतुः}}
{\sunmoonrsdata{05:59}{18:27}{17:02}{04:46*}{12:13}
{\kalas{04:27 05:13 09:19 08:29 10:09 16:48 10:59 13:28 15:58 17:37 19:14 21:20 22:47 01:40*}}}
{\tnykdata{\anga{\tithi{13}{शुक्ल-त्रयोदशी}}{\time{18-43}{13:46}}\hspace{1ex}}%
{\anga{उत्तराषाढा}{\time{58-6}{05:16*}}\hspace{1ex}}{चन्द्रराशिः—\mbox{धनुः\RIGHTarrow{09:24}}}%
{\anga{प्रीतिः}{\time{10-59}{10:34}}\hspace{1ex}\uanga{आयुष्मान्}}%
{\anga{तैतिलम्}{\time{18-43}{13:46}}\hspace{1ex}\anga{गरजा}{\time{51-30}{02:44*}}\hspace{1ex}\uanga{वणिजा}}{}
}
{अनङ्ग-त्रयोदशी}
{Tue} 
\cfoot{\rygdata{15:20--16:54}{09:06--10:40}{12:13--13:47}}
\caldata{AUGUST}{14}{\sunmonth{कर्कटः}{29}{}{श्रावणः}{वर्षऋतुः}{बुधः}{विकारी}{दक्षिणायनम्}{ग्रीष्मऋतुः}}
{\sunmoonrsdata{06:00}{18:27}{17:46}{05:36*}{12:13}
{\kalas{04:27 05:13 09:19 08:29 10:09 16:47 10:58 13:28 15:57 17:37 19:13 21:20 22:47 01:40*}}}
{\tnykdata{\anga{\tithi{14}{शुक्ल-चतुर्दशी}}{\time{23-30}{15:45}}\hspace{1ex}}%
{\prev{\anga{उत्तराषाढा}{\time{*58-6}{05:16}}}\hspace{1ex}\fullanga{श्रवणः}}{चन्द्रराशिः—\mbox{मकरः}}%
{\anga{आयुष्मान्}{\time{12-23}{11:08}}\hspace{1ex}\uanga{सौभाग्यः}}%
{\anga{वणिजा}{\time{23-30}{15:45}}\hspace{1ex}\anga{भद्रा}{\time{57-0}{04:51*}}\hspace{1ex}\uanga{बवम्}}{}
}
{अनध्यायः\eventsep पञ्च-पर्व-पूजा (पूर्णिमा)\eventsep वेङ्कटाचले पूर्णिमा-गरुड-सेवा\eventsep श्रवण-व्रतम्}
{Wed} 
\cfoot{\rygdata{12:13--13:47}{07:33--09:06}{10:40--12:13}}
\caldata{AUGUST}{15}{\sunmonth{कर्कटः}{30}{}{श्रावणः}{वर्षऋतुः}{गुरुः}{विकारी}{दक्षिणायनम्}{ग्रीष्मऋतुः}}
{\sunmoonrsdata{06:00}{18:26}{18:27}{---}{12:13}
{\kalas{04:27 05:13 09:19 08:29 10:08 16:47 10:58 13:28 15:57 17:37 19:13 21:20 22:46 01:40*}}}
{\tnykdata{\anga{\tithi{15}{पौर्णमासी}}{\time{28-53}{17:59}}\hspace{1ex}}%
{\anga{श्रवणः}{\time{4-48}{07:59}}\hspace{1ex}}{चन्द्रराशिः—\mbox{मकरः\RIGHTarrow{21:25}}}%
{\anga{सौभाग्यः}{\time{14-15}{11:54}}\hspace{1ex}\uanga{शोभनः}}%
{\prev{\anga{भद्रा}{\time{*57-0}{04:51}}}\hspace{1ex}\anga{बवम्}{\time{28-53}{17:59}}\hspace{1ex}\uanga{बालवम्}}{}
}
{ऋग्वेद-उपाकर्म\eventsep अनध्यायः\eventsep गायत्री-जयन्ती\eventsep हयग्रीव-जयन्ती\eventsep काञ्ची २० जगद्गुरु श्री-मूकशङ्करेन्द्र सरस्वती आराधना~\#{१५८३}\eventsep महाश्रावणी-योगः\eventsep नारिकेल-पूर्णिमा\eventsep पार्वणव्रतम् पूर्णिमायाम्\eventsep पूर्णिमा-व्रतम्\eventsep रक्षाबन्धनम्\eventsep संस्कृत-दिवसः\eventsep सर्प-बलि-प्रारम्भः\eventsep श्रावण्युपवासः प्रायश्चित्तार्थः\eventsep वैखानस-महर्षि-जयन्ती\eventsep यजुर्वेद-उपाकर्म}
{Thu} 
\cfoot{\rygdata{13:46--15:20}{06:00--07:33}{09:06--10:40}}
\caldata{AUGUST}{16}{\sunmonth{कर्कटः}{31}{}{श्रावणः}{वर्षऋतुः}{शुक्रः}{विकारी}{दक्षिणायनम्}{ग्रीष्मऋतुः}}
{\sunmoonsrdata{06:00}{18:26}{19:06}{06:23}{12:13}
{\kalas{04:27 05:14 09:19 08:29 10:08 16:46 10:58 13:27 15:57 17:36 19:12 21:19 22:46 01:40*}}}
{\tnykdata{\anga{\tithi{16}{कृष्ण-प्रथमा}}{\time{34-59}{20:21}}\hspace{1ex}}%
{\anga{श्रविष्ठा}{\time{11-47}{10:53}}\hspace{1ex}}{चन्द्रराशिः—\mbox{कुम्भः}}%
{\anga{शोभनः}{\time{16-25}{12:48}}\hspace{1ex}\uanga{अतिगण्डः}}%
{\anga{बालवम्}{\time{2-47}{07:09}}\hspace{1ex}\anga{कौलवम्}{\time{34-59}{20:21}}\hspace{1ex}\uanga{तैतिलम्}}{}
}
{\tamil{ஆடி~வெள்ளிக்கிழமை}\eventsep अनध्यायः\eventsep काञ्ची ६९ जगद्गुरु श्री-जयेन्द्र सरस्वती जयन्ती~\#{८५}\eventsep पार्वण-प्रायश्चित्तावकाशः पौर्णमास्याम्\eventsep पूर्णमासेष्टिः\eventsep सहस्रगायत्रीजपः प्रायश्चित्तार्थः\eventsep पूर्ण-स्थालीपाकः}
{Fri} 
\cfoot{\rygdata{10:40--12:13}{15:19--16:53}{07:33--09:06}}
\caldata{AUGUST}{17}{\sunmonth{सिंहः}{1}{\mbox{कर्कटः{\tiny\RIGHTarrow}{12:33}}}{श्रावणः}{वर्षऋतुः}{शनिः}{विकारी}{दक्षिणायनम्}{वर्षऋतुः}}
{\sunmoonsrdata{06:00}{18:25}{19:43}{07:10}{12:13}
{\kalas{04:27 05:14 09:19 08:29 10:08 16:46 10:58 13:27 15:56 17:36 19:12 21:19 22:46 01:39*}}}
{\tnykdata{\anga{\tithi{17}{कृष्ण-द्वितीया}}{\time{41-21}{22:48}}\hspace{1ex}}%
{\anga{शतभिषक्}{\time{19-0}{13:52}}\hspace{1ex}}{चन्द्रराशिः—\mbox{कुम्भः}}%
{\anga{अतिगण्डः}{\time{18-46}{13:46}}\hspace{1ex}\uanga{सुकर्म}}%
{\anga{तैतिलम्}{\time{8-39}{09:35}}\hspace{1ex}\anga{गरजा}{\time{41-21}{22:48}}\hspace{1ex}\uanga{वणिजा}}{}
}
{अशून्यशयन-व्रतम्\eventsep बृहती-वृक्षक-पूजा\eventsep भीम-चण्डी-जयन्ती\eventsep सङ्क्रमण-दिन-अपराह्ण-पुण्यकालः~12:13\RIGHTarrow{}18:25\eventsep सिंह-रवि-सङ्क्रमण-विष्णुपदी-पुण्यकालः~06:09\RIGHTarrow{}18:25\eventsep त्रिपुष्कर-योगः~13:52\RIGHTarrow{}22:48\eventsep श्री-राघवेन्द्र-स्वामि-आराधना~\#{३४८}}
{Sat} 
\cfoot{\rygdata{09:06--10:39}{13:46--15:19}{06:00--07:33}}
\caldata{AUGUST}{18}{\sunmonth{सिंहः}{2}{}{श्रावणः}{वर्षऋतुः}{भानुः}{विकारी}{दक्षिणायनम्}{वर्षऋतुः}}
{\sunmoonsrdata{06:00}{18:25}{20:19}{07:56}{12:12}
{\kalas{04:27 05:14 09:19 08:29 10:08 16:45 10:58 13:27 15:56 17:35 19:11 21:19 22:45 01:39*}}}
{\tnykdata{\anga{\tithi{18}{कृष्ण-तृतीया}}{\time{47-37}{01:13*}}\hspace{1ex}}%
{\anga{पूर्वप्रोष्ठपदा}{\time{26-15}{16:52}}\hspace{1ex}}{चन्द्रराशिः—\mbox{कुम्भः\RIGHTarrow{10:07}}}%
{\anga{सुकर्म}{\time{21-8}{14:45}}\hspace{1ex}\uanga{धृतिः}}%
{\anga{वणिजा}{\time{14-33}{12:01}}\hspace{1ex}\anga{भद्रा}{\time{47-37}{01:13*}}\hspace{1ex}\uanga{बवम्}}{}
}
{\tamil{ஆவணி~ஞாயிற்றுக்கிழமை}\eventsep कज्जली-तृतीया\eventsep तुष्टि-प्राप्ति-तृतीया}
{Sun} 
\cfoot{\rygdata{16:52--18:25}{12:12--13:45}{15:19--16:52}}
\caldata{AUGUST}{19}{\sunmonth{सिंहः}{3}{}{श्रावणः}{वर्षऋतुः}{सोमः}{विकारी}{दक्षिणायनम्}{वर्षऋतुः}}
{\sunmoonsrdata{06:00}{18:24}{20:55}{08:41}{12:12}
{\kalas{04:27 05:14 09:19 08:29 10:08 16:45 10:58 13:26 15:55 17:35 19:11 21:18 22:45 01:39*}}}
{\tnykdata{\anga{\tithi{19}{कृष्ण-चतुर्थी}}{\time{53-31}{03:30*}}\hspace{1ex}}%
{\anga{उत्तरप्रोष्ठपदा}{\time{33-30}{19:45}}\hspace{1ex}}{चन्द्रराशिः—\mbox{मीनः}}%
{\anga{धृतिः}{\time{23-19}{15:39}}\hspace{1ex}\uanga{शूलः}}%
{\anga{बवम्}{\time{20-16}{14:23}}\hspace{1ex}\anga{बालवम्}{\time{53-31}{03:30*}}\hspace{1ex}\uanga{कौलवम्}}{}
}
{बहुला-चतुर्थी\eventsep हेरम्ब-महागणपति महासङ्कटहर-चतुर्थी-व्रतम्}
{Mon} 
\cfoot{\rygdata{07:33--09:06}{10:39--12:12}{13:45--15:18}}
\caldata{AUGUST}{20}{\sunmonth{सिंहः}{4}{}{श्रावणः}{वर्षऋतुः}{मङ्गलः}{विकारी}{दक्षिणायनम्}{वर्षऋतुः}}
{\sunmoonsrdata{06:00}{18:24}{21:33}{09:27}{12:12}
{\kalas{04:27 05:14 09:18 08:29 10:08 16:44 10:58 13:26 15:55 17:34 19:10 21:18 22:45 01:39*}}}
{\tnykdata{\anga{\tithi{20}{कृष्ण-पञ्चमी}}{\time{58-42}{05:30*}}\hspace{1ex}}%
{\anga{रेवती}{\time{40-25}{22:26}}\hspace{1ex}}{चन्द्रराशिः—\mbox{मीनः\RIGHTarrow{22:26}}}%
{\anga{शूलः}{\time{25-7}{16:23}}\hspace{1ex}\uanga{गण्डः}}%
{\anga{कौलवम्}{\time{25-30}{16:33}}\hspace{1ex}\anga{तैतिलम्}{\time{58-42}{05:30*}}\hspace{1ex}\uanga{गरजा}}{}
}
{रक्षा-पञ्चमी}
{Tue} 
\cfoot{\rygdata{15:18--16:51}{09:06--10:39}{12:12--13:45}}
\caldata{AUGUST}{21}{\sunmonth{सिंहः}{5}{}{श्रावणः}{वर्षऋतुः}{बुधः}{विकारी}{दक्षिणायनम्}{वर्षऋतुः}}
{\sunmoonsrdata{06:00}{18:23}{22:12}{10:14}{12:12}
{\kalas{04:27 05:14 09:18 08:29 10:08 16:44 10:57 13:26 15:55 17:34 19:10 21:17 22:44 01:39*}}}
{\tnykdata{\prev{\anga{\tithi{20}{कृष्ण-पञ्चमी}}{\time{*58-42}{05:30}}}\hspace{1ex}\fulltithi{\tithi{21}{कृष्ण-षष्ठी}}}%
{\anga{अश्विनी}{\time{46-24}{00:44*}}\hspace{1ex}}{चन्द्रराशिः—\mbox{मेषः}}%
{\anga{गण्डः}{\time{26-17}{16:51}}\hspace{1ex}\uanga{वृद्धिः}}%
{\prev{\anga{तैतिलम्}{\time{*58-42}{05:30}}}\hspace{1ex}\anga{गरजा}{\time{29-56}{18:22}}\hspace{1ex}\uanga{वणिजा}}{}
}
{हल-षष्ठी}
{Wed} 
\cfoot{\rygdata{12:12--13:44}{07:33--09:06}{10:39--12:12}}
\caldata{AUGUST}{22}{\sunmonth{सिंहः}{6}{}{श्रावणः}{वर्षऋतुः}{गुरुः}{विकारी}{दक्षिणायनम्}{वर्षऋतुः}}
{\sunmoonsrdata{06:00}{18:23}{22:54}{11:04}{12:11}
{\kalas{04:27 05:14 09:18 08:29 10:08 16:44 10:57 13:26 15:54 17:33 19:09 21:17 22:44 01:39*}}}
{\tnykdata{\anga{\tithi{21}{कृष्ण-षष्ठी}}{\time{2-39}{07:06}}\hspace{1ex}}%
{\anga{अपभरणी}{\time{51-5}{02:33*}}\hspace{1ex}}{चन्द्रराशिः—\mbox{मेषः}}%
{\anga{वृद्धिः}{\time{26-35}{16:58}}\hspace{1ex}\uanga{ध्रुवः}}%
{\anga{वणिजा}{\time{2-39}{07:06}}\hspace{1ex}\anga{भद्रा}{\time{33-24}{19:42}}\hspace{1ex}\uanga{बवम्}}{}
}
{}
{Thu} 
\cfoot{\rygdata{13:44--15:17}{06:00--07:33}{09:06--10:39}}
\caldata{AUGUST}{23}{\sunmonth{सिंहः}{7}{}{श्रावणः}{वर्षऋतुः}{शुक्रः}{विकारी}{दक्षिणायनम्}{वर्षऋतुः}}
{\sunmoonsrdata{06:00}{18:22}{23:41}{11:55}{12:11}
{\kalas{04:27 05:14 09:18 08:29 10:07 16:43 10:57 13:25 15:54 17:32 19:08 21:16 22:44 01:38*}}}
{\tnykdata{\anga{\tithi{22}{कृष्ण-सप्तमी}}{\time{5-11}{08:09}}\hspace{1ex}}%
{\anga{कृत्तिका}{\time{54-10}{03:45*}}\hspace{1ex}}{चन्द्रराशिः—\mbox{मेषः\RIGHTarrow{08:55}}}%
{\anga{ध्रुवः}{\time{25-44}{16:37}}\hspace{1ex}\uanga{व्याघातः}}%
{\anga{बवम्}{\time{5-11}{08:09}}\hspace{1ex}\anga{बालवम्}{\time{35-18}{20:26}}\hspace{1ex}\uanga{कौलवम्}}{}
}
{(सायन) षडशीति-पुण्यकालः~15:32\RIGHTarrow{}18:22\eventsep अनध्यायः\eventsep काञ्ची २१ जगद्गुरु श्री-सार्वभौमगुरुः चन्द्रचूडेन्द्र सरस्वती आराधना~\#{१५७३}\eventsep कृत्तिका-व्रतम्\eventsep मन्वादिः-(दक्षः-[९])\eventsep नभस्य-मासः~15:32\RIGHTarrow{}\eventsep पञ्च-पर्व-पूजा (अष्टमी)\eventsep सायन-रवि-सङ्क्रमण-पुण्यकालः~09:08\RIGHTarrow{}18:22\eventsep सायन-सङ्क्रमण-दिन-अपराह्ण-पुण्यकालः~12:11\RIGHTarrow{}18:22\eventsep श्रीकृष्णजन्माष्टमी}
{Fri} 
\cfoot{\rygdata{10:38--12:11}{15:16--16:49}{07:33--09:06}}
\caldata{AUGUST}{24}{\sunmonth{सिंहः}{8}{}{श्रावणः}{वर्षऋतुः}{शनिः}{विकारी}{दक्षिणायनम्}{वर्षऋतुः}}
{\sunmoonsrdata{06:00}{18:21}{00:32*}{12:50}{12:11}
{\kalas{04:27 05:14 09:18 08:29 10:07 16:43 10:57 13:25 15:53 17:32 19:08 21:16 22:43 01:38*}}}
{\tnykdata{\anga{\tithi{23}{कृष्ण-अष्टमी}}{\time{6-7}{08:32}}\hspace{1ex}}%
{\prev{\anga{कृत्तिका}{\time{*54-10}{03:45}}}\hspace{1ex}\anga{रोहिणी}{\time{55-23}{04:13*}}\hspace{1ex}}{चन्द्रराशिः—\mbox{वृषभः}}%
{\anga{व्याघातः}{\time{23-34}{15:43}}\hspace{1ex}\uanga{हर्षणः}}%
{\anga{कौलवम्}{\time{6-7}{08:32}}\hspace{1ex}\anga{तैतिलम्}{\time{35-22}{20:27}}\hspace{1ex}\uanga{गरजा}}{}
}
{एकविंशति-दिवस-गणपति-व्रत-समापनम्\eventsep अनध्यायः\eventsep काञ्ची २४ जगद्गुरु श्री-चित्सुखेन्द्र सरस्वती आराधना~\#{१४९३}\eventsep नन्दोत्सवः\eventsep \tamil{திருச்செந்தூர் முருகன் ஆவணித் திருவிழா தொடக்கம்/கொடியேற்றம்}\eventsep \tamil{வரகூர் உறியடி உத்ஸவம்}\eventsep शनिरोहिणी-योगः\eventsep श्री-जयन्ती\eventsep श्रीकृष्णदेवराय-राज्याभिषेकः}
{Sat} 
\cfoot{\rygdata{09:06--10:38}{13:43--15:16}{06:00--07:33}}
\caldata{AUGUST}{25}{\sunmonth{सिंहः}{9}{}{श्रावणः}{वर्षऋतुः}{भानुः}{विकारी}{दक्षिणायनम्}{वर्षऋतुः}}
{\sunmoonsrdata{06:01}{18:21}{01:27*}{13:47}{12:11}
{\kalas{04:27 05:14 09:18 08:29 10:07 16:42 10:57 13:25 15:53 17:31 19:07 21:16 22:43 01:38*}}}
{\tnykdata{\anga{\tithi{24}{कृष्ण-नवमी}}{\time{5-15}{08:10}}\hspace{1ex}}%
{\prev{\anga{रोहिणी}{\time{*55-23}{04:13}}}\hspace{1ex}\anga{मृगशीर्षम्}{\time{54-40}{03:56*}}\hspace{1ex}}{चन्द्रराशिः—\mbox{वृषभः\RIGHTarrow{16:10}}}%
{\anga{हर्षणः}{\time{19-57}{14:13}}\hspace{1ex}\uanga{वज्रम्}}%
{\anga{गरजा}{\time{5-15}{08:10}}\hspace{1ex}\anga{वणिजा}{\time{33-29}{19:42}}\hspace{1ex}\uanga{भद्रा}}{}
}
{\tamil{ஆவணி~ஞாயிற்றுக்கிழமை}\eventsep अरविन्द-जयन्ती\eventsep चण्डिका-पूजा\eventsep कौमार-पूजा\eventsep \tamil{திருச்செந்தூர் முருகன் ஆவணித் திருவிழா 2ம் நாள்}}
{Sun} 
\cfoot{\rygdata{16:48--18:21}{12:11--13:43}{15:16--16:48}}
\caldata{AUGUST}{26}{\sunmonth{सिंहः}{10}{}{श्रावणः}{वर्षऋतुः}{सोमः}{विकारी}{दक्षिणायनम्}{वर्षऋतुः}}
{\sunmoonsrdata{06:01}{18:20}{02:27*}{14:45}{12:10}
{\kalas{04:27 05:14 09:18 08:28 10:07 16:41 10:56 13:24 15:52 17:31 19:07 21:15 22:43 01:38*}}}
{\tnykdata{\anga{\tithi{25}{कृष्ण-दशमी}}{\time{2-30}{07:02}}\hspace{1ex}\anga{\tithi{26}{कृष्ण-एकादशी}}{\time{57-48}{05:10*}}\hspace{1ex}\avamA{}}%
{\prev{\anga{मृगशीर्षम्}{\time{*54-40}{03:56}}}\hspace{1ex}\anga{आर्द्रा}{\time{52-0}{02:54*}}\hspace{1ex}}{चन्द्रराशिः—\mbox{मिथुनम्}}%
{\anga{वज्रम्}{\time{14-48}{12:05}}\hspace{1ex}\uanga{सिद्धिः}}%
{\anga{भद्रा}{\time{2-30}{07:02}}\hspace{1ex}\anga{बवम्}{\time{29-38}{18:11}}\hspace{1ex}\anga{बालवम्}{\time{57-48}{05:10*}}\hspace{1ex}\uanga{कौलवम्}}{}
}
{दिनक्षयः\eventsep स्मार्त-अजा-एकादशी\eventsep \tamil{திருச்செந்தூர் முருகன் ஆவணித் திருவிழா 3ம் நாள்—முருகன் பவனி}}
{Mon} 
\cfoot{\rygdata{07:33--09:05}{10:38--12:10}{13:43--15:15}}
\caldata{AUGUST}{27}{\sunmonth{सिंहः}{11}{}{श्रावणः}{वर्षऋतुः}{मङ्गलः}{विकारी}{दक्षिणायनम्}{वर्षऋतुः}}
{\sunmoonsrdata{06:01}{18:20}{03:29*}{15:42}{12:10}
{\kalas{04:27 05:14 09:18 08:28 10:07 16:41 10:56 13:24 15:52 17:30 19:06 21:15 22:42 01:38*}}}
{\tnykdata{\prev{\anga{\tithi{26}{कृष्ण-एकादशी}}{\time{*57-48}{05:10}}}\hspace{1ex}\anga{\tithi{27}{कृष्ण-द्वादशी}}{\time{51-14}{02:36*}}\hspace{1ex}}%
{\anga{पुनर्वसुः}{\time{47-35}{01:11*}}\hspace{1ex}}{चन्द्रराशिः—\mbox{मिथुनम्\RIGHTarrow{19:40}}}%
{\anga{सिद्धिः}{\time{8-10}{09:22}}\hspace{1ex}\uanga{व्यतीपातः}}%
{\prev{\anga{बालवम्}{\time{*57-48}{05:10}}}\hspace{1ex}\anga{कौलवम्}{\time{24-13}{15:57}}\hspace{1ex}\anga{तैतिलम्}{\time{51-14}{02:36*}}\hspace{1ex}\uanga{गरजा}}{}
}
{हरिवासरः\RIGHTarrow{}10:35\eventsep जयन्ती-महाद्वादशी\eventsep रोहिणी-द्वादशी\eventsep \tamil{திருச்செந்தூர் முருகன் ஆவணித் திருவிழா 4ம் நாள்—யானை வாஹநத்தில் முருகன்-அம்பாள் பவனி}\eventsep त्रिपुष्कर-योगः~06:00\RIGHTarrow{}01:11*\eventsep वैष्णव-अजा-एकादशी\eventsep व्यतीपात-श्राद्धम्}
{Tue} 
\cfoot{\rygdata{15:15--16:47}{09:05--10:38}{12:10--13:42}}
\caldata{AUGUST}{28}{\sunmonth{सिंहः}{12}{}{श्रावणः}{वर्षऋतुः}{बुधः}{विकारी}{दक्षिणायनम्}{वर्षऋतुः}}
{\sunmoonsrdata{06:01}{18:19}{04:32*}{16:38}{12:10}
{\kalas{04:27 05:14 09:18 08:28 10:07 16:40 10:56 13:24 15:51 17:30 19:06 21:14 22:42 01:38*}}}
{\tnykdata{\anga{\tithi{28}{कृष्ण-त्रयोदशी}}{\time{43-13}{23:28}}\hspace{1ex}}%
{\anga{पुष्यः}{\time{41-41}{22:53}}\hspace{1ex}}{चन्द्रराशिः—\mbox{कर्कटः}}%
{\anga{व्यतीपातः}{\time{0-12}{06:06}}\hspace{1ex}\anga{वरीयान्}{\time{50-41}{02:23*}}\hspace{1ex}\uanga{परिघः}}%
{\anga{गरजा}{\time{17-16}{13:06}}\hspace{1ex}\anga{वणिजा}{\time{43-13}{23:28}}\hspace{1ex}\uanga{भद्रा}}{}
}
{\tamil{செருத்துணை நாயன்மார் (54) குருபூஜை}\eventsep मासशिवरात्रिः\eventsep पञ्च-पर्व-पूजा (चतुर्दशी)\eventsep प्रदोष-व्रतम्~18:19\RIGHTarrow{}19:47\eventsep \tamil{திருச்செந்தூர் முருகன் ஆவணித் திருவிழா 5ம் நாள்}}
{Wed} 
\cfoot{\rygdata{12:10--13:42}{07:33--09:05}{10:37--12:10}}
\caldata{AUGUST}{29}{\sunmonth{सिंहः}{13}{}{श्रावणः}{वर्षऋतुः}{गुरुः}{विकारी}{दक्षिणायनम्}{वर्षऋतुः}}
{\sunmoonsrdata{06:01}{18:18}{05:35*}{17:31}{12:09}
{\kalas{04:27 05:14 09:17 08:28 10:06 16:40 10:56 13:23 15:51 17:29 19:05 21:14 22:42 01:37*}}}
{\tnykdata{\anga{\tithi{29}{कृष्ण-चतुर्दशी}}{\time{34-8}{19:55}}\hspace{1ex}}%
{\anga{आश्रेषा}{\time{34-42}{20:09}}\hspace{1ex}}{चन्द्रराशिः—\mbox{कर्कटः\RIGHTarrow{20:09}}}%
{\anga{परिघः}{\time{40-17}{22:19}}\hspace{1ex}\uanga{शिवः}}%
{\anga{भद्रा}{\time{9-5}{09:44}}\hspace{1ex}\anga{शकुनिः}{\time{34-8}{19:55}}\hspace{1ex}\uanga{चतुष्पात्}}{}
}
{अघोर-चतुर्दशी\eventsep अनध्यायः\eventsep \tamil{அதிபத்த நாயன்மார் (42) குருபூஜை}\eventsep बोधायन-कात्यायन-श्रावण-अमावास्या\eventsep देवी-पर्व-५\eventsep पञ्च-पर्व-पूजा (अमावास्या)\eventsep \tamil{திருச்செந்தூர் முருகன் ஆவணித் திருவிழா 6ம் நாள்—வெள்ளித் தேர் பவனி}}
{Thu} 
\cfoot{\rygdata{13:42--15:14}{06:01--07:33}{09:05--10:37}}
\caldata{AUGUST}{30}{\sunmonth{सिंहः}{14}{}{श्रावणः}{वर्षऋतुः}{शुक्रः}{विकारी}{दक्षिणायनम्}{वर्षऋतुः}}
{\sunmoonsrdata{06:01}{18:18}{---}{18:22}{12:09}
{\kalas{04:27 05:14 09:17 08:28 10:06 16:39 10:55 13:23 15:50 17:29 19:04 21:13 22:41 01:37*}}}
{\tnykdata{\anga{\tithi{30}{अमावास्या}}{\time{24-40}{16:07}}\hspace{1ex}}%
{\anga{मघा}{\time{27-12}{17:09}}\hspace{1ex}}{चन्द्रराशिः—\mbox{सिंहः}}%
{\anga{शिवः}{\time{29-25}{18:04}}\hspace{1ex}\uanga{सिद्धः}}%
{\anga{चतुष्पात्}{\time{0-3}{06:02}}\hspace{1ex}\anga{नाग}{\time{24-40}{16:07}}\hspace{1ex}\anga{किंस्तुघ्नः}{\time{50-9}{02:10*}}\hspace{1ex}\uanga{बवम्}}{}
}
{६४ योगिनी-पूजा\eventsep अनध्यायः\eventsep बोधायन-कात्यायन-इष्टिः\eventsep दर्भ-सङ्ग्रहः\eventsep दर्भ-सङ्ग्रह आर्तव-चान्द्रः ५\eventsep \tamil{இளையான்குடி மாற நாயன்மார் (4) குருபூஜை}\eventsep पार्वणव्रतम् अमावास्यायाम्\eventsep पिण्ड-पितृ-यज्ञः\eventsep \tamil{திருச்செந்தூர் முருகன் ஆவணித் திருவிழா 7ம் நாள்—சிகப்பு சாத்தி அலங்காரம்}\eventsep वृषभ-पूजा\eventsep श्रावण-अमावास्या}
{Fri} 
\cfoot{\rygdata{10:37--12:09}{15:13--16:45}{07:33--09:05}}
\caldata{AUGUST}{31}{\sunmonth{सिंहः}{15}{}{भाद्रपदः}{वर्षऋतुः}{शनिः}{विकारी}{दक्षिणायनम्}{वर्षऋतुः}}
{\sunmoonrsdata{06:01}{18:17}{06:36}{19:11}{12:09}
{\kalas{04:27 05:14 09:17 08:28 10:06 16:39 10:55 13:22 15:50 17:28 19:04 21:13 22:41 01:37*}}}
{\tnykdata{\anga{\tithi{1}{शुक्ल-प्रथमा}}{\time{15-10}{12:13}}\hspace{1ex}}%
{\anga{पूर्वफल्गुनी}{\time{19-43}{14:05}}\hspace{1ex}}{चन्द्रराशिः—\mbox{सिंहः\RIGHTarrow{19:20}}}%
{\anga{सिद्धः}{\time{18-55}{13:45}}\hspace{1ex}\uanga{साध्यः}}%
{\anga{बवम्}{\time{15-10}{12:13}}\hspace{1ex}\anga{बालवम्}{\time{40-17}{22:18}}\hspace{1ex}\uanga{कौलवम्}}{}
}
{अनध्यायः\eventsep भाद्रपद-चन्द्र-दर्शनम्~18:17\RIGHTarrow{}19:11\eventsep दर्शेष्टिः\eventsep मृगशीर्ष-व्रतम्\eventsep पार्वण-प्रायश्चित्तावकाशः दर्शे\eventsep दर्श-स्थालीपाकः\eventsep \tamil{திருச்செந்தூர் முருகன் ஆவணித் திருவிழா 8ம் நாள்—பச்சை சாத்தி அலங்காரம்}\eventsep त्रिपुष्कर-योगः~14:05\RIGHTarrow{}06:01*}
{Sat} 
\cfoot{\rygdata{09:05--10:37}{13:41--15:13}{06:01--07:33}}
\caldata{SEPTEMBER}{1}{\sunmonth{सिंहः}{16}{}{भाद्रपदः}{वर्षऋतुः}{भानुः}{विकारी}{दक्षिणायनम्}{वर्षऋतुः}}
{\sunmoonrsdata{06:01}{18:16}{07:36}{19:59}{12:09}
{\kalas{04:27 05:14 09:17 08:28 10:06 16:38 10:55 13:22 15:49 17:27 19:03 21:12 22:40 01:37*}}}
{\tnykdata{\anga{\tithi{2}{शुक्ल-द्वितीया}}{\time{5-55}{08:26}}\hspace{1ex}\anga{\tithi{3}{शुक्ल-तृतीया}}{\time{57-14}{04:56*}}\hspace{1ex}\avamA{}}%
{\anga{उत्तरफल्गुनी}{\time{12-32}{11:08}}\hspace{1ex}}{चन्द्रराशिः—\mbox{कन्या}}%
{\anga{साध्यः}{\time{8-37}{09:32}}\hspace{1ex}\anga{शुभः}{\time{58-51}{05:34*}}\hspace{1ex}\uanga{शुक्लः}}%
{\anga{कौलवम्}{\time{5-55}{08:26}}\hspace{1ex}\anga{तैतिलम्}{\time{30-56}{18:38}}\hspace{1ex}\anga{गरजा}{\time{57-14}{04:56*}}\hspace{1ex}\uanga{वणिजा}}{}
}
{आदित्यहस्त-योगः~11:08\RIGHTarrow{}\eventsep \tamil{ஆவணி~ஞாயிற்றுக்கிழமை}\eventsep अङ्गारक-जयन्ती\eventsep अनध्यायः\eventsep दिनक्षयः\eventsep हरितालिका-व्रतम्\eventsep कल्कि-जयन्ती\eventsep मन्वादिः-(तामसः-[४])\eventsep \tamil{திருச்செந்தூர் முருகன் ஆவணித் திருவிழா 9ம் நாள்}\eventsep त्रिपुष्कर-योगः~06:01\RIGHTarrow{}08:26\eventsep विपत्तार-गौरी-व्रतम्}
{Sun} 
\cfoot{\rygdata{16:44--18:16}{12:09--13:40}{15:12--16:44}}
\caldata{SEPTEMBER}{2}{\sunmonth{सिंहः}{17}{}{भाद्रपदः}{वर्षऋतुः}{सोमः}{विकारी}{दक्षिणायनम्}{वर्षऋतुः}}
{\sunmoonrsdata{06:01}{18:16}{08:35}{20:46}{12:08}
{\kalas{04:27 05:14 09:17 08:28 10:06 16:38 10:55 13:22 15:49 17:27 19:03 21:12 22:40 01:36*}}}
{\tnykdata{\prev{\anga{\tithi{3}{शुक्ल-तृतीया}}{\time{*57-14}{04:56}}}\hspace{1ex}\anga{\tithi{4}{शुक्ल-चतुर्थी}}{\time{49-28}{01:53*}}\hspace{1ex}}%
{\anga{हस्तः}{\time{6-6}{08:30}}\hspace{1ex}}{चन्द्रराशिः—\mbox{कन्या\RIGHTarrow{19:22}}}%
{\prev{\anga{शुभः}{\time{*58-51}{05:34}}}\hspace{1ex}\anga{शुक्लः}{\time{49-42}{01:59*}}\hspace{1ex}\uanga{ब्राह्मः}}%
{\prev{\anga{गरजा}{\time{*57-14}{04:56}}}\hspace{1ex}\anga{वणिजा}{\time{22-51}{15:21}}\hspace{1ex}\anga{भद्रा}{\time{49-28}{01:53*}}\hspace{1ex}\uanga{बवम्}}{}
}
{अनध्यायः\eventsep सामवेद-उपाकर्म\eventsep \tamil{திருச்செந்தூர் முருகன் ஆவணித் திருவிழா 10ம் நாள்—தேர்}\eventsep श्रीविनायक-चतुर्थी\eventsep शुक्ल-चतुर्थी-व्रतम्}
{Mon} 
\cfoot{\rygdata{07:33--09:05}{10:36--12:08}{13:40--15:12}}
\caldata{SEPTEMBER}{3}{\sunmonth{सिंहः}{18}{}{भाद्रपदः}{वर्षऋतुः}{मङ्गलः}{विकारी}{दक्षिणायनम्}{वर्षऋतुः}}
{\sunmoonrsdata{06:01}{18:15}{09:34}{21:34}{12:08}
{\kalas{04:27 05:14 09:17 08:28 10:06 16:37 10:55 13:21 15:48 17:26 19:02 21:11 22:40 01:36*}}}
{\tnykdata{\anga{\tithi{5}{शुक्ल-पञ्चमी}}{\time{43-16}{23:27}}\hspace{1ex}}%
{\anga{चित्रा}{\time{0-50}{06:22}}\hspace{1ex}\anga{स्वाती}{\time{57-1}{04:51*}}\hspace{1ex}\avamA{}}{चन्द्रराशिः—\mbox{तुला}}%
{\anga{ब्राह्मः}{\time{41-51}{22:54}}\hspace{1ex}\uanga{माहेन्द्रः}}%
{\anga{बवम्}{\time{16-7}{12:35}}\hspace{1ex}\anga{बालवम्}{\time{43-16}{23:27}}\hspace{1ex}\uanga{कौलवम्}}{}
}
{ऋषि-पञ्चमी-व्रतम्\eventsep अनध्यायः\eventsep भाद्रपदिक-नाग-पञ्चमी\eventsep \tamil{திருச்செந்தூர் முருகன் ஆவணித் திருவிழா 11ம் நாள்}}
{Tue} 
\cfoot{\rygdata{15:11--16:43}{09:04--10:36}{12:08--13:40}}
\caldata{SEPTEMBER}{4}{\sunmonth{सिंहः}{19}{}{भाद्रपदः}{वर्षऋतुः}{बुधः}{विकारी}{दक्षिणायनम्}{वर्षऋतुः}}
{\sunmoonrsdata{06:01}{18:14}{10:33}{22:24}{12:08}
{\kalas{04:27 05:14 09:16 08:28 10:05 16:37 10:54 13:21 15:48 17:25 19:01 21:11 22:39 01:36*}}}
{\tnykdata{\anga{\tithi{6}{शुक्ल-षष्ठी}}{\time{38-54}{21:44}}\hspace{1ex}}%
{\prev{\anga{स्वाती}{\time{*57-1}{04:51}}}\hspace{1ex}\anga{विशाखा}{\time{55-3}{04:05*}}\hspace{1ex}}{चन्द्रराशिः—\mbox{तुला\RIGHTarrow{22:12}}}%
{\anga{माहेन्द्रः}{\time{35-31}{20:24}}\hspace{1ex}\uanga{वैधृतिः}}%
{\anga{कौलवम्}{\time{11-0}{10:30}}\hspace{1ex}\anga{तैतिलम्}{\time{38-54}{21:44}}\hspace{1ex}\uanga{गरजा}}{}
}
{षष्ठीदेवी-षष्ठी-व्रतम्\eventsep अनध्यायः\eventsep कुमारिका-स्वपनम्\eventsep ललिता-षष्ठी\eventsep मन्थन-षष्ठी\eventsep सूर्य-षष्ठी\eventsep \tamil{திருச்செந்தூர் ஆவணித் திருவிழா நிறைவு}}
{Wed} 
\cfoot{\rygdata{12:08--13:39}{07:33--09:04}{10:36--12:08}}
\caldata{SEPTEMBER}{5}{\sunmonth{सिंहः}{20}{}{भाद्रपदः}{वर्षऋतुः}{गुरुः}{विकारी}{दक्षिणायनम्}{वर्षऋतुः}}
{\sunmoonrsdata{06:01}{18:14}{11:30}{23:15}{12:07}
{\kalas{04:27 05:14 09:16 08:27 10:05 16:36 10:54 13:21 15:47 17:25 19:01 21:10 22:39 01:36*}}}
{\tnykdata{\anga{\tithi{7}{शुक्ल-सप्तमी}}{\time{36-36}{20:49}}\hspace{1ex}}%
{\prev{\anga{विशाखा}{\time{*55-3}{04:05}}}\hspace{1ex}\anga{अनूराधा}{\time{55-7}{04:06*}}\hspace{1ex}}{चन्द्रराशिः—\mbox{वृश्चिकः}}%
{\anga{वैधृतिः}{\time{30-49}{18:33}}\hspace{1ex}\uanga{विष्कम्भः}}%
{\anga{गरजा}{\time{7-46}{09:11}}\hspace{1ex}\anga{वणिजा}{\time{36-36}{20:49}}\hspace{1ex}\uanga{भद्रा}}{}
}
{अमुक्ताभरण-सप्तमी\eventsep अनन्तफल-सप्तमी\eventsep देवी-पर्व-६\eventsep कुक्कुटी-व्रतम्\eventsep \tamil{குலச்சிறை நாயன்மார் (22) குருபூஜை}\eventsep वैधृति-श्राद्धम्}
{Thu} 
\cfoot{\rygdata{13:39--15:10}{06:01--07:33}{09:04--10:36}}
\caldata{SEPTEMBER}{6}{\sunmonth{सिंहः}{21}{}{भाद्रपदः}{वर्षऋतुः}{शुक्रः}{विकारी}{दक्षिणायनम्}{वर्षऋतुः}}
{\sunmoonrsdata{06:01}{18:13}{12:27}{00:07*}{12:07}
{\kalas{04:27 05:14 09:16 08:27 10:05 16:35 10:54 13:20 15:46 17:24 19:00 21:10 22:38 01:35*}}}
{\tnykdata{\anga{\tithi{8}{शुक्ल-अष्टमी}}{\time{36-20}{20:43}}\hspace{1ex}}%
{\prev{\anga{अनूराधा}{\time{*55-7}{04:06}}}\hspace{1ex}\anga{ज्येष्ठा}{\time{57-11}{04:55*}}\hspace{1ex}}{चन्द्रराशिः—\mbox{वृश्चिकः\RIGHTarrow{04:55*}}}%
{\anga{विष्कम्भः}{\time{27-51}{17:20}}\hspace{1ex}\uanga{प्रीतिः}}%
{\anga{भद्रा}{\time{6-31}{08:40}}\hspace{1ex}\anga{बवम्}{\time{36-20}{20:43}}\hspace{1ex}\uanga{बालवम्}}{}
}
{अनध्यायः\eventsep दूर्वाष्टमी\eventsep दधीचि-महर्षि-जयन्ति\eventsep राधाष्टमी}
{Fri} 
\cfoot{\rygdata{10:35--12:07}{15:10--16:41}{07:32--09:04}}
\caldata{SEPTEMBER}{7}{\sunmonth{सिंहः}{22}{}{भाद्रपदः}{वर्षऋतुः}{शनिः}{विकारी}{दक्षिणायनम्}{वर्षऋतुः}}
{\sunmoonrsdata{06:01}{18:12}{13:21}{00:59*}{12:07}
{\kalas{04:26 05:14 09:16 08:27 10:05 16:35 10:53 13:20 15:46 17:24 18:59 21:09 22:38 01:35*}}}
{\tnykdata{\anga{\tithi{9}{शुक्ल-नवमी}}{\time{38-1}{21:22}}\hspace{1ex}}%
{\prev{\anga{ज्येष्ठा}{\time{*57-11}{04:55}}}\hspace{1ex}\fullanga{मूला}}{चन्द्रराशिः—\mbox{धनुः}}%
{\anga{प्रीतिः}{\time{26-23}{16:44}}\hspace{1ex}\uanga{आयुष्मान्}}%
{\anga{बालवम्}{\time{7-13}{08:57}}\hspace{1ex}\anga{कौलवम्}{\time{38-1}{21:22}}\hspace{1ex}\uanga{तैतिलम्}}{}
}
{अदुःखनवमी\eventsep गोधूमा-नवमी\eventsep गजेन्द्र-मोक्षः\eventsep \tamil{குங்கிலியக்கலய நாயன்மார் (11) குருபூஜை}\eventsep महालक्ष्मी-व्रत-आरम्भः\eventsep नन्दा-नवमी\eventsep \tamil{பிட்டுக்கு மண் சுமந்த லீலை}\eventsep तालनवमी}
{Sat} 
\cfoot{\rygdata{09:04--10:35}{13:38--15:09}{06:01--07:32}}
\caldata{SEPTEMBER}{8}{\sunmonth{सिंहः}{23}{}{भाद्रपदः}{वर्षऋतुः}{भानुः}{विकारी}{दक्षिणायनम्}{वर्षऋतुः}}
{\sunmoonrsdata{06:01}{18:12}{14:12}{01:52*}{12:06}
{\kalas{04:26 05:14 09:16 08:27 10:04 16:34 10:53 13:19 15:45 17:23 18:59 21:09 22:38 01:35*}}}
{\tnykdata{\anga{\tithi{10}{शुक्ल-दशमी}}{\time{41-22}{22:41}}\hspace{1ex}}%
{\anga{मूला}{\time{1-2}{06:26}}\hspace{1ex}}{चन्द्रराशिः—\mbox{धनुः}}%
{\anga{आयुष्मान्}{\time{26-13}{16:40}}\hspace{1ex}\uanga{सौभाग्यः}}%
{\anga{तैतिलम्}{\time{9-41}{09:57}}\hspace{1ex}\anga{गरजा}{\time{41-22}{22:41}}\hspace{1ex}\uanga{वणिजा}}{}
}
{\tamil{ஆவணி~ஞாயிற்றுக்கிழமை}\eventsep दशावतार-व्रतम्\eventsep वितस्तोत्सवः}
{Sun} 
\cfoot{\rygdata{16:40--18:12}{12:06--13:38}{15:09--16:40}}
\caldata{SEPTEMBER}{9}{\sunmonth{सिंहः}{24}{}{भाद्रपदः}{वर्षऋतुः}{सोमः}{विकारी}{दक्षिणायनम्}{वर्षऋतुः}}
{\sunmoonrsdata{06:01}{18:11}{15:00}{02:43*}{12:06}
{\kalas{04:26 05:14 09:16 08:27 10:04 16:34 10:53 13:19 15:45 17:22 18:58 21:08 22:37 01:35*}}}
{\tnykdata{\anga{\tithi{11}{शुक्ल-एकादशी}}{\time{46-2}{00:31*}}\hspace{1ex}}%
{\anga{पूर्वाषाढा}{\time{6-14}{08:33}}\hspace{1ex}}{चन्द्रराशिः—\mbox{धनुः\RIGHTarrow{15:09}}}%
{\anga{सौभाग्यः}{\time{27-6}{17:01}}\hspace{1ex}\uanga{शोभनः}}%
{\anga{वणिजा}{\time{13-37}{11:32}}\hspace{1ex}\anga{भद्रा}{\time{46-2}{00:31*}}\hspace{1ex}\uanga{बवम्}}{}
}
{भुवनेश्वरी-जयन्ती\eventsep कटदानोत्सवः\eventsep सर्व-परिवर्तिनी-एकादशी}
{Mon} 
\cfoot{\rygdata{07:32--09:03}{10:35--12:06}{13:37--15:08}}
\caldata{SEPTEMBER}{10}{\sunmonth{सिंहः}{25}{}{भाद्रपदः}{वर्षऋतुः}{मङ्गलः}{विकारी}{दक्षिणायनम्}{वर्षऋतुः}}
{\sunmoonrsdata{06:01}{18:10}{15:45}{03:32*}{12:06}
{\kalas{04:26 05:14 09:15 08:27 10:04 16:33 10:53 13:18 15:44 17:21 18:58 21:08 22:37 01:34*}}}
{\tnykdata{\anga{\tithi{12}{शुक्ल-द्वादशी}}{\time{51-36}{02:42*}}\hspace{1ex}}%
{\anga{उत्तराषाढा}{\time{12-33}{11:06}}\hspace{1ex}}{चन्द्रराशिः—\mbox{मकरः}}%
{\anga{शोभनः}{\time{28-45}{17:40}}\hspace{1ex}\uanga{अतिगण्डः}}%
{\anga{बवम्}{\time{18-39}{13:34}}\hspace{1ex}\anga{बालवम्}{\time{51-36}{02:42*}}\hspace{1ex}\uanga{कौलवम्}}{}
}
{अगस्त्यार्घ्यम्\eventsep अनध्यायः\eventsep अनन्त-द्वादशी\eventsep दधि-व्रत-समापनम्\eventsep हरिवासरः\RIGHTarrow{}07:02\eventsep पयोव्रत-आरम्भः\eventsep त्रिपुष्कर-योगः~06:01\RIGHTarrow{}11:06\eventsep वामन-जयन्ती\eventsep विजया/श्रवण-महाद्वादशी\eventsep शक्रध्वजोत्थापनम्}
{Tue} 
\cfoot{\rygdata{15:08--16:39}{09:03--10:34}{12:06--13:37}}
\caldata{SEPTEMBER}{11}{\sunmonth{सिंहः}{26}{}{भाद्रपदः}{वर्षऋतुः}{बुधः}{विकारी}{दक्षिणायनम्}{वर्षऋतुः}}
{\sunmoonrsdata{06:01}{18:09}{16:27}{04:21*}{12:05}
{\kalas{04:26 05:14 09:15 08:27 10:04 16:32 10:52 13:18 15:44 17:21 18:57 21:07 22:36 01:34*}}}
{\tnykdata{\anga{\tithi{13}{शुक्ल-त्रयोदशी}}{\time{57-41}{05:06*}}\hspace{1ex}}%
{\anga{श्रवणः}{\time{19-34}{13:56}}\hspace{1ex}}{चन्द्रराशिः—\mbox{मकरः\RIGHTarrow{03:25*}}}%
{\anga{अतिगण्डः}{\time{30-54}{18:31}}\hspace{1ex}\uanga{सुकर्म}}%
{\anga{कौलवम्}{\time{24-23}{15:53}}\hspace{1ex}\anga{तैतिलम्}{\time{57-41}{05:06*}}\hspace{1ex}\uanga{गरजा}}{}
}
{\tamil{ஓணம்}\eventsep दूर्व-त्रि-व्रतम्\eventsep गो-त्रिरात्र-व्रतम्\eventsep प्रदोष-व्रतम्~18:09\RIGHTarrow{}19:38\eventsep विवेकानन्द-भाषणं चिकागोनगरे~\#{१२६}\eventsep श्रवण-व्रतम्}
{Wed} 
\cfoot{\rygdata{12:05--13:36}{07:32--09:03}{10:34--12:05}}
\caldata{SEPTEMBER}{12}{\sunmonth{सिंहः}{27}{}{भाद्रपदः}{वर्षऋतुः}{गुरुः}{विकारी}{दक्षिणायनम्}{वर्षऋतुः}}
{\sunmoonrsdata{06:01}{18:09}{17:06}{05:07*}{12:05}
{\kalas{04:26 05:14 09:15 08:27 10:04 16:32 10:52 13:18 15:43 17:20 18:56 21:07 22:36 01:34*}}}
{\tnykdata{\prev{\anga{\tithi{13}{शुक्ल-त्रयोदशी}}{\time{*57-41}{05:06}}}\hspace{1ex}\fulltithi{\tithi{14}{शुक्ल-चतुर्दशी}}}%
{\anga{श्रविष्ठा}{\time{26-58}{16:55}}\hspace{1ex}}{चन्द्रराशिः—\mbox{कुम्भः}}%
{\anga{सुकर्म}{\time{33-19}{19:27}}\hspace{1ex}\uanga{धृतिः}}%
{\prev{\anga{तैतिलम्}{\time{*57-41}{05:06}}}\hspace{1ex}\anga{गरजा}{\time{30-29}{18:20}}\hspace{1ex}\uanga{वणिजा}}{}
}
{अनध्यायः\eventsep अनन्त-चतुर्दशी\eventsep अनन्त-पद्मनाभ-व्रतम्\eventsep \tamil{நடராஜர் மஹாபிஷேகம்}}
{Thu} 
\cfoot{\rygdata{13:36--15:07}{06:01--07:32}{09:03--10:34}}
\caldata{SEPTEMBER}{13}{\sunmonth{सिंहः}{28}{}{भाद्रपदः}{वर्षऋतुः}{शुक्रः}{विकारी}{दक्षिणायनम्}{वर्षऋतुः}}
{\sunmoonrsdata{06:01}{18:08}{17:43}{05:53*}{12:04}
{\kalas{04:26 05:13 09:15 08:26 10:03 16:31 10:52 13:17 15:43 17:19 18:56 21:06 22:35 01:34*}}}
{\tnykdata{\anga{\tithi{14}{शुक्ल-चतुर्दशी}}{\time{3-52}{07:35}}\hspace{1ex}}%
{\anga{शतभिषक्}{\time{34-31}{19:56}}\hspace{1ex}}{चन्द्रराशिः—\mbox{कुम्भः}}%
{\anga{धृतिः}{\time{35-45}{20:25}}\hspace{1ex}\uanga{शूलः}}%
{\anga{वणिजा}{\time{3-52}{07:35}}\hspace{1ex}\anga{भद्रा}{\time{36-46}{20:49}}\hspace{1ex}\uanga{बवम्}}{}
}
{काञ्ची ५९ जगद्गुरु श्री-भगवन्नाम बोधेन्द्र सरस्वती आराधना~\#{३२८}\eventsep पार्वणव्रतम् पूर्णिमायाम्\eventsep पञ्च-पर्व-पूजा (पूर्णिमा)\eventsep वेङ्कटाचले पूर्णिमा-गरुड-सेवा}
{Fri} 
\cfoot{\rygdata{10:34--12:04}{15:06--16:37}{07:32--09:03}}
\caldata{SEPTEMBER}{14}{\sunmonth{सिंहः}{29}{}{भाद्रपदः}{वर्षऋतुः}{शनिः}{विकारी}{दक्षिणायनम्}{वर्षऋतुः}}
{\sunmoonrsdata{06:01}{18:07}{18:20}{---}{12:04}
{\kalas{04:26 05:13 09:15 08:26 10:03 16:30 10:52 13:17 15:42 17:19 18:55 21:06 22:35 01:33*}}}
{\tnykdata{\anga{\tithi{15}{पौर्णमासी}}{\time{9-58}{10:02}}\hspace{1ex}}%
{\anga{पूर्वप्रोष्ठपदा}{\time{41-59}{22:52}}\hspace{1ex}}{चन्द्रराशिः—\mbox{कुम्भः\RIGHTarrow{16:09}}}%
{\anga{शूलः}{\time{38-4}{21:19}}\hspace{1ex}\uanga{गण्डः}}%
{\anga{बवम्}{\time{9-58}{10:02}}\hspace{1ex}\anga{बालवम्}{\time{42-53}{23:14}}\hspace{1ex}\uanga{कौलवम्}}{}
}
{अगस्त्यार्घ्यम्\eventsep अनध्यायः\eventsep दिक्पाल-पूजा\eventsep महालय-पक्ष-आरम्भः\eventsep पार्वण-प्रायश्चित्तावकाशः पौर्णमास्याम्\eventsep पूर्णमासेष्टिः\eventsep पूर्णिमा-व्रतम्\eventsep पूर्ण-स्थालीपाकः\eventsep उमा-महेश्वर-व्रतम्\eventsep उपाङ्ग-ललिता-गौरी-व्रतम्\eventsep विश्वरूप-यात्रा\eventsep यतिचातुर्मास्यव्रत-समापनम्}
{Sat} 
\cfoot{\rygdata{09:03--10:33}{13:35--15:06}{06:01--07:32}}
\caldata{SEPTEMBER}{15}{\sunmonth{सिंहः}{30}{}{भाद्रपदः}{वर्षऋतुः}{भानुः}{विकारी}{दक्षिणायनम्}{वर्षऋतुः}}
{\sunmoonsrdata{06:01}{18:07}{18:56}{06:39}{12:04}
{\kalas{04:26 05:13 09:14 08:26 10:03 16:30 10:51 13:16 15:41 17:18 18:54 21:05 22:34 01:33*}}}
{\tnykdata{\anga{\tithi{16}{कृष्ण-प्रथमा}}{\time{15-49}{12:24}}\hspace{1ex}}%
{\anga{उत्तरप्रोष्ठपदा}{\time{49-6}{01:41*}}\hspace{1ex}}{चन्द्रराशिः—\mbox{मीनः}}%
{\anga{गण्डः}{\time{40-8}{22:08}}\hspace{1ex}\uanga{वृद्धिः}}%
{\anga{कौलवम्}{\time{15-49}{12:24}}\hspace{1ex}\anga{तैतिलम्}{\time{48-39}{01:31*}}\hspace{1ex}\uanga{गरजा}}{}
}
{\tamil{ஆவணி~ஞாயிற்றுக்கிழமை}\eventsep अगस्त्यार्घ्यम्\eventsep अनध्यायः\eventsep अशून्यशयन-व्रतम्}
{Sun} 
\cfoot{\rygdata{16:36--18:07}{12:04--13:34}{15:05--16:36}}
\caldata{SEPTEMBER}{16}{\sunmonth{सिंहः}{31}{}{भाद्रपदः}{वर्षऋतुः}{सोमः}{विकारी}{दक्षिणायनम्}{वर्षऋतुः}}
{\sunmoonsrdata{06:01}{18:06}{19:33}{07:25}{12:03}
{\kalas{04:26 05:13 09:14 08:26 10:03 16:29 10:51 13:16 15:41 17:17 18:54 21:05 22:34 01:33*}}}
{\tnykdata{\anga{\tithi{17}{कृष्ण-द्वितीया}}{\time{21-16}{14:35}}\hspace{1ex}}%
{\anga{रेवती}{\time{55-43}{04:19*}}\hspace{1ex}}{चन्द्रराशिः—\mbox{मीनः\RIGHTarrow{04:19*}}}%
{\anga{वृद्धिः}{\time{41-49}{22:48}}\hspace{1ex}\uanga{ध्रुवः}}%
{\anga{गरजा}{\time{21-16}{14:35}}\hspace{1ex}\anga{वणिजा}{\time{53-54}{03:36*}}\hspace{1ex}\uanga{भद्रा}}{}
}
{अगस्त्यार्घ्यम्}
{Mon} 
\cfoot{\rygdata{07:32--09:02}{10:33--12:03}{13:34--15:05}}
\caldata{SEPTEMBER}{17}{\sunmonth{कन्या}{1}{\mbox{सिंहः{\tiny\RIGHTarrow}{12:31}}}{भाद्रपदः}{वर्षऋतुः}{मङ्गलः}{विकारी}{दक्षिणायनम्}{वर्षऋतुः}}
{\sunmoonsrdata{06:01}{18:05}{20:12}{08:12}{12:03}
{\kalas{04:26 05:13 09:14 08:26 10:02 16:29 10:51 13:15 15:40 17:17 18:53 21:04 22:34 01:33*}}}
{\tnykdata{\anga{\tithi{18}{कृष्ण-तृतीया}}{\time{26-9}{16:33}}\hspace{1ex}}%
{\prev{\anga{रेवती}{\time{*55-43}{04:19}}}\hspace{1ex}\fullanga{अश्विनी}}{चन्द्रराशिः—\mbox{मेषः}}%
{\anga{ध्रुवः}{\time{43-0}{23:16}}\hspace{1ex}\uanga{व्याघातः}}%
{\anga{भद्रा}{\time{26-9}{16:33}}\hspace{1ex}\anga{बवम्}{\time{58-28}{05:25*}}\hspace{1ex}\uanga{बालवम्}}{}
}
{अङ्गारकी विघ्नराज-महागणपति सङ्कटहर-चतुर्थी-व्रतम्\eventsep भौमाश्विनी-योगः\eventsep गौरी-व्रतम्\eventsep कजरी-तृतीया\eventsep कन्या-रवि-सङ्क्रमण-षडशीति-पुण्यकालः~12:31\RIGHTarrow{}18:05\eventsep रवि-सङ्क्रमण-पुण्यकालः~06:07\RIGHTarrow{}18:05\eventsep सङ्क्रमण-दिन-अपराह्ण-पुण्यकालः~12:03\RIGHTarrow{}18:05\eventsep सुखा~अङ्गारकी-चतुर्थी\eventsep विश्वकर्म-जयन्ती}
{Tue} 
\cfoot{\rygdata{15:04--16:35}{09:02--10:33}{12:03--13:34}}
\caldata{SEPTEMBER}{18}{\sunmonth{कन्या}{2}{}{भाद्रपदः}{वर्षऋतुः}{बुधः}{विकारी}{दक्षिणायनम्}{वर्षऋतुः}}
{\sunmoonsrdata{06:01}{18:04}{20:53}{09:00}{12:03}
{\kalas{04:25 05:13 09:14 08:26 10:02 16:28 10:50 13:15 15:40 17:16 18:52 21:04 22:33 01:32*}}}
{\tnykdata{\anga{\tithi{19}{कृष्ण-चतुर्थी}}{\time{30-17}{18:11}}\hspace{1ex}}%
{\anga{अश्विनी}{\time{1-39}{06:41}}\hspace{1ex}}{चन्द्रराशिः—\mbox{मेषः}}%
{\anga{व्याघातः}{\time{43-33}{23:28}}\hspace{1ex}\uanga{हर्षणः}}%
{\prev{\anga{बवम्}{\time{*58-28}{05:25}}}\hspace{1ex}\anga{बालवम्}{\time{30-17}{18:11}}\hspace{1ex}\uanga{कौलवम्}}{}
}
{अनध्यायः\eventsep दिक्पाल-पूजा\eventsep महाभरणी}
{Wed} 
\cfoot{\rygdata{12:03--13:33}{07:31--09:02}{10:32--12:03}}
\caldata{SEPTEMBER}{19}{\sunmonth{कन्या}{3}{}{भाद्रपदः}{वर्षऋतुः}{गुरुः}{विकारी}{दक्षिणायनम्}{वर्षऋतुः}}
{\sunmoonsrdata{06:01}{18:04}{21:37}{09:50}{12:02}
{\kalas{04:25 05:13 09:14 08:26 10:02 16:27 10:50 13:15 15:39 17:16 18:51 21:03 22:33 01:32*}}}
{\tnykdata{\anga{\tithi{20}{कृष्ण-पञ्चमी}}{\time{33-27}{19:26}}\hspace{1ex}}%
{\anga{अपभरणी}{\time{6-41}{08:42}}\hspace{1ex}}{चन्द्रराशिः—\mbox{मेषः\RIGHTarrow{15:08}}}%
{\anga{हर्षणः}{\time{43-17}{23:21}}\hspace{1ex}\uanga{वज्रम्}}%
{\anga{कौलवम्}{\time{2-7}{06:52}}\hspace{1ex}\anga{तैतिलम्}{\time{33-27}{19:26}}\hspace{1ex}\uanga{गरजा}}{}
}
{चन्द्र-षष्ठी\eventsep कृत्तिका-व्रतम्\eventsep नाग-पूजा\eventsep सप्तर्षि-पूजा/अर्घ्यम्}
{Thu} 
\cfoot{\rygdata{13:33--15:03}{06:01--07:31}{09:02--10:32}}
\caldata{SEPTEMBER}{20}{\sunmonth{कन्या}{4}{}{भाद्रपदः}{वर्षऋतुः}{शुक्रः}{विकारी}{दक्षिणायनम्}{वर्षऋतुः}}
{\sunmoonsrdata{06:01}{18:03}{22:25}{10:43}{12:02}
{\kalas{04:25 05:13 09:13 08:25 10:02 16:27 10:50 13:14 15:39 17:15 18:51 21:03 22:32 01:32*}}}
{\tnykdata{\anga{\tithi{21}{कृष्ण-षष्ठी}}{\time{35-21}{20:11}}\hspace{1ex}}%
{\anga{कृत्तिका}{\time{10-37}{10:17}}\hspace{1ex}}{चन्द्रराशिः—\mbox{वृषभः}}%
{\anga{वज्रम्}{\time{42-0}{22:51}}\hspace{1ex}\uanga{सिद्धिः}}%
{\anga{गरजा}{\time{4-38}{07:53}}\hspace{1ex}\anga{वणिजा}{\time{35-21}{20:11}}\hspace{1ex}\uanga{भद्रा}}{}
}
{काञ्ची ३३ जगद्गुरु श्री-सच्चिदानन्दघनेन्द्र सरस्वती २ आराधना~\#{१३२८}\eventsep कपिल-षष्ठी}
{Fri} 
\cfoot{\rygdata{10:32--12:02}{15:03--16:33}{07:31--09:02}}
\caldata{SEPTEMBER}{21}{\sunmonth{कन्या}{5}{}{भाद्रपदः}{वर्षऋतुः}{शनिः}{विकारी}{दक्षिणायनम्}{वर्षऋतुः}}
{\sunmoonsrdata{06:01}{18:02}{23:18}{11:38}{12:02}
{\kalas{04:25 05:13 09:13 08:25 10:01 16:26 10:50 13:14 15:38 17:14 18:50 21:02 22:32 01:32*}}}
{\tnykdata{\anga{\tithi{22}{कृष्ण-सप्तमी}}{\time{35-46}{20:20}}\hspace{1ex}}%
{\anga{रोहिणी}{\time{13-13}{11:19}}\hspace{1ex}}{चन्द्रराशिः—\mbox{वृषभः\RIGHTarrow{23:36}}}%
{\anga{सिद्धिः}{\time{39-34}{21:51}}\hspace{1ex}\uanga{व्यतीपातः}}%
{\anga{भद्रा}{\time{5-48}{08:21}}\hspace{1ex}\anga{बवम्}{\time{35-46}{20:20}}\hspace{1ex}\uanga{बालवम्}}{}
}
{द्विपुष्कर-योगः~11:19\RIGHTarrow{}20:20\eventsep महाकाली-जयन्ती\eventsep महालक्ष्मी-व्रत-समापनम्\eventsep पञ्च-पर्व-पूजा (अष्टमी)\eventsep \tamil{புரட்டாசி~சனிக்கிழமை}\eventsep \tamil{திருநாளைப்போவார் நாயன்மார் (18) குருபூஜை}\eventsep शृङ्गेरी ३५ जगद्गुरु श्री-अभिनव विद्यातीर्थ महास्वामी आराधना\eventsep शनिरोहिणी-योगः\RIGHTarrow{}11:19}
{Sat} 
\cfoot{\rygdata{09:01--10:31}{13:32--15:02}{06:01--07:31}}
\caldata{SEPTEMBER}{22}{\sunmonth{कन्या}{6}{}{भाद्रपदः}{वर्षऋतुः}{भानुः}{विकारी}{दक्षिणायनम्}{वर्षऋतुः}}
{\sunmoonsrdata{06:01}{18:02}{00:14*}{12:34}{12:01}
{\kalas{04:25 05:13 09:13 08:25 10:01 16:25 10:49 13:13 15:37 17:14 18:49 21:01 22:31 01:31*}}}
{\tnykdata{\anga{\tithi{23}{कृष्ण-अष्टमी}}{\time{34-31}{19:50}}\hspace{1ex}}%
{\anga{मृगशीर्षम्}{\time{14-15}{11:43}}\hspace{1ex}}{चन्द्रराशिः—\mbox{मिथुनम्}}%
{\anga{व्यतीपातः}{\time{35-47}{20:20}}\hspace{1ex}\uanga{वरीयान्}}%
{\anga{बालवम्}{\time{5-23}{08:10}}\hspace{1ex}\anga{कौलवम्}{\time{34-31}{19:50}}\hspace{1ex}\uanga{तैतिलम्}}{}
}
{अनध्यायः\eventsep अशोकाष्टमी-व्रत-आरम्भः\eventsep जीवपुत्रिकाष्टमी/जीमूतवाहन-पूजा\eventsep मध्याष्टमी\eventsep महाव्यतीपात-श्राद्धम्\eventsep पातार्क-योगः\eventsep वन-रक्षक-वैष्णव-हत्या~\#{२८९}}
{Sun} 
\cfoot{\rygdata{16:32--18:02}{12:01--13:31}{15:01--16:32}}
\caldata{SEPTEMBER}{23}{\sunmonth{कन्या}{7}{}{भाद्रपदः}{वर्षऋतुः}{सोमः}{विकारी}{दक्षिणायनम्}{वर्षऋतुः}}
{\sunmoonsrdata{06:01}{18:01}{01:13*}{13:29}{12:01}
{\kalas{04:25 05:13 09:13 08:25 10:01 16:25 10:49 13:13 15:37 17:13 18:49 21:01 22:31 01:31*}}}
{\tnykdata{\anga{\tithi{24}{कृष्ण-नवमी}}{\time{31-30}{18:37}}\hspace{1ex}}%
{\anga{आर्द्रा}{\time{13-34}{11:27}}\hspace{1ex}}{चन्द्रराशिः—\mbox{मिथुनम्\RIGHTarrow{04:47*}}}%
{\anga{वरीयान्}{\time{30-37}{18:16}}\hspace{1ex}\uanga{परिघः}}%
{\anga{तैतिलम्}{\time{3-14}{07:19}}\hspace{1ex}\anga{गरजा}{\time{31-30}{18:37}}\hspace{1ex}\anga{वणिजा}{\time{59-19}{05:45*}}\hspace{1ex}\uanga{भद्रा}}{}
}
{अनध्यायः\eventsep अविधवा-नवमी\eventsep दक्षिण-विषुव-दिनम्\eventsep दुर्गा/गौरी-पूजा\eventsep इष-मासः/शरदृतुः~13:20\RIGHTarrow{}\eventsep सायन-रवि-सङ्क्रमण-पुण्यकालः~06:56\RIGHTarrow{}18:01\eventsep सायन-सङ्क्रमण-दिन-अपराह्ण-पुण्यकालः~12:01\RIGHTarrow{}18:01\eventsep विषु-पुण्यकालः~09:20\RIGHTarrow{}17:20}
{Mon} 
\cfoot{\rygdata{07:31--09:01}{10:31--12:01}{13:31--15:01}}
\caldata{SEPTEMBER}{24}{\sunmonth{कन्या}{8}{}{भाद्रपदः}{वर्षऋतुः}{मङ्गलः}{विकारी}{दक्षिणायनम्}{वर्षऋतुः}}
{\sunmoonsrdata{06:01}{18:00}{02:14*}{14:24}{12:01}
{\kalas{04:25 05:13 09:13 08:25 10:01 16:24 10:49 13:13 15:36 17:12 18:48 21:00 22:31 01:31*}}}
{\tnykdata{\anga{\tithi{25}{कृष्ण-दशमी}}{\time{26-45}{16:42}}\hspace{1ex}}%
{\anga{पुनर्वसुः}{\time{11-9}{10:28}}\hspace{1ex}}{चन्द्रराशिः—\mbox{कर्कटः}}%
{\anga{परिघः}{\time{24-2}{15:37}}\hspace{1ex}\uanga{शिवः}}%
{\prev{\anga{वणिजा}{\time{*59-19}{05:45}}}\hspace{1ex}\anga{भद्रा}{\time{26-45}{16:42}}\hspace{1ex}\anga{बवम्}{\time{53-42}{03:30*}}\hspace{1ex}\uanga{बालवम्}}{}
}
{}
{Tue} 
\cfoot{\rygdata{15:00--16:30}{09:01--10:31}{12:01--13:31}}
\caldata{SEPTEMBER}{25}{\sunmonth{कन्या}{9}{}{भाद्रपदः}{वर्षऋतुः}{बुधः}{विकारी}{दक्षिणायनम्}{वर्षऋतुः}}
{\sunmoonsrdata{06:01}{17:59}{03:15*}{15:17}{12:00}
{\kalas{04:25 05:13 09:13 08:25 10:00 16:24 10:48 13:12 15:36 17:12 18:47 21:00 22:30 01:30*}}}
{\tnykdata{\anga{\tithi{26}{कृष्ण-एकादशी}}{\time{20-22}{14:09}}\hspace{1ex}}%
{\anga{पुष्यः}{\time{7-3}{08:50}}\hspace{1ex}}{चन्द्रराशिः—\mbox{कर्कटः}}%
{\anga{शिवः}{\time{16-6}{12:27}}\hspace{1ex}\uanga{सिद्धः}}%
{\anga{बालवम्}{\time{20-22}{14:09}}\hspace{1ex}\anga{कौलवम्}{\time{46-37}{00:39*}}\hspace{1ex}\uanga{तैतिलम्}}{}
}
{सर्व-इन्दिरा-एकादशी\eventsep यति-महालयम्}
{Wed} 
\cfoot{\rygdata{12:00--13:30}{07:31--09:01}{10:30--12:00}}
\caldata{SEPTEMBER}{26}{\sunmonth{कन्या}{10}{}{भाद्रपदः}{वर्षऋतुः}{गुरुः}{विकारी}{दक्षिणायनम्}{वर्षऋतुः}}
{\sunmoonsrdata{06:01}{17:59}{04:16*}{16:08}{12:00}
{\kalas{04:25 05:13 09:12 08:25 10:00 16:23 10:48 13:12 15:35 17:11 18:47 20:59 22:30 01:30*}}}
{\tnykdata{\anga{\tithi{27}{कृष्ण-द्वादशी}}{\time{12-36}{11:02}}\hspace{1ex}}%
{\anga{आश्रेषा}{\time{1-31}{06:38}}\hspace{1ex}\anga{मघा}{\time{54-53}{03:58*}}\hspace{1ex}\avamA{}}{चन्द्रराशिः—\mbox{कर्कटः\RIGHTarrow{06:38}}}%
{\anga{सिद्धः}{\time{7-0}{08:49}}\hspace{1ex}\anga{साध्यः}{\time{57-0}{04:49*}}\hspace{1ex}\uanga{शुभः}}%
{\anga{तैतिलम्}{\time{12-36}{11:02}}\hspace{1ex}\anga{गरजा}{\time{38-20}{21:20}}\hspace{1ex}\uanga{वणिजा}}{}
}
{अनध्यायः\eventsep द्वापरयुगादिः\eventsep काञ्ची ४४ जगद्गुरु श्री-पूर्णबोधेन्द्र सरस्वती २ आराधना~\#{९८०}\eventsep प्रदोष-व्रतम्~17:59\RIGHTarrow{}19:29}
{Thu} 
\cfoot{\rygdata{13:30--14:59}{06:01--07:31}{09:00--10:30}}
\caldata{SEPTEMBER}{27}{\sunmonth{कन्या}{11}{}{भाद्रपदः}{वर्षऋतुः}{शुक्रः}{विकारी}{दक्षिणायनम्}{वर्षऋतुः}}
{\sunmoonsrdata{06:01}{17:58}{05:16*}{16:57}{12:00}
{\kalas{04:25 05:13 09:12 08:24 10:00 16:22 10:48 13:11 15:35 17:10 18:46 20:59 22:29 01:30*}}}
{\tnykdata{\anga{\tithi{28}{कृष्ण-त्रयोदशी}}{\time{3-47}{07:32}}\hspace{1ex}\anga{\tithi{29}{कृष्ण-चतुर्दशी}}{\time{54-23}{03:46*}}\hspace{1ex}\avamA{}}%
{\prev{\anga{मघा}{\time{*54-53}{03:58}}}\hspace{1ex}\anga{पूर्वफल्गुनी}{\time{47-35}{01:02*}}\hspace{1ex}}{चन्द्रराशिः—\mbox{सिंहः}}%
{\prev{\anga{साध्यः}{\time{*57-0}{04:49}}}\hspace{1ex}\anga{शुभः}{\time{46-31}{00:36*}}\hspace{1ex}\uanga{शुक्लः}}%
{\anga{वणिजा}{\time{3-47}{07:32}}\hspace{1ex}\anga{भद्रा}{\time{29-14}{17:40}}\hspace{1ex}\anga{शकुनिः}{\time{54-23}{03:46*}}\hspace{1ex}\uanga{चतुष्पात्}}{}
}
{अनध्यायः\eventsep दिनक्षयः\eventsep कात्यायनी-जयन्ती\eventsep मासशिवरात्रिः\eventsep पञ्च-पर्व-पूजा (चतुर्दशी)\eventsep शस्त्रहतचतुर्दशी}
{Fri} 
\cfoot{\rygdata{10:30--12:00}{14:59--16:28}{07:31--09:00}}
\caldata{SEPTEMBER}{28}{\sunmonth{कन्या}{12}{}{भाद्रपदः}{वर्षऋतुः}{शनिः}{विकारी}{दक्षिणायनम्}{वर्षऋतुः}}
{\sunmoonsrdata{06:01}{17:57}{---}{17:45}{11:59}
{\kalas{04:25 05:13 09:12 08:24 10:00 16:22 10:48 13:11 15:34 17:10 18:46 20:58 22:29 01:30*}}}
{\tnykdata{\prev{\anga{\tithi{29}{कृष्ण-चतुर्दशी}}{\time{*54-23}{03:46}}}\hspace{1ex}\anga{\tithi{30}{अमावास्या}}{\time{44-51}{23:56}}\hspace{1ex}}%
{\anga{उत्तरफल्गुनी}{\time{40-3}{22:00}}\hspace{1ex}}{चन्द्रराशिः—\mbox{सिंहः\RIGHTarrow{06:17}}}%
{\anga{शुक्लः}{\time{35-50}{20:18}}\hspace{1ex}\uanga{ब्राह्मः}}%
{\prev{\anga{शकुनिः}{\time{*54-23}{03:46}}}\hspace{1ex}\anga{चतुष्पात्}{\time{19-40}{13:51}}\hspace{1ex}\anga{नाग}{\time{44-51}{23:56}}\hspace{1ex}\uanga{किंस्तुघ्नः}}{}
}
{अनध्यायः\eventsep अश्वशिरो-देव-पूजा\eventsep गजच्छाया-योगः~22:00\RIGHTarrow{}23:56\eventsep महालय-पक्ष-समापनम्\eventsep पार्वणव्रतम् अमावास्यायाम्\eventsep पञ्च-पर्व-पूजा (अमावास्या)\eventsep पिण्ड-पितृ-यज्ञः\eventsep \tamil{புரட்டாசி~சனிக்கிழமை}\eventsep सर्व-(भाद्रपद) महालय अमावास्या\eventsep सुजन्मप्राप्ति-व्रतम्\eventsep शृङ्गेरी ३४ जगद्गुरु श्री-चन्द्रशेखर भारती आराधना}
{Sat} 
\cfoot{\rygdata{09:00--10:30}{13:29--14:58}{06:01--07:31}}
\caldata{SEPTEMBER}{29}{\sunmonth{कन्या}{13}{}{आश्वयुजः}{शरदृतुः}{भानुः}{विकारी}{दक्षिणायनम्}{वर्षऋतुः}}
{\sunmoonrsdata{06:01}{17:57}{06:16}{18:33}{11:59}
{\kalas{04:25 05:13 09:12 08:24 10:00 16:21 10:47 13:10 15:33 17:09 18:45 20:58 22:28 01:29*}}}
{\tnykdata{\anga{\tithi{1}{शुक्ल-प्रथमा}}{\time{35-39}{20:13}}\hspace{1ex}}%
{\anga{हस्तः}{\time{32-47}{19:04}}\hspace{1ex}}{चन्द्रराशिः—\mbox{कन्या\RIGHTarrow{05:42*}}}%
{\anga{ब्राह्मः}{\time{25-17}{16:04}}\hspace{1ex}\uanga{माहेन्द्रः}}%
{\anga{किंस्तुघ्नः}{\time{10-8}{10:03}}\hspace{1ex}\anga{बवम्}{\time{35-39}{20:13}}\hspace{1ex}\uanga{बालवम्}}{}
}
{आदित्यहस्त-योगः\eventsep अनध्यायः\eventsep दर्शेष्टिः\eventsep दौहित्र-प्रतिपत्\eventsep द्विपुष्कर-योगः~20:13\RIGHTarrow{}06:01*\eventsep गृहदेवी-पूजा\eventsep महालय-पक्ष-तर्पण-पूर्तिः\eventsep पार्वण-प्रायश्चित्तावकाशः दर्शे\eventsep स्तन्यवृद्धि-गौरी-व्रतम्\eventsep दर्श-स्थालीपाकः\eventsep शरन्नवरात्र-आरम्भः}
{Sun} 
\cfoot{\rygdata{16:27--17:57}{11:59--13:28}{14:58--16:27}}
\caldata{SEPTEMBER}{30}{\sunmonth{कन्या}{14}{}{आश्वयुजः}{शरदृतुः}{सोमः}{विकारी}{दक्षिणायनम्}{वर्षऋतुः}}
{\sunmoonrsdata{06:01}{17:56}{07:16}{19:22}{11:58}
{\kalas{04:24 05:13 09:12 08:24 09:59 16:21 10:47 13:10 15:33 17:08 18:44 20:57 22:28 01:29*}}}
{\tnykdata{\anga{\tithi{2}{शुक्ल-द्वितीया}}{\time{27-11}{16:49}}\hspace{1ex}}%
{\anga{चित्रा}{\time{26-14}{16:26}}\hspace{1ex}}{चन्द्रराशिः—\mbox{तुला}}%
{\anga{माहेन्द्रः}{\time{15-13}{12:04}}\hspace{1ex}\uanga{वैधृतिः}}%
{\anga{बालवम्}{\time{1-8}{06:28}}\hspace{1ex}\anga{कौलवम्}{\time{27-11}{16:49}}\hspace{1ex}\anga{तैतिलम्}{\time{53-13}{03:18*}}\hspace{1ex}\uanga{गरजा}}{}
}
{अनध्यायः\eventsep चन्द्र-दर्शनम्~17:56\RIGHTarrow{}19:22\eventsep वैधृति-श्राद्धम्}
{Mon} 
\cfoot{\rygdata{07:31--09:00}{10:29--11:58}{13:28--14:57}}
\caldata{OCTOBER}{1}{\sunmonth{कन्या}{15}{}{आश्वयुजः}{शरदृतुः}{मङ्गलः}{विकारी}{दक्षिणायनम्}{वर्षऋतुः}}
{\sunmoonrsdata{06:01}{17:55}{08:17}{20:13}{11:58}
{\kalas{04:24 05:13 09:12 08:24 09:59 16:20 10:47 13:10 15:32 17:08 18:44 20:57 22:27 01:29*}}}
{\tnykdata{\anga{\tithi{3}{शुक्ल-तृतीया}}{\time{19-53}{13:55}}\hspace{1ex}}%
{\anga{स्वाती}{\time{20-52}{14:18}}\hspace{1ex}}{चन्द्रराशिः—\mbox{तुला}}%
{\anga{वैधृतिः}{\time{6-5}{08:26}}\hspace{1ex}\anga{विष्कम्भः}{\time{58-15}{05:19*}}\hspace{1ex}\uanga{प्रीतिः}}%
{\anga{गरजा}{\time{19-53}{13:55}}\hspace{1ex}\anga{वणिजा}{\time{46-47}{00:42*}}\hspace{1ex}\uanga{भद्रा}}{}
}
{मेघपालीय-तृतीया\eventsep सुखा~अङ्गारकी-चतुर्थी}
{Tue} 
\cfoot{\rygdata{14:57--16:26}{09:00--10:29}{11:58--13:27}}
\caldata{OCTOBER}{2}{\sunmonth{कन्या}{16}{}{आश्वयुजः}{शरदृतुः}{बुधः}{विकारी}{दक्षिणायनम्}{वर्षऋतुः}}
{\sunmoonrsdata{06:01}{17:55}{09:17}{21:05}{11:58}
{\kalas{04:24 05:13 09:11 08:24 09:59 16:19 10:46 13:09 15:32 17:07 18:43 20:56 22:27 01:29*}}}
{\tnykdata{\anga{\tithi{4}{शुक्ल-चतुर्थी}}{\time{14-13}{11:39}}\hspace{1ex}}%
{\anga{विशाखा}{\time{17-9}{12:49}}\hspace{1ex}}{चन्द्रराशिः—\mbox{तुला\RIGHTarrow{07:07}}}%
{\prev{\anga{विष्कम्भः}{\time{*58-15}{05:19}}}\hspace{1ex}\anga{प्रीतिः}{\time{52-1}{02:48*}}\hspace{1ex}\uanga{आयुष्मान्}}%
{\anga{भद्रा}{\time{14-13}{11:39}}\hspace{1ex}\anga{बवम्}{\time{42-9}{22:49}}\hspace{1ex}\uanga{बालवम्}}{}
}
{बुधानुराधा-योगः~12:49\RIGHTarrow{}\eventsep देवता-सुवासिनी-पूजा\eventsep ललिता-पञ्चमी\eventsep शुक्ल-चतुर्थी-व्रतम्}
{Wed} 
\cfoot{\rygdata{11:58--13:27}{07:30--09:00}{10:29--11:58}}
\caldata{OCTOBER}{3}{\sunmonth{कन्या}{17}{}{आश्वयुजः}{शरदृतुः}{गुरुः}{विकारी}{दक्षिणायनम्}{वर्षऋतुः}}
{\sunmoonrsdata{06:01}{17:54}{10:16}{21:58}{11:58}
{\kalas{04:24 05:13 09:11 08:24 09:59 16:19 10:46 13:09 15:31 17:06 18:42 20:56 22:27 01:28*}}}
{\tnykdata{\anga{\tithi{5}{शुक्ल-पञ्चमी}}{\time{10-32}{10:12}}\hspace{1ex}}%
{\anga{अनूराधा}{\time{15-24}{12:07}}\hspace{1ex}}{चन्द्रराशिः—\mbox{वृश्चिकः}}%
{\anga{आयुष्मान्}{\time{47-28}{00:58*}}\hspace{1ex}\uanga{सौभाग्यः}}%
{\anga{बालवम्}{\time{10-32}{10:12}}\hspace{1ex}\anga{कौलवम्}{\time{39-35}{21:47}}\hspace{1ex}\uanga{तैतिलम्}}{}
}
{आश्विन-नाग-पञ्चमी\eventsep षष्ठी-व्रतम्\eventsep उपाङ्ग-ललिता-व्रतम्\eventsep शान्ति-पञ्चमी-व्रतम्}
{Thu} 
\cfoot{\rygdata{13:27--14:56}{06:01--07:30}{08:59--10:28}}
\caldata{OCTOBER}{4}{\sunmonth{कन्या}{18}{}{आश्वयुजः}{शरदृतुः}{शुक्रः}{विकारी}{दक्षिणायनम्}{वर्षऋतुः}}
{\sunmoonrsdata{06:01}{17:53}{11:13}{22:52}{11:57}
{\kalas{04:24 05:13 09:11 08:24 09:59 16:18 10:46 13:08 15:31 17:06 18:42 20:55 22:26 01:28*}}}
{\tnykdata{\anga{\tithi{6}{शुक्ल-षष्ठी}}{\time{9-0}{09:35}}\hspace{1ex}}%
{\anga{ज्येष्ठा}{\time{15-47}{12:16}}\hspace{1ex}}{चन्द्रराशिः—\mbox{वृश्चिकः\RIGHTarrow{12:16}}}%
{\anga{सौभाग्यः}{\time{44-38}{23:48}}\hspace{1ex}\uanga{शोभनः}}%
{\anga{तैतिलम्}{\time{9-0}{09:35}}\hspace{1ex}\anga{गरजा}{\time{39-11}{21:36}}\hspace{1ex}\uanga{वणिजा}}{}
}
{काञ्ची ४५ जगद्गुरु श्री-परमशिवेन्द्र सरस्वती १ आराधना~\#{९५९}\eventsep सरस्वती-आवाहनम्}
{Fri} 
\cfoot{\rygdata{10:28--11:57}{14:55--16:24}{07:30--08:59}}
\caldata{OCTOBER}{5}{\sunmonth{कन्या}{19}{}{आश्वयुजः}{शरदृतुः}{शनिः}{विकारी}{दक्षिणायनम्}{वर्षऋतुः}}
{\sunmoonrsdata{06:01}{17:53}{12:07}{23:46}{11:57}
{\kalas{04:24 05:13 09:11 08:24 09:58 16:18 10:46 13:08 15:30 17:05 18:41 20:55 22:26 01:28*}}}
{\tnykdata{\anga{\tithi{7}{शुक्ल-सप्तमी}}{\time{9-39}{09:50}}\hspace{1ex}}%
{\anga{मूला}{\time{18-18}{13:15}}\hspace{1ex}}{चन्द्रराशिः—\mbox{धनुः}}%
{\anga{शोभनः}{\time{43-24}{23:18}}\hspace{1ex}\uanga{अतिगण्डः}}%
{\anga{वणिजा}{\time{9-39}{09:50}}\hspace{1ex}\anga{भद्रा}{\time{40-51}{22:17}}\hspace{1ex}\uanga{बवम्}}{}
}
{पत्रिका-प्रवेश-पूजा\eventsep \tamil{புரட்டாசி~சனிக்கிழமை}\eventsep शुभ-सप्तमी}
{Sat} 
\cfoot{\rygdata{08:59--10:28}{13:26--14:55}{06:01--07:30}}
\caldata{OCTOBER}{6}{\sunmonth{कन्या}{20}{}{आश्वयुजः}{शरदृतुः}{भानुः}{विकारी}{दक्षिणायनम्}{वर्षऋतुः}}
{\sunmoonrsdata{06:01}{17:52}{12:57}{00:38*}{11:57}
{\kalas{04:24 05:13 09:11 08:24 09:58 16:17 10:46 13:08 15:30 17:04 18:41 20:54 22:25 01:28*}}}
{\tnykdata{\anga{\tithi{8}{शुक्ल-अष्टमी}}{\time{12-21}{10:54}}\hspace{1ex}}%
{\anga{पूर्वाषाढा}{\time{22-45}{15:00}}\hspace{1ex}}{चन्द्रराशिः—\mbox{धनुः\RIGHTarrow{21:33}}}%
{\anga{अतिगण्डः}{\time{43-34}{23:22}}\hspace{1ex}\uanga{सुकर्म}}%
{\anga{बवम्}{\time{12-21}{10:54}}\hspace{1ex}\anga{बालवम्}{\time{44-22}{23:42}}\hspace{1ex}\uanga{कौलवम्}}{}
}
{अनध्यायः\eventsep अनध्यायः\eventsep भद्रकाळी-पूजा\eventsep दुर्गाष्टमी\eventsep काञ्ची १९ जगद्गुरु श्री-मार्तण्ड विद्याघनेन्द्र सरस्वती आराधना~\#{१६२२}\eventsep काल-त्रिरात्र-व्रतम्\eventsep मन्वादिः-(स्वायम्भुवः-[१])\eventsep सरस्वती-पूजा}
{Sun} 
\cfoot{\rygdata{16:23--17:52}{11:57--13:25}{14:54--16:23}}
\caldata{OCTOBER}{7}{\sunmonth{कन्या}{21}{}{आश्वयुजः}{शरदृतुः}{सोमः}{विकारी}{दक्षिणायनम्}{वर्षऋतुः}}
{\sunmoonrsdata{06:02}{17:51}{13:43}{01:28*}{11:56}
{\kalas{04:24 05:13 09:11 08:23 09:58 16:17 10:45 13:07 15:29 17:04 18:40 20:54 22:25 01:28*}}}
{\tnykdata{\anga{\tithi{9}{शुक्ल-नवमी}}{\time{16-44}{12:38}}\hspace{1ex}}%
{\anga{उत्तराषाढा}{\time{28-45}{17:22}}\hspace{1ex}}{चन्द्रराशिः—\mbox{मकरः}}%
{\anga{सुकर्म}{\time{44-49}{23:52}}\hspace{1ex}\uanga{धृतिः}}%
{\anga{कौलवम्}{\time{16-44}{12:38}}\hspace{1ex}\anga{तैतिलम्}{\time{49-17}{01:41*}}\hspace{1ex}\uanga{गरजा}}{}
}
{आयुध-पूजा\eventsep \tamil{ஏனாதிநாத நாயன்மார் (9) குருபூஜை}\eventsep अनध्यायः\eventsep भद्रकाळी-व्रतम्\eventsep बुद्ध-जयन्ती\eventsep महानवमी/सरस्वती-पूजा\eventsep सोमश्रावणी-योगः~17:22\RIGHTarrow{}\eventsep सरस्वती-विसर्जनम्\eventsep विजयदशमी\eventsep शमी-पूजा/अपराजिता-पूजा\eventsep शरन्नवरात्र-समापनम्}
{Mon} 
\cfoot{\rygdata{07:30--08:59}{10:28--11:56}{13:25--14:54}}
\caldata{OCTOBER}{8}{\sunmonth{कन्या}{22}{}{आश्वयुजः}{शरदृतुः}{मङ्गलः}{विकारी}{दक्षिणायनम्}{वर्षऋतुः}}
{\sunmoonrsdata{06:02}{17:51}{14:26}{02:17*}{11:56}
{\kalas{04:24 05:13 09:11 08:23 09:58 16:16 10:45 13:07 15:29 17:03 18:39 20:53 22:25 01:27*}}}
{\tnykdata{\anga{\tithi{10}{शुक्ल-दशमी}}{\time{22-21}{14:50}}\hspace{1ex}}%
{\anga{श्रवणः}{\time{35-40}{20:09}}\hspace{1ex}}{चन्द्रराशिः—\mbox{मकरः}}%
{\anga{धृतिः}{\time{46-46}{00:39*}}\hspace{1ex}\uanga{शूलः}}%
{\anga{गरजा}{\time{22-21}{14:50}}\hspace{1ex}\anga{वणिजा}{\time{55-7}{04:03*}}\hspace{1ex}\uanga{भद्रा}}{}
}
{दशहरा\eventsep दुर्गा-पूजा\eventsep गङ्गावतरणम्\eventsep कूष्माण्ड-दशमी\eventsep मध्वाचार्य-जयन्ती~\#{७८२}\eventsep युद्धदेवता-आराधना\eventsep श्रवण-व्रतम्}
{Tue} 
\cfoot{\rygdata{14:53--16:22}{08:59--10:27}{11:56--13:25}}
\caldata{OCTOBER}{9}{\sunmonth{कन्या}{23}{}{आश्वयुजः}{शरदृतुः}{बुधः}{विकारी}{दक्षिणायनम्}{वर्षऋतुः}}
{\sunmoonrsdata{06:02}{17:50}{15:06}{03:04*}{11:56}
{\kalas{04:24 05:13 09:10 08:23 09:58 16:16 10:45 13:07 15:28 17:03 18:39 20:53 22:24 01:27*}}}
{\tnykdata{\anga{\tithi{11}{शुक्ल-एकादशी}}{\time{28-39}{17:18}}\hspace{1ex}}%
{\anga{श्रविष्ठा}{\time{43-4}{23:09}}\hspace{1ex}}{चन्द्रराशिः—\mbox{मकरः\RIGHTarrow{09:38}}}%
{\anga{शूलः}{\time{49-5}{01:36*}}\hspace{1ex}\uanga{गण्डः}}%
{\prev{\anga{वणिजा}{\time{*55-7}{04:03}}}\hspace{1ex}\anga{भद्रा}{\time{28-39}{17:18}}\hspace{1ex}\uanga{बवम्}}{}
}
{सर्व-पापाङ्कुशा-एकादशी}
{Wed} 
\cfoot{\rygdata{11:56--13:24}{07:30--08:59}{10:27--11:56}}
\caldata{OCTOBER}{10}{\sunmonth{कन्या}{24}{}{आश्वयुजः}{शरदृतुः}{गुरुः}{विकारी}{दक्षिणायनम्}{वर्षऋतुः}}
{\sunmoonrsdata{06:02}{17:49}{15:43}{03:50*}{11:55}
{\kalas{04:24 05:13 09:10 08:23 09:58 16:15 10:45 13:06 15:28 17:02 18:38 20:52 22:24 01:27*}}}
{\tnykdata{\anga{\tithi{12}{शुक्ल-द्वादशी}}{\time{35-0}{19:52}}\hspace{1ex}}%
{\anga{शतभिषक्}{\time{50-32}{02:11*}}\hspace{1ex}}{चन्द्रराशिः—\mbox{कुम्भः}}%
{\anga{गण्डः}{\time{51-25}{02:33*}}\hspace{1ex}\uanga{वृद्धिः}}%
{\anga{बवम्}{\time{1-24}{06:35}}\hspace{1ex}\anga{बालवम्}{\time{35-0}{19:52}}\hspace{1ex}\uanga{कौलवम्}}{}
}
{अनध्यायः\eventsep द्विदल-व्रत-आरम्भः\eventsep \tamil{நரசிங்கமுனையரைய நாயன்மார் (41) குருபூஜை}\eventsep पयोव्रत-समापनम्\eventsep शक्रध्वजपातः}
{Thu} 
\cfoot{\rygdata{13:24--14:52}{06:02--07:30}{08:59--10:27}}
\caldata{OCTOBER}{11}{\sunmonth{कन्या}{25}{}{आश्वयुजः}{शरदृतुः}{शुक्रः}{विकारी}{दक्षिणायनम्}{वर्षऋतुः}}
{\sunmoonrsdata{06:02}{17:49}{16:20}{04:36*}{11:55}
{\kalas{04:24 05:13 09:10 08:23 09:57 16:14 10:44 13:06 15:27 17:02 18:38 20:52 22:24 01:27*}}}
{\tnykdata{\anga{\tithi{13}{शुक्ल-त्रयोदशी}}{\time{41-5}{22:20}}\hspace{1ex}}%
{\anga{पूर्वप्रोष्ठपदा}{\time{57-43}{05:06*}}\hspace{1ex}}{चन्द्रराशिः—\mbox{कुम्भः\RIGHTarrow{22:23}}}%
{\anga{वृद्धिः}{\time{53-33}{03:24*}}\hspace{1ex}\uanga{ध्रुवः}}%
{\anga{कौलवम्}{\time{7-50}{09:07}}\hspace{1ex}\anga{तैतिलम्}{\time{41-5}{22:20}}\hspace{1ex}\uanga{गरजा}}{}
}
{शुक्रवार-शुक्ल-प्रदोष-व्रतम्~17:49\RIGHTarrow{}19:20}
{Fri} 
\cfoot{\rygdata{10:27--11:55}{14:52--16:20}{07:30--08:58}}
\caldata{OCTOBER}{12}{\sunmonth{कन्या}{26}{}{आश्वयुजः}{शरदृतुः}{शनिः}{विकारी}{दक्षिणायनम्}{वर्षऋतुः}}
{\sunmoonrsdata{06:02}{17:48}{16:56}{05:22*}{11:55}
{\kalas{04:24 05:13 09:10 08:23 09:57 16:14 10:44 13:06 15:27 17:01 18:37 20:52 22:23 01:27*}}}
{\tnykdata{\anga{\tithi{14}{शुक्ल-चतुर्दशी}}{\time{46-41}{00:36*}}\hspace{1ex}}%
{\prev{\anga{पूर्वप्रोष्ठपदा}{\time{*57-43}{05:06}}}\hspace{1ex}\fullanga{उत्तरप्रोष्ठपदा}}{चन्द्रराशिः—\mbox{मीनः}}%
{\anga{ध्रुवः}{\time{55-17}{04:07*}}\hspace{1ex}\uanga{व्याघातः}}%
{\anga{गरजा}{\time{13-55}{11:30}}\hspace{1ex}\anga{वणिजा}{\time{46-41}{00:36*}}\hspace{1ex}\uanga{भद्रा}}{}
}
{अनध्यायः\eventsep \tamil{நடராஜர் மஹாபிஷேகம்}\eventsep \tamil{புரட்டாசி~சனிக்கிழமை}}
{Sat} 
\cfoot{\rygdata{08:58--10:27}{13:23--14:52}{06:02--07:30}}
\caldata{OCTOBER}{13}{\sunmonth{कन्या}{27}{}{आश्वयुजः}{शरदृतुः}{भानुः}{विकारी}{दक्षिणायनम्}{वर्षऋतुः}}
{\sunmoonrsdata{06:02}{17:48}{17:33}{---}{11:55}
{\kalas{04:24 05:13 09:10 08:23 09:57 16:13 10:44 13:05 15:26 17:00 18:37 20:51 22:23 01:27*}}}
{\tnykdata{\anga{\tithi{15}{पौर्णमासी}}{\time{51-38}{02:37*}}\hspace{1ex}}%
{\anga{उत्तरप्रोष्ठपदा}{\time{4-34}{07:50}}\hspace{1ex}}{चन्द्रराशिः—\mbox{मीनः}}%
{\prev{\anga{ध्रुवः}{\time{*55-17}{04:07}}}\hspace{1ex}\anga{व्याघातः}{\time{56-30}{04:37*}}\hspace{1ex}\uanga{हर्षणः}}%
{\anga{भद्रा}{\time{19-25}{13:39}}\hspace{1ex}\anga{बवम्}{\time{51-38}{02:37*}}\hspace{1ex}\uanga{बालवम्}}{}
}
{अनध्यायः\eventsep अपत्य-नीराजनम्\eventsep काञ्ची ३६ जगद्गुरु श्री-चित्सुखानन्देन्द्र सरस्वती आराधना~\#{१२६२}\eventsep को-जागर्ति-व्रतम्\eventsep कौमुदी-उत्सवः\eventsep कुमार-पूर्णिमा/महा-अश्विनी\eventsep कुन्ती-(पार्वती)-व्रतम्\eventsep लक्ष्मी-इन्द्र-कुबेर-पूजा\eventsep मीराबाई-जयन्ती~\#{५२२}\eventsep पार्वणव्रतम् पूर्णिमायाम्\eventsep पूर्णिमा-व्रतम्\eventsep पञ्च-पर्व-पूजा (पूर्णिमा)\eventsep वाल्मीकि-महर्षि-जयन्ती\eventsep वेङ्कटाचले पूर्णिमा-गरुड-सेवा\eventsep शरत्-पूर्णिमा/नवान्न-पूर्णिमा}
{Sun} 
\cfoot{\rygdata{16:19--17:48}{11:55--13:23}{14:51--16:19}}
\caldata{OCTOBER}{14}{\sunmonth{कन्या}{28}{}{आश्वयुजः}{शरदृतुः}{सोमः}{विकारी}{दक्षिणायनम्}{वर्षऋतुः}}
{\sunmoonsrdata{06:02}{17:47}{18:12}{06:09}{11:54}
{\kalas{04:24 05:13 09:10 08:23 09:57 16:13 10:44 13:05 15:26 17:00 18:36 20:51 22:23 01:26*}}}
{\tnykdata{\anga{\tithi{16}{कृष्ण-प्रथमा}}{\time{55-51}{04:21*}}\hspace{1ex}}%
{\anga{रेवती}{\time{10-51}{10:17}}\hspace{1ex}}{चन्द्रराशिः—\mbox{मीनः\RIGHTarrow{10:17}}}%
{\prev{\anga{व्याघातः}{\time{*56-30}{04:37}}}\hspace{1ex}\anga{हर्षणः}{\time{57-10}{04:53*}}\hspace{1ex}\uanga{वज्रम्}}%
{\anga{बालवम्}{\time{24-13}{15:31}}\hspace{1ex}\anga{कौलवम्}{\time{55-51}{04:21*}}\hspace{1ex}\uanga{तैतिलम्}}{}
}
{अनध्यायः\eventsep अप्पय्य-दीक्षित-जयन्ती~\#{५०१}\eventsep जयावाप्ति-व्रतम्\eventsep पार्वण-प्रायश्चित्तावकाशः पौर्णमास्याम्\eventsep पूर्णमासेष्टिः\eventsep पूर्ण-स्थालीपाकः}
{Mon} 
\cfoot{\rygdata{07:30--08:58}{10:26--11:54}{13:23--14:51}}
\caldata{OCTOBER}{15}{\sunmonth{कन्या}{29}{}{आश्वयुजः}{शरदृतुः}{मङ्गलः}{विकारी}{दक्षिणायनम्}{वर्षऋतुः}}
{\sunmoonsrdata{06:02}{17:46}{18:52}{06:57}{11:54}
{\kalas{04:24 05:13 09:10 08:23 09:57 16:12 10:44 13:05 15:25 16:59 18:35 20:50 22:22 01:26*}}}
{\tnykdata{\prev{\anga{\tithi{16}{कृष्ण-प्रथमा}}{\time{*55-51}{04:21}}}\hspace{1ex}\anga{\tithi{17}{कृष्ण-द्वितीया}}{\time{59-16}{05:45*}}\hspace{1ex}}%
{\anga{अश्विनी}{\time{16-24}{12:27}}\hspace{1ex}}{चन्द्रराशिः—\mbox{मेषः}}%
{\prev{\anga{हर्षणः}{\time{*57-10}{04:53}}}\hspace{1ex}\anga{वज्रम्}{\time{57-14}{04:55*}}\hspace{1ex}\uanga{सिद्धिः}}%
{\prev{\anga{कौलवम्}{\time{*55-51}{04:21}}}\hspace{1ex}\anga{तैतिलम्}{\time{28-15}{17:05}}\hspace{1ex}\anga{गरजा}{\time{59-16}{05:45*}}\hspace{1ex}\uanga{वणिजा}}{}
}
{अशून्यशयन-व्रतम्\eventsep भौमाश्विनी-योगः\RIGHTarrow{}12:27\eventsep \tamil{ருத்ர~பஶுபதி நாயன்மார் (17) குருபூஜை}}
{Tue} 
\cfoot{\rygdata{14:50--16:18}{08:58--10:26}{11:54--13:22}}
\caldata{OCTOBER}{16}{\sunmonth{कन्या}{30}{}{आश्वयुजः}{शरदृतुः}{बुधः}{विकारी}{दक्षिणायनम्}{वर्षऋतुः}}
{\sunmoonsrdata{06:02}{17:46}{19:36}{07:47}{11:54}
{\kalas{04:24 05:13 09:10 08:23 09:57 16:12 10:44 13:04 15:25 16:59 18:35 20:50 22:22 01:26*}}}
{\tnykdata{\prev{\anga{\tithi{17}{कृष्ण-द्वितीया}}{\time{*59-16}{05:45}}}\hspace{1ex}\fulltithi{\tithi{18}{कृष्ण-तृतीया}}}%
{\anga{अपभरणी}{\time{21-8}{14:18}}\hspace{1ex}}{चन्द्रराशिः—\mbox{मेषः\RIGHTarrow{20:43}}}%
{\prev{\anga{वज्रम्}{\time{*57-14}{04:55}}}\hspace{1ex}\anga{सिद्धिः}{\time{56-38}{04:40*}}\hspace{1ex}\uanga{व्यतीपातः}}%
{\prev{\anga{गरजा}{\time{*59-16}{05:45}}}\hspace{1ex}\anga{वणिजा}{\time{31-21}{18:19}}\hspace{1ex}\uanga{भद्रा}}{}
}
{चन्द्रोदय-गौरी-व्रतम्\eventsep कृत्तिका-व्रतम्\eventsep कनक-गणेश-व्रतम्\eventsep ललिता-गौरी-व्रतम्}
{Wed} 
\cfoot{\rygdata{11:54--13:22}{07:30--08:58}{10:26--11:54}}
\caldata{OCTOBER}{17}{\sunmonth{कन्या}{31}{\mbox{कन्या{\tiny\RIGHTarrow}{00:31*}}}{आश्वयुजः}{शरदृतुः}{गुरुः}{विकारी}{दक्षिणायनम्}{वर्षऋतुः}}
{\sunmoonsrdata{06:03}{17:45}{20:23}{08:39}{11:54}
{\kalas{04:24 05:13 09:10 08:23 09:57 16:12 10:44 13:04 15:25 16:58 18:34 20:50 22:22 01:26*}}}
{\tnykdata{\anga{\tithi{18}{कृष्ण-तृतीया}}{\time{1-56}{06:48}}\hspace{1ex}}%
{\anga{कृत्तिका}{\time{25-1}{15:49}}\hspace{1ex}}{चन्द्रराशिः—\mbox{वृषभः}}%
{\prev{\anga{सिद्धिः}{\time{*56-38}{04:40}}}\hspace{1ex}\anga{व्यतीपातः}{\time{55-21}{04:08*}}\hspace{1ex}\uanga{वरीयान्}}%
{\anga{भद्रा}{\time{1-56}{06:48}}\hspace{1ex}\anga{बवम्}{\time{33-29}{19:11}}\hspace{1ex}\uanga{बालवम्}}{}
}
{करक-चतुर्थी\eventsep वक्रतुण्ड-महागणपति-सङ्कटहर-चतुर्थी-व्रतम्\eventsep व्यतीपात-श्राद्धम्}
{Thu} 
\cfoot{\rygdata{13:22--14:50}{06:03--07:30}{08:58--10:26}}
\caldata{OCTOBER}{18}{\sunmonth{तुला}{1}{}{आश्वयुजः}{शरदृतुः}{शुक्रः}{विकारी}{दक्षिणायनम्}{शरदृतुः}}
{\sunmoonsrdata{06:03}{17:45}{21:14}{09:33}{11:54}
{\kalas{04:24 05:13 09:10 08:23 09:57 16:11 10:43 13:04 15:24 16:58 18:34 20:49 22:21 01:26*}}}
{\tnykdata{\anga{\tithi{19}{कृष्ण-चतुर्थी}}{\time{3-40}{07:28}}\hspace{1ex}}%
{\anga{रोहिणी}{\time{27-55}{16:56}}\hspace{1ex}}{चन्द्रराशिः—\mbox{वृषभः\RIGHTarrow{05:20*}}}%
{\prev{\anga{व्यतीपातः}{\time{*55-21}{04:08}}}\hspace{1ex}\anga{वरीयान्}{\time{53-16}{03:17*}}\hspace{1ex}\uanga{परिघः}}%
{\anga{बालवम्}{\time{3-40}{07:28}}\hspace{1ex}\anga{कौलवम्}{\time{34-39}{19:39}}\hspace{1ex}\uanga{तैतिलम्}}{}
}
{आकाशदीप-आरम्भः\eventsep भृगुवार-सुब्रह्मण्य-व्रतम्\eventsep रवि-सङ्क्रमण-पुण्यकालः~06:03\RIGHTarrow{}06:55\eventsep सङ्क्रमण-दिन-पूर्वाह्ण-पुण्यकालः~06:03\RIGHTarrow{}11:54\eventsep तुला-कावेरी-स्नान-आरम्भः\eventsep तुला-सङ्क्रमण-पुण्यकालः}
{Fri} 
\cfoot{\rygdata{10:26--11:54}{14:49--16:17}{07:30--08:58}}
\caldata{OCTOBER}{19}{\sunmonth{तुला}{2}{}{आश्वयुजः}{शरदृतुः}{शनिः}{विकारी}{दक्षिणायनम्}{शरदृतुः}}
{\sunmoonsrdata{06:03}{17:44}{22:08}{10:28}{11:53}
{\kalas{04:24 05:14 09:10 08:23 09:57 16:11 10:43 13:04 15:24 16:57 18:33 20:49 22:21 01:26*}}}
{\tnykdata{\anga{\tithi{20}{कृष्ण-पञ्चमी}}{\time{4-18}{07:43}}\hspace{1ex}}%
{\anga{मृगशीर्षम्}{\time{29-42}{17:37}}\hspace{1ex}}{चन्द्रराशिः—\mbox{मिथुनम्}}%
{\anga{परिघः}{\time{50-17}{02:04*}}\hspace{1ex}\uanga{शिवः}}%
{\anga{तैतिलम्}{\time{4-18}{07:43}}\hspace{1ex}\anga{गरजा}{\time{34-43}{19:40}}\hspace{1ex}\uanga{वणिजा}}{}
}
{घोटक-पञ्चमी\eventsep सेङ्गालिपुरम् अनन्तराम-दीक्षित-आराधना~\#{५०}}
{Sat} 
\cfoot{\rygdata{08:58--10:26}{13:21--14:49}{06:03--07:30}}
\caldata{OCTOBER}{20}{\sunmonth{तुला}{3}{}{आश्वयुजः}{शरदृतुः}{भानुः}{विकारी}{दक्षिणायनम्}{शरदृतुः}}
{\sunmoonsrdata{06:03}{17:44}{23:05}{11:23}{11:53}
{\kalas{04:24 05:14 09:10 08:23 09:57 16:10 10:43 13:03 15:23 16:57 18:33 20:48 22:21 01:26*}}}
{\tnykdata{\anga{\tithi{21}{कृष्ण-षष्ठी}}{\time{3-42}{07:30}}\hspace{1ex}}%
{\anga{आर्द्रा}{\time{30-14}{17:49}}\hspace{1ex}}{चन्द्रराशिः—\mbox{मिथुनम्}}%
{\anga{शिवः}{\time{46-19}{00:26*}}\hspace{1ex}\uanga{सिद्धः}}%
{\anga{वणिजा}{\time{3-42}{07:30}}\hspace{1ex}\anga{भद्रा}{\time{33-33}{19:11}}\hspace{1ex}\uanga{बवम्}}{}
}
{भानुसप्तमी\eventsep त्रिपुष्कर-योगः~17:49\RIGHTarrow{}06:03*}
{Sun} 
\cfoot{\rygdata{16:16--17:44}{11:53--13:21}{14:48--16:16}}
\caldata{OCTOBER}{21}{\sunmonth{तुला}{4}{}{आश्वयुजः}{शरदृतुः}{सोमः}{विकारी}{दक्षिणायनम्}{शरदृतुः}}
{\sunmoonsrdata{06:03}{17:43}{00:03*}{12:17}{11:53}
{\kalas{04:24 05:14 09:10 08:23 09:56 16:10 10:43 13:03 15:23 16:56 18:32 20:48 22:21 01:26*}}}
{\tnykdata{\anga{\tithi{22}{कृष्ण-सप्तमी}}{\time{1-45}{06:44}}\hspace{1ex}\anga{\tithi{23}{कृष्ण-अष्टमी}}{\time{58-27}{05:25*}}\hspace{1ex}\avamA{}}%
{\anga{पुनर्वसुः}{\time{29-24}{17:29}}\hspace{1ex}}{चन्द्रराशिः—\mbox{मिथुनम्\RIGHTarrow{11:37}}}%
{\anga{सिद्धः}{\time{41-16}{22:21}}\hspace{1ex}\uanga{साध्यः}}%
{\anga{बवम्}{\time{1-45}{06:44}}\hspace{1ex}\anga{बालवम्}{\time{31-2}{18:09}}\hspace{1ex}\anga{कौलवम्}{\time{58-27}{05:25*}}\hspace{1ex}\uanga{तैतिलम्}}{}
}
{अनध्यायः\eventsep दिनक्षयः\eventsep मङ्गल-व्रतम्\eventsep महालक्ष्मी-व्रतम्\eventsep पञ्च-पर्व-पूजा (अष्टमी)}
{Mon} 
\cfoot{\rygdata{07:31--08:58}{10:26--11:53}{13:21--14:48}}
\caldata{OCTOBER}{22}{\sunmonth{तुला}{5}{}{आश्वयुजः}{शरदृतुः}{मङ्गलः}{विकारी}{दक्षिणायनम्}{शरदृतुः}}
{\sunmoonsrdata{06:03}{17:43}{01:02*}{13:09}{11:53}
{\kalas{04:25 05:14 09:10 08:23 09:56 16:09 10:43 13:03 15:23 16:56 18:32 20:48 22:20 01:26*}}}
{\tnykdata{\prev{\anga{\tithi{23}{कृष्ण-अष्टमी}}{\time{*58-27}{05:25}}}\hspace{1ex}\anga{\tithi{24}{कृष्ण-नवमी}}{\time{53-53}{03:32*}}\hspace{1ex}}%
{\anga{पुष्यः}{\time{27-8}{16:36}}\hspace{1ex}}{चन्द्रराशिः—\mbox{कर्कटः}}%
{\anga{साध्यः}{\time{35-9}{19:50}}\hspace{1ex}\uanga{शुभः}}%
{\prev{\anga{कौलवम्}{\time{*58-27}{05:25}}}\hspace{1ex}\anga{तैतिलम्}{\time{27-0}{16:33}}\hspace{1ex}\anga{गरजा}{\time{53-53}{03:32*}}\hspace{1ex}\uanga{वणिजा}}{}
}
{भीमसेन-जयन्ती}
{Tue} 
\cfoot{\rygdata{14:48--16:15}{08:58--10:25}{11:53--13:20}}
\caldata{OCTOBER}{23}{\sunmonth{तुला}{6}{}{आश्वयुजः}{शरदृतुः}{बुधः}{विकारी}{दक्षिणायनम्}{शरदृतुः}}
{\sunmoonsrdata{06:03}{17:42}{02:01*}{13:58}{11:53}
{\kalas{04:25 05:14 09:10 08:23 09:56 16:09 10:43 13:03 15:22 16:55 18:31 20:47 22:20 01:26*}}}
{\tnykdata{\anga{\tithi{25}{कृष्ण-दशमी}}{\time{48-3}{01:09*}}\hspace{1ex}}%
{\anga{आश्रेषा}{\time{23-28}{15:10}}\hspace{1ex}}{चन्द्रराशिः—\mbox{कर्कटः\RIGHTarrow{15:10}}}%
{\anga{शुभः}{\time{27-52}{16:52}}\hspace{1ex}\uanga{शुक्लः}}%
{\anga{वणिजा}{\time{21-30}{14:24}}\hspace{1ex}\anga{भद्रा}{\time{48-3}{01:09*}}\hspace{1ex}\uanga{बवम्}}{}
}
{(सायन) विष्णुपदी-पुण्यकालः~16:25\RIGHTarrow{}17:42\eventsep ऊर्ज-मासः~22:49\RIGHTarrow{}\eventsep सायन-सङ्क्रमण-दिन-अपराह्ण-पुण्यकालः~11:53\RIGHTarrow{}17:42}
{Wed} 
\cfoot{\rygdata{11:53--13:20}{07:31--08:58}{10:25--11:53}}
\caldata{OCTOBER}{24}{\sunmonth{तुला}{7}{}{आश्वयुजः}{शरदृतुः}{गुरुः}{विकारी}{दक्षिणायनम्}{शरदृतुः}}
{\sunmoonsrdata{06:04}{17:42}{02:59*}{14:46}{11:53}
{\kalas{04:25 05:14 09:10 08:23 09:56 16:08 10:43 13:02 15:22 16:55 18:31 20:47 22:20 01:25*}}}
{\tnykdata{\anga{\tithi{26}{कृष्ण-एकादशी}}{\time{41-10}{22:18}}\hspace{1ex}}%
{\anga{मघा}{\time{18-33}{13:15}}\hspace{1ex}}{चन्द्रराशिः—\mbox{सिंहः}}%
{\anga{शुक्लः}{\time{19-14}{13:32}}\hspace{1ex}\uanga{ब्राह्मः}}%
{\anga{बवम्}{\time{14-43}{11:46}}\hspace{1ex}\anga{बालवम्}{\time{41-10}{22:18}}\hspace{1ex}\uanga{कौलवम्}}{}
}
{सर्व-रमा-एकादशी\eventsep ताम्रपर्णी-अन्त्य-पुष्कर-आरम्भः\eventsep शृङ्गेरी ३४ जगद्गुरु श्री-चन्द्रशेखर भारती-३ जयन्ती}
{Thu} 
\cfoot{\rygdata{13:20--14:47}{06:04--07:31}{08:58--10:25}}
\caldata{OCTOBER}{25}{\sunmonth{तुला}{8}{}{आश्वयुजः}{शरदृतुः}{शुक्रः}{विकारी}{दक्षिणायनम्}{शरदृतुः}}
{\sunmoonsrdata{06:04}{17:41}{03:58*}{15:33}{11:52}
{\kalas{04:25 05:14 09:10 08:23 09:56 16:08 10:43 13:02 15:22 16:55 18:31 20:47 22:20 01:25*}}}
{\tnykdata{\anga{\tithi{27}{कृष्ण-द्वादशी}}{\time{33-30}{19:08}}\hspace{1ex}}%
{\anga{पूर्वफल्गुनी}{\time{12-38}{10:58}}\hspace{1ex}}{चन्द्रराशिः—\mbox{सिंहः\RIGHTarrow{16:20}}}%
{\anga{ब्राह्मः}{\time{9-47}{09:51}}\hspace{1ex}\anga{माहेन्द्रः}{\time{59-47}{05:59*}}\hspace{1ex}\uanga{वैधृतिः}}%
{\anga{कौलवम्}{\time{6-56}{08:45}}\hspace{1ex}\anga{तैतिलम्}{\time{33-30}{19:08}}\hspace{1ex}\anga{गरजा}{\time{58-32}{05:28*}}\hspace{1ex}\uanga{वणिजा}}{}
}
{\tamil{சத்தி நாயன்மார் (45) குருபூஜை}\eventsep धन्वन्तरि-जयन्ती\eventsep गोवत्स-द्वादशी\eventsep कामाक्षी-आविर्भावः\eventsep प्रदोष-व्रतम्~17:41\RIGHTarrow{}19:14\eventsep वसुदेव-पूजा\eventsep व्याघ्र-द्वादशी}
{Fri} 
\cfoot{\rygdata{10:25--11:52}{14:47--16:14}{07:31--08:58}}
\caldata{OCTOBER}{26}{\sunmonth{तुला}{9}{}{आश्वयुजः}{शरदृतुः}{शनिः}{विकारी}{दक्षिणायनम्}{शरदृतुः}}
{\sunmoonsrdata{06:04}{17:41}{04:57*}{16:20}{11:52}
{\kalas{04:25 05:15 09:10 08:23 09:56 16:08 10:43 13:02 15:21 16:54 18:30 20:47 22:20 01:25*}}}
{\tnykdata{\anga{\tithi{28}{कृष्ण-त्रयोदशी}}{\time{25-4}{15:46}}\hspace{1ex}}%
{\anga{उत्तरफल्गुनी}{\time{6-3}{08:25}}\hspace{1ex}\anga{हस्तः}{\time{59-17}{05:47*}}\hspace{1ex}\avamA{}}{चन्द्रराशिः—\mbox{कन्या}}%
{\prev{\anga{माहेन्द्रः}{\time{*59-47}{05:59}}}\hspace{1ex}\anga{वैधृतिः}{\time{50-10}{02:01*}}\hspace{1ex}\uanga{विष्कम्भः}}%
{\prev{\anga{गरजा}{\time{*58-32}{05:28}}}\hspace{1ex}\anga{वणिजा}{\time{25-4}{15:46}}\hspace{1ex}\anga{भद्रा}{\time{50-18}{02:04*}}\hspace{1ex}\uanga{शकुनिः}}{}
}
{(यम)-दीप-त्रयोदशी\eventsep देवी-पर्व-७\eventsep धन-त्रयोदशी\eventsep गो-त्रिरात्र-व्रतम्\eventsep मासशिवरात्रिः\eventsep पञ्च-पर्व-पूजा (चतुर्दशी)\eventsep वैधृति-श्राद्धम्}
{Sat} 
\cfoot{\rygdata{08:58--10:25}{13:19--14:47}{06:04--07:31}}
\caldata{OCTOBER}{27}{\sunmonth{तुला}{10}{}{आश्वयुजः}{शरदृतुः}{भानुः}{विकारी}{दक्षिणायनम्}{शरदृतुः}}
{\sunmoonsrdata{06:04}{17:40}{05:57*}{17:08}{11:52}
{\kalas{04:25 05:15 09:10 08:24 09:56 16:07 10:43 13:02 15:21 16:54 18:30 20:46 22:19 01:25*}}}
{\tnykdata{\anga{\tithi{29}{कृष्ण-चतुर्दशी}}{\time{16-18}{12:23}}\hspace{1ex}}%
{\prev{\anga{हस्तः}{\time{*59-17}{05:47}}}\hspace{1ex}\anga{चित्रा}{\time{53-7}{03:14*}}\hspace{1ex}}{चन्द्रराशिः—\mbox{कन्या\RIGHTarrow{16:29}}}%
{\anga{विष्कम्भः}{\time{40-41}{22:06}}\hspace{1ex}\uanga{प्रीतिः}}%
{\anga{शकुनिः}{\time{16-18}{12:23}}\hspace{1ex}\anga{चतुष्पात्}{\time{42-13}{22:44}}\hspace{1ex}\uanga{नाग}}{}
}
{अनध्यायः\eventsep दीपावली/लक्ष्मी-कुबेर-पूजा\eventsep दीपोत्सव-चतुर्दशी/यम-तर्पणम्\eventsep नरक-चतुर्दशी-स्नानम्\eventsep पार्वणव्रतम् अमावास्यायाम्\eventsep पञ्च-पर्व-पूजा (अमावास्या)\eventsep प्रेत-चतुर्दशी\eventsep सर्व-आश्वयुज-अमावास्या\eventsep शृङ्गेरी ३५ जगद्गुरु श्री-अभिनव विद्यातीर्थ महास्वामी जयन्ती}
{Sun} 
\cfoot{\rygdata{16:13--17:40}{11:52--13:19}{14:46--16:13}}
\caldata{OCTOBER}{28}{\sunmonth{तुला}{11}{}{आश्वयुजः}{शरदृतुः}{सोमः}{विकारी}{दक्षिणायनम्}{शरदृतुः}}
{\sunmoonsrdata{06:05}{17:40}{---}{17:58}{11:52}
{\kalas{04:25 05:15 09:10 08:24 09:56 16:07 10:43 13:02 15:21 16:54 18:29 20:46 22:19 01:25*}}}
{\tnykdata{\anga{\tithi{30}{अमावास्या}}{\time{7-54}{09:08}}\hspace{1ex}}%
{\anga{स्वाती}{\time{47-37}{00:58*}}\hspace{1ex}}{चन्द्रराशिः—\mbox{तुला}}%
{\anga{प्रीतिः}{\time{31-40}{18:22}}\hspace{1ex}\uanga{आयुष्मान्}}%
{\anga{नाग}{\time{7-54}{09:08}}\hspace{1ex}\anga{किंस्तुघ्नः}{\time{34-44}{19:37}}\hspace{1ex}\uanga{बवम्}}{}
}
{आग्रयण-होमः द्राविडेषु\eventsep अनध्यायः\eventsep अनध्यायः\eventsep दर्शेष्टिः\eventsep गोवर्धन-पूजा\eventsep केदार-गौरी-व्रतम्\eventsep पार्वण-प्रायश्चित्तावकाशः दर्शे\eventsep पिण्ड-पितृ-यज्ञः\eventsep सोमवती अमावास्या\eventsep दर्श-स्थालीपाकः\eventsep विक्रमादित्य-पट्टाभिषेकः\eventsep श्रीराम-पट्टाभिषेकः}
{Mon} 
\cfoot{\rygdata{07:32--08:58}{10:25--11:52}{13:19--14:46}}
\caldata{OCTOBER}{29}{\sunmonth{तुला}{12}{}{कार्त्तिकः}{शरदृतुः}{मङ्गलः}{विकारी}{दक्षिणायनम्}{शरदृतुः}}
{\sunmoonrsdata{06:05}{17:39}{06:58}{18:50}{11:52}
{\kalas{04:25 05:15 09:10 08:24 09:56 16:07 10:43 13:02 15:21 16:53 18:29 20:46 22:19 01:25*}}}
{\tnykdata{\anga{\tithi{1}{शुक्ल-प्रथमा}}{\time{0-20}{06:13}}\hspace{1ex}\anga{\tithi{2}{शुक्ल-द्वितीया}}{\time{54-27}{03:47*}}\hspace{1ex}\avamA{}}%
{\anga{विशाखा}{\time{43-14}{23:09}}\hspace{1ex}}{चन्द्रराशिः—\mbox{तुला\RIGHTarrow{17:33}}}%
{\anga{आयुष्मान्}{\time{22-59}{14:57}}\hspace{1ex}\uanga{सौभाग्यः}}%
{\anga{बवम्}{\time{0-20}{06:13}}\hspace{1ex}\anga{बालवम्}{\time{28-6}{16:56}}\hspace{1ex}\anga{कौलवम्}{\time{54-27}{03:47*}}\hspace{1ex}\uanga{तैतिलम्}}{}
}
{चन्द्र-दर्शनम्~17:39\RIGHTarrow{}18:50\eventsep दिनक्षयः\eventsep त्रिपुष्कर-योगः~06:13\RIGHTarrow{}23:09\eventsep यम/भ्रातृ-द्वितीया}
{Tue} 
\cfoot{\rygdata{14:46--16:13}{08:58--10:25}{11:52--13:19}}
\caldata{OCTOBER}{30}{\sunmonth{तुला}{13}{}{कार्त्तिकः}{शरदृतुः}{बुधः}{विकारी}{दक्षिणायनम्}{शरदृतुः}}
{\sunmoonrsdata{06:05}{17:39}{07:58}{19:44}{11:52}
{\kalas{04:26 05:15 09:10 08:24 09:56 16:06 10:43 13:01 15:20 16:53 18:29 20:46 22:19 01:25*}}}
{\tnykdata{\prev{\anga{\tithi{2}{शुक्ल-द्वितीया}}{\time{*54-27}{03:47}}}\hspace{1ex}\anga{\tithi{3}{शुक्ल-तृतीया}}{\time{50-10}{02:01*}}\hspace{1ex}}%
{\anga{अनूराधा}{\time{40-20}{21:56}}\hspace{1ex}}{चन्द्रराशिः—\mbox{वृश्चिकः}}%
{\anga{सौभाग्यः}{\time{15-21}{12:00}}\hspace{1ex}\uanga{शोभनः}}%
{\prev{\anga{कौलवम्}{\time{*54-27}{03:47}}}\hspace{1ex}\anga{तैतिलम्}{\time{22-38}{14:49}}\hspace{1ex}\anga{गरजा}{\time{50-10}{02:01*}}\hspace{1ex}\uanga{वणिजा}}{}
}
{बुधानुराधा-योगः\eventsep \tamil{பூசலார் நாயன்மார் (58) குருபூஜை}}
{Wed} 
\cfoot{\rygdata{11:52--13:19}{07:32--08:58}{10:25--11:52}}
\caldata{OCTOBER}{31}{\sunmonth{तुला}{14}{}{कार्त्तिकः}{शरदृतुः}{गुरुः}{विकारी}{दक्षिणायनम्}{शरदृतुः}}
{\sunmoonrsdata{06:05}{17:39}{08:58}{20:39}{11:52}
{\kalas{04:26 05:16 09:10 08:24 09:57 16:06 10:43 13:01 15:20 16:52 18:28 20:45 22:19 01:25*}}}
{\tnykdata{\anga{\tithi{4}{शुक्ल-चतुर्थी}}{\time{47-45}{01:01*}}\hspace{1ex}}%
{\anga{ज्येष्ठा}{\time{39-13}{21:28}}\hspace{1ex}}{चन्द्रराशिः—\mbox{वृश्चिकः\RIGHTarrow{21:28}}}%
{\anga{शोभनः}{\time{9-9}{09:37}}\hspace{1ex}\uanga{अतिगण्डः}}%
{\anga{वणिजा}{\time{19-1}{13:25}}\hspace{1ex}\anga{भद्रा}{\time{47-45}{01:01*}}\hspace{1ex}\uanga{बवम्}}{}
}
{शुक्ल-चतुर्थी-व्रतम्}
{Thu} 
\cfoot{\rygdata{13:19--14:45}{06:05--07:32}{08:59--10:25}}
\caldata{NOVEMBER}{1}{\sunmonth{तुला}{15}{}{कार्त्तिकः}{शरदृतुः}{शुक्रः}{विकारी}{दक्षिणायनम्}{शरदृतुः}}
{\sunmoonrsdata{06:06}{17:38}{09:55}{21:35}{11:52}
{\kalas{04:26 05:16 09:10 08:24 09:57 16:06 10:43 13:01 15:20 16:52 18:28 20:45 22:19 01:26*}}}
{\tnykdata{\anga{\tithi{5}{शुक्ल-पञ्चमी}}{\time{47-21}{00:51*}}\hspace{1ex}}%
{\anga{मूला}{\time{40-3}{21:49}}\hspace{1ex}}{चन्द्रराशिः—\mbox{धनुः}}%
{\anga{अतिगण्डः}{\time{4-35}{07:52}}\hspace{1ex}\uanga{सुकर्म}}%
{\anga{बवम्}{\time{17-29}{12:49}}\hspace{1ex}\anga{बालवम्}{\time{47-21}{00:51*}}\hspace{1ex}\uanga{कौलवम्}}{}
}
{\tamil{ஐயடிகள் காடவர்கோன் நாயன்மார் (46) குருபூஜை}\eventsep देवसेना-पञ्चमी\eventsep जया-व्रतम्\eventsep पाण्डव-(लाभ)-पञ्चमी\eventsep सर्प-पूजा}
{Fri} 
\cfoot{\rygdata{10:25--11:52}{14:45--16:12}{07:32--08:59}}
\caldata{NOVEMBER}{2}{\sunmonth{तुला}{16}{}{कार्त्तिकः}{शरदृतुः}{शनिः}{विकारी}{दक्षिणायनम्}{शरदृतुः}}
{\sunmoonrsdata{06:06}{17:38}{10:48}{22:29}{11:52}
{\kalas{04:26 05:16 09:10 08:24 09:57 16:06 10:43 13:01 15:20 16:52 18:28 20:45 22:19 01:26*}}}
{\tnykdata{\anga{\tithi{6}{शुक्ल-षष्ठी}}{\time{48-56}{01:31*}}\hspace{1ex}}%
{\anga{पूर्वाषाढा}{\time{42-50}{22:58}}\hspace{1ex}}{चन्द्रराशिः—\mbox{धनुः\RIGHTarrow{05:23*}}}%
{\anga{सुकर्म}{\time{1-43}{06:46}}\hspace{1ex}\uanga{धृतिः}}%
{\anga{कौलवम्}{\time{18-8}{13:05}}\hspace{1ex}\anga{तैतिलम्}{\time{48-56}{01:31*}}\hspace{1ex}\uanga{गरजा}}{}
}
{स्कन्दषष्ठी-व्रतम्\eventsep त्रिपुष्कर-योगः~01:31*\RIGHTarrow{}06:06*}
{Sat} 
\cfoot{\rygdata{08:59--10:25}{13:18--14:45}{06:06--07:32}}
\caldata{NOVEMBER}{3}{\sunmonth{तुला}{17}{}{कार्त्तिकः}{शरदृतुः}{भानुः}{विकारी}{दक्षिणायनम्}{शरदृतुः}}
{\sunmoonrsdata{06:06}{17:38}{11:37}{23:21}{11:52}
{\kalas{04:26 05:16 09:11 08:25 09:57 16:05 10:43 13:01 15:19 16:52 18:28 20:45 22:19 01:26*}}}
{\tnykdata{\anga{\tithi{7}{शुक्ल-सप्तमी}}{\time{52-21}{02:56*}}\hspace{1ex}}%
{\anga{उत्तराषाढा}{\time{47-23}{00:52*}}\hspace{1ex}}{चन्द्रराशिः—\mbox{मकरः}}%
{\anga{धृतिः}{\time{0-30}{06:18}}\hspace{1ex}\uanga{शूलः}}%
{\anga{गरजा}{\time{20-54}{14:08}}\hspace{1ex}\anga{वणिजा}{\time{52-21}{02:56*}}\hspace{1ex}\uanga{भद्रा}}{}
}
{त्रिपुष्कर-योगः~06:06\RIGHTarrow{}00:52*\eventsep विजया-भानुसप्तमी}
{Sun} 
\cfoot{\rygdata{16:11--17:38}{11:52--13:18}{14:45--16:11}}
\caldata{NOVEMBER}{4}{\sunmonth{तुला}{18}{}{कार्त्तिकः}{शरदृतुः}{सोमः}{विकारी}{दक्षिणायनम्}{शरदृतुः}}
{\sunmoonrsdata{06:07}{17:37}{12:22}{00:12*}{11:52}
{\kalas{04:27 05:17 09:11 08:25 09:57 16:05 10:43 13:01 15:19 16:51 18:27 20:45 22:18 01:26*}}}
{\tnykdata{\anga{\tithi{8}{शुक्ल-अष्टमी}}{\time{57-11}{04:57*}}\hspace{1ex}}%
{\anga{श्रवणः}{\time{53-19}{03:20*}}\hspace{1ex}}{चन्द्रराशिः—\mbox{मकरः}}%
{\anga{शूलः}{\time{0-43}{06:23}}\hspace{1ex}\uanga{गण्डः}}%
{\anga{भद्रा}{\time{25-26}{15:52}}\hspace{1ex}\anga{बवम्}{\time{57-11}{04:57*}}\hspace{1ex}\uanga{बालवम्}}{}
}
{अनध्यायः\eventsep गोपाष्टमी\eventsep कार्तवीर्यार्जुन-जयन्ती\eventsep कार्त्तिक-सोमवासरः\eventsep \tamil{பொய்கையாழ்வார் திருநக்ஷத்திரம்}\eventsep पुष्कर-मेला-प्रारम्भः\eventsep सोमश्रावणी-योगः\eventsep ताम्रपर्णी-अन्त्य-पुष्कर-समापनम्\eventsep श्रवण-व्रतम्}
{Mon} 
\cfoot{\rygdata{07:33--08:59}{10:26--11:52}{13:18--14:45}}
\caldata{NOVEMBER}{5}{\sunmonth{तुला}{19}{}{कार्त्तिकः}{शरदृतुः}{मङ्गलः}{विकारी}{दक्षिणायनम्}{शरदृतुः}}
{\sunmoonrsdata{06:07}{17:37}{13:03}{01:00*}{11:52}
{\kalas{04:27 05:17 09:11 08:25 09:57 16:05 10:43 13:01 15:19 16:51 18:27 20:45 22:18 01:26*}}}
{\tnykdata{\prev{\anga{\tithi{8}{शुक्ल-अष्टमी}}{\time{*57-11}{04:57}}}\hspace{1ex}\fulltithi{\tithi{9}{शुक्ल-नवमी}}}%
{\fullanga{श्रविष्ठा}}{चन्द्रराशिः—\mbox{मकरः\RIGHTarrow{16:44}}}%
{\anga{गण्डः}{\time{2-4}{06:55}}\hspace{1ex}\uanga{वृद्धिः}}%
{\prev{\anga{बवम्}{\time{*57-11}{04:57}}}\hspace{1ex}\anga{बालवम्}{\time{31-11}{18:07}}\hspace{1ex}\uanga{कौलवम्}}{}
}
{अक्षया-नवमी\eventsep अनध्यायः\eventsep \tamil{பூதத்தாழ்வார் திருநக்ஷத்திரம்}\eventsep देवी-पर्व-८\eventsep गुरु-सङ्क्रान्तिः~(वृश्चिकः\To{}धनुः)~02:41\RIGHTarrow{}\eventsep जगद्धात्री-पूजा\eventsep काञ्ची २२ जगद्गुरु श्री-परिपूर्णबोधेन्द्र सरस्वती आराधना~\#{१५३९}\eventsep कूष्माण्ड-नवमी\eventsep सिन्धु-आद्य-पुष्कर-आरम्भः\eventsep त्रेतायुगादिः\eventsep तुलसी-विवाहोत्सव-आरम्भः}
{Tue} 
\cfoot{\rygdata{14:45--16:11}{08:59--10:26}{11:52--13:18}}
\caldata{NOVEMBER}{6}{\sunmonth{तुला}{20}{}{कार्त्तिकः}{शरदृतुः}{बुधः}{विकारी}{दक्षिणायनम्}{शरदृतुः}}
{\sunmoonrsdata{06:07}{17:37}{13:42}{01:46*}{11:52}
{\kalas{04:27 05:17 09:11 08:25 09:57 16:05 10:43 13:01 15:19 16:51 18:27 20:45 22:18 01:26*}}}
{\tnykdata{\anga{\tithi{9}{शुक्ल-नवमी}}{\time{3-13}{07:21}}\hspace{1ex}}%
{\anga{श्रविष्ठा}{\time{0-11}{06:12}}\hspace{1ex}}{चन्द्रराशिः—\mbox{कुम्भः}}%
{\anga{वृद्धिः}{\time{4-8}{07:42}}\hspace{1ex}\uanga{ध्रुवः}}%
{\anga{कौलवम्}{\time{3-13}{07:21}}\hspace{1ex}\anga{तैतिलम्}{\time{37-13}{20:38}}\hspace{1ex}\uanga{गरजा}}{}
}
{}
{Wed} 
\cfoot{\rygdata{11:52--13:18}{07:33--09:00}{10:26--11:52}}
\caldata{NOVEMBER}{7}{\sunmonth{तुला}{21}{}{कार्त्तिकः}{शरदृतुः}{गुरुः}{विकारी}{दक्षिणायनम्}{शरदृतुः}}
{\sunmoonrsdata{06:08}{17:37}{14:19}{02:32*}{11:52}
{\kalas{04:27 05:17 09:11 08:25 09:57 16:05 10:43 13:01 15:19 16:51 18:27 20:44 22:18 01:26*}}}
{\tnykdata{\anga{\tithi{10}{शुक्ल-दशमी}}{\time{9-53}{09:55}}\hspace{1ex}}%
{\anga{शतभिषक्}{\time{8-2}{09:12}}\hspace{1ex}}{चन्द्रराशिः—\mbox{कुम्भः\RIGHTarrow{05:26*}}}%
{\anga{ध्रुवः}{\time{6-28}{08:36}}\hspace{1ex}\uanga{व्याघातः}}%
{\anga{गरजा}{\time{9-53}{09:55}}\hspace{1ex}\anga{वणिजा}{\time{43-20}{23:11}}\hspace{1ex}\uanga{भद्रा}}{}
}
{कंस-वधः\eventsep \tamil{பேயாழ்வார் திருநக்ஷத்திரம்}\eventsep राजराज-चोऴ-दानोत्सवः~\#{१०३५}}
{Thu} 
\cfoot{\rygdata{13:18--14:44}{06:08--07:34}{09:00--10:26}}
\caldata{NOVEMBER}{8}{\sunmonth{तुला}{22}{}{कार्त्तिकः}{शरदृतुः}{शुक्रः}{विकारी}{दक्षिणायनम्}{शरदृतुः}}
{\sunmoonrsdata{06:08}{17:36}{14:55}{03:17*}{11:52}
{\kalas{04:28 05:18 09:11 08:26 09:57 16:05 10:43 13:01 15:19 16:50 18:26 20:44 22:18 01:26*}}}
{\tnykdata{\anga{\tithi{11}{शुक्ल-एकादशी}}{\time{16-23}{12:24}}\hspace{1ex}}%
{\anga{पूर्वप्रोष्ठपदा}{\time{15-44}{12:09}}\hspace{1ex}}{चन्द्रराशिः—\mbox{मीनः}}%
{\anga{व्याघातः}{\time{8-42}{09:28}}\hspace{1ex}\uanga{हर्षणः}}%
{\anga{भद्रा}{\time{16-23}{12:24}}\hspace{1ex}\anga{बवम्}{\time{49-3}{01:34*}}\hspace{1ex}\uanga{बालवम्}}{}
}
{आदि-शङ्कर मानसिक-सन्न्यास-दिनम्\eventsep अनध्यायः\eventsep भीष्म-पञ्चक-व्रत-आरम्भः\eventsep मन्वादिः-(स्वारोचिषः-[२])\eventsep सर्व-उत्थान-एकादशी\eventsep तुलसी-विवाहः}
{Fri} 
\cfoot{\rygdata{10:26--11:52}{14:44--16:10}{07:34--09:00}}
\caldata{NOVEMBER}{9}{\sunmonth{तुला}{23}{}{कार्त्तिकः}{शरदृतुः}{शनिः}{विकारी}{दक्षिणायनम्}{शरदृतुः}}
{\sunmoonrsdata{06:08}{17:36}{15:32}{04:04*}{11:52}
{\kalas{04:28 05:18 09:12 08:26 09:58 16:04 10:43 13:01 15:19 16:50 18:26 20:44 22:18 01:26*}}}
{\tnykdata{\anga{\tithi{12}{शुक्ल-द्वादशी}}{\time{22-17}{14:39}}\hspace{1ex}}%
{\anga{उत्तरप्रोष्ठपदा}{\time{22-52}{14:53}}\hspace{1ex}}{चन्द्रराशिः—\mbox{मीनः}}%
{\anga{हर्षणः}{\time{10-30}{10:09}}\hspace{1ex}\uanga{वज्रम्}}%
{\anga{बालवम्}{\time{22-17}{14:39}}\hspace{1ex}\anga{कौलवम्}{\time{54-2}{03:39*}}\hspace{1ex}\uanga{तैतिलम्}}{}
}
{अनध्यायः\eventsep बृन्दावन-द्वादशी\eventsep चातुर्मास्यव्रत-समापनम्\eventsep द्विदल-व्रत-समापनम्\eventsep गोपद्म-व्रत-समापनम्\eventsep प्रबोधोत्सवः\eventsep याज्ञवल्क्य-जयन्ती\eventsep शनिवार-शुक्ल-प्रदोष-व्रतम्~17:36\RIGHTarrow{}19:10}
{Sat} 
\cfoot{\rygdata{09:00--10:26}{13:18--14:44}{06:08--07:34}}
\caldata{NOVEMBER}{10}{\sunmonth{तुला}{24}{}{कार्त्तिकः}{शरदृतुः}{भानुः}{विकारी}{दक्षिणायनम्}{शरदृतुः}}
{\sunmoonrsdata{06:09}{17:36}{16:10}{04:52*}{11:52}
{\kalas{04:28 05:18 09:12 08:26 09:58 16:04 10:43 13:01 15:18 16:50 18:26 20:44 22:18 01:27*}}}
{\tnykdata{\anga{\tithi{13}{शुक्ल-त्रयोदशी}}{\time{27-14}{16:33}}\hspace{1ex}}%
{\anga{रेवती}{\time{29-6}{17:15}}\hspace{1ex}}{चन्द्रराशिः—\mbox{मीनः\RIGHTarrow{17:15}}}%
{\anga{वज्रम्}{\time{11-38}{10:35}}\hspace{1ex}\uanga{सिद्धिः}}%
{\prev{\anga{कौलवम्}{\time{*54-2}{03:39}}}\hspace{1ex}\anga{तैतिलम्}{\time{27-14}{16:33}}\hspace{1ex}\anga{गरजा}{\time{58-3}{05:20*}}\hspace{1ex}\uanga{वणिजा}}{}
}
{कार्त्तिक-मास-अन्तिमत्रयतिथि-व्रत-आरम्भः}
{Sun} 
\cfoot{\rygdata{16:10--17:36}{11:52--13:18}{14:44--16:10}}
\caldata{NOVEMBER}{11}{\sunmonth{तुला}{25}{}{कार्त्तिकः}{शरदृतुः}{सोमः}{विकारी}{दक्षिणायनम्}{शरदृतुः}}
{\sunmoonrsdata{06:09}{17:36}{16:50}{05:42*}{11:52}
{\kalas{04:29 05:19 09:12 08:26 09:58 16:04 10:44 13:01 15:18 16:50 18:26 20:44 22:18 01:27*}}}
{\tnykdata{\anga{\tithi{14}{शुक्ल-चतुर्दशी}}{\time{31-1}{18:01}}\hspace{1ex}}%
{\anga{अश्विनी}{\time{33-55}{19:14}}\hspace{1ex}}{चन्द्रराशिः—\mbox{मेषः}}%
{\anga{सिद्धिः}{\time{11-57}{10:43}}\hspace{1ex}\uanga{व्यतीपातः}}%
{\prev{\anga{गरजा}{\time{*58-3}{05:20}}}\hspace{1ex}\anga{वणिजा}{\time{31-1}{18:01}}\hspace{1ex}\uanga{भद्रा}}{}
}
{अनध्यायः\eventsep कार्त्तिक-सोमवासरः\eventsep मणिकर्णिका-स्नानम्/वैकुण्ठ-चतुर्दशी\eventsep पञ्च-पर्व-पूजा (पूर्णिमा)\eventsep \tamil{திருமூல நாயன்மார் (30) குருபூஜை}\eventsep त्रिपुरोत्सवः\eventsep व्यतीपात-श्राद्धम्}
{Mon} 
\cfoot{\rygdata{07:35--09:01}{10:26--11:52}{13:18--14:44}}
\caldata{NOVEMBER}{12}{\sunmonth{तुला}{26}{}{कार्त्तिकः}{शरदृतुः}{मङ्गलः}{विकारी}{दक्षिणायनम्}{शरदृतुः}}
{\sunmoonrsdata{06:09}{17:36}{17:33}{---}{11:53}
{\kalas{04:29 05:19 09:12 08:27 09:58 16:04 10:44 13:01 15:18 16:50 18:26 20:44 22:18 01:27*}}}
{\tnykdata{\anga{\tithi{15}{पौर्णमासी}}{\time{33-30}{19:04}}\hspace{1ex}}%
{\anga{अपभरणी}{\time{37-39}{20:48}}\hspace{1ex}}{चन्द्रराशिः—\mbox{मेषः\RIGHTarrow{03:08*}}}%
{\anga{व्यतीपातः}{\time{11-25}{10:31}}\hspace{1ex}\uanga{वरीयान्}}%
{\anga{भद्रा}{\time{1-9}{06:36}}\hspace{1ex}\anga{बवम्}{\time{33-30}{19:04}}\hspace{1ex}\uanga{बालवम्}}{}
}
{आग्रयण-होमः द्राविडेषु\eventsep अनध्यायः\eventsep अनध्यायः\eventsep भीष्म-पञ्चक-व्रत-समापनम्\eventsep कार्त्तिक-मास-अन्तिमत्रयतिथि-व्रत-समापनम्\eventsep कार्त्तिक-पूर्णिमा-स्नानम्\eventsep महा-अन्नाभिषेकः\eventsep मन्वादिः-(धर्म-सावर्णिः-[११])\eventsep \tamil{நின்றசீர் நெடுமாற நாயன்மார் (49) குருபூஜை}\eventsep पार्वणव्रतम् पूर्णिमायाम्\eventsep पूर्णिमा-व्रतम्\eventsep पद्मक-योगः\eventsep पुष्कर-मेला-समाप्तिः\eventsep तिरुवनन्तपुर-देवायतन-प्रवेश-घोषणा~\#{८४}\eventsep वेङ्कटाचले पूर्णिमा-गरुड-सेवा\eventsep श्री-गोविन्द भगवत्पाद आराधना}
{Tue} 
\cfoot{\rygdata{14:44--16:10}{09:01--10:27}{11:52--13:18}}
\caldata{NOVEMBER}{13}{\sunmonth{तुला}{27}{}{कार्त्तिकः}{शरदृतुः}{बुधः}{विकारी}{दक्षिणायनम्}{शरदृतुः}}
{\sunmoonsrdata{06:10}{17:35}{18:19}{06:34}{11:53}
{\kalas{04:29 05:19 09:13 08:27 09:58 16:04 10:44 13:01 15:18 16:50 18:26 20:44 22:18 01:27*}}}
{\tnykdata{\anga{\tithi{16}{कृष्ण-प्रथमा}}{\time{34-59}{19:41}}\hspace{1ex}}%
{\anga{कृत्तिका}{\time{40-25}{21:58}}\hspace{1ex}}{चन्द्रराशिः—\mbox{वृषभः}}%
{\anga{वरीयान्}{\time{10-1}{09:59}}\hspace{1ex}\uanga{परिघः}}%
{\anga{बालवम्}{\time{3-19}{07:26}}\hspace{1ex}\anga{कौलवम्}{\time{34-59}{19:41}}\hspace{1ex}\uanga{तैतिलम्}}{}
}
{अनध्यायः\eventsep अनध्यायः\eventsep \tamil{இடங்கழி நாயன்மார் (55) குருபூஜை}\eventsep काञ्ची ६४ जगद्गुरु श्री-चन्द्रशेखरेन्द्र सरस्वती ५ आराधना~\#{१६९}\eventsep कृत्तिका-व्रतम्\eventsep पार्वण-प्रायश्चित्तावकाशः पौर्णमास्याम्\eventsep पूर्णमासेष्टिः\eventsep पूर्ण-स्थालीपाकः}
{Wed} 
\cfoot{\rygdata{11:53--13:18}{07:35--09:01}{10:27--11:53}}
\caldata{NOVEMBER}{14}{\sunmonth{तुला}{28}{}{कार्त्तिकः}{शरदृतुः}{गुरुः}{विकारी}{दक्षिणायनम्}{शरदृतुः}}
{\sunmoonsrdata{06:10}{17:35}{19:10}{07:28}{11:53}
{\kalas{04:30 05:20 09:13 08:27 09:59 16:04 10:44 13:01 15:18 16:50 18:26 20:44 22:19 01:27*}}}
{\tnykdata{\anga{\tithi{17}{कृष्ण-द्वितीया}}{\time{35-31}{19:55}}\hspace{1ex}}%
{\anga{रोहिणी}{\time{42-16}{22:44}}\hspace{1ex}}{चन्द्रराशिः—\mbox{वृषभः}}%
{\anga{परिघः}{\time{7-46}{09:08}}\hspace{1ex}\uanga{शिवः}}%
{\anga{तैतिलम्}{\time{4-24}{07:51}}\hspace{1ex}\anga{गरजा}{\time{35-31}{19:55}}\hspace{1ex}\uanga{वणिजा}}{}
}
{अनध्यायः\eventsep अशून्यशयन-व्रतम्\eventsep चातुर्मास्य-द्वितीया}
{Thu} 
\cfoot{\rygdata{13:18--14:44}{06:10--07:36}{09:02--10:27}}
\caldata{NOVEMBER}{15}{\sunmonth{तुला}{29}{}{कार्त्तिकः}{शरदृतुः}{शुक्रः}{विकारी}{दक्षिणायनम्}{शरदृतुः}}
{\sunmoonsrdata{06:11}{17:35}{20:04}{08:24}{11:53}
{\kalas{04:30 05:20 09:13 08:28 09:59 16:04 10:44 13:01 15:18 16:50 18:26 20:44 22:19 01:28*}}}
{\tnykdata{\anga{\tithi{18}{कृष्ण-तृतीया}}{\time{35-10}{19:45}}\hspace{1ex}}%
{\anga{मृगशीर्षम्}{\time{43-15}{23:09}}\hspace{1ex}}{चन्द्रराशिः—\mbox{वृषभः\RIGHTarrow{10:59}}}%
{\anga{शिवः}{\time{4-45}{07:59}}\hspace{1ex}\uanga{सिद्धः}}%
{\anga{वणिजा}{\time{4-28}{07:53}}\hspace{1ex}\anga{भद्रा}{\time{35-10}{19:45}}\hspace{1ex}\uanga{बवम्}}{}
}
{अनध्यायः\eventsep गणाधिप-महागणपति-सङ्कटहर-चतुर्थी-व्रतम्\eventsep काञ्ची ९ जगद्गुरु श्री-कृपाशङ्करेन्द्र सरस्वती आराधना~\#{१९५१}\eventsep सौभाग्य-सुन्दरी-व्रतम्}
{Fri} 
\cfoot{\rygdata{10:27--11:53}{14:44--16:10}{07:36--09:02}}
\caldata{NOVEMBER}{16}{\sunmonth{तुला}{30}{\mbox{तुला{\tiny\RIGHTarrow}{00:21*}}}{कार्त्तिकः}{शरदृतुः}{शनिः}{विकारी}{दक्षिणायनम्}{शरदृतुः}}
{\sunmoonsrdata{06:11}{17:35}{21:00}{09:20}{11:53}
{\kalas{04:30 05:21 09:13 08:28 09:59 16:04 10:45 13:01 15:18 16:50 18:26 20:44 22:19 01:28*}}}
{\tnykdata{\anga{\tithi{19}{कृष्ण-चतुर्थी}}{\time{33-57}{19:15}}\hspace{1ex}}%
{\anga{आर्द्रा}{\time{43-24}{23:13}}\hspace{1ex}}{चन्द्रराशिः—\mbox{मिथुनम्}}%
{\anga{सिद्धः}{\time{0-58}{06:33}}\hspace{1ex}\anga{साध्यः}{\time{56-47}{04:51*}}\hspace{1ex}\uanga{शुभः}}%
{\anga{बवम्}{\time{3-34}{07:33}}\hspace{1ex}\anga{बालवम्}{\time{33-57}{19:15}}\hspace{1ex}\uanga{कौलवम्}}{}
}
{आकाशदीप-समापनम्\eventsep सिन्धु-आद्य-पुष्कर-समापनम्\eventsep तुला-कावेरी-स्नान-समापनम्}
{Sat} 
\cfoot{\rygdata{09:02--10:27}{13:19--14:44}{06:11--07:37}}
\caldata{NOVEMBER}{17}{\sunmonth{वृश्चिकः}{1}{}{कार्त्तिकः}{शरदृतुः}{भानुः}{विकारी}{दक्षिणायनम्}{शरदृतुः}}
{\sunmoonsrdata{06:11}{17:35}{21:58}{10:14}{11:53}
{\kalas{04:31 05:21 09:14 08:28 09:59 16:04 10:45 13:02 15:18 16:49 18:25 20:44 22:19 01:28*}}}
{\tnykdata{\anga{\tithi{20}{कृष्ण-पञ्चमी}}{\time{31-53}{18:23}}\hspace{1ex}}%
{\anga{पुनर्वसुः}{\time{42-43}{22:56}}\hspace{1ex}}{चन्द्रराशिः—\mbox{मिथुनम्\RIGHTarrow{17:02}}}%
{\prev{\anga{साध्यः}{\time{*56-47}{04:51}}}\hspace{1ex}\anga{शुभः}{\time{52-2}{02:51*}}\hspace{1ex}\uanga{शुक्लः}}%
{\anga{कौलवम्}{\time{1-45}{06:51}}\hspace{1ex}\anga{तैतिलम्}{\time{31-53}{18:23}}\hspace{1ex}\anga{गरजा}{\time{59-4}{05:49*}}\hspace{1ex}\uanga{वणिजा}}{}
}
{\tamil{கார்த்திகை~ஞாயிற்றுக்கிழமை}\eventsep कृत्तिका-मण्डल-पारायण-आरम्भः\eventsep \tamil{முடவன் முழுக்கு}\eventsep सङ्क्रमण-दिन-पूर्वाह्ण-पुण्यकालः~06:11\RIGHTarrow{}11:53\eventsep तिरुविशलूर् गङ्गाकर्षण-महोत्सव-आरम्भः\eventsep वृश्चिक-रवि-सङ्क्रमण-विष्णुपदी-पुण्यकालः~06:11\RIGHTarrow{}06:45}
{Sun} 
\cfoot{\rygdata{16:10--17:35}{11:53--13:19}{14:44--16:10}}
\caldata{NOVEMBER}{18}{\sunmonth{वृश्चिकः}{2}{}{कार्त्तिकः}{शरदृतुः}{सोमः}{विकारी}{दक्षिणायनम्}{शरदृतुः}}
{\sunmoonsrdata{06:12}{17:35}{22:56}{11:06}{11:53}
{\kalas{04:31 05:21 09:14 08:29 10:00 16:04 10:45 13:02 15:18 16:49 18:25 20:44 22:19 01:28*}}}
{\tnykdata{\anga{\tithi{21}{कृष्ण-षष्ठी}}{\time{28-52}{17:09}}\hspace{1ex}}%
{\anga{पुष्यः}{\time{41-12}{22:18}}\hspace{1ex}}{चन्द्रराशिः—\mbox{कर्कटः}}%
{\anga{शुक्लः}{\time{46-37}{00:35*}}\hspace{1ex}\uanga{ब्राह्मः}}%
{\prev{\anga{गरजा}{\time{*59-4}{05:49}}}\hspace{1ex}\anga{वणिजा}{\time{28-52}{17:09}}\hspace{1ex}\anga{भद्रा}{\time{55-44}{04:25*}}\hspace{1ex}\uanga{बवम्}}{}
}
{कार्त्तिक-सोमवासरः}
{Mon} 
\cfoot{\rygdata{07:37--09:03}{10:28--11:53}{13:19--14:44}}
\caldata{NOVEMBER}{19}{\sunmonth{वृश्चिकः}{3}{}{कार्त्तिकः}{शरदृतुः}{मङ्गलः}{विकारी}{दक्षिणायनम्}{शरदृतुः}}
{\sunmoonsrdata{06:12}{17:35}{23:54}{11:55}{11:54}
{\kalas{04:31 05:22 09:14 08:29 10:00 16:04 10:45 13:02 15:18 16:49 18:25 20:44 22:19 01:29*}}}
{\tnykdata{\anga{\tithi{22}{कृष्ण-सप्तमी}}{\time{24-44}{15:35}}\hspace{1ex}}%
{\anga{आश्रेषा}{\time{38-54}{21:20}}\hspace{1ex}}{चन्द्रराशिः—\mbox{कर्कटः\RIGHTarrow{21:20}}}%
{\anga{ब्राह्मः}{\time{40-33}{22:02}}\hspace{1ex}\uanga{माहेन्द्रः}}%
{\prev{\anga{भद्रा}{\time{*55-44}{04:25}}}\hspace{1ex}\anga{बवम्}{\time{24-44}{15:35}}\hspace{1ex}\anga{बालवम्}{\time{51-35}{02:40*}}\hspace{1ex}\uanga{कौलवम्}}{}
}
{कालभैरवाष्टमी\eventsep पञ्च-पर्व-पूजा (अष्टमी)\eventsep सावित्री-कल्पादिः}
{Tue} 
\cfoot{\rygdata{14:44--16:10}{09:03--10:28}{11:54--13:19}}
\caldata{NOVEMBER}{20}{\sunmonth{वृश्चिकः}{4}{}{कार्त्तिकः}{शरदृतुः}{बुधः}{विकारी}{दक्षिणायनम्}{शरदृतुः}}
{\sunmoonsrdata{06:13}{17:35}{00:50*}{12:42}{11:54}
{\kalas{04:32 05:22 09:15 08:29 10:00 16:04 10:46 13:02 15:19 16:49 18:25 20:45 22:19 01:29*}}}
{\tnykdata{\anga{\tithi{23}{कृष्ण-अष्टमी}}{\time{19-42}{13:41}}\hspace{1ex}}%
{\anga{मघा}{\time{35-48}{20:02}}\hspace{1ex}}{चन्द्रराशिः—\mbox{सिंहः}}%
{\anga{माहेन्द्रः}{\time{33-52}{19:13}}\hspace{1ex}\uanga{वैधृतिः}}%
{\anga{कौलवम्}{\time{19-42}{13:41}}\hspace{1ex}\anga{तैतिलम्}{\time{46-40}{00:37*}}\hspace{1ex}\uanga{गरजा}}{}
}
{अनध्यायः\eventsep काञ्ची ४२ जगद्गुरु श्री-ब्रह्मानन्दघनेन्द्र सरस्वती २ आराधना~\#{१०४२}\eventsep काञ्ची ४९ जगद्गुरु श्री-महादेवेन्द्र सरस्वती ३ आराधना~\#{७७३}\eventsep काञ्ची ५८ जगद्गुरु श्री-आत्मबोधेन्द्र सरस्वती आराधना~\#{३८२}\eventsep कालाष्टमी\eventsep महादेवाष्टमी}
{Wed} 
\cfoot{\rygdata{11:54--13:19}{07:38--09:03}{10:29--11:54}}
\caldata{NOVEMBER}{21}{\sunmonth{वृश्चिकः}{5}{}{कार्त्तिकः}{शरदृतुः}{गुरुः}{विकारी}{दक्षिणायनम्}{शरदृतुः}}
{\sunmoonsrdata{06:13}{17:35}{01:47*}{13:28}{11:54}
{\kalas{04:32 05:23 09:15 08:30 10:01 16:04 10:46 13:02 15:19 16:50 18:26 20:45 22:20 01:29*}}}
{\tnykdata{\anga{\tithi{24}{कृष्ण-नवमी}}{\time{13-52}{11:28}}\hspace{1ex}}%
{\anga{पूर्वफल्गुनी}{\time{32-2}{18:27}}\hspace{1ex}}{चन्द्रराशिः—\mbox{सिंहः\RIGHTarrow{00:01*}}}%
{\anga{वैधृतिः}{\time{26-15}{16:10}}\hspace{1ex}\uanga{विष्कम्भः}}%
{\anga{गरजा}{\time{13-52}{11:28}}\hspace{1ex}\anga{वणिजा}{\time{41-7}{22:16}}\hspace{1ex}\uanga{भद्रा}}{}
}
{काञ्ची २८ जगद्गुरु श्री-महादेवेन्द्र सरस्वती १ आराधना~\#{१४१९}\eventsep वैधृति-श्राद्धम्}
{Thu} 
\cfoot{\rygdata{13:19--14:45}{06:13--07:39}{09:04--10:29}}
\caldata{NOVEMBER}{22}{\sunmonth{वृश्चिकः}{6}{}{कार्त्तिकः}{शरदृतुः}{शुक्रः}{विकारी}{दक्षिणायनम्}{शरदृतुः}}
{\sunmoonsrdata{06:14}{17:35}{02:44*}{14:13}{11:54}
{\kalas{04:33 05:23 09:15 08:30 10:01 16:04 10:46 13:02 15:19 16:50 18:26 20:45 22:20 01:30*}}}
{\tnykdata{\anga{\tithi{25}{कृष्ण-दशमी}}{\time{7-21}{09:01}}\hspace{1ex}}%
{\anga{उत्तरफल्गुनी}{\time{27-30}{16:38}}\hspace{1ex}}{चन्द्रराशिः—\mbox{कन्या}}%
{\anga{विष्कम्भः}{\time{17-41}{12:55}}\hspace{1ex}\uanga{प्रीतिः}}%
{\anga{भद्रा}{\time{7-21}{09:01}}\hspace{1ex}\anga{बवम्}{\time{35-4}{19:43}}\hspace{1ex}\uanga{बालवम्}}{}
}
{(सायन) षडशीति-पुण्यकालः\eventsep \tamil{மெய்ப்பொருள் நாயன்மார் (5) குருபூஜை}\eventsep सायन-रवि-सङ्क्रमण-पुण्यकालः~14:04\RIGHTarrow{}17:35\eventsep सायन-सङ्क्रमण-दिन-अपराह्ण-पुण्यकालः~11:54\RIGHTarrow{}17:35\eventsep सहो-मासः/हेमन्तऋतुः~20:28\RIGHTarrow{}\eventsep स्मार्त-उत्पन्ना-एकादशी (गृहस्थ)}
{Fri} 
\cfoot{\rygdata{10:29--11:54}{14:45--16:10}{07:39--09:04}}
\caldata{NOVEMBER}{23}{\sunmonth{वृश्चिकः}{7}{}{कार्त्तिकः}{शरदृतुः}{शनिः}{विकारी}{दक्षिणायनम्}{शरदृतुः}}
{\sunmoonsrdata{06:14}{17:35}{03:41*}{14:59}{11:55}
{\kalas{04:33 05:24 09:16 08:30 10:01 16:04 10:47 13:03 15:19 16:50 18:26 20:45 22:20 01:30*}}}
{\tnykdata{\anga{\tithi{26}{कृष्ण-एकादशी}}{\time{0-24}{06:24}}\hspace{1ex}\anga{\tithi{27}{कृष्ण-द्वादशी}}{\time{54-0}{03:43*}}\hspace{1ex}\avamA{}}%
{\anga{हस्तः}{\time{22-22}{14:42}}\hspace{1ex}}{चन्द्रराशिः—\mbox{कन्या\RIGHTarrow{01:43*}}}%
{\anga{प्रीतिः}{\time{8-47}{09:34}}\hspace{1ex}\anga{आयुष्मान्}{\time{59-48}{06:10*}}\hspace{1ex}\uanga{सौभाग्यः}}%
{\anga{बालवम्}{\time{0-24}{06:24}}\hspace{1ex}\anga{कौलवम्}{\time{28-36}{17:03}}\hspace{1ex}\anga{तैतिलम्}{\time{54-0}{03:43*}}\hspace{1ex}\uanga{गरजा}}{}
}
{\tamil{ஆனாய நாயன்மார் (14) குருபூஜை}\eventsep दिनक्षयः\eventsep द्विपुष्कर-योगः~14:42\RIGHTarrow{}03:43*\eventsep हरिवासरः\RIGHTarrow{}11:44\eventsep सहोमास-उषःकाल-पूजारम्भः\eventsep स्मार्त-उत्पन्ना-एकादशी (सन्न्यस्त)\eventsep त्रिस्पृशा-महाद्वादशी\eventsep वैष्णव-उत्पन्ना-एकादशी}
{Sat} 
\cfoot{\rygdata{09:04--10:30}{13:20--14:45}{06:14--07:39}}
\caldata{NOVEMBER}{24}{\sunmonth{वृश्चिकः}{8}{}{कार्त्तिकः}{शरदृतुः}{भानुः}{विकारी}{दक्षिणायनम्}{शरदृतुः}}
{\sunmoonsrdata{06:15}{17:35}{04:40*}{15:46}{11:55}
{\kalas{04:34 05:24 09:16 08:31 10:02 16:04 10:47 13:03 15:19 16:50 18:26 20:45 22:20 01:30*}}}
{\tnykdata{\anga{\tithi{28}{कृष्ण-त्रयोदशी}}{\time{47-46}{01:05*}}\hspace{1ex}}%
{\anga{चित्रा}{\time{17-12}{12:45}}\hspace{1ex}}{चन्द्रराशिः—\mbox{तुला}}%
{\prev{\anga{आयुष्मान्}{\time{*59-48}{06:10}}}\hspace{1ex}\anga{सौभाग्यः}{\time{51-53}{02:50*}}\hspace{1ex}\uanga{शोभनः}}%
{\anga{गरजा}{\time{21-32}{14:23}}\hspace{1ex}\anga{वणिजा}{\time{47-46}{01:05*}}\hspace{1ex}\uanga{भद्रा}}{}
}
{\tamil{கார்த்திகை~ஞாயிற்றுக்கிழமை}\eventsep मासशिवरात्रिः\eventsep प्रदोष-व्रतम्~17:35\RIGHTarrow{}19:10}
{Sun} 
\cfoot{\rygdata{16:10--17:35}{11:55--13:20}{14:45--16:10}}
\caldata{NOVEMBER}{25}{\sunmonth{वृश्चिकः}{9}{}{कार्त्तिकः}{शरदृतुः}{सोमः}{विकारी}{दक्षिणायनम्}{शरदृतुः}}
{\sunmoonsrdata{06:15}{17:35}{05:40*}{16:36}{11:55}
{\kalas{04:34 05:25 09:17 08:31 10:02 16:04 10:47 13:03 15:19 16:50 18:26 20:45 22:20 01:31*}}}
{\tnykdata{\anga{\tithi{29}{कृष्ण-चतुर्दशी}}{\time{42-1}{22:40}}\hspace{1ex}}%
{\anga{स्वाती}{\time{12-19}{10:55}}\hspace{1ex}}{चन्द्रराशिः—\mbox{तुला\RIGHTarrow{03:42*}}}%
{\anga{शोभनः}{\time{44-23}{23:40}}\hspace{1ex}\uanga{अतिगण्डः}}%
{\anga{भद्रा}{\time{14-48}{11:51}}\hspace{1ex}\anga{शकुनिः}{\time{42-1}{22:40}}\hspace{1ex}\uanga{चतुष्पात्}}{}
}
{अनध्यायः\eventsep कार्त्तिक-सोमवासरः\eventsep पञ्च-पर्व-पूजा (अमावास्या)\eventsep पञ्च-पर्व-पूजा (चतुर्दशी)}
{Mon} 
\cfoot{\rygdata{07:40--09:05}{10:30--11:55}{13:20--14:45}}
\caldata{NOVEMBER}{26}{\sunmonth{वृश्चिकः}{10}{}{कार्त्तिकः}{शरदृतुः}{मङ्गलः}{विकारी}{दक्षिणायनम्}{शरदृतुः}}
{\sunmoonsrdata{06:16}{17:35}{---}{17:29}{11:56}
{\kalas{04:34 05:25 09:17 08:32 10:02 16:05 10:48 13:03 15:19 16:50 18:26 20:46 22:21 01:31*}}}
{\tnykdata{\anga{\tithi{30}{अमावास्या}}{\time{37-5}{20:35}}\hspace{1ex}}%
{\anga{विशाखा}{\time{8-8}{09:20}}\hspace{1ex}}{चन्द्रराशिः—\mbox{वृश्चिकः}}%
{\anga{अतिगण्डः}{\time{37-32}{20:47}}\hspace{1ex}\uanga{सुकर्म}}%
{\anga{चतुष्पात्}{\time{8-46}{09:35}}\hspace{1ex}\anga{नाग}{\time{37-5}{20:35}}\hspace{1ex}\uanga{किंस्तुघ्नः}}{}
}
{आग्रयण-होमः द्राविडेषु\eventsep अनध्यायः\eventsep कार्त्तिक-स्नानपूर्तिः\eventsep पार्वणव्रतम् अमावास्यायाम्\eventsep पिण्ड-पितृ-यज्ञः\eventsep सर्व-कार्त्तिक-अमावास्या (अलभ्यम्–अनूराधा, पुष्कला)\eventsep तिरुविशलूर् गङ्गाकर्षण-महोत्सव-समापनम्}
{Tue} 
\cfoot{\rygdata{14:45--16:10}{09:06--10:31}{11:55--13:20}}
\caldata{NOVEMBER}{27}{\sunmonth{वृश्चिकः}{11}{}{मार्गशीर्षः}{हेमन्तऋतुः}{बुधः}{विकारी}{दक्षिणायनम्}{शरदृतुः}}
{\sunmoonrsdata{06:16}{17:35}{06:40}{18:24}{11:56}
{\kalas{04:35 05:26 09:17 08:32 10:03 16:05 10:48 13:04 15:20 16:50 18:26 20:46 22:21 01:31*}}}
{\tnykdata{\anga{\tithi{1}{शुक्ल-प्रथमा}}{\time{33-17}{18:59}}\hspace{1ex}}%
{\anga{अनूराधा}{\time{5-0}{08:10}}\hspace{1ex}}{चन्द्रराशिः—\mbox{वृश्चिकः}}%
{\anga{सुकर्म}{\time{31-36}{18:16}}\hspace{1ex}\uanga{धृतिः}}%
{\anga{किंस्तुघ्नः}{\time{3-49}{07:43}}\hspace{1ex}\anga{बवम्}{\time{33-17}{18:59}}\hspace{1ex}\uanga{बालवम्}}{}
}
{अनध्यायः\eventsep बुधानुराधा-योगः\RIGHTarrow{}08:09\eventsep दर्शेष्टिः\eventsep धन-व्रतम्\eventsep काञ्ची १८ जगद्गुरु श्री-योगतिलक सुरेन्द्र सरस्वती आराधना~\#{१६३५}\eventsep पार्वण-प्रायश्चित्तावकाशः दर्शे\eventsep दर्श-स्थालीपाकः\eventsep वनदुर्गानवरात्र-आरम्भः}
{Wed} 
\cfoot{\rygdata{11:56--13:21}{07:41--09:06}{10:31--11:56}}
\caldata{NOVEMBER}{28}{\sunmonth{वृश्चिकः}{12}{}{मार्गशीर्षः}{हेमन्तऋतुः}{गुरुः}{विकारी}{दक्षिणायनम्}{शरदृतुः}}
{\sunmoonrsdata{06:17}{17:35}{07:39}{19:20}{11:56}
{\kalas{04:35 05:26 09:18 08:33 10:03 16:05 10:48 13:04 15:20 16:50 18:26 20:46 22:21 01:32*}}}
{\tnykdata{\anga{\tithi{2}{शुक्ल-द्वितीया}}{\time{30-54}{17:58}}\hspace{1ex}}%
{\anga{ज्येष्ठा}{\time{3-16}{07:31}}\hspace{1ex}}{चन्द्रराशिः—\mbox{वृश्चिकः\RIGHTarrow{07:31}}}%
{\anga{धृतिः}{\time{26-23}{16:14}}\hspace{1ex}\uanga{शूलः}}%
{\anga{बालवम्}{\time{0-17}{06:24}}\hspace{1ex}\anga{कौलवम्}{\time{30-54}{17:58}}\hspace{1ex}\anga{तैतिलम्}{\time{58-39}{05:43*}}\hspace{1ex}\uanga{गरजा}}{}
}
{चन्द्र-दर्शनम्~17:35\RIGHTarrow{}19:20\eventsep \tamil{மூர்க்க நாயன்மார் (32) குருபூஜை}\eventsep तिन्त्रिणी-गौरी-व्रतम्}
{Thu} 
\cfoot{\rygdata{13:21--14:46}{06:17--07:42}{09:07--10:31}}
\caldata{NOVEMBER}{29}{\sunmonth{वृश्चिकः}{13}{}{मार्गशीर्षः}{हेमन्तऋतुः}{शुक्रः}{विकारी}{दक्षिणायनम्}{शरदृतुः}}
{\sunmoonrsdata{06:17}{17:36}{08:35}{20:16}{11:56}
{\kalas{04:36 05:27 09:18 08:33 10:03 16:05 10:49 13:04 15:20 16:50 18:26 20:46 22:21 01:32*}}}
{\tnykdata{\anga{\tithi{3}{शुक्ल-तृतीया}}{\time{30-8}{17:39}}\hspace{1ex}}%
{\anga{मूला}{\time{3-15}{07:31}}\hspace{1ex}}{चन्द्रराशिः—\mbox{धनुः}}%
{\anga{शूलः}{\time{22-24}{14:44}}\hspace{1ex}\uanga{गण्डः}}%
{\prev{\anga{तैतिलम्}{\time{*58-39}{05:43}}}\hspace{1ex}\anga{गरजा}{\time{30-8}{17:39}}\hspace{1ex}\anga{वणिजा}{\time{58-45}{05:46*}}\hspace{1ex}\uanga{भद्रा}}{}
}
{\tamil{சிறப்புலி நாயன்மார் (35) குருபூஜை}}
{Fri} 
\cfoot{\rygdata{10:32--11:56}{14:46--16:11}{07:42--09:07}}
\caldata{NOVEMBER}{30}{\sunmonth{वृश्चिकः}{14}{}{मार्गशीर्षः}{हेमन्तऋतुः}{शनिः}{विकारी}{दक्षिणायनम्}{शरदृतुः}}
{\sunmoonrsdata{06:18}{17:36}{09:27}{21:10}{11:57}
{\kalas{04:36 05:27 09:19 08:33 10:04 16:05 10:49 13:05 15:20 16:51 18:27 20:47 22:22 01:33*}}}
{\tnykdata{\anga{\tithi{4}{शुक्ल-चतुर्थी}}{\time{31-7}{18:05}}\hspace{1ex}}%
{\anga{पूर्वाषाढा}{\time{5-5}{08:13}}\hspace{1ex}}{चन्द्रराशिः—\mbox{धनुः\RIGHTarrow{14:30}}}%
{\anga{गण्डः}{\time{19-54}{13:48}}\hspace{1ex}\uanga{वृद्धिः}}%
{\prev{\anga{वणिजा}{\time{*58-45}{05:46}}}\hspace{1ex}\anga{भद्रा}{\time{31-7}{18:05}}\hspace{1ex}\uanga{बवम्}}{}
}
{बदरी-गौरी-व्रतम्\eventsep देवी-पर्व-९\eventsep शुक्ल-चतुर्थी-व्रतम्}
{Sat} 
\cfoot{\rygdata{09:07--10:32}{13:22--14:46}{06:18--07:43}}
\caldata{DECEMBER}{1}{\sunmonth{वृश्चिकः}{15}{}{मार्गशीर्षः}{हेमन्तऋतुः}{भानुः}{विकारी}{दक्षिणायनम्}{शरदृतुः}}
{\sunmoonrsdata{06:19}{17:36}{10:15}{22:02}{11:57}
{\kalas{04:37 05:28 09:19 08:34 10:04 16:06 10:50 13:05 15:21 16:51 18:27 20:47 22:22 01:33*}}}
{\tnykdata{\anga{\tithi{5}{शुक्ल-पञ्चमी}}{\time{33-48}{19:13}}\hspace{1ex}}%
{\anga{उत्तराषाढा}{\time{8-47}{09:37}}\hspace{1ex}}{चन्द्रराशिः—\mbox{मकरः}}%
{\anga{वृद्धिः}{\time{18-53}{13:25}}\hspace{1ex}\uanga{ध्रुवः}}%
{\anga{बवम्}{\time{0-39}{06:33}}\hspace{1ex}\anga{बालवम्}{\time{33-48}{19:13}}\hspace{1ex}\uanga{कौलवम्}}{}
}
{\tamil{கார்த்திகை~ஞாயிற்றுக்கிழமை}\eventsep श्रवण-व्रतम्}
{Sun} 
\cfoot{\rygdata{16:11--17:36}{11:57--13:22}{14:47--16:11}}
\caldata{DECEMBER}{2}{\sunmonth{वृश्चिकः}{16}{}{मार्गशीर्षः}{हेमन्तऋतुः}{सोमः}{विकारी}{दक्षिणायनम्}{शरदृतुः}}
{\sunmoonrsdata{06:19}{17:36}{10:59}{22:52}{11:58}
{\kalas{04:37 05:28 09:20 08:34 10:05 16:06 10:50 13:05 15:21 16:51 18:27 20:47 22:22 01:33*}}}
{\tnykdata{\anga{\tithi{6}{शुक्ल-षष्ठी}}{\time{37-58}{20:59}}\hspace{1ex}}%
{\anga{श्रवणः}{\time{14-13}{11:40}}\hspace{1ex}}{चन्द्रराशिः—\mbox{मकरः\RIGHTarrow{00:54*}}}%
{\anga{ध्रुवः}{\time{19-11}{13:32}}\hspace{1ex}\uanga{व्याघातः}}%
{\anga{कौलवम्}{\time{4-33}{08:02}}\hspace{1ex}\anga{तैतिलम्}{\time{37-58}{20:59}}\hspace{1ex}\uanga{गरजा}}{}
}
{काञ्ची ३२ जगद्गुरु श्री-चिदानन्दघनेन्द्र सरस्वती आराधना~\#{१३४८}\eventsep मार्गशीर्ष-शिवलिङ्ग-षष्ठी\eventsep सोमश्रावणी-योगः\RIGHTarrow{}11:40\eventsep सुब्रह्मण्य-षष्ठी-व्रतम्}
{Mon} 
\cfoot{\rygdata{07:44--09:08}{10:33--11:58}{13:22--14:47}}
\caldata{DECEMBER}{3}{\sunmonth{वृश्चिकः}{17}{}{मार्गशीर्षः}{हेमन्तऋतुः}{मङ्गलः}{विकारी}{दक्षिणायनम्}{शरदृतुः}}
{\sunmoonrsdata{06:20}{17:36}{11:39}{23:39}{11:58}
{\kalas{04:38 05:29 09:20 08:35 10:05 16:06 10:50 13:06 15:21 16:51 18:27 20:47 22:23 01:34*}}}
{\tnykdata{\anga{\tithi{7}{शुक्ल-सप्तमी}}{\time{43-14}{23:14}}\hspace{1ex}}%
{\anga{श्रविष्ठा}{\time{21-0}{14:14}}\hspace{1ex}}{चन्द्रराशिः—\mbox{कुम्भः}}%
{\anga{व्याघातः}{\time{20-31}{14:03}}\hspace{1ex}\uanga{हर्षणः}}%
{\anga{गरजा}{\time{9-55}{10:04}}\hspace{1ex}\anga{वणिजा}{\time{43-14}{23:14}}\hspace{1ex}\uanga{भद्रा}}{}
}
{द्विपुष्कर-योगः~06:19\RIGHTarrow{}14:14\eventsep काञ्ची ५ जगद्गुरु श्री-ज्ञानानन्देन्द्र सरस्वती आराधना~\#{२२२४}\eventsep मित्र-सप्तमी\eventsep नन्दा-सप्तमी}
{Tue} 
\cfoot{\rygdata{14:47--16:12}{09:09--10:33}{11:58--13:23}}
\caldata{DECEMBER}{4}{\sunmonth{वृश्चिकः}{18}{}{मार्गशीर्षः}{हेमन्तऋतुः}{बुधः}{विकारी}{दक्षिणायनम्}{शरदृतुः}}
{\sunmoonrsdata{06:20}{17:37}{12:16}{00:26*}{11:58}
{\kalas{04:38 05:29 09:21 08:35 10:06 16:06 10:51 13:06 15:21 16:52 18:28 20:48 22:23 01:34*}}}
{\tnykdata{\anga{\tithi{8}{शुक्ल-अष्टमी}}{\time{49-7}{01:44*}}\hspace{1ex}}%
{\anga{शतभिषक्}{\time{28-38}{17:06}}\hspace{1ex}}{चन्द्रराशिः—\mbox{कुम्भः}}%
{\anga{हर्षणः}{\time{22-32}{14:48}}\hspace{1ex}\uanga{वज्रम्}}%
{\anga{भद्रा}{\time{16-17}{12:28}}\hspace{1ex}\anga{बवम्}{\time{49-7}{01:44*}}\hspace{1ex}\uanga{बालवम्}}{}
}
{अनध्यायः\eventsep बुधाष्टमी}
{Wed} 
\cfoot{\rygdata{11:58--13:23}{07:45--09:09}{10:34--11:58}}
\caldata{DECEMBER}{5}{\sunmonth{वृश्चिकः}{19}{}{मार्गशीर्षः}{हेमन्तऋतुः}{गुरुः}{विकारी}{दक्षिणायनम्}{शरदृतुः}}
{\sunmoonrsdata{06:21}{17:37}{12:53}{01:11*}{11:59}
{\kalas{04:39 05:30 09:21 08:36 10:06 16:07 10:51 13:06 15:22 16:52 18:28 20:48 22:24 01:35*}}}
{\tnykdata{\anga{\tithi{9}{शुक्ल-नवमी}}{\time{55-2}{04:15*}}\hspace{1ex}}%
{\anga{पूर्वप्रोष्ठपदा}{\time{35-46}{20:04}}\hspace{1ex}}{चन्द्रराशिः—\mbox{कुम्भः\RIGHTarrow{13:20}}}%
{\anga{वज्रम्}{\time{24-46}{15:39}}\hspace{1ex}\uanga{सिद्धिः}}%
{\anga{बालवम्}{\time{23-2}{15:00}}\hspace{1ex}\anga{कौलवम्}{\time{55-2}{04:15*}}\hspace{1ex}\uanga{तैतिलम्}}{}
}
{प्रलय-कल्पादिः\eventsep वनदुर्गानवरात्र-समापनम्}
{Thu} 
\cfoot{\rygdata{13:23--14:48}{06:21--07:45}{09:10--10:34}}
\caldata{DECEMBER}{6}{\sunmonth{वृश्चिकः}{20}{}{मार्गशीर्षः}{हेमन्तऋतुः}{शुक्रः}{विकारी}{दक्षिणायनम्}{शरदृतुः}}
{\sunmoonrsdata{06:21}{17:37}{13:29}{01:57*}{11:59}
{\kalas{04:39 05:30 09:22 08:36 10:07 16:07 10:52 13:07 15:22 16:52 18:28 20:48 22:24 01:35*}}}
{\tnykdata{\prev{\anga{\tithi{9}{शुक्ल-नवमी}}{\time{*55-2}{04:15}}}\hspace{1ex}\fulltithi{\tithi{10}{शुक्ल-दशमी}}}%
{\anga{उत्तरप्रोष्ठपदा}{\time{42-26}{22:54}}\hspace{1ex}}{चन्द्रराशिः—\mbox{मीनः}}%
{\anga{सिद्धिः}{\time{26-48}{16:25}}\hspace{1ex}\uanga{व्यतीपातः}}%
{\prev{\anga{कौलवम्}{\time{*55-2}{04:15}}}\hspace{1ex}\anga{तैतिलम्}{\time{29-32}{17:27}}\hspace{1ex}\uanga{गरजा}}{}
}
{}
{Fri} 
\cfoot{\rygdata{10:35--11:59}{14:48--16:13}{07:46--09:10}}
\caldata{DECEMBER}{7}{\sunmonth{वृश्चिकः}{21}{}{मार्गशीर्षः}{हेमन्तऋतुः}{शनिः}{विकारी}{दक्षिणायनम्}{शरदृतुः}}
{\sunmoonrsdata{06:22}{17:38}{14:06}{02:44*}{12:00}
{\kalas{04:40 05:31 09:22 08:37 10:07 16:07 10:52 13:07 15:22 16:53 18:29 20:49 22:24 01:36*}}}
{\tnykdata{\anga{\tithi{10}{शुक्ल-दशमी}}{\time{0-32}{06:34}}\hspace{1ex}}%
{\anga{रेवती}{\time{48-19}{01:25*}}\hspace{1ex}}{चन्द्रराशिः—\mbox{मीनः\RIGHTarrow{01:25*}}}%
{\anga{व्यतीपातः}{\time{28-14}{16:58}}\hspace{1ex}\uanga{वरीयान्}}%
{\anga{गरजा}{\time{0-32}{06:34}}\hspace{1ex}\anga{वणिजा}{\time{34-36}{19:35}}\hspace{1ex}\uanga{भद्रा}}{}
}
{व्यतीपात-श्राद्धम्}
{Sat} 
\cfoot{\rygdata{09:11--10:35}{13:24--14:49}{06:22--07:46}}
\caldata{DECEMBER}{8}{\sunmonth{वृश्चिकः}{22}{}{मार्गशीर्षः}{हेमन्तऋतुः}{भानुः}{विकारी}{दक्षिणायनम्}{शरदृतुः}}
{\sunmoonrsdata{06:22}{17:38}{14:45}{03:33*}{12:00}
{\kalas{04:40 05:31 09:22 08:37 10:07 16:08 10:53 13:08 15:23 16:53 18:29 20:49 22:25 01:36*}}}
{\tnykdata{\anga{\tithi{11}{शुक्ल-एकादशी}}{\time{5-37}{08:29}}\hspace{1ex}}%
{\anga{अश्विनी}{\time{53-7}{03:27*}}\hspace{1ex}}{चन्द्रराशिः—\mbox{मेषः}}%
{\anga{वरीयान्}{\time{28-47}{17:11}}\hspace{1ex}\uanga{परिघः}}%
{\anga{भद्रा}{\time{5-37}{08:29}}\hspace{1ex}\anga{बवम्}{\time{38-32}{21:15}}\hspace{1ex}\uanga{बालवम्}}{}
}
{गीता-जयन्ती\eventsep गुरुवायुपुर-एकादशी\eventsep \tamil{கார்த்திகை~ஞாயிற்றுக்கிழமை}\eventsep कैशिक-एकादशी\eventsep सर्व-मोक्षदा-एकादशी}
{Sun} 
\cfoot{\rygdata{16:13--17:38}{12:00--13:25}{14:49--16:13}}
\caldata{DECEMBER}{9}{\sunmonth{वृश्चिकः}{23}{}{मार्गशीर्षः}{हेमन्तऋतुः}{सोमः}{विकारी}{दक्षिणायनम्}{शरदृतुः}}
{\sunmoonrsdata{06:23}{17:38}{15:27}{04:24*}{12:01}
{\kalas{04:41 05:32 09:23 08:38 10:08 16:08 10:53 13:08 15:23 16:53 18:29 20:50 22:25 01:36*}}}
{\tnykdata{\anga{\tithi{12}{शुक्ल-द्वादशी}}{\time{9-21}{09:54}}\hspace{1ex}}%
{\anga{अपभरणी}{\time{56-38}{04:58*}}\hspace{1ex}}{चन्द्रराशिः—\mbox{मेषः}}%
{\anga{परिघः}{\time{28-14}{16:59}}\hspace{1ex}\uanga{शिवः}}%
{\anga{बालवम्}{\time{9-21}{09:54}}\hspace{1ex}\anga{कौलवम्}{\time{41-9}{22:23}}\hspace{1ex}\uanga{तैतिलम्}}{}
}
{\tamil{பரணீ~தீபம்}\eventsep \tamil{கார்த்திகை}\eventsep कैशिक-द्वादशी\eventsep सोम-प्रदोष-व्रतम्~17:38\RIGHTarrow{}19:14\eventsep \tamil{திருவண்ணாமலை~தீபம்}}
{Mon} 
\cfoot{\rygdata{07:47--09:12}{10:36--12:01}{13:25--14:49}}
\caldata{DECEMBER}{10}{\sunmonth{वृश्चिकः}{24}{}{मार्गशीर्षः}{हेमन्तऋतुः}{मङ्गलः}{विकारी}{दक्षिणायनम्}{शरदृतुः}}
{\sunmoonrsdata{06:23}{17:39}{16:12}{05:18*}{12:01}
{\kalas{04:41 05:32 09:23 08:38 10:08 16:08 10:53 13:08 15:24 16:54 18:30 20:50 22:26 01:37*}}}
{\tnykdata{\anga{\tithi{13}{शुक्ल-त्रयोदशी}}{\time{11-33}{10:43}}\hspace{1ex}}%
{\prev{\anga{अपभरणी}{\time{*56-38}{04:58}}}\hspace{1ex}\anga{कृत्तिका}{\time{58-50}{05:54*}}\hspace{1ex}}{चन्द्रराशिः—\mbox{मेषः\RIGHTarrow{11:15}}}%
{\anga{शिवः}{\time{26-30}{16:20}}\hspace{1ex}\uanga{सिद्धः}}%
{\anga{तैतिलम्}{\time{11-33}{10:43}}\hspace{1ex}\anga{गरजा}{\time{42-25}{22:55}}\hspace{1ex}\uanga{वणिजा}}{}
}
{कृत्तिका-व्रतम्\eventsep \tamil{கணம்புல்ல நாயன்மார் (47) குருபூஜை}\eventsep \tamil{திருமங்கையாழ்வார் திருநக்ஷத்திரம்}}
{Tue} 
\cfoot{\rygdata{14:50--16:14}{09:12--10:37}{12:01--13:25}}
\caldata{DECEMBER}{11}{\sunmonth{वृश्चिकः}{25}{}{मार्गशीर्षः}{हेमन्तऋतुः}{बुधः}{विकारी}{दक्षिणायनम्}{शरदृतुः}}
{\sunmoonrsdata{06:24}{17:39}{17:01}{06:14*}{12:01}
{\kalas{04:42 05:33 09:24 08:39 10:09 16:09 10:54 13:09 15:24 16:54 18:30 20:50 22:26 01:37*}}}
{\tnykdata{\anga{\tithi{14}{शुक्ल-चतुर्दशी}}{\time{12-12}{10:59}}\hspace{1ex}}%
{\prev{\anga{कृत्तिका}{\time{*58-50}{05:54}}}\hspace{1ex}\anga{रोहिणी}{\time{59-47}{06:19*}}\hspace{1ex}}{चन्द्रराशिः—\mbox{वृषभः}}%
{\anga{सिद्धः}{\time{23-37}{15:15}}\hspace{1ex}\uanga{साध्यः}}%
{\anga{वणिजा}{\time{12-12}{10:59}}\hspace{1ex}\anga{भद्रा}{\time{42-21}{22:54}}\hspace{1ex}\uanga{बवम्}}{}
}
{अनध्यायः\eventsep दत्तात्रेय-जयन्ती\eventsep मार्गशीर्ष-पूर्णिमा\eventsep पार्वणव्रतम् पूर्णिमायाम्\eventsep पञ्च-पर्व-पूजा (पूर्णिमा)\eventsep \tamil{ஸர்வாலய~தீபம்}\eventsep त्रिपुर-भैरवी-जयन्ती\eventsep वेङ्कटाचले पूर्णिमा-गरुड-सेवा}
{Wed} 
\cfoot{\rygdata{12:01--13:26}{07:48--09:13}{10:37--12:01}}
\caldata{DECEMBER}{12}{\sunmonth{वृश्चिकः}{26}{}{मार्गशीर्षः}{हेमन्तऋतुः}{गुरुः}{विकारी}{दक्षिणायनम्}{शरदृतुः}}
{\sunmoonrsdata{06:25}{17:39}{17:55}{---}{12:02}
{\kalas{04:42 05:33 09:24 08:39 10:09 16:09 10:54 13:09 15:24 16:54 18:30 20:51 22:26 01:38*}}}
{\tnykdata{\anga{\tithi{15}{पौर्णमासी}}{\time{11-26}{10:42}}\hspace{1ex}}%
{\prev{\anga{रोहिणी}{\time{*59-47}{06:19}}}\hspace{1ex}\anga{मृगशीर्षम्}{\time{59-38}{06:16*}}\hspace{1ex}}{चन्द्रराशिः—\mbox{वृषभः\RIGHTarrow{18:21}}}%
{\anga{साध्यः}{\time{19-37}{13:46}}\hspace{1ex}\uanga{शुभः}}%
{\anga{बवम्}{\time{11-26}{10:42}}\hspace{1ex}\anga{बालवम्}{\time{41-5}{22:22}}\hspace{1ex}\uanga{कौलवम्}}{}
}
{आग्रयण-होमः द्राविडेषु\eventsep अनध्यायः\eventsep अनध्यायः\eventsep अन्नपूर्णा-जयन्ती\eventsep काञ्ची १३ जगद्गुरु श्री-सच्चिद्घनेन्द्र सरस्वती आराधना~\#{१७४८}\eventsep महामार्गशीर्षी-योगः\eventsep पार्वण-प्रायश्चित्तावकाशः पौर्णमास्याम्\eventsep पूर्णमासेष्टिः\eventsep पूर्णिमा-व्रतम्\eventsep सर्प-बल्युत्सर्जनम्\eventsep पूर्ण-स्थालीपाकः}
{Thu} 
\cfoot{\rygdata{13:26--14:51}{06:25--07:49}{09:13--10:38}}
\caldata{DECEMBER}{13}{\sunmonth{वृश्चिकः}{27}{}{मार्गशीर्षः}{हेमन्तऋतुः}{शुक्रः}{विकारी}{दक्षिणायनम्}{शरदृतुः}}
{\sunmoonsrdata{06:25}{17:40}{18:52}{07:12}{12:02}
{\kalas{04:43 05:34 09:25 08:40 10:10 16:10 10:55 13:10 15:25 16:55 18:31 20:51 22:27 01:38*}}}
{\tnykdata{\anga{\tithi{16}{कृष्ण-प्रथमा}}{\time{9-23}{09:56}}\hspace{1ex}}%
{\prev{\anga{मृगशीर्षम्}{\time{*59-38}{06:16}}}\hspace{1ex}\anga{आर्द्रा}{\time{58-31}{05:48*}}\hspace{1ex}}{चन्द्रराशिः—\mbox{मिथुनम्}}%
{\anga{शुभः}{\time{14-39}{11:55}}\hspace{1ex}\uanga{शुक्लः}}%
{\anga{कौलवम्}{\time{9-23}{09:56}}\hspace{1ex}\anga{तैतिलम्}{\time{38-47}{21:24}}\hspace{1ex}\uanga{गरजा}}{}
}
{}
{Fri} 
\cfoot{\rygdata{10:38--12:02}{14:51--16:15}{07:49--09:14}}
\caldata{DECEMBER}{14}{\sunmonth{वृश्चिकः}{28}{}{मार्गशीर्षः}{हेमन्तऋतुः}{शनिः}{विकारी}{दक्षिणायनम्}{शरदृतुः}}
{\sunmoonsrdata{06:26}{17:40}{19:51}{08:08}{12:03}
{\kalas{04:43 05:35 09:25 08:40 10:10 16:10 10:55 13:10 15:25 16:55 18:31 20:52 22:27 01:39*}}}
{\tnykdata{\anga{\tithi{17}{कृष्ण-द्वितीया}}{\time{6-16}{08:47}}\hspace{1ex}}%
{\prev{\anga{आर्द्रा}{\time{*58-31}{05:48}}}\hspace{1ex}\anga{पुनर्वसुः}{\time{56-39}{05:01*}}\hspace{1ex}}{चन्द्रराशिः—\mbox{मिथुनम्\RIGHTarrow{23:14}}}%
{\anga{शुक्लः}{\time{8-52}{09:45}}\hspace{1ex}\uanga{ब्राह्मः}}%
{\anga{गरजा}{\time{6-16}{08:47}}\hspace{1ex}\anga{वणिजा}{\time{35-39}{20:04}}\hspace{1ex}\uanga{भद्रा}}{}
}
{नारायणीयं-जयन्ती~\#{४३४}\eventsep परशुराम-जयन्ती\eventsep त्रिपुष्कर-योगः~06:26\RIGHTarrow{}08:47}
{Sat} 
\cfoot{\rygdata{09:14--10:39}{13:27--14:51}{06:26--07:50}}
\caldata{DECEMBER}{15}{\sunmonth{वृश्चिकः}{29}{}{मार्गशीर्षः}{हेमन्तऋतुः}{भानुः}{विकारी}{दक्षिणायनम्}{शरदृतुः}}
{\sunmoonsrdata{06:26}{17:40}{20:51}{09:02}{12:03}
{\kalas{04:44 05:35 09:26 08:41 10:11 16:11 10:56 13:11 15:26 16:56 18:32 20:52 22:28 01:39*}}}
{\tnykdata{\anga{\tithi{18}{कृष्ण-तृतीया}}{\time{2-18}{07:18}}\hspace{1ex}\anga{\tithi{19}{कृष्ण-चतुर्थी}}{\time{57-56}{05:34*}}\hspace{1ex}\avamA{}}%
{\prev{\anga{पुनर्वसुः}{\time{*56-39}{05:01}}}\hspace{1ex}\anga{पुष्यः}{\time{54-10}{03:58*}}\hspace{1ex}}{चन्द्रराशिः—\mbox{कर्कटः}}%
{\anga{ब्राह्मः}{\time{2-23}{07:20}}\hspace{1ex}\anga{माहेन्द्रः}{\time{55-56}{04:43*}}\hspace{1ex}\uanga{वैधृतिः}}%
{\anga{भद्रा}{\time{2-18}{07:18}}\hspace{1ex}\anga{बवम्}{\time{31-50}{18:28}}\hspace{1ex}\anga{बालवम्}{\time{57-56}{05:34*}}\hspace{1ex}\uanga{कौलवम्}}{}
}
{दिनक्षयः\eventsep \tamil{கார்த்திகை~ஞாயிற்றுக்கிழமை}\eventsep रविपुष्य-योगः\eventsep रविवार-आखुरथ-महागणपति सङ्कटहर-चतुर्थी-व्रतम्}
{Sun} 
\cfoot{\rygdata{16:16--17:40}{12:03--13:28}{14:52--16:16}}
\caldata{DECEMBER}{16}{\sunmonth{धनुः}{1}{\mbox{वृश्चिकः{\tiny\RIGHTarrow}{15:01}}}{मार्गशीर्षः}{हेमन्तऋतुः}{सोमः}{विकारी}{दक्षिणायनम्}{हेमन्तऋतुः}}
{\sunmoonsrdata{06:27}{17:41}{21:49}{09:53}{12:04}
{\kalas{04:44 05:36 09:26 08:42 10:11 16:11 10:56 13:11 15:26 16:56 18:32 20:52 22:28 01:40*}}}
{\tnykdata{\prev{\anga{\tithi{19}{कृष्ण-चतुर्थी}}{\time{*57-56}{05:34}}}\hspace{1ex}\anga{\tithi{20}{कृष्ण-पञ्चमी}}{\time{53-25}{03:39*}}\hspace{1ex}}%
{\prev{\anga{पुष्यः}{\time{*54-10}{03:58}}}\hspace{1ex}\anga{आश्रेषा}{\time{51-17}{02:45*}}\hspace{1ex}}{चन्द्रराशिः—\mbox{कर्कटः\RIGHTarrow{02:45*}}}%
{\prev{\anga{माहेन्द्रः}{\time{*55-56}{04:43}}}\hspace{1ex}\anga{वैधृतिः}{\time{49-26}{01:57*}}\hspace{1ex}\uanga{विष्कम्भः}}%
{\prev{\anga{बालवम्}{\time{*57-56}{05:34}}}\hspace{1ex}\anga{कौलवम्}{\time{27-11}{16:38}}\hspace{1ex}\anga{तैतिलम्}{\time{53-25}{03:39*}}\hspace{1ex}\uanga{गरजा}}{}
}
{धनूरवि-सङ्क्रमण-षडशीति-पुण्यकालः~15:01\RIGHTarrow{}17:41\eventsep रवि-सङ्क्रमण-पुण्यकालः~08:37\RIGHTarrow{}17:41\eventsep सङ्क्रमण-दिन-अपराह्ण-पुण्यकालः~12:04\RIGHTarrow{}17:41\eventsep वैधृति-श्राद्धम्}
{Mon} 
\cfoot{\rygdata{07:51--09:15}{10:40--12:04}{13:28--14:52}}
\caldata{DECEMBER}{17}{\sunmonth{धनुः}{2}{}{मार्गशीर्षः}{हेमन्तऋतुः}{मङ्गलः}{विकारी}{दक्षिणायनम्}{हेमन्तऋतुः}}
{\sunmoonsrdata{06:27}{17:41}{22:46}{10:41}{12:04}
{\kalas{04:45 05:36 09:27 08:42 10:12 16:12 10:57 13:12 15:27 16:56 18:32 20:53 22:29 01:40*}}}
{\tnykdata{\anga{\tithi{21}{कृष्ण-षष्ठी}}{\time{48-37}{01:37*}}\hspace{1ex}}%
{\anga{मघा}{\time{48-5}{01:23*}}\hspace{1ex}}{चन्द्रराशिः—\mbox{सिंहः}}%
{\anga{विष्कम्भः}{\time{42-40}{23:05}}\hspace{1ex}\uanga{प्रीतिः}}%
{\anga{गरजा}{\time{21-52}{14:39}}\hspace{1ex}\anga{वणिजा}{\time{48-37}{01:37*}}\hspace{1ex}\uanga{भद्रा}}{}
}
{धनुर्मास-उषःकाल-पूजारम्भः}
{Tue} 
\cfoot{\rygdata{14:53--16:17}{09:16--10:40}{12:04--13:28}}
\caldata{DECEMBER}{18}{\sunmonth{धनुः}{3}{}{मार्गशीर्षः}{हेमन्तऋतुः}{बुधः}{विकारी}{दक्षिणायनम्}{हेमन्तऋतुः}}
{\sunmoonsrdata{06:28}{17:42}{23:42}{11:27}{12:05}
{\kalas{04:45 05:37 09:27 08:43 10:12 16:12 10:57 13:12 15:27 16:57 18:33 20:53 22:29 01:41*}}}
{\tnykdata{\anga{\tithi{22}{कृष्ण-सप्तमी}}{\time{43-39}{23:31}}\hspace{1ex}}%
{\anga{पूर्वफल्गुनी}{\time{44-43}{23:58}}\hspace{1ex}}{चन्द्रराशिः—\mbox{सिंहः\RIGHTarrow{05:36*}}}%
{\anga{प्रीतिः}{\time{35-46}{20:09}}\hspace{1ex}\uanga{आयुष्मान्}}%
{\anga{भद्रा}{\time{16-18}{12:34}}\hspace{1ex}\anga{बवम्}{\time{43-39}{23:31}}\hspace{1ex}\uanga{बालवम्}}{}
}
{अनध्यायः\eventsep कुचेल-दिनम्\eventsep मार्गशीर्ष-अष्टका-पूर्वेद्युः}
{Wed} 
\cfoot{\rygdata{12:05--13:29}{07:52--09:16}{10:40--12:05}}
\caldata{DECEMBER}{19}{\sunmonth{धनुः}{4}{}{मार्गशीर्षः}{हेमन्तऋतुः}{गुरुः}{विकारी}{दक्षिणायनम्}{हेमन्तऋतुः}}
{\sunmoonsrdata{06:28}{17:42}{00:38*}{12:12}{12:05}
{\kalas{04:46 05:37 09:28 08:43 10:13 16:12 10:58 13:13 15:27 16:57 18:33 20:54 22:30 01:41*}}}
{\tnykdata{\anga{\tithi{23}{कृष्ण-अष्टमी}}{\time{38-37}{21:23}}\hspace{1ex}}%
{\anga{उत्तरफल्गुनी}{\time{41-19}{22:31}}\hspace{1ex}}{चन्द्रराशिः—\mbox{कन्या}}%
{\anga{आयुष्मान्}{\time{28-39}{17:12}}\hspace{1ex}\uanga{सौभाग्यः}}%
{\anga{बालवम्}{\time{10-36}{10:27}}\hspace{1ex}\anga{कौलवम्}{\time{38-37}{21:23}}\hspace{1ex}\uanga{तैतिलम्}}{}
}
{अनध्यायः\eventsep अनध्यायः\eventsep \tamil{இயற்பகை நாயன்மார் (3) குருபூஜை}\eventsep काञ्ची ४ जगद्गुरु श्री-सत्यबोधेन्द्र सरस्वती आराधना~\#{२२८७}\eventsep मार्गशीर्ष-अष्टका-श्राद्धम्\eventsep पञ्च-पर्व-पूजा (अष्टमी)}
{Thu} 
\cfoot{\rygdata{13:30--14:54}{06:28--07:53}{09:17--10:41}}
\caldata{DECEMBER}{20}{\sunmonth{धनुः}{5}{}{मार्गशीर्षः}{हेमन्तऋतुः}{शुक्रः}{विकारी}{दक्षिणायनम्}{हेमन्तऋतुः}}
{\sunmoonsrdata{06:29}{17:43}{01:34*}{12:56}{12:06}
{\kalas{04:47 05:38 09:28 08:44 10:13 16:13 10:58 13:13 15:28 16:58 18:34 20:54 22:30 01:42*}}}
{\tnykdata{\anga{\tithi{24}{कृष्ण-नवमी}}{\time{33-39}{19:16}}\hspace{1ex}}%
{\anga{हस्तः}{\time{37-58}{21:06}}\hspace{1ex}}{चन्द्रराशिः—\mbox{कन्या}}%
{\anga{सौभाग्यः}{\time{20-46}{14:16}}\hspace{1ex}\uanga{शोभनः}}%
{\anga{तैतिलम्}{\time{4-54}{08:19}}\hspace{1ex}\anga{गरजा}{\time{33-39}{19:16}}\hspace{1ex}\anga{वणिजा}{\time{59-25}{06:15*}}\hspace{1ex}\uanga{भद्रा}}{}
}
{अनध्यायः\eventsep मार्गशीर्ष-अन्वष्टका-श्राद्धम्}
{Fri} 
\cfoot{\rygdata{10:41--12:06}{14:54--16:19}{07:53--09:17}}
\caldata{DECEMBER}{21}{\sunmonth{धनुः}{6}{}{मार्गशीर्षः}{हेमन्तऋतुः}{शनिः}{विकारी}{दक्षिणायनम्}{हेमन्तऋतुः}}
{\sunmoonsrdata{06:29}{17:43}{02:30*}{13:41}{12:06}
{\kalas{04:47 05:38 09:29 08:44 10:14 16:13 10:59 13:14 15:28 16:58 18:34 20:55 22:31 01:42*}}}
{\tnykdata{\anga{\tithi{25}{कृष्ण-दशमी}}{\time{28-44}{17:15}}\hspace{1ex}}%
{\anga{चित्रा}{\time{34-49}{19:46}}\hspace{1ex}}{चन्द्रराशिः—\mbox{कन्या\RIGHTarrow{08:26}}}%
{\anga{शोभनः}{\time{13-2}{11:22}}\hspace{1ex}\uanga{अतिगण्डः}}%
{\prev{\anga{वणिजा}{\time{*59-25}{06:15}}}\hspace{1ex}\anga{भद्रा}{\time{28-44}{17:15}}\hspace{1ex}\anga{बवम्}{\time{54-48}{04:17*}}\hspace{1ex}\uanga{बालवम्}}{}
}
{}
{Sat} 
\cfoot{\rygdata{09:18--10:42}{13:31--14:55}{06:29--07:54}}
\caldata{DECEMBER}{22}{\sunmonth{धनुः}{7}{}{मार्गशीर्षः}{हेमन्तऋतुः}{भानुः}{विकारी}{दक्षिणायनम्}{हेमन्तऋतुः}}
{\sunmoonsrdata{06:30}{17:44}{03:28*}{14:28}{12:07}
{\kalas{04:48 05:39 09:29 08:45 10:14 16:14 10:59 13:14 15:29 16:59 18:35 20:55 22:31 01:43*}}}
{\tnykdata{\anga{\tithi{26}{कृष्ण-एकादशी}}{\time{23-41}{15:22}}\hspace{1ex}}%
{\anga{स्वाती}{\time{32-1}{18:35}}\hspace{1ex}}{चन्द्रराशिः—\mbox{तुला}}%
{\anga{अतिगण्डः}{\time{5-34}{08:35}}\hspace{1ex}\anga{सुकर्म}{\time{58-40}{05:57*}}\hspace{1ex}\uanga{धृतिः}}%
{\prev{\anga{बवम्}{\time{*54-48}{04:17}}}\hspace{1ex}\anga{बालवम्}{\time{23-41}{15:22}}\hspace{1ex}\anga{कौलवम्}{\time{50-35}{02:30*}}\hspace{1ex}\uanga{तैतिलम्}}{}
}
{अनध्यायः\eventsep गणितज्ञ-रामानुज-जन्म~\#{१३२}\eventsep \tamil{மானக்கஞ்சாற நாயன்மார் (12) குருபூஜை}\eventsep सायन-रवि-सङ्क्रमण-पुण्यकालः~06:30\RIGHTarrow{}16:13\eventsep सायन-सङ्क्रमण-दिन-पूर्वाह्ण-पुण्यकालः~06:30\RIGHTarrow{}12:07\eventsep सहोमास-उषःकाल-पूजा-समापनम्\eventsep सहस्य-मासः/उत्तरायणम्~09:49\RIGHTarrow{}\eventsep सर्व-सफला-एकादशी\eventsep त्रिपुष्कर-योगः~18:35\RIGHTarrow{}06:30*\eventsep उत्तरायणारम्भः\eventsep उत्तरायण-पुण्यकालः~09:49\RIGHTarrow{}17:44}
{Sun} 
\cfoot{\rygdata{16:20--17:44}{12:07--13:31}{14:55--16:20}}
\caldata{DECEMBER}{23}{\sunmonth{धनुः}{8}{}{मार्गशीर्षः}{हेमन्तऋतुः}{सोमः}{विकारी}{दक्षिणायनम्}{हेमन्तऋतुः}}
{\sunmoonsrdata{06:30}{17:44}{04:27*}{15:18}{12:07}
{\kalas{04:48 05:39 09:30 08:45 10:15 16:14 11:00 13:15 15:29 16:59 18:35 20:56 22:32 01:43*}}}
{\tnykdata{\anga{\tithi{27}{कृष्ण-द्वादशी}}{\time{19-11}{13:42}}\hspace{1ex}}%
{\anga{विशाखा}{\time{29-41}{17:37}}\hspace{1ex}}{चन्द्रराशिः—\mbox{तुला\RIGHTarrow{11:50}}}%
{\prev{\anga{सुकर्म}{\time{*58-40}{05:57}}}\hspace{1ex}\anga{धृतिः}{\time{52-56}{03:31*}}\hspace{1ex}\uanga{शूलः}}%
{\anga{तैतिलम्}{\time{19-11}{13:42}}\hspace{1ex}\anga{गरजा}{\time{46-57}{00:58*}}\hspace{1ex}\uanga{वणिजा}}{}
}
{काञ्ची ६८ जगद्गुरु श्री-चन्द्रशेखरेन्द्र सरस्वती ७ आराधना~\#{२६}\eventsep सोम-प्रदोष-व्रतम्~17:44\RIGHTarrow{}19:20}
{Mon} 
\cfoot{\rygdata{07:55--09:19}{10:43--12:07}{13:32--14:56}}
\caldata{DECEMBER}{24}{\sunmonth{धनुः}{9}{}{मार्गशीर्षः}{हेमन्तऋतुः}{मङ्गलः}{विकारी}{दक्षिणायनम्}{हेमन्तऋतुः}}
{\sunmoonsrdata{06:31}{17:45}{05:25*}{16:11}{12:08}
{\kalas{04:49 05:40 09:30 08:46 10:15 16:15 11:00 13:15 15:30 17:00 18:36 20:56 22:32 01:44*}}}
{\tnykdata{\anga{\tithi{28}{कृष्ण-त्रयोदशी}}{\time{15-28}{12:18}}\hspace{1ex}}%
{\anga{अनूराधा}{\time{27-51}{16:57}}\hspace{1ex}}{चन्द्रराशिः—\mbox{वृश्चिकः}}%
{\anga{शूलः}{\time{47-50}{01:20*}}\hspace{1ex}\uanga{गण्डः}}%
{\anga{वणिजा}{\time{15-28}{12:18}}\hspace{1ex}\anga{भद्रा}{\time{44-5}{23:45}}\hspace{1ex}\uanga{शकुनिः}}{}
}
{मासशिवरात्रिः\eventsep पञ्च-पर्व-पूजा (चतुर्दशी)}
{Tue} 
\cfoot{\rygdata{14:56--16:21}{09:19--10:43}{12:08--13:32}}
\caldata{DECEMBER}{25}{\sunmonth{धनुः}{10}{}{मार्गशीर्षः}{हेमन्तऋतुः}{बुधः}{विकारी}{दक्षिणायनम्}{हेमन्तऋतुः}}
{\sunmoonsrdata{06:31}{17:45}{06:22*}{17:06}{12:08}
{\kalas{04:49 05:40 09:31 08:46 10:16 16:15 11:01 13:16 15:30 17:00 18:36 20:57 22:33 01:44*}}}
{\tnykdata{\anga{\tithi{29}{कृष्ण-चतुर्दशी}}{\time{12-43}{11:17}}\hspace{1ex}}%
{\anga{ज्येष्ठा}{\time{27-1}{16:39}}\hspace{1ex}}{चन्द्रराशिः—\mbox{वृश्चिकः\RIGHTarrow{16:39}}}%
{\anga{गण्डः}{\time{43-28}{23:30}}\hspace{1ex}\uanga{वृद्धिः}}%
{\anga{शकुनिः}{\time{12-43}{11:17}}\hspace{1ex}\anga{चतुष्पात्}{\time{42-10}{22:56}}\hspace{1ex}\uanga{नाग}}{}
}
{अनध्यायः\eventsep काञ्ची १४ जगद्गुरु श्री-विद्याघनेन्द्र सरस्वती आराधना~\#{१७०३}\eventsep काञ्ची ३४ जगद्गुरु श्री-चन्द्रशेखरेन्द्र सरस्वती २ आराधना~\#{१३१०}\eventsep कमला-जयन्ती\eventsep ख्येचिमावस पोष्त/खिचडी-अमावास्या\eventsep पार्वणव्रतम् अमावास्यायाम्\eventsep पञ्च-पर्व-पूजा (अमावास्या)\eventsep सर्व-मार्गशीर्ष-अमावास्या\eventsep \tamil{தொண்டரடிப்பொடியாழ்வார் திருநக்ஷத்திரம்}}
{Wed} 
\cfoot{\rygdata{12:08--13:33}{07:56--09:20}{10:44--12:08}}
\caldata{DECEMBER}{26}{\sunmonth{धनुः}{11}{}{मार्गशीर्षः}{हेमन्तऋतुः}{गुरुः}{विकारी}{दक्षिणायनम्}{हेमन्तऋतुः}}
{\sunmoonsrdata{06:32}{17:46}{---}{18:02}{12:09}
{\kalas{04:50 05:41 09:31 08:47 10:16 16:16 11:01 13:16 15:31 17:01 18:37 20:57 22:33 01:45*}}}
{\tnykdata{\anga{\tithi{30}{अमावास्या}}{\time{11-9}{10:43}}\hspace{1ex}}%
{\anga{मूला}{\time{27-24}{16:48}}\hspace{1ex}}{चन्द्रराशिः—\mbox{धनुः}}%
{\anga{वृद्धिः}{\time{40-0}{22:01}}\hspace{1ex}\uanga{ध्रुवः}}%
{\anga{नाग}{\time{11-9}{10:43}}\hspace{1ex}\anga{किंस्तुघ्नः}{\time{41-23}{22:37}}\hspace{1ex}\uanga{बवम्}}{}
}
{अनध्यायः\eventsep अनध्यायः\eventsep दर्शेष्टिः\eventsep पार्वण-प्रायश्चित्तावकाशः दर्शे\eventsep पिण्ड-पितृ-यज्ञः\eventsep सूर्य-ग्रहणं (केतुग्रस्त)~08:08\RIGHTarrow{}11:19\eventsep दर्श-स्थालीपाकः\eventsep श्री-हनूमत्-जयन्ती}
{Thu} 
\cfoot{\rygdata{13:33--14:57}{06:32--07:56}{09:20--10:44}}
\caldata{DECEMBER}{27}{\sunmonth{धनुः}{12}{}{पौषः}{हेमन्तऋतुः}{शुक्रः}{विकारी}{दक्षिणायनम्}{हेमन्तऋतुः}}
{\sunmoonrsdata{06:32}{17:46}{07:16}{18:58}{12:09}
{\kalas{04:50 05:41 09:32 08:47 10:17 16:16 11:02 13:17 15:31 17:01 18:37 20:58 22:34 01:45*}}}
{\tnykdata{\anga{\tithi{1}{शुक्ल-प्रथमा}}{\time{10-59}{10:39}}\hspace{1ex}}%
{\anga{पूर्वाषाढा}{\time{29-9}{17:27}}\hspace{1ex}}{चन्द्रराशिः—\mbox{धनुः\RIGHTarrow{23:43}}}%
{\anga{ध्रुवः}{\time{37-31}{20:58}}\hspace{1ex}\uanga{व्याघातः}}%
{\anga{बवम्}{\time{10-59}{10:39}}\hspace{1ex}\anga{बालवम्}{\time{41-53}{22:50}}\hspace{1ex}\uanga{कौलवम्}}{}
}
{\tamil{சாக்கிய நாயன்மார் (34) குருபூஜை}\eventsep चन्द्र-दर्शनम्~17:46\RIGHTarrow{}18:57}
{Fri} 
\cfoot{\rygdata{10:45--12:09}{14:58--16:22}{07:56--09:21}}
\caldata{DECEMBER}{28}{\sunmonth{धनुः}{13}{}{पौषः}{हेमन्तऋतुः}{शनिः}{विकारी}{दक्षिणायनम्}{हेमन्तऋतुः}}
{\sunmoonrsdata{06:33}{17:47}{08:06}{19:51}{12:10}
{\kalas{04:51 05:42 09:32 08:48 10:17 16:17 11:02 13:17 15:32 17:02 18:38 20:58 22:34 01:46*}}}
{\tnykdata{\anga{\tithi{2}{शुक्ल-द्वितीया}}{\time{12-19}{11:10}}\hspace{1ex}}%
{\anga{उत्तराषाढा}{\time{32-6}{18:41}}\hspace{1ex}}{चन्द्रराशिः—\mbox{मकरः}}%
{\anga{व्याघातः}{\time{36-4}{20:22}}\hspace{1ex}\uanga{हर्षणः}}%
{\anga{कौलवम्}{\time{12-19}{11:10}}\hspace{1ex}\anga{तैतिलम्}{\time{43-45}{23:38}}\hspace{1ex}\uanga{गरजा}}{}
}
{त्रिपुष्कर-योगः~06:33\RIGHTarrow{}11:10}
{Sat} 
\cfoot{\rygdata{09:21--10:45}{13:34--14:58}{06:33--07:57}}
\caldata{DECEMBER}{29}{\sunmonth{धनुः}{14}{}{पौषः}{हेमन्तऋतुः}{भानुः}{विकारी}{दक्षिणायनम्}{हेमन्तऋतुः}}
{\sunmoonrsdata{06:33}{17:47}{08:52}{20:43}{12:10}
{\kalas{04:51 05:42 09:33 08:48 10:18 16:18 11:03 13:18 15:33 17:02 18:39 20:59 22:35 01:46*}}}
{\tnykdata{\anga{\tithi{3}{शुक्ल-तृतीया}}{\time{15-13}{12:15}}\hspace{1ex}}%
{\anga{श्रवणः}{\time{36-15}{20:27}}\hspace{1ex}}{चन्द्रराशिः—\mbox{मकरः}}%
{\anga{हर्षणः}{\time{35-40}{20:12}}\hspace{1ex}\uanga{वज्रम्}}%
{\anga{गरजा}{\time{15-13}{12:15}}\hspace{1ex}\anga{वणिजा}{\time{46-58}{01:01*}}\hspace{1ex}\uanga{भद्रा}}{}
}
{श्रवण-व्रतम्\eventsep शुक्ल-चतुर्थी-व्रतम्}
{Sun} 
\cfoot{\rygdata{16:23--17:47}{12:10--13:35}{14:59--16:23}}
\caldata{DECEMBER}{30}{\sunmonth{धनुः}{15}{}{पौषः}{हेमन्तऋतुः}{सोमः}{विकारी}{दक्षिणायनम्}{हेमन्तऋतुः}}
{\sunmoonrsdata{06:33}{17:48}{09:34}{21:31}{12:11}
{\kalas{04:51 05:42 09:33 08:48 10:18 16:18 11:03 13:18 15:33 17:03 18:39 21:00 22:35 01:47*}}}
{\tnykdata{\anga{\tithi{4}{शुक्ल-चतुर्थी}}{\time{19-36}{13:54}}\hspace{1ex}}%
{\anga{श्रविष्ठा}{\time{41-35}{22:44}}\hspace{1ex}}{चन्द्रराशिः—\mbox{मकरः\RIGHTarrow{09:32}}}%
{\anga{वज्रम्}{\time{36-11}{20:26}}\hspace{1ex}\uanga{सिद्धिः}}%
{\anga{भद्रा}{\time{19-36}{13:54}}\hspace{1ex}\anga{बवम्}{\time{51-24}{02:55*}}\hspace{1ex}\uanga{बालवम्}}{}
}
{}
{Mon} 
\cfoot{\rygdata{07:58--09:22}{10:46--12:11}{13:35--14:59}}
\caldata{DECEMBER}{31}{\sunmonth{धनुः}{16}{}{पौषः}{हेमन्तऋतुः}{मङ्गलः}{विकारी}{दक्षिणायनम्}{हेमन्तऋतुः}}
{\sunmoonrsdata{06:34}{17:49}{10:13}{22:19}{12:11}
{\kalas{04:52 05:43 09:34 08:49 10:19 16:19 11:04 13:19 15:34 17:04 18:40 21:00 22:36 01:47*}}}
{\tnykdata{\anga{\tithi{5}{शुक्ल-पञ्चमी}}{\time{25-13}{16:01}}\hspace{1ex}}%
{\anga{शतभिषक्}{\time{47-52}{01:25*}}\hspace{1ex}}{चन्द्रराशिः—\mbox{कुम्भः}}%
{\anga{सिद्धिः}{\time{37-28}{21:00}}\hspace{1ex}\uanga{व्यतीपातः}}%
{\anga{बालवम्}{\time{25-13}{16:01}}\hspace{1ex}\anga{कौलवम्}{\time{56-47}{05:13*}}\hspace{1ex}\uanga{तैतिलम्}}{}
}
{}
{Tue} 
\cfoot{\rygdata{15:00--16:24}{09:23--10:47}{12:11--13:36}}
\end{document}
